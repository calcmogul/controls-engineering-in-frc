\section{Curvilinear motion}

\subsection{Two-wheeled vehicle kinematics}

For this derivation, we'll assume positive $\hat{i}$ is forward, positive
$\hat{j}$ is to the right, and positive $\hat{k}$ is up and that the robot is
facing in the $\hat{i}$ direction.

The mapping from $v$ and $\omega$ to the left and right wheel velocities $v_L$
and $v_R$ are derived as follows. Let $\vec{v}_C$ be the velocity vector of the
center of rotation, $\vec{v}_L$ be the velocity vector of the left wheel,
$\vec{v}_R$ be the velocity vector of the right wheel, $r_b$ is the distance
from the center of rotation to each wheel, and $\omega$ is the counterclockwise
turning rate around the center of rotation.

The main equation we'll need is the following.

\begin{equation*}
  \vec{v}_B = \vec{v}_A + \omega_A \times \vec{r}_{B|A}
\end{equation*}

where $\vec{v}_B$ is the velocity vector at point B, $\vec{v}_A$ is the velocity
vector at point A, $\omega_A$ is the angular velocity vector at point A, and
$\vec{r}_{B|A}$ is the distance vector from point A to point B (also described
as the ``distance to B relative to A").

First, we'll derive $v_l$.

\begin{align}
  \vec{v}_l &= v_c \hat{i} + \omega \hat{k} \times r_b \hat{j} \nonumber \\
  \vec{v}_l &= v_c \hat{i} - \omega r_b \hat{i} \nonumber \\
  \vec{v}_l &= (v_c - \omega r_b) \hat{i} \nonumber \\
  \lvert\vec{v}_l\rvert &= v_c - \omega r_b \nonumber \\
  v_l &= v_c - \omega r_b \label{eq:diff_vl}
\end{align}

Next, we'll derive $v_r$.

\begin{align}
  \vec{v}_r &= v_c \hat{i} + \omega \hat{k} \times r_b \hat{j} \nonumber
    \\
  \vec{v}_r &= v_c \hat{i} + \omega r_b \hat{i} \nonumber \\
  \vec{v}_r &= (v_c + \omega r_b) \hat{i} \nonumber \\
  \lvert\vec{v}_r\rvert &= v_c + \omega r_b \nonumber \\
  v_r &= v_c + \omega r_b \label{eq:diff_vr}
\end{align}
