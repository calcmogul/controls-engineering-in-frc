\chapterimage{motion-planning.jpg}{OPERS field at UCSC}

\chapter{Configuration spaces}

\section{Introduction}

When a robot can interfere with itself, we need a motion profiling solution that
ensures the robot doesn't destroy itself. Now that we've demonstrated various
methods of motion profiling, how can we incorporate safe zones into the planning
process or implement mechanism collision avoidance?

1-DOF systems are straightforward. We define the minimum and maximum position
constraints for the arm, elevator, etc. and ensure we stay between them. To do
so, we might use some combination of:
\begin{enumerate}
  \item Rejecting, coercing, or clamping setpoints so that they are always
    within the valid range
  \item Soft limits on the speed controller
  \item Limit switches, and/or physical hard stops
\end{enumerate}

Multi-DOF systems are more complex. In the 2-DOF case (e.g., an arm with a
shoulder and a wrist, or an elevator and an arm), each DOF has some limits that
are independent of the rest of the system (e.g., min and max angles), but there
are also dependent constraints (e.g., if the shoulder is angled down, the wrist
must be angled up to avoid hitting the ground). Examples of such constraints
could include:
\begin{itemize}
  \item Minimum/maximum angles
  \item Maximum extension rules
  \item Robot chassis
  \item Angles that require an infeasible amount of holding torque
\end{itemize}

One intuitive way to think about this is to envision an N-D space (2D for 2
DOFs, etc.) where the $x$ and $y$ axes are the positions of the degrees of
freedom. A given point $(x, y)$ can either be a valid configuration or an
invalid configuration (meaning it violates one or more constraints). So, our 2D
plot may have regions (half-planes, rectangles, etc.) representing constraints.

\section{Waypoint planning}

Our first example has 2 DOFs where all of the constraints are independent
(minimum/maximum limits on each axis), as shown in figure
\ref{fig:config_spaces_1}.
\begin{svg}{build/\partpath/configuration_spaces_fig1}
  \caption{Elevator-arm configuration space with independent constraints}
  \label{fig:config_spaces_1}
\end{svg}

Motion planning and control in this example is simple; just make sure all
controller references are always inside the box of valid states.

Now let's make it more interesting. Say that if the arm is near the middle of
travel (pointing straight down), the elevator must be above a certain height.
Otherwise, the arm intersects the robot chassis. This configuration space would
look like figure \ref{fig:config_spaces_2}.
\begin{svg}{build/\partpath/configuration_spaces_fig2}
  \caption{Elevator-arm configuration space requiring elevator to be above a
    certain height when arm is pointing down}
  \label{fig:config_spaces_2}
\end{svg}

Let's consider motion planning in the second example. Say the current state of
our system is $x$ and our goal state is $r$ as shown in figure
\ref{fig:config_spaces_3}.
\begin{svg}{build/\partpath/configuration_spaces_fig3}
  \caption{Elevator-arm configuration space with initial state $x$ and goal
    state $r$}
  \label{fig:config_spaces_3}
\end{svg}

Notice that the elevator height at the current state and the goal is identical.
Do we just have to move the arm? Moving this way will sweep the system into the
constraint we defined for preventing self-intersection with the chassis.
Instead, we need to plan a sequence of references to execute that only pass
through valid states. Figure \ref{fig:config_spaces_4} shows an example.
\begin{svg}{build/\partpath/configuration_spaces_fig4}
  \caption{Elevator-arm configuration space with path between initial state $x$
    and final state $r$}
  \label{fig:config_spaces_4}
\end{svg}

$a$ and $b$ are intermediate waypoints we added to ensure that the whole system
moves in a safe manner. Considering again our toy example, the elevator goes up,
then the arm rotates underneath, then the elevator goes back down. There are
many possible ways to generate this path (or other valid paths), including some
sort of search (discretize the plot into a grid and find a valid path using
breadth-first search or A*) or hard-coded rules to handle different regions of
the constraint space.

\section{Waypoint traversal}

Once we have a sequence of waypoints, there are a few options for traversing
them:
\begin{enumerate}
  \item Generate trajectories for each DOF and time-synchronize them. This
    ensures the mechanism follows the ``diagonal lines" in the path above. The
    upside of this approach is it can yield the fastest (time-optimal)
    coordinated movements possible. The downside is that the motion planning is
    ``open loop" -- if one of the DOFs lags behind its trajectory, it's possible
    that the system may still end up in an invalid state. In practice, we
    include some buffer around our references to ensure that tracking error
    doesn’t immediately lead to violated constraints. We are making our
    trajectories more robust to uncertainty.
  \item Perform movements sequentially. That is, move from $x$ to $a$, and only
    start moving toward $b$ once the path from the current system state to $b$
    is a straight line that is collision-free. This gives robustness to control
    error, but may result in unnecessary stops or jerky movement compared to the
    optimized, open-loop trajectory.
\end{enumerate}

This planning approach generalizes to an arbitrary number of DOFs and is
essentially how industrial 7-DOF arms are operated.

\section{State machine validation with safe sets}

Configuration spaces can also be used for state machine validation. The states
of all the mechanisms, including solenoid actuation states, would be axes in a
configuration space. A state machine unit test would define a valid path through
the configuration space, send the desired events to the state machine, then
ensure the robot took the correct path in response to each event for all
iterations.
