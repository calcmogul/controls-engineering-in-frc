\section{Drivetrain motion planning software}

There's a broad spectrum of drivetrain motion planning methods suitable for
different team requirements and capabilities. On a tight schedule, consider
writing and testing a simple fallback method before attempting one of the more
complex and time-intensive methods discussed here.

\subsection{Model-free}

Model-free path planning methods use geometric relationships to enforce
convergence to a desired pose. They're ideal for those who want to avoid system
modeling and don't care about the exact path the robot takes between two points.
Methods in this category include pure
pursuit\footnote{\url{https://file.tavsys.net/control/pure-pursuit-considered-harmful.pdf}}
and guiding vector
fields\footnote{\url{https://research.rug.nl/en/publications/guiding-vector-fields-for-robot-motion-control/}}.

Model-free methods satisfy the needs of most teams and save them tuning time,
but they do impose a performance ceiling; higher performance necessarily
requires reasoning about the robot's dynamics and physical constraints.

\subsection{Some modeling required}

By investing a bit of effort into system modeling, one can obtain trajectories
with reasonable default performance and limited feasibility guarantees.

PathPlanner\footnote{\url{https://pathplanner.dev/home.html}} is in this
category. PathPlanner plans the shape of the path with a series of cubic or
quintic Hermite splines, time-parameterizes each spline with a chassis velocity
trapezoid profile, then iteratively constrains the profile based on differential
or swerve drivetrain kinematics and user-defined velocity and acceleration
limits. Trajectories can be replanned on-demand (useful for auto-aim/auto-score
strategies).

\textit{A Dive into WPILib Trajectories} by Declan
Freeman-Gleason\footnote{\url{https://pietroglyph.github.io/trajectory-presentation/}}
describes 2-DOF trajectory planning with Hermite splines in more detail.
\textit{Planning Motion Trajectories for Mobile Robots Using Splines} by
Christoph Sprunk provides a more general treatment of spline-based trajectory
generation\footnote{\url{http://www2.informatik.uni-freiburg.de/~lau/students/Sprunk2008.pdf}}.

\subsection{High-fidelity modeling}

High-fidelity modeling allows us to use more sophisticated mathematical tools to
achieve peak drivetrain performance, sometimes at the expense of
\gls{robustness}. Successful applications require accurate modeling and
sufficient control input margin to compensate for disturbances. Replanning after
large disturbances is theoretically possible but not always computationally
tractible.

Choreo\footnote{\url{https://choreo.autos/}} is in this category. Choreo poses
and solves a nonlinear constrained optimization problem (see chapter
\ref{ch:trajectory_optimization}). It finds the time-minimizing trajectory
through a set of waypoints subject to differential or swerve drivetrain
dynamics, control input limits, velocity and acceleration limits, keep-in and
keep-out regions, and pointing constraints. The dynamics are derived via
sum-of-forces and sum-of-torques with user-provided drivetrain geometry and mass
properties. Most constraints can be applied to either the whole trajectory or
specific segments.

Choreo requires a lot of physical information about the drivetrain, but it tends
to generate feasible trajectories that robots can execute well on their first
try (assuming well-tuned wheel velocity controllers). These trajectories are too
expensive to replan on-demand.
