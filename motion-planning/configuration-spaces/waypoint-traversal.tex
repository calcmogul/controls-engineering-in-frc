\section{Waypoint traversal}

Once we have a sequence of waypoints, there are a few options for traversing
them:
\begin{enumerate}
  \item Generate trajectories for each DOF and time-synchronize them. This
    ensures the mechanism follows the ``diagonal lines" in the path above. The
    upside of this approach is it can yield the fastest (time-optimal)
    coordinated movements possible. The downside is that the motion planning is
    ``open loop" -- if one of the DOFs lags behind its trajectory, it's possible
    that the system may still end up in an invalid state. In practice, we
    include some buffer around our references to ensure that tracking error
    doesn't immediately lead to violated constraints. We are making our
    trajectories more robust to uncertainty.
  \item Perform movements sequentially. That is, move from $x$ to $a$, and only
    start moving toward $b$ once the path from the current system state to $b$
    is a straight line that is collision-free. This gives robustness to control
    error, but may result in unnecessary stops or jerky movement compared to the
    optimized, open-loop trajectory.
\end{enumerate}

This planning approach generalizes to an arbitrary number of DOFs and is
essentially how industrial 7-DOF arms are operated.
