\section{Introduction}

When a robot can interfere with itself, we need a motion profiling solution that
ensures the robot doesn't destroy itself. Now that we've demonstrated various
methods of motion profiling, how can we incorporate safe zones into the planning
process or implement mechanism collision avoidance?

1-DOF systems are straightforward. We define the minimum and maximum position
constraints for the arm, elevator, etc. and ensure we stay between them. To do
so, we might use some combination of:
\begin{enumerate}
  \item Rejecting, coercing, or clamping setpoints so that they are always
    within the valid range
  \item Soft limits on the speed controller
  \item Limit switches, and/or physical hard stops
\end{enumerate}

Multi-DOF systems are more complex. In the 2-DOF case (e.g., an arm with a
shoulder and a wrist, or an elevator and an arm), each DOF has some limits that
are independent of the rest of the system (e.g., min and max angles), but there
are also dependent constraints (e.g., if the shoulder is angled down, the wrist
must be angled up to avoid hitting the ground). Examples of such constraints
could include:
\begin{itemize}
  \item Minimum/maximum angles
  \item Maximum extension rules
  \item Robot chassis
  \item Angles that require an infeasible amount of holding torque
\end{itemize}

One intuitive way to think about this is to envision an N-D space (2D for 2
DOFs, etc.) where the $x$ and $y$ axes are the positions of the degrees of
freedom. A given point $(x, y)$ can either be a valid configuration or an
invalid configuration (meaning it violates one or more constraints). So, our 2D
plot may have regions (half-planes, rectangles, etc.) representing constraints.
