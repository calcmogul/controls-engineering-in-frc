\chapterimage{notes-to-the-reader.jpg}

\setcounter{chapter}{-1}
\chapter{Notes to the reader}

\section{The structure of this book}

This book consists of three parts and a collection of appendices:

\begin{itemize}
  \item Part I, "Classical control," introduces the basics of control theory,
    reviews the dynamics of PID controllers, describes what a transfer function
    is, and shows how they can be used to analyze dynamical systems. Emphasis is
    placed on the geometric intuition of this analysis rather than the frequency
    domain math.
  \item Part II, "Modern control," builds on the intuition gained in part I to
    describe and control multiple-input, multiple-output (MIMO) systems. It
    provides a crash course in the geometric intuition behind linear algebra and
    covers enough of the mechanics of evaluating matrix algebra for the reader
    to follow along in later chapters. State-space representation, canonical
    forms of state-space models, discretization, LQR controller design, and LQE
    observer design are covered.
  \item Part III, "Controls design/implementation," describes how to apply the
    concepts learned in the earlier parts to design and implement controllers
    for real systems. It walks through several examples of common FRC subsystems
    from deriving the model using kinematics to implementing and testing a
    digital controller. Finally, Kalman filters are implemented and the models
    are augmented with $u_{error}$ estimators to help handle model uncertainty.
  \item The appendices provide further enrichment that isn't required for a
    passing understanding of the material. This includes derivations for many of
    the results presented and used in the main body of the book.
\end{itemize}

This book is intended as both a tutorial for new students and as a reference
manual for more experienced readers who need to review a thing or two. While it
isn't comprehensive, the reader will hopefully learn enough to either implement
these concepts themselves or know where to look for more information.

The sections on classical control theory are intended to provide a geometric
intuition into the mathematical machinery of modern control theory. Some topics
have been oversimplified to make them easier to grasp. For more detail, please
see the Wikibook on control systems at
\url{https://en.wikibooks.org/wiki/Control_Systems}.

\section{Request for feedback}

While we have tried to write a book that makes the topics of control theory
approachable, it's still very high level for the subject it covers as well as
very dense and fast-paced (it covers three classes of feedback control, two of
which are for graduate students, in one short book). Please send us feedback,
corrections, or suggestions through the GitHub link listed on the copyright
page. New examples that demonstrate key concepts and make them more accessible
are also appreciated.
