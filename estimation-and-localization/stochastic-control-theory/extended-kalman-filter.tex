\section{Extended Kalman filter}
\label{sec:ekf}
\index{nonlinear control!extended Kalman filter}
\index{state-space observers!Extended Kalman filter}

In this book, we have covered the Kalman filter, which is the optimal unbiased
estimator for linear \glspl{system}. It isn't optimal for nonlinear
\glspl{system}, but several extensions to it have been developed to make it more
accurate.

The extended Kalman filter \glslink{linearization}{linearizes} the matrices used
during the prediction step. $\mat{A}$, $\mat{B}$, $\mat{C}$, and $\mat{D}$ are
\glslink{linearization}{linearized} as follows:
\begin{align*}
  \mat{A} &\approx
    \frac{\partial f(\mat{x}, \mat{u})}{\partial \mat{x}} &
  \mat{B} &\approx
    \frac{\partial f(\mat{x}, \mat{u})}{\partial \mat{u}} &
  \mat{C} &\approx
    \frac{\partial h(\mat{x}, \mat{u})}{\partial \mat{x}} &
  \mat{D} &\approx
    \frac{\partial h(\mat{x}, \mat{u})}{\partial \mat{u}}
\end{align*}

From there, the continuous Kalman filter equations are used like normal to
compute the error covariance matrix $\mat{P}$ and Kalman gain matrix. The
\gls{state} estimate update can still use the function $h(\mat{x})$ for
accuracy.
\begin{equation*}
  \hat{\mat{x}}_{k+1}^+ = \hat{\mat{x}}_{k+1}^- +
    \mat{K}_{k+1}(\mat{y}_{k+1} - h(\hat{\mat{x}}_{k+1}^-))
\end{equation*}
