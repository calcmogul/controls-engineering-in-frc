\section{Unscented Kalman filter}
\label{sec:ukf}
\index{nonlinear control!Unscented Kalman filter}
\index{state-space observers!Unscented Kalman filter}

In this book, we have covered the Kalman filter, which is the optimal unbiased
estimator for linear \glspl{system}. One method for extending it to nonlinear
systems is the unscented Kalman filter.

The unscented Kalman filter (UKF) propagates carefully chosen points called
sigma points through the nonlinear state and measurement models to obtain
estimates of the true covariances (as opposed to linearized versions of them).
Read \textit{Kalman and Bayesian Filters in Python} by Roger
Labbe\footnote{\url{https://github.com/rlabbe/Kalman-and-Bayesian-Filters-in-Python/blob/master/10-Unscented-Kalman-Filter.ipynb}} or the original paper\footfullcite{bib:ukf} for more on UKFs.

Here's an interview about the origin of the UKF with its
creator.\footnote{\url{https://ethw.org/First-Hand:The_Unscented_Transform}}

\subsection{Sigma point selection}

There's several selection heuristics for sigma points. The original UKF paper
uses Merwe scaled sigma points,\footfullcite{bib:ukf_merwe_scaled_sigma_points}
but Scaled Spherical Simplex sigma points require fewer points for the same
accuracy.\footfullcite{bib:ukf_s3_sigma_points}

\subsection{Square-root UKF}

The UKF uses a matrix square root (Cholesky decomposition) to compute the
scaling for the sigma points. This operation can introduce numerical
instability. The square-root UKF\footfullcite{bib:ukf_square_root} avoids this
by propagating the square-root of the error covariance directly instead of the
error covariance; the Cholesky decomposition is replaced with a QR decomposition
and a Cholesky rank-1 update.

\subsection{Case study: orientation tracking}

Here's a paper on a quaternion-based Unscented Kalman filter for orientation
tracking.\footfullcite{bib:ukf_state_tracking}
