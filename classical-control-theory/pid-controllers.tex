\chapterimage{pid-controllers.jpg}{On trail between McHenry Library and Media Theater at UCSC}

\chapter{PID controllers}
\index{PID control}

The PID controller is a commonly used feedback controller consisting of
proportional, integral, and derivative terms, hence the name. This chapter will
build up the definition of a PID controller term by term while trying to provide
intuition for how each of them behaves.

For those already familiar with PID control, this book's interpretation won't be
consistent with the classical intuition of "past", "present", and "future"
error. We will be approaching it from the viewpoint of modern control theory
with proportional controllers applied to different physical quantities we care
about. This will provide a more complete explanation of the derivative term's
behavior for constant and moving \glspl{setpoint}, and this intuition will carry
over to the modern control methods covered later in this book.

For PID controllers, the \gls{reference} is called the \gls{setpoint} (the
desired position) and the \gls{output} is called the \gls{process variable} (the
measured position). Below are some common variable naming conventions for
relevant quantities.

\begin{figurekey}
  \begin{tabular}{llll}
    $r(t)$ & \gls{setpoint} & $u(t)$ & \gls{control input} \\
    $e(t)$ & \gls{error} & $y(t)$ & \gls{output}
  \end{tabular}
\end{figurekey}

The \gls{error} $e(t)$ is $r(t) - y(t)$.

\renewcommand*{\chapterpath}{\partpath/pid-controllers}
\section{Proportional gain}

The \textit{Proportional} term drives the position error to zero.

\begin{definition}[Proportional controller]
  \begin{equation}
    u(t) = K_p e(t)
  \end{equation}

  where $K_p$ is the proportional gain and $e(t)$ is the error at the current
  time $t$.
\end{definition}

Figure \ref{fig:p_ctrl_diag} shows a block diagram for a \gls{system}
controlled by a P controller.

\begin{bookfigure}
  \begin{tikzpicture}[auto, >=latex']
    \fontsize{9pt}{10pt}

    % Place the blocks
    \node [name=input] {$r(t)$};
    \node [sum, right=0.5cm of input] (errorsum) {};
    \node [coordinate, right=0.75cm of errorsum] (branch) {};
    \node [block, right=0.5cm of branch] (I) { $K_p e(t)$ };
    \node [coordinate, right=0.5cm of I] (ctrlsum) {};
    \node [block, right=0.75cm of ctrlsum] (plant) {Plant};
    \node [right=0.75cm of plant] (output) {};
    \node [coordinate, below=0.5cm of I] (measurements) {};

    % Connect the nodes
    \draw [arrow] (input) -- node[pos=0.9] {$+$} (errorsum);
    \draw [-] (errorsum) -- node {$e(t)$} (branch);
    \draw [arrow] (branch) -- (I);
    \draw [arrow] (I) -- node {$u(t)$} (plant);
    \draw [arrow] (plant) -- node [name=y] {$y(t)$} (output);
    \draw [-] (y) |- (measurements);
    \draw [arrow] (measurements) -| node[pos=0.99, right] {$-$} (errorsum);
  \end{tikzpicture}

  \caption{P controller block diagram}
  \label{fig:p_ctrl_diag}
\end{bookfigure}

Proportional gains act like ``software-defined springs" that pull the
\gls{system} toward the desired position. Recall from physics that we model
springs as $F = -kx$ where $F$ is the force applied, $k$ is a proportional
constant, and $x$ is the displacement from the equilibrium point. This can be
written another way as $F = k(0 - x)$ where $0$ is the equilibrium point.
If we let the equilibrium point be our feedback controller's \gls{setpoint}, the
equations have a one-to-one correspondence.

\begin{align*}
  F &= k(r - x) \\
  u(t) &= K_p e(t) = K_p(r(t) - y(t))
\end{align*}

so the ``force" with which the proportional controller pulls the \gls{system}'s
\gls{output} toward the \gls{setpoint} is proportional to the \gls{error}, just
like a spring.

\input{\chapterpath/derivative-gain}
\input{\chapterpath/integral-gain}
\section{PID controller definition}

When these three terms are combined, one gets the typical definition for a PID
controller.

\begin{definition}[PID controller]
  \begin{equation}
    u(t) = K_p e(t) + K_i \int_0^t e(\tau) \,d\tau + K_d \frac{de}{dt}
  \end{equation}

  where $K_p$ is the proportional gain, $K_i$ is the integral gain, $K_d$ is the
  derivative gain, $e(t)$ is the error at the current time $t$, and $\tau$ is
  the integration variable.
\end{definition}

Figure \ref{fig:pid_ctrl_diag} shows a block diagram for a \gls{system}
controlled by a PID controller.

\begin{bookfigure}
  \begin{tikzpicture}[auto, >=latex']
    \fontsize{9pt}{10pt}

    % Place the blocks
    \node [name=input] {$r(t)$};
    \node [sum, right=0.5cm of input] (errorsum) {};
    \node [coordinate, right=0.75cm of errorsum] (branch) {};
    \node [block, right=0.5cm of branch] (I) { $K_i \int_0^t e(\tau) \,d\tau$ };
    \node [block, above=0.5cm of I] (P) { $K_p e(t)$ };
    \node [block, below=0.5cm of I] (D) { $K_d \frac{de(t)}{dt}$ };
    \node [sum, right=0.5cm of I] (ctrlsum) {};
    \node [block, right=0.75cm of ctrlsum] (plant) {Plant};
    \node [right=0.75cm of plant] (output) {};
    \node [coordinate, below=0.5cm of D] (measurements) {};

    % Connect the nodes
    \draw [arrow] (input) -- node[pos=0.9] {$+$} (errorsum);
    \draw [-] (errorsum) -- node {$e(t)$} (branch);
    \draw [arrow] (branch) |- (P);
    \draw [arrow] (branch) -- (I);
    \draw [arrow] (branch) |- (D);
    \draw [arrow] (P) -| node[pos=0.95, left] {$+$} (ctrlsum);
    \draw [arrow] (I) -- node[pos=0.9, below] {$+$} (ctrlsum);
    \draw [arrow] (D) -| node[pos=0.95, right] {$+$} (ctrlsum);
    \draw [arrow] (ctrlsum) -- node {$u(t)$} (plant);
    \draw [arrow] (plant) -- node [name=y] {$y(t)$} (output);
    \draw [-] (y) |- (measurements);
    \draw [arrow] (measurements) -| node[pos=0.99, right] {$-$} (errorsum);
  \end{tikzpicture}

  \caption{PID controller block diagram}
  \label{fig:pid_ctrl_diag}
\end{bookfigure}

\section{Response types}

A \gls{system} driven by a PID controller generally has three types of
responses: underdamped, overdamped, and critically damped. These are shown in
figure \ref{fig:pid_responses}.
\begin{svg}{build/\chapterpath/pid_responses}
  \caption{PID controller response types}
  \label{fig:pid_responses}
\end{svg}

For the \glspl{step response} in figure \ref{fig:pid_responses}, \gls{rise time}
is the time the \gls{system} takes to initially reach the \gls{reference} after
applying the \gls{step input}. \Gls{settling time} is the time the \gls{system}
takes to settle at the \gls{reference} after the \gls{step input} is applied.

An \textit{underdamped} response oscillates around the \gls{reference} before
settling. An \textit{overdamped} response is slow to rise and does not overshoot
the \gls{reference}. A \textit{critically damped} response has the shortest
\gls{rise time} without oscillating around the \gls{reference} (i.e.,
overshooting then undershooting).

\section{Manual tuning}

These steps apply to position PID controllers. Velocity PID controllers
typically don't need $K_d$.

\begin{enumerate}
  \item Set $K_p$, $K_i$, and $K_d$ to zero.
  \item Increase $K_p$ until the \gls{output} starts to oscillate around the
    \gls{setpoint}.
  \item Increase $K_d$ as much as possible without introducing jittering in the
    \gls{system response}.
\end{enumerate}

If the \gls{setpoint} follows a trapezoidal motion profile (see chapter
\ref{ch:1_dof_motion_profiles}), tuning becomes a lot easier. Plot the position
\gls{setpoint}, velocity \gls{setpoint}, measured position, and measured
velocity. The velocity \gls{setpoint} can be obtained by numerical integration
$v_{des,k} = \frac{r_k - r_{k-1}}{\Delta t}$. Increase $K_p$ until the position
tracks well, then increase $K_d$ until the velocity tracks well.

If the \gls{controller} settles at an \gls{output} above or below the
\gls{setpoint}, one can increase $K_i$ such that the \gls{controller} reaches
the \gls{setpoint} in a reasonable amount of time. However, a steady-state
feedforward is strongly preferred over integral control (especially for velocity
PID control).

\begin{remark}
  \textit{Note:} Adding an integral gain to the \gls{controller} is an incorrect
  way to eliminate \gls{steady-state error}. A better approach would be to tune
  it with an integrator added to the \gls{plant}, but this requires a
  \gls{model}. Since we are doing output-based rather than model-based control,
  our only option is to add an integrator to the \gls{controller}.
\end{remark}

Beware that if $K_i$ is too large, integral windup can occur. Following a large
change in \gls{setpoint}, the integral term can accumulate an error larger than
the maximal \gls{control input}. As a result, the system overshoots and
continues to increase until this accumulated error is unwound.

\section{Actuator saturation}
\index{Controller design!actuator saturation}

Recall that a controller calculates its output based on the error between the
\gls{reference} and the current \gls{state}. \Gls{plant} in the real world don't
have unlimited control authority available for the controller to apply. When the
actuator limits are reached, the controller acts as if the gain has been
temporarily reduced.

We'll try to explain this through a bit of math. Let's say we have a controller
$u = k(r - x)$ where $u$ is the \gls{control effort}, $k$ is the gain, $r$ is
the \gls{reference}, and $x$ is the current \gls{state}. Let $u_{max}$ be the
limit of the actuator's output which is less than the uncapped value of $u$ and
$k_{max}$ be the associated maximum gain. We will now compare the capped and
uncapped controllers for the same \gls{reference} and current \gls{state}.

\begin{align*}
  u_{max} &< u \\
  k_{max}(r - x) &< k(r - x) \\
  k_{max} &< k
\end{align*}

For the inequality to hold, $k_{max}$ must be less than the original value for
$k$. This reduced gain is evident in a \gls{system response} when there is a
linear change in state instead of an exponential one as it approaches the
\gls{reference}. This is due to the \gls{control effort} no longer following a
decaying exponential plot. Once the \gls{system} is closer to the
\gls{reference}, the controller will stop saturating and produce realistic
controller values again.

\section{Limitations}

PID's heuristic method of tuning is a reasonable choice when there is no
\textit{a priori} knowledge of the \gls{system} dynamics. However, controllers
with much better response can be developed if a \glslink{model}{dynamical model}
of the \gls{system} is known. Furthermore, PID only applies to single-input,
single-output (SISO) \glspl{system}; we'll cover methods for multiple-input,
multiple-output (MIMO) control in part \ref{part:modern_control_theory} of this
book.

