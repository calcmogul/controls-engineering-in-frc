\chapterimage{transfer-functions.jpg}{Road next to Stevenson Academic building at UCSC}

\chapter{Transfer functions}

\section{What is a transfer function?}

The Laplace domain is a two-dimensional coordinate system whose coordinates are
represented by a complex number $s = \sigma + j\omega$. The real part $\sigma$
cooresponds to the x-axis and the imaginary part $j\omega$ cooresponds to the
y-axis.

A transfer function maps an input coordinate to an output coordinate in the
Laplace domain. These can be obtained by applying the Laplace transform to a
differential equation and rearranging the terms to obtain a ratio of the output
variable to the input variable. Equation (\ref{eq:transfer_func}) is an example
of a transfer function.

\begin{equation} \label{eq:transfer_func}
  H(s) = \frac{(s-9+9i)(s-9-9i)}{s(s+10)}
\end{equation}

The factors of the numerator and denominator of a transfer function are called
residues. The roots of residues in the numerator are called zeroes while the
roots of residues in the denominator are called poles. We call them zeroes
because they make the residue approach zero, which makes the transfer function
also approach zero. Likewise, poles are called such because they make the
residue approach zero, which makes the transfer function approach infinity. On a
3D graph, they look like the poles of a circus tent.

Imaginary poles and zeroes always come in complex conjugate pairs (e.g.,
$-2 + 3i$, $-2 - 3i$).

\section{Transfer functions in feedback}

For \glspl{controller} to \glslink{regulator}{regulate} a system or
\glslink{tracking}{track} a reference, they must be placed in positive or
negative feedback with the \gls{plant} (whether to use positive or negative
depends on the \gls{plant} in question). Given the following block diagram

\begin{figure}[H]
  \centering

  \begin{tikzpicture}[auto, >=latex']
    % Place the blocks
    \node [name=input] {$X(s)$};
    \node [sum, right=of input] (sum) {};
    \node [block, right=of sum] (K) {$K$};
    \node [block, right=of K] (G) {$G$};
    \node [right=of G] (output) {$Y(s)$};
    \node [block, below=of $(K)!0.5!(G)$] (H) {$H$};

    % Connect the nodes
    \draw [arrow] (input) -- node[pos=0.85] {$+$} (sum);
    \draw [arrow] (sum) -- node {} (K);
    \draw [arrow] (K) -- node {} (G);
    \draw [arrow] (G) -- node[name=y] {} (output);
    \draw [arrow] (y) |- (H);
    \draw [arrow] (H) -| node[pos=0.97, right] {$-$} (sum);
  \end{tikzpicture}

  \caption{Feedback controller block diagram}
  \label{fig:feedback_controller_block_diagram}
\end{figure}

where $K$ is the controller gain, $G$ is the plant transfer function, and
$H$ is the measurement transfer function. The transfer function of figure
\ref{fig:feedback_controller_block_diagram}, a control system diagram with
feedback, from input to output is

\begin{equation}
  G_{cl}(s) = \frac{Y(s)}{X(s)} = \frac{KG}{1 + KGH}
\end{equation}

The numerator is the \gls{open-loop gain} and the denominator is one plus the
gain around the feedback loop, which may include parts of the
\gls{open-loop gain} (see appendix \ref{ch:app-tf-feedback-deriv} for a
derivation). As another example, the transfer function from the input to the
error is

\begin{equation}
  G_{cl}(s) = \frac{E(s)}{X(s)} = \frac{1}{1 + KGH}
\end{equation}

The roots of the denominator of $G_{cl}(s)$ are different from those of the
open-loop transfer function $KG(s)$. These are called the closed-loop poles.
