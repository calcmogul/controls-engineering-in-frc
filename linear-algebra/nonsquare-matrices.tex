\section{Nonsquare matrices as transformations between dimensions}

When we've talked about linear transformations so far, we've only really talked
about transformations from 2D vectors to other 2D vectors, represented with
$2 \times 2$ matrices; or from 3D vectors to other 3D vectors, represented with
$3 \times 3$ matrices. What about nonsquare matrices? We'll take a moment to
discuss what those mean geometrically.

By now, you have most of the background you need to start pondering a question
like this on your own, but we'll start talking through it, just to give a little
mental momentum.

It's perfectly reasonable to talk about transformations between dimensions, such
as one that takes 2D vectors to 3D vectors. Again, what makes one of these
linear is that grid lines remain parallel and evenly spaced, and that the origin
maps to the origin.

Encoding one of these transformations with a matrix the same as what we've done
before. You look at where each basis vector lands and write the coordinates of
the landing spots as the coordinates of the landing spots as the columns of a
matrix. For example, the following is a transformation that takes $\hat{i}$ to
the coordinates $(2, -1, -2)$ and $\hat{j}$ to the coordinates $(0, 1, 1)$.

\begin{equation*}
  \begin{bmatrix}
    2 & 0 \\
    -1 & 1 \\
    -2 & 1
  \end{bmatrix}
\end{equation*}

Notice, this means the matrix encoding our transformation has three rows and two
columns, which, to use standard terminology, makes it a $3 \times 2$ matrix. In
the language of last section, the column space of this matrix, the place where
all the vectors land, is a 2D plane slicing through the origin of 3D space. The
matrix is still full rank since the number of dimensions in this column space is
the same as the number of dimensions of the input space.

If you see a $3 \times 2$ matrix out in the wild, you can know that it has the
geometric interpretation of mapping two dimensions to three dimensions since the
two columns indicate that the input space has two basis vectors, and the three
rows indicate that the landing spots for each of those basis vectors is
described with three separate coordinates.

For a $2 \times 3$ matrix, the three columns indicate a starting space that has
three basis vectors, so it starts in three dimensions; and the two rows indicate
that the landing spot for each of those three basis vectors is described with
only two coordinates, so they must be landing in two dimensions. It's a
transformation from 3D space onto the 2D plane.

You could also have a transformation from two dimensions to one dimension.
One-dimensional space is really just the number line, so a transformation like
this takes in 2D vectors and returns numbers. Thinking about gridlines remaining
parallel and evenly spaced is messy due to all the compression happening here,
so in this case, the visual understanding for what linearity means is that if
you have a line of evenly spaced dots, it would remain evenly spaced once
they're mapped onto the number line.

One of these transformations is encoded with a $1 \times 2$ matrix, each of
whose two columns has just a single entry. The two columns represent where the
basis vectors land, and each one of those columns requires just one number, the
number that that basis vector landed on.

\begin{remark}
  See the corresponding \textit{Essence of Linear Algebra} video for a more
  visual presentation (4 minutes)
  \cite{bib:linalg_nonsquare_matrices_as_transformations_between_dimensions}.
\end{remark}
