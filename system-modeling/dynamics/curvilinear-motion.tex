\section{Curvilinear motion}

Curvilinear motion describes the motion of an object along a fixed curve. This
motion has both linear and angular components. For derivations involving
curvilinear motion, we'll assume positive $x$ ($\hat{i}$) is forward, positive
$y$ ($\hat{j}$) is to the left, positive $z$ ($\hat{k}$) is up, and the robot is
facing in the $x$ direction. This axes convention is known as North-West-Up
(NWU), and is shown in figure \ref{fig:nwu_axes_convention}.
\begin{bookfigure}
  \begin{tikzpicture}[auto, >=latex']
    %\draw [help lines] (-4,-3) grid (4,3);

    % Main axes
    \draw[<->] (4,0) -- (-4,0) node[below] {\small +y};
    \draw[<->] (0,-3) -- (0,3) node[right] {\small +x};
  \end{tikzpicture}

  \caption{2D projection of North-West-Up (NWU) axes convention. The positive
    $z$ axis is pointed out of the page toward the reader.}
  \label{fig:nwu_axes_convention}
\end{bookfigure}

The main equation we'll need is the following.
\begin{equation*}
  \vec{v}_B = \vec{v}_A + \omega_A \times \vec{r}_{B|A}
\end{equation*}

where $\vec{v}_B$ is the velocity vector at point B, $\vec{v}_A$ is the velocity
vector at point A, $\omega_A$ is the angular velocity vector at point A, and
$\vec{r}_{B|A}$ is the distance vector from point A to point B (also described
as the ``distance to B relative to A").
