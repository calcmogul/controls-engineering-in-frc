\section{Swerve drive kinematics}

A swerve drive has an arbitrary number of wheels which can rotate in place
independent of the chassis. The forces they generate are shown in figure
\ref{fig:swerve_drive_fbd}.
\begin{bookfigure}
  \begin{tikzpicture}[auto, >=latex']
    %\draw [help lines] (-4,-3) grid (4,3);

    % Coordinate axes
    \begin{scope}[xshift=2cm,yshift=-2cm]
      \draw[->] (0,0) -- (-1,0) node[below] {\small +$y$};
      \draw[->] (0,0) -- (0,1) node[right] {\small +$x$};
    \end{scope}

    % Chassis
    \filldraw[draw=black,fill=white] (-1,-1) rectangle (1,1);

    % Front-left wheel
    \begin{scope}[xshift=-0.75cm,yshift=0.75cm]
      % Wheel vector
      \draw[->,thick,rotate around={-45:(0,0)}] (0,0) -- (0,1)
        node[left] {\small $v_1$};

      % Wheel
      \draw[black] (0,0) circle (0.25);
      \filldraw[draw=black,fill=lightgray,rotate around={-45:(0,0)}]
        (-0.1,-0.2) rectangle (0.1,0.2);
    \end{scope}

    % Front-right wheel
    \begin{scope}[xshift=0.75cm,yshift=0.75cm]
      % Wheel vector
      \draw[->,thick,rotate around={-135:(0,0)}] (0,0) -- (0,1)
        node[right] {\small $v_2$};

      % Wheel
      \draw[black] (0,0) circle (0.25);
      \filldraw[draw=black,fill=lightgray,rotate around={-135:(0,0)}]
        (-0.1,-0.2) rectangle (0.1,0.2);
    \end{scope}

    % Rear-left wheel
    \begin{scope}[xshift=-0.75cm,yshift=-0.75cm]
      % Wheel vector
      \draw[->,thick,rotate around={45:(0,0)}] (0,0) -- (0,1)
        node[left] {\small $v_3$};

      % Wheel
      \draw[black] (0,0) circle (0.25);
      \filldraw[draw=black,fill=lightgray,rotate around={45:(0,0)}]
        (-0.1,-0.2) rectangle (0.1,0.2);
    \end{scope}

    % Rear-right wheel
    \begin{scope}[xshift=0.75cm,yshift=-0.75cm]
      % Wheel vector
      \draw[->,thick,rotate around={135:(0,0)}] (0,0) -- (0,1)
        node[right] {\small $v_4$};

      % Wheel
      \draw[black] (0,0) circle (0.25);
      \filldraw[draw=black,fill=lightgray,rotate around={135:(0,0)}]
        (-0.1,-0.2) rectangle (0.1,0.2);
    \end{scope}
  \end{tikzpicture}

  \caption{Swerve drive free body diagram}
  \label{fig:swerve_drive_fbd}
\end{bookfigure}

\subsection{Inverse kinematics}
\begin{align*}
  \vec{v}_{wheel1} &= \vec{v}_{robot} + \vec{\omega}_{robot} \times
    \vec{r}_{robot2wheel1} \\
  \vec{v}_{wheel2} &= \vec{v}_{robot} + \vec{\omega}_{robot} \times
    \vec{r}_{robot2wheel2} \\
  \vec{v}_{wheel3} &= \vec{v}_{robot} + \vec{\omega}_{robot} \times
    \vec{r}_{robot2wheel3} \\
  \vec{v}_{wheel4} &= \vec{v}_{robot} + \vec{\omega}_{robot} \times
    \vec{r}_{robot2wheel4}
\end{align*}

where $\vec{v}_{wheel}$ is the wheel velocity vector, $\vec{v}_{robot}$ is the
robot velocity vector, $\vec{\omega}_{robot}$ is the robot angular velocity
vector, $\vec{r}_{robot2wheel}$ is the displacement vector from the robot's
center of rotation to the wheel\footnote{The robot's center of rotation need not
coincide with the robot's geometric center.},
$\vec{v}_{robot} = v_x \hat{i} + v_y \hat{j}$, and
$\vec{r}_{robot2wheel} = r_x \hat{i} + r_y \hat{j}$. The number suffixes denote
a specific wheel in figure \ref{fig:swerve_drive_fbd}.
\begin{align*}
  \vec{v}_1 &= v_x \hat{i} + v_y \hat{j} + \omega \hat{k} \times
    (r_{1x} \hat{i} + r_{1y} \hat{j}) \\
  \vec{v}_2 &= v_x \hat{i} + v_y \hat{j} + \omega \hat{k} \times
    (r_{2x} \hat{i} + r_{2y} \hat{j}) \\
  \vec{v}_3 &= v_x \hat{i} + v_y \hat{j} + \omega \hat{k} \times
    (r_{3x} \hat{i} + r_{3y} \hat{j}) \\
  \vec{v}_4 &= v_x \hat{i} + v_y \hat{j} + \omega \hat{k} \times
    (r_{4x} \hat{i} + r_{4y} \hat{j})
\end{align*}
\begin{align*}
  \vec{v}_1 &= v_x \hat{i} + v_y \hat{j} +
    (\omega r_{1x} \hat{j} - \omega r_{1y} \hat{i}) \\
  \vec{v}_2 &= v_x \hat{i} + v_y \hat{j} +
    (\omega r_{2x} \hat{j} - \omega r_{2y} \hat{i}) \\
  \vec{v}_3 &= v_x \hat{i} + v_y \hat{j} +
    (\omega r_{3x} \hat{j} - \omega r_{3y} \hat{i}) \\
  \vec{v}_4 &= v_x \hat{i} + v_y \hat{j} +
    (\omega r_{4x} \hat{j} - \omega r_{4y} \hat{i})
\end{align*}
\begin{align*}
  \vec{v}_1 &= v_x \hat{i} + v_y \hat{j} +
    \omega r_{1x} \hat{j} - \omega r_{1y} \hat{i} \\
  \vec{v}_2 &= v_x \hat{i} + v_y \hat{j} +
    \omega r_{2x} \hat{j} - \omega r_{2y} \hat{i} \\
  \vec{v}_3 &= v_x \hat{i} + v_y \hat{j} +
    \omega r_{3x} \hat{j} - \omega r_{3y} \hat{i} \\
  \vec{v}_4 &= v_x \hat{i} + v_y \hat{j} +
    \omega r_{4x} \hat{j} - \omega r_{4y} \hat{i}
\end{align*}
\begin{align*}
  \vec{v}_1 &= v_x \hat{i} - \omega r_{1y} \hat{i} +
    v_y \hat{j} + \omega r_{1x} \hat{j} \\
  \vec{v}_2 &= v_x \hat{i} - \omega r_{2y} \hat{i} +
    v_y \hat{j} + \omega r_{2x} \hat{j} \\
  \vec{v}_3 &= v_x \hat{i} - \omega r_{3y} \hat{i} +
    v_y \hat{j} + \omega r_{3x} \hat{j} \\
  \vec{v}_4 &= v_x \hat{i} - \omega r_{4y} \hat{i} +
    v_y \hat{j} + \omega r_{4x} \hat{j}
\end{align*}
\begin{align*}
  \vec{v}_1 &= (v_x - \omega r_{1y}) \hat{i} + (v_y + \omega r_{1x}) \hat{j} \\
  \vec{v}_2 &= (v_x - \omega r_{2y}) \hat{i} + (v_y + \omega r_{2x}) \hat{j} \\
  \vec{v}_3 &= (v_x - \omega r_{3y}) \hat{i} + (v_y + \omega r_{3x}) \hat{j} \\
  \vec{v}_4 &= (v_x - \omega r_{4y}) \hat{i} + (v_y + \omega r_{4x}) \hat{j}
\end{align*}

Separate the i-hat components into independent equations.
\begin{align*}
  v_{1x} &= v_x - \omega r_{1y} \\
  v_{2x} &= v_x - \omega r_{2y} \\
  v_{3x} &= v_x - \omega r_{3y} \\
  v_{4x} &= v_x - \omega r_{4y}
\end{align*}

Separate the j-hat components into independent equations.
\begin{align*}
  v_{1y} &= v_y + \omega r_{1x} \\
  v_{2y} &= v_y + \omega r_{2x} \\
  v_{3y} &= v_y + \omega r_{3x} \\
  v_{4y} &= v_y + \omega r_{4x}
\end{align*}

Now we'll factor them out into matrices.
\begin{equation*}
  \begin{bmatrix}
    v_{1x} \\
    v_{2x} \\
    v_{3x} \\
    v_{4x} \\
    v_{1y} \\
    v_{2y} \\
    v_{3y} \\
    v_{4y}
  \end{bmatrix} =
  \begin{bmatrix}
    1 & 0 & -r_{1y} \\
    1 & 0 & -r_{2y} \\
    1 & 0 & -r_{3y} \\
    1 & 0 & -r_{4y} \\
    0 & 1 &  r_{1x} \\
    0 & 1 &  r_{2x} \\
    0 & 1 &  r_{3x} \\
    0 & 1 &  r_{4x}
  \end{bmatrix}
  \begin{bmatrix}
    v_x \\
    v_y \\
    \omega
  \end{bmatrix}
\end{equation*}

Rearrange the rows so the $x$ and $y$ components are in adjacent rows.
\begin{equation}
  \label{eq:swerve_ik}
  \begin{bmatrix}
    v_{1x} \\
    v_{1y} \\
    v_{2x} \\
    v_{2y} \\
    v_{3x} \\
    v_{3y} \\
    v_{4x} \\
    v_{4y}
  \end{bmatrix} =
  \begin{bmatrix}
    1 & 0 & -r_{1y} \\
    0 & 1 &  r_{1x} \\
    1 & 0 & -r_{2y} \\
    0 & 1 &  r_{2x} \\
    1 & 0 & -r_{3y} \\
    0 & 1 &  r_{3x} \\
    1 & 0 & -r_{4y} \\
    0 & 1 &  r_{4x}
  \end{bmatrix}
  \begin{bmatrix}
    v_x \\
    v_y \\
    \omega
  \end{bmatrix}
\end{equation}

To convert the swerve module $x$ and $y$ velocity components to a velocity and
heading, use the Pythagorean theorem and arctangent respectively. Here's an
example for module 1.
\begin{align}
  v_1 &= \sqrt{v_{1x}^2 + v_{1y}^2} \\
  \theta_1 &= \tan^{-1}\left(\frac{v_{1y}}{v_{1x}}\right)
\end{align}

\subsection{Forward kinematics}

Let $\mat{M}$ be the $8 \times 3$ inverse kinematics matrix from equation
\eqref{eq:swerve_ik}. The forward kinematics are
\begin{equation}
  \begin{bmatrix}
    v_x \\
    v_y \\
    \omega
  \end{bmatrix} =
  \mat{M}^\dagger
  \begin{bmatrix}
    v_{1x} \\
    v_{1y} \\
    v_{2x} \\
    v_{2y} \\
    v_{3x} \\
    v_{3y} \\
    v_{4x} \\
    v_{4y}
  \end{bmatrix}
\end{equation}

where $\mat{M}^\dagger$ is the pseudoinverse of $\mat{M}$.
