\section{Mecanum drive kinematics}

A mecanum drive has four wheels, one on each corner of a rectangular chassis.
The wheels have rollers offset at $45$ degrees (whether it's clockwise or not
varies per wheel). The forces they generate when moving forward are shown in
figure \ref{fig:mecanum_drive_fbd}.
\begin{bookfigure}
  \begin{tikzpicture}[auto, >=latex']
    %\draw [help lines] (-4,-3) grid (4,3);

    % Coordinate axes
    \begin{scope}[xshift=2cm,yshift=-2cm]
      \draw[->] (0,0) -- (-1,0) node[below] {\small +$y$};
      \draw[->] (0,0) -- (0,1) node[right] {\small +$x$};
    \end{scope}

    % Shared front axle
    \begin{scope}[xshift=0cm,yshift=0.7cm]
      \fill[black] (-1.2,-0.1) rectangle (1.2,0.1);
    \end{scope}

    % Shared rear axle
    \begin{scope}[xshift=0cm,yshift=-0.7cm]
      \fill[black] (-1.2,-0.1) rectangle (1.2,0.1);
    \end{scope}

    % Chassis
    \filldraw[draw=black,fill=white] (-1,-1) rectangle (1,1);

    % Front-left wheel
    \begin{scope}[xshift=-1.3cm,yshift=0.7cm]
      % Wheel vector
      \draw[->,thick,rotate around={-45:(0,0)}] (0,0) -- (0,1)
        node[left] {\small $v_{fl}$};

      % Wheel
      \filldraw[draw=black,fill=lightgray] (-0.2,-0.45) rectangle (0.2,0.45);
      \begin{scope}[xshift=0cm,yshift=-0.3cm]
        \filldraw[draw=black,fill=lightgray,rotate around={45:(0,0)}]
          (-0.1,-0.2) rectangle (0.1,0.2);
      \end{scope}
      \filldraw[draw=black,fill=lightgray,rotate around={45:(0,0)}]
        (-0.1,-0.2) rectangle (0.1,0.2);
      \begin{scope}[xshift=0cm,yshift=0.3cm]
        \filldraw[draw=black,fill=lightgray,rotate around={45:(0,0)}]
          (-0.1,-0.2) rectangle (0.1,0.2);
      \end{scope}
    \end{scope}

    % Front-right wheel
    \begin{scope}[xshift=1.3cm,yshift=0.7cm]
      % Wheel vector
      \draw[->,thick,rotate around={45:(0,0)}] (0,0) -- (0,1)
        node[right] {\small $v_{fr}$};

      % Wheel
      \filldraw[draw=black,fill=lightgray] (-0.2,-0.45) rectangle (0.2,0.45);
      \begin{scope}[xshift=0cm,yshift=-0.3cm]
        \filldraw[draw=black,fill=lightgray,rotate around={-45:(0,0)}]
          (-0.1,-0.2) rectangle (0.1,0.2);
      \end{scope}
      \filldraw[draw=black,fill=lightgray,rotate around={-45:(0,0)}]
        (-0.1,-0.2) rectangle (0.1,0.2);
      \begin{scope}[xshift=0cm,yshift=0.3cm]
        \filldraw[draw=black,fill=lightgray,rotate around={-45:(0,0)}]
          (-0.1,-0.2) rectangle (0.1,0.2);
      \end{scope}
    \end{scope}

    % Rear-left wheel
    \begin{scope}[xshift=-1.3cm,yshift=-0.7cm]
      % Wheel vector
      \draw[->,thick,rotate around={45:(0,0)}] (0,0) -- (0,1)
        node[left] {\small $v_{rl}$};

      % Wheel
      \filldraw[draw=black,fill=lightgray] (-0.2,-0.45) rectangle (0.2,0.45);
      \begin{scope}[xshift=0cm,yshift=-0.3cm]
        \filldraw[draw=black,fill=lightgray,rotate around={-45:(0,0)}]
          (-0.1,-0.2) rectangle (0.1,0.2);
      \end{scope}
      \filldraw[draw=black,fill=lightgray,rotate around={-45:(0,0)}]
        (-0.1,-0.2) rectangle (0.1,0.2);
      \begin{scope}[xshift=0cm,yshift=0.3cm]
        \filldraw[draw=black,fill=lightgray,rotate around={-45:(0,0)}]
          (-0.1,-0.2) rectangle (0.1,0.2);
      \end{scope}
    \end{scope}

    % Rear-right wheel
    \begin{scope}[xshift=1.3cm,yshift=-0.7cm]
      % Wheel vector
      \draw[->,thick,rotate around={-45:(0,0)}] (0,0) -- (0,1)
        node[right] {\small $v_{rr}$};

      % Wheel
      \filldraw[draw=black,fill=lightgray] (-0.2,-0.45) rectangle (0.2,0.45);
      \begin{scope}[xshift=0cm,yshift=-0.3cm]
        \filldraw[draw=black,fill=lightgray,rotate around={45:(0,0)}]
          (-0.1,-0.2) rectangle (0.1,0.2);
      \end{scope}
      \filldraw[draw=black,fill=lightgray,rotate around={45:(0,0)}]
        (-0.1,-0.2) rectangle (0.1,0.2);
      \begin{scope}[xshift=0cm,yshift=0.3cm]
        \filldraw[draw=black,fill=lightgray,rotate around={45:(0,0)}]
          (-0.1,-0.2) rectangle (0.1,0.2);
      \end{scope}
    \end{scope}
  \end{tikzpicture}

  \caption{Mecanum drive free body diagram}
  \label{fig:mecanum_drive_fbd}
\end{bookfigure}

Note that the velocity of the wheel is the same as the velocity in the diagram
for the purposes of feedback control. The rollers on the wheel redirect the
velocity vector.

\subsection{Inverse kinematics}

First, we'll derive the front-left wheel kinematics.
\begin{align}
  \vec{v}_{fl} &= v_x \hat{i} + v_y \hat{j} +
    \omega \hat{k} \times (r_{fl_x} \hat{i} + r_{fl_y} \hat{j}) \nonumber \\
  \vec{v}_{fl} &= v_x \hat{i} + v_y \hat{j} +
    \omega r_{fl_x} \hat{j} - \omega r_{fl_y} \hat{i} \nonumber \\
  \vec{v}_{fl} &= (v_x - \omega r_{fl_y}) \hat{i} +
    (v_y + \omega r_{fl_x}) \hat{j} \nonumber
  \intertext{Project the front-left wheel onto its wheel vector.}
  v_{fl} &= ((v_x - \omega r_{fl_y}) \hat{i} + (v_y + \omega r_{fl_x}) \hat{j})
    \cdot \frac{\hat{i} - \hat{j}}{\sqrt{2}} \nonumber \\
  v_{fl} &= ((v_x - \omega r_{fl_y}) - (v_y + \omega r_{fl_x}))
    \frac{1}{\sqrt{2}} \nonumber \\
  v_{fl} &= (v_x - \omega r_{fl_y} - v_y - \omega r_{fl_x})
    \frac{1}{\sqrt{2}} \nonumber \\
  v_{fl} &= (v_x - v_y - \omega r_{fl_y} - \omega r_{fl_x})
    \frac{1}{\sqrt{2}} \nonumber \\
  v_{fl} &= (v_x - v_y - \omega (r_{fl_x} + r_{fl_y}))
    \frac{1}{\sqrt{2}} \nonumber \\
  v_{fl} &= \frac{1}{\sqrt{2}} v_x - \frac{1}{\sqrt{2}} v_y -
    \frac{1}{\sqrt{2}} \omega (r_{fl_x} + r_{fl_y})
\end{align}

Next, we'll derive the front-right wheel kinematics.
\begin{align}
  \vec{v}_{fr} &= v_x \hat{i} + v_y \hat{j} +
    \omega \hat{k} \times (r_{fr_x} \hat{i} + r_{fr_y} \hat{j}) \nonumber \\
  \vec{v}_{fr} &= v_x \hat{i} + v_y \hat{j} +
    \omega r_{fr_x} \hat{j} - \omega r_{fr_y} \hat{i} \nonumber \\
  \vec{v}_{fr} &= (v_x - \omega r_{fr_y}) \hat{i} +
    (v_y + \omega r_{fr_x}) \hat{j} \nonumber
  \intertext{Project the front-right wheel onto its wheel vector.}
  v_{fr} &= ((v_x - \omega r_{fr_y}) \hat{i} + (v_y + \omega r_{fr_x}) \hat{j})
    \cdot (\hat{i} + \hat{j}) \frac{1}{\sqrt{2}} \nonumber \\
  v_{fr} &= ((v_x - \omega r_{fr_y}) + (v_y + \omega r_{fr_x}))
    \frac{1}{\sqrt{2}} \nonumber \\
  v_{fr} &= (v_x - \omega r_{fr_y} + v_y + \omega r_{fr_x})
    \frac{1}{\sqrt{2}} \nonumber \\
  v_{fr} &= (v_x + v_y - \omega r_{fr_y} + \omega r_{fr_x})
    \frac{1}{\sqrt{2}} \nonumber \\
  v_{fr} &= (v_x + v_y + \omega (r_{fr_x} - r_{fr_y}))
    \frac{1}{\sqrt{2}} \nonumber \\
  v_{fr} &= \frac{1}{\sqrt{2}} v_x + \frac{1}{\sqrt{2}} v_y +
    \frac{1}{\sqrt{2}} \omega (r_{fr_x} - r_{fr_y})
\end{align}

Next, we'll derive the rear-left wheel kinematics.
\begin{align}
  \vec{v}_{rl} &= v_x \hat{i} + v_y \hat{j} +
    \omega \hat{k} \times (r_{rl_x} \hat{i} + r_{rl_y} \hat{j}) \nonumber \\
  \vec{v}_{rl} &= v_x \hat{i} + v_y \hat{j} +
    \omega r_{rl_x} \hat{j} - \omega r_{rl_y} \hat{i} \nonumber \\
  \vec{v}_{rl} &= (v_x - \omega r_{rl_y}) \hat{i} +
    (v_y + \omega r_{rl_x}) \hat{j} \nonumber
  \intertext{Project the rear-left wheel onto its wheel vector.}
  v_{rl} &= ((v_x - \omega r_{rl_y}) \hat{i} + (v_y + \omega r_{rl_x}) \hat{j})
    \cdot (\hat{i} + \hat{j}) \frac{1}{\sqrt{2}} \nonumber \\
  v_{rl} &= ((v_x - \omega r_{rl_y}) + (v_y + \omega r_{rl_x}))
    \frac{1}{\sqrt{2}} \nonumber \\
  v_{rl} &= (v_x - \omega r_{rl_y} + v_y + \omega r_{rl_x})
    \frac{1}{\sqrt{2}} \nonumber \\
  v_{rl} &= (v_x + v_y - \omega r_{rl_y} + \omega r_{rl_x})
    \frac{1}{\sqrt{2}} \nonumber \\
  v_{rl} &= (v_x + v_y + \omega (r_{rl_x} - r_{rl_y}))
    \frac{1}{\sqrt{2}} \nonumber \\
  v_{rl} &= \frac{1}{\sqrt{2}} v_x + \frac{1}{\sqrt{2}} v_y +
    \frac{1}{\sqrt{2}} \omega (r_{rl_x} - r_{rl_y})
\end{align}

Next, we'll derive the rear-right wheel kinematics.
\begin{align}
  \vec{v}_{rr} &= v_x \hat{i} + v_y \hat{j} +
    \omega \hat{k} \times (r_{rr_x} \hat{i} + r_{rr_y} \hat{j}) \nonumber \\
  \vec{v}_{rr} &= v_x \hat{i} + v_y \hat{j} +
    \omega r_{rr_x} \hat{j} - \omega r_{rr_y} \hat{i} \nonumber \\
  \vec{v}_{rr} &= (v_x - \omega r_{rr_y}) \hat{i} +
    (v_y + \omega r_{rr_x}) \hat{j} \nonumber
  \intertext{Project the rear-right wheel onto its wheel vector.}
  v_{rr} &= ((v_x - \omega r_{rr_y}) \hat{i} + (v_y + \omega r_{rr_x}) \hat{j})
    \cdot \frac{\hat{i} - \hat{j}}{\sqrt{2}} \nonumber \\
  v_{rr} &= ((v_x - \omega r_{rr_y}) - (v_y + \omega r_{rr_x}))
    \frac{1}{\sqrt{2}} \nonumber \\
  v_{rr} &= (v_x - \omega r_{rr_y} - v_y - \omega r_{rr_x})
    \frac{1}{\sqrt{2}} \nonumber \\
  v_{rr} &= (v_x - v_y - \omega r_{rr_y} - \omega r_{rr_x})
    \frac{1}{\sqrt{2}} \nonumber \\
  v_{rr} &= (v_x - v_y - \omega (r_{rr_x} + r_{rr_y}))
    \frac{1}{\sqrt{2}} \nonumber \\
  v_{rr} &= \frac{1}{\sqrt{2}} v_x - \frac{1}{\sqrt{2}} v_y -
    \frac{1}{\sqrt{2}} \omega (r_{rr_x} + r_{rr_y})
\end{align}

This gives the following inverse kinematic equations.
\begin{align}
  v_{fl} &= \frac{1}{\sqrt{2}} v_x - \frac{1}{\sqrt{2}} v_y -
    \frac{1}{\sqrt{2}} \omega (r_{fl_x} + r_{fl_y}) \nonumber \\
  v_{fr} &= \frac{1}{\sqrt{2}} v_x + \frac{1}{\sqrt{2}} v_y +
    \frac{1}{\sqrt{2}} \omega (r_{fr_x} - r_{fr_y}) \nonumber \\
  v_{rl} &= \frac{1}{\sqrt{2}} v_x + \frac{1}{\sqrt{2}} v_y +
    \frac{1}{\sqrt{2}} \omega (r_{rl_x} - r_{rl_y}) \nonumber \\
  v_{rr} &= \frac{1}{\sqrt{2}} v_x - \frac{1}{\sqrt{2}} v_y -
    \frac{1}{\sqrt{2}} \omega (r_{rr_x} + r_{rr_y}) \nonumber
  \intertext{Now we'll factor them out into matrices.}
  \begin{bmatrix}
    v_{fl} \\
    v_{fr} \\
    v_{rl} \\
    v_{rr}
  \end{bmatrix} &=
  \begin{bmatrix}
    \frac{1}{\sqrt{2}} & -\frac{1}{\sqrt{2}} &
      -\frac{1}{\sqrt{2}}(r_{fl_x} + r_{fl_y}) \\
    \frac{1}{\sqrt{2}} &  \frac{1}{\sqrt{2}} &
      \frac{1}{\sqrt{2}}(r_{fr_x} - r_{fr_y}) \\
    \frac{1}{\sqrt{2}} &  \frac{1}{\sqrt{2}} &
      \frac{1}{\sqrt{2}}(r_{rl_x} - r_{rl_y}) \\
    \frac{1}{\sqrt{2}} & -\frac{1}{\sqrt{2}} &
      -\frac{1}{\sqrt{2}}(r_{rr_x} + r_{rr_y})
  \end{bmatrix}
  \begin{bmatrix}
    v_x \\
    v_y \\
    \omega
  \end{bmatrix} \nonumber \\
  \begin{bmatrix}
    v_{fl} \\
    v_{fr} \\
    v_{rl} \\
    v_{rr}
  \end{bmatrix} &= \frac{1}{\sqrt{2}}
  \begin{bmatrix}
    1 & -1  & -(r_{fl_x} + r_{fl_y}) \\
    1 &  1  &  (r_{fr_x} - r_{fr_y}) \\
    1 &  1  &  (r_{rl_x} - r_{rl_y}) \\
    1 & -1  & -(r_{rr_x} + r_{rr_y})
  \end{bmatrix}
  \begin{bmatrix}
    v_x \\
    v_y \\
    \omega
  \end{bmatrix}
\end{align}

\subsection{Forward kinematics}

Let $\mat{M}$ be the $4 \times 3$ inverse kinematics matrix above including the
$\frac{1}{\sqrt{2}}$ factor. The forward kinematics are
\begin{equation}
  \begin{bmatrix}
    v_x \\
    v_y \\
    \omega
  \end{bmatrix} =
  \mat{M}^+
  \begin{bmatrix}
    v_{fl} \\
    v_{fr} \\
    v_{rl} \\
    v_{rr}
  \end{bmatrix}
\end{equation}

where $\mat{M}^+$ is the pseudoinverse of $\mat{M}$.
