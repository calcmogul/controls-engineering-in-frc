\section{Drivetrain velocity model}
\begin{equation*}
  \dot{\mtx{x}} = \mtx{A}\mtx{x} + \mtx{B}\mtx{u}
\end{equation*}

If $\mtx{u} =
\begin{bmatrix}
  K_{v1} v \\
  K_{v1} v
\end{bmatrix}$, then $\mtx{x} =
\begin{bmatrix}
  v \\
  v
\end{bmatrix}$ and $\dot{\mtx{x}} =
\begin{bmatrix}
  0 \\
  0
\end{bmatrix}$.
\begin{equation*}
  \begin{bmatrix}
    0 \\
    0
  \end{bmatrix} =
  \begin{bmatrix}
    A_1 & A_2 \\
    A_2 & A_1
  \end{bmatrix}
  \begin{bmatrix}
    v \\
    v
  \end{bmatrix} +
  \begin{bmatrix}
    B_1 & B_2 \\
    B_2 & B_1
  \end{bmatrix}
  \begin{bmatrix}
    K_{v1} v \\
    K_{v1} v
  \end{bmatrix}
\end{equation*}

Since the column vectors contain the same element, the elements in the second
row can be rearranged.
\begin{equation*}
  \begin{bmatrix}
    0 \\
    0
  \end{bmatrix} =
  \begin{bmatrix}
    A_1 & A_2 \\
    A_1 & A_2
  \end{bmatrix}
  \begin{bmatrix}
    v \\
    v
  \end{bmatrix} +
  \begin{bmatrix}
    B_1 & B_2 \\
    B_1 & B_2
  \end{bmatrix}
  \begin{bmatrix}
    K_{v1} v \\
    K_{v1} v
  \end{bmatrix}
\end{equation*}

Since the rows are linearly dependent, we can use just one of them.
\begin{align*}
  \mtx{0} &=
    \begin{bmatrix}
      A_1 & A_2
    \end{bmatrix} v +
    \begin{bmatrix}
      B_1 & B_2
    \end{bmatrix} K_{v1} v \\
  \mtx{0} &=
    \begin{bmatrix}
      v & v & K_{v1} v & K_{v1} v
    \end{bmatrix}
    \begin{bmatrix}
      A_1 \\
      A_2 \\
      B_1 \\
      B_2
    \end{bmatrix} \\
  \mtx{0} &=
    \begin{bmatrix}
      1 & 1 & K_{v1} & K_{v1}
    \end{bmatrix}
    \begin{bmatrix}
      A_1 \\
      A_2 \\
      B_1 \\
      B_2
    \end{bmatrix}
\end{align*}

If $\mtx{u} =
\begin{bmatrix}
  K_{v1} v + K_{a1} a
\end{bmatrix}$, then $\mtx{x} =
\begin{bmatrix}
  v \\
  v
\end{bmatrix}$ and $\dot{\mtx{x}} =
\begin{bmatrix}
  a \\
  a
\end{bmatrix}$.
\begin{equation*}
  \begin{bmatrix}
    a \\
    a
  \end{bmatrix} =
  \begin{bmatrix}
    A_1 & A_2 \\
    A_2 & A_1
  \end{bmatrix}
  \begin{bmatrix}
    v \\
    v
  \end{bmatrix} +
  \begin{bmatrix}
    B_1 & B_2 \\
    B_2 & B_1
  \end{bmatrix}
  \begin{bmatrix}
    K_{v1} v + K_{a1} a \\
    K_{v1} v + K_{a1} a
  \end{bmatrix}
\end{equation*}

Since the column vectors contain the same element, the elements in the second
row can be rearranged.
\begin{equation*}
  \begin{bmatrix}
    a \\
    a
  \end{bmatrix} =
  \begin{bmatrix}
    A_1 & A_2 \\
    A_1 & A_2
  \end{bmatrix}
  \begin{bmatrix}
    v \\
    v
  \end{bmatrix} +
  \begin{bmatrix}
    B_1 & B_2 \\
    B_1 & B_2
  \end{bmatrix}
  \begin{bmatrix}
    K_{v1} v + K_{a1} a \\
    K_{v1} v + K_{a1} a
  \end{bmatrix}
\end{equation*}

Since the rows are linearly dependent, we can use just one of them.
\begin{align*}
  a &=
    \begin{bmatrix}
      A_1 & A_2
    \end{bmatrix} v +
    \begin{bmatrix}
      B_1 & B_2
    \end{bmatrix}
    \begin{bmatrix}
      K_{v1} v + K_{a1} a
    \end{bmatrix} \\
  a &=
    \begin{bmatrix}
      v & v & K_{v1} v + K_{a1} a & K_{v1} + K_{a1} a
    \end{bmatrix}
    \begin{bmatrix}
      A_1 \\
      A_2 \\
      B_1 \\
      B_2
    \end{bmatrix}
\end{align*}

If $\mtx{u} =
\begin{bmatrix}
  -K_{v2} v \\
  K_{v2} v
\end{bmatrix}$, then $\mtx{x} =
\begin{bmatrix}
  -v \\
  v
\end{bmatrix}$ and $\dot{\mtx{x}} =
\begin{bmatrix}
  0 \\
  0
\end{bmatrix}$.
\begin{align*}
  \begin{bmatrix}
    0 \\
    0
  \end{bmatrix} &=
    \begin{bmatrix}
      A_1 & A_2 \\
      A_2 & A_1
    \end{bmatrix}
    \begin{bmatrix}
      -v \\
      v
    \end{bmatrix} +
    \begin{bmatrix}
      B_1 & B_2 \\
      B_2 & B_1
    \end{bmatrix}
    \begin{bmatrix}
      -K_{v2} v \\
      K_{v2} v
    \end{bmatrix} \\
  \begin{bmatrix}
    0 \\
    0
  \end{bmatrix} &=
    \begin{bmatrix}
      -A_1 & A_2 \\
      -A_2 & A_1
    \end{bmatrix}
    \begin{bmatrix}
      v \\
      v
    \end{bmatrix} +
    \begin{bmatrix}
      -B_1 & B_2 \\
      -B_2 & B_1
    \end{bmatrix}
    \begin{bmatrix}
      K_{v2} v \\
      K_{v2} v
    \end{bmatrix} \\
  \begin{bmatrix}
    0 \\
    0
  \end{bmatrix} &=
    \begin{bmatrix}
      -A_1 & A_2 \\
      A_1 & -A_2
    \end{bmatrix}
    \begin{bmatrix}
      v \\
      v
    \end{bmatrix} +
    \begin{bmatrix}
      -B_1 & B_2 \\
      B_1 & -B_2
    \end{bmatrix}
    \begin{bmatrix}
      K_{v2} v \\
      K_{v2} v
    \end{bmatrix}
\end{align*}

Since the column vectors contain the same element, the elements in the second
row can be rearranged.
\begin{equation*}
  \begin{bmatrix}
    0 \\
    0
  \end{bmatrix} =
  \begin{bmatrix}
    -A_1 & A_2 \\
    -A_1 & A_2
  \end{bmatrix}
  \begin{bmatrix}
    v \\
    v
  \end{bmatrix} +
  \begin{bmatrix}
    -B_1 & B_2 \\
    -B_1 & B_2
  \end{bmatrix}
  \begin{bmatrix}
    K_{v2} v \\
    K_{v2} v
  \end{bmatrix}
\end{equation*}

Since the rows are linearly dependent, we can use just one of them.
\begin{align*}
  0 &=
    \begin{bmatrix}
      -A_1 & A_2
    \end{bmatrix} v +
    \begin{bmatrix}
      -B_1 & B_2
    \end{bmatrix} K_{v2} v \\
  0 &= -v A_1 + v A_2 - K_{v2} v B_1 + K_{v2} v B_2 \\
  0 &=
    \begin{bmatrix}
      -v & v & -K_{v2} v & K_{v2} v
    \end{bmatrix}
    \begin{bmatrix}
      A_1 \\
      A_2 \\
      B_1 \\
      B_2
    \end{bmatrix} \\
  0 &=
    \begin{bmatrix}
      -1 & 1 & -K_{v2} & K_{v2}
    \end{bmatrix}
    \begin{bmatrix}
      A_1 \\
      A_2 \\
      B_1 \\
      B_2
    \end{bmatrix}
\end{align*}

If $\mtx{u} =
\begin{bmatrix}
  -K_{v2} v - K_{a2} a \\
  K_{v2} v + K_{a2} a
\end{bmatrix}$, then $\mtx{x} =
\begin{bmatrix}
  -v \\
  v
\end{bmatrix}$ and $\dot{\mtx{x}} =
\begin{bmatrix}
  -a \\
  a
\end{bmatrix}$.
\begin{align*}
  \begin{bmatrix}
    -a \\
    a
  \end{bmatrix} &=
    \begin{bmatrix}
      A_1 & A_2 \\
      A_2 & A_1
    \end{bmatrix}
    \begin{bmatrix}
      -v \\
      v
    \end{bmatrix} +
    \begin{bmatrix}
      B_1 & B_2 \\
      B_2 & B_1
    \end{bmatrix}
    \begin{bmatrix}
      -K_{v2} v - K_{a2} a \\
      K_{v2} v + K_{a2} a
    \end{bmatrix} \\
  \begin{bmatrix}
    -a \\
    a
  \end{bmatrix} &=
    \begin{bmatrix}
      -A_1 & A_2 \\
      -A_2 & A_1
    \end{bmatrix}
    \begin{bmatrix}
      v \\
      v
    \end{bmatrix} +
    \begin{bmatrix}
      -B_1 & B_2 \\
      -B_2 & B_1
    \end{bmatrix}
    \begin{bmatrix}
      K_{v2} v + K_{a2} a \\
      K_{v2} v + K_{a2} a
    \end{bmatrix} \\
  \begin{bmatrix}
    -a \\
    a
  \end{bmatrix} &=
    \begin{bmatrix}
      -A_1 & A_2 \\
      A1 & -A_2
    \end{bmatrix}
    \begin{bmatrix}
      v \\
      v
    \end{bmatrix} +
    \begin{bmatrix}
      -B_1 & B_2 \\
      B_1 & -B_2
    \end{bmatrix}
    \begin{bmatrix}
      K_{v2} v + K_{a2} a \\
      K_{v2} v + K_{a2} a
    \end{bmatrix}
\end{align*}

Since the column vectors contain the same element, the elements in the second
row can be rearranged.
\begin{equation*}
  \begin{bmatrix}
    -a \\
    -a
  \end{bmatrix} =
  \begin{bmatrix}
    -A_1 & A_2 \\
    -A_1 & A_2
  \end{bmatrix}
  \begin{bmatrix}
    v \\
    v
  \end{bmatrix} +
  \begin{bmatrix}
    -B_1 & B_2 \\
    -B_1 & B_2
  \end{bmatrix}
  \begin{bmatrix}
    K_{v2} v + K_{a2} a \\
    K_{v2} v + K_{a2} a
  \end{bmatrix}
\end{equation*}

Since the rows are linearly dependent, we can use just one of them.
\begin{align*}
  -a &=
    \begin{bmatrix}
      -A_1 & A_2
    \end{bmatrix} v +
    \begin{bmatrix}
      -B_1 & B_2
    \end{bmatrix}
    \begin{bmatrix}
      K_{v2} v + K_{a2} a
    \end{bmatrix} \\
  -a &= -v A_1 + v A_2 - (K_{v2} v + K_{a2} a) B_1 + K_{v2} v + K_{a2} a) B_2 \\
  -a &=
    \begin{bmatrix}
      -v & v & -(K_{v2} v + K_{a2} a) & K_{v2} v+ K_{a2} a
    \end{bmatrix}
    \begin{bmatrix}
      A_1 \\
      A_2 \\
      B_1 \\
      B_2
    \end{bmatrix} \\
  a &=
    \begin{bmatrix}
      v & -v & K_{v2} v + K_{a2} a & -(K_{v2} v + K_{a2} a)
    \end{bmatrix}
    \begin{bmatrix}
      A_1 \\
      A_2 \\
      B_1 \\
      B_2
    \end{bmatrix}
\end{align*}

Now stack the rows.
\begin{equation*}
  \begin{bmatrix}
    0 \\
    a \\
    0 \\
    a
  \end{bmatrix} =
  \begin{bmatrix}
    1 & 1 & K_{v1} & K_{v1} \\
    v & v & K_{v1} v + K_{a1} a & K_{v1} v + K_{a1} a \\
    -1 & 1 & -K_{v2} & K_{v2} \\
    v & -v & K_{v2} v + K_{a2} a & -(K_{v2} v + K_{a2} a)
  \end{bmatrix}
  \begin{bmatrix}
    A_1 \\
    A_2 \\
    B_1 \\
    B_2
  \end{bmatrix}
\end{equation*}

Solve for matrix elements with Wolfram Alpha. Let
$b = K_{v1}$, $c = K_{a1}$, $d = K_{v2}$, and $f = K_{a2}$.
\begin{verbatim}
inverse of {{1, 1, b, b}, {v, v, b v + c a, b v + c a},
  {-1, 1, -d, d}, {v, -v, d v + f a, -(d v + f a)}} * {{0}, {a}, {0}, {a}}
\end{verbatim}
\begin{align*}
  \begin{bmatrix}
    A_1 \\
    A_2 \\
    B_1 \\
    B_2
  \end{bmatrix} &= \frac{1}{2cf}
  \begin{bmatrix}
    -cd - bf \\
    cd - bf \\
    c + f \\
    f - c
  \end{bmatrix} \\
  \begin{bmatrix}
    A_1 \\
    A_2 \\
    B_1 \\
    B_2
  \end{bmatrix} &= \frac{1}{2 K_{a1} K_{a2}}
  \begin{bmatrix}
    -K_{a1} K_{v2} - K_{v1} K_{a2} \\
    K_{a1} K_{v2} - K_{v1} K_{a2} \\
    K_{a1} + K_{a2} \\
    K_{a2} - K_{a1}
  \end{bmatrix}
\end{align*}

To summarize,
\begin{theorem}[Drivetrain velocity model]
  \begin{equation*}
    \dot{\mtx{x}} = \mtx{A}\mtx{x} + \mtx{B}\mtx{u}
  \end{equation*}
  \begin{equation*}
    \begin{array}{cc}
    \mtx{x} =
      \begin{bmatrix}
        \text{left velocity} \\
        \text{right velocity}
      \end{bmatrix} &
    \mtx{u} =
      \begin{bmatrix}
        \text{left voltage} \\
        \text{right voltage}
      \end{bmatrix}
    \end{array}
  \end{equation*}
  \begin{equation*}
    \dot{\mtx{x}} =
      \begin{bmatrix}
        A_1 & A_2 \\
        A_2 & A_1
      \end{bmatrix} \mtx{x} +
      \begin{bmatrix}
        B_1 & B_2 \\
        B_2 & B_1
      \end{bmatrix} \mtx{u}
  \end{equation*}
  \begin{equation}
    \begin{bmatrix}
      A_1 \\
      A_2 \\
      B_1 \\
      B_2
    \end{bmatrix} = \frac{1}{2 K_{a,linear} K_{a,angular}}
      \begin{bmatrix}
        -K_{a,linear} K_{v,angular} - K_{v,linear} K_{a,angular} \\
        K_{a,linear} K_{v,angular} - K_{v,linear} K_{a,angular} \\
        K_{a,linear} + K_{a,angular} \\
        K_{a,angular} - K_{a,linear}
      \end{bmatrix}
  \end{equation}
\end{theorem}

If $K_v$ and $K_a$ are the same for both the linear and angular cases, it devolves to the one-dimensional case. This means the left and right sides are decoupled.
\begin{align*}
  \begin{bmatrix}
    A_1 \\
    A_2 \\
    B_1 \\
    B_2
  \end{bmatrix} &= \frac{1}{2 K_a K_a}
    \begin{bmatrix}
      -K_a K_v - K_v K_a \\
      K_a K_v - K_v K_a \\
      K_a + K_a \\
      K_a - K_a
    \end{bmatrix} \\
  \begin{bmatrix}
    A_1 \\
    A_2 \\
    B_1 \\
    B_2
  \end{bmatrix} &= \frac{1}{2 K_a K_a}
    \begin{bmatrix}
      -2K_v K_a \\
      0 \\
      2 K_a \\
      0
    \end{bmatrix} \\
  \begin{bmatrix}
    A_1 \\
    A_2 \\
    B_1 \\
    B_2
  \end{bmatrix} &=
    \begin{bmatrix}
      -\frac{K_v}{K_a} \\
      0 \\
      \frac{1}{K_a} \\
      0
    \end{bmatrix}
\end{align*}
