\section{Drivetrain velocity model}
\begin{equation*}
  \dot{\mtx{x}} = \mtx{A}\mtx{x} + \mtx{B}\mtx{u}
\end{equation*}
\begin{equation*}
  \begin{array}{ccc}
    \mtx{x} = \begin{bmatrix}
      \text{left velocity} \\
      \text{right velocity}
    \end{bmatrix} &
    \dot{\mtx{x}} = \begin{bmatrix}
      \text{left acceleration} \\
      \text{right acceleration}
    \end{bmatrix} &
    \mtx{u} = \begin{bmatrix}
      \text{left voltage} \\
      \text{right voltage}
    \end{bmatrix}
  \end{array}
\end{equation*}

We want to derive what $\mtx{A}$ and $\mtx{B}$ are from linear and angular
feedforward models. Since the left and right dynamics are symmetric, we'll guess
the model has the form
\begin{equation*}
  \begin{array}{cc}
    \mtx{A} = \begin{bmatrix}
      A_1 & A_2 \\
      A_2 & A_1
    \end{bmatrix} &
    \mtx{B} = \begin{bmatrix}
      B_1 & B_2 \\
      B_2 & B_1
    \end{bmatrix}
  \end{array}
\end{equation*}

Let $\mtx{u} =
\begin{bmatrix}
  K_{v,linear} v & K_{v,linear} v
\end{bmatrix}^T$ be the input that makes both sides of the drivetrain move at a
constant velocity $v$. Therefore, $\mtx{x} =
\begin{bmatrix}
  v & v
\end{bmatrix}^T$ and $\dot{\mtx{x}} =
\begin{bmatrix}
  0 & 0
\end{bmatrix}^T$. Substitute these into the state-space model.
\begin{equation*}
  \begin{bmatrix}
    0 \\
    0
  \end{bmatrix} =
  \begin{bmatrix}
    A_1 & A_2 \\
    A_2 & A_1
  \end{bmatrix}
  \begin{bmatrix}
    v \\
    v
  \end{bmatrix} +
  \begin{bmatrix}
    B_1 & B_2 \\
    B_2 & B_1
  \end{bmatrix}
  \begin{bmatrix}
    K_{v,linear} v \\
    K_{v,linear} v
  \end{bmatrix}
\end{equation*}

Since the column vectors contain the same element, the elements in the second
row can be rearranged.
\begin{equation*}
  \begin{bmatrix}
    0 \\
    0
  \end{bmatrix} =
  \begin{bmatrix}
    A_1 & A_2 \\
    A_1 & A_2
  \end{bmatrix}
  \begin{bmatrix}
    v \\
    v
  \end{bmatrix} +
  \begin{bmatrix}
    B_1 & B_2 \\
    B_1 & B_2
  \end{bmatrix}
  \begin{bmatrix}
    K_{v,linear} v \\
    K_{v,linear} v
  \end{bmatrix}
\end{equation*}

Since the rows are linearly dependent, we can use just one of them.
\begin{align*}
  0 &=
    \begin{bmatrix}
      A_1 & A_2
    \end{bmatrix} v +
    \begin{bmatrix}
      B_1 & B_2
    \end{bmatrix} K_{v,linear} v \\
  0 &=
    \begin{bmatrix}
      v & v & K_{v,linear} v & K_{v,linear} v
    \end{bmatrix}
    \begin{bmatrix}
      A_1 \\
      A_2 \\
      B_1 \\
      B_2
    \end{bmatrix} \\
  0 &=
    \begin{bmatrix}
      1 & 1 & K_{v,linear} & K_{v,linear}
    \end{bmatrix}
    \begin{bmatrix}
      A_1 \\
      A_2 \\
      B_1 \\
      B_2
    \end{bmatrix}
\end{align*}

Let $\mtx{u} =
\begin{bmatrix}
  K_{v,linear} v + K_{a,linear} a & K_{v,linear} v + K_{a,linear} a
\end{bmatrix}^T$ be the input that accelerates both sides of the drivetrain by
$a$ from an initial velocity of $v$. Therefore, $\mtx{x} =
\begin{bmatrix}
  v & v
\end{bmatrix}^T$ and $\dot{\mtx{x}} =
\begin{bmatrix}
  a & a
\end{bmatrix}^T$. Substitute these into the state-space model.
\begin{equation*}
  \begin{bmatrix}
    a \\
    a
  \end{bmatrix} =
  \begin{bmatrix}
    A_1 & A_2 \\
    A_2 & A_1
  \end{bmatrix}
  \begin{bmatrix}
    v \\
    v
  \end{bmatrix} +
  \begin{bmatrix}
    B_1 & B_2 \\
    B_2 & B_1
  \end{bmatrix}
  \begin{bmatrix}
    K_{v,linear} v + K_{a,linear} a \\
    K_{v,linear} v + K_{a,linear} a
  \end{bmatrix}
\end{equation*}

Since the column vectors contain the same element, the elements in the second
row can be rearranged.
\begin{equation*}
  \begin{bmatrix}
    a \\
    a
  \end{bmatrix} =
  \begin{bmatrix}
    A_1 & A_2 \\
    A_1 & A_2
  \end{bmatrix}
  \begin{bmatrix}
    v \\
    v
  \end{bmatrix} +
  \begin{bmatrix}
    B_1 & B_2 \\
    B_1 & B_2
  \end{bmatrix}
  \begin{bmatrix}
    K_{v,linear} v + K_{a,linear} a \\
    K_{v,linear} v + K_{a,linear} a
  \end{bmatrix}
\end{equation*}

Since the rows are linearly dependent, we can use just one of them.
\begin{align*}
  a &=
    \begin{bmatrix}
      A_1 & A_2
    \end{bmatrix} v +
    \begin{bmatrix}
      B_1 & B_2
    \end{bmatrix}
    \begin{bmatrix}
      K_{v,linear} v + K_{a,linear} a
    \end{bmatrix} \\
  a &=
    \begin{bmatrix}
      v & v & K_{v,linear} v + K_{a,linear} a & K_{v,linear} + K_{a,linear} a
    \end{bmatrix}
    \begin{bmatrix}
      A_1 \\
      A_2 \\
      B_1 \\
      B_2
    \end{bmatrix}
\end{align*}

Let $\mtx{u} =
\begin{bmatrix}
  -K_{v,angular} v & K_{v,angular} v
\end{bmatrix}^T$ be the input that rotates the drivetrain in place where each
wheel has a constant velocity $v$. Therefore, $\mtx{x} =
\begin{bmatrix}
  -v & v
\end{bmatrix}^T$ and $\dot{\mtx{x}} =
\begin{bmatrix}
  0 & 0
\end{bmatrix}^T$.
\begin{align*}
  \begin{bmatrix}
    0 \\
    0
  \end{bmatrix} &=
    \begin{bmatrix}
      A_1 & A_2 \\
      A_2 & A_1
    \end{bmatrix}
    \begin{bmatrix}
      -v \\
      v
    \end{bmatrix} +
    \begin{bmatrix}
      B_1 & B_2 \\
      B_2 & B_1
    \end{bmatrix}
    \begin{bmatrix}
      -K_{v,angular} v \\
      K_{v,angular} v
    \end{bmatrix} \\
  \begin{bmatrix}
    0 \\
    0
  \end{bmatrix} &=
    \begin{bmatrix}
      -A_1 & A_2 \\
      -A_2 & A_1
    \end{bmatrix}
    \begin{bmatrix}
      v \\
      v
    \end{bmatrix} +
    \begin{bmatrix}
      -B_1 & B_2 \\
      -B_2 & B_1
    \end{bmatrix}
    \begin{bmatrix}
      K_{v,angular} v \\
      K_{v,angular} v
    \end{bmatrix} \\
  \begin{bmatrix}
    0 \\
    0
  \end{bmatrix} &=
    \begin{bmatrix}
      -A_1 & A_2 \\
      A_1 & -A_2
    \end{bmatrix}
    \begin{bmatrix}
      v \\
      v
    \end{bmatrix} +
    \begin{bmatrix}
      -B_1 & B_2 \\
      B_1 & -B_2
    \end{bmatrix}
    \begin{bmatrix}
      K_{v,angular} v \\
      K_{v,angular} v
    \end{bmatrix}
\end{align*}

Since the column vectors contain the same element, the elements in the second
row can be rearranged.
\begin{equation*}
  \begin{bmatrix}
    0 \\
    0
  \end{bmatrix} =
  \begin{bmatrix}
    -A_1 & A_2 \\
    -A_1 & A_2
  \end{bmatrix}
  \begin{bmatrix}
    v \\
    v
  \end{bmatrix} +
  \begin{bmatrix}
    -B_1 & B_2 \\
    -B_1 & B_2
  \end{bmatrix}
  \begin{bmatrix}
    K_{v,angular} v \\
    K_{v,angular} v
  \end{bmatrix}
\end{equation*}

Since the rows are linearly dependent, we can use just one of them.
\begin{align*}
  0 &=
    \begin{bmatrix}
      -A_1 & A_2
    \end{bmatrix} v +
    \begin{bmatrix}
      -B_1 & B_2
    \end{bmatrix} K_{v,angular} v \\
  0 &= -v A_1 + v A_2 - K_{v,angular} v B_1 + K_{v,angular} v B_2 \\
  0 &=
    \begin{bmatrix}
      -v & v & -K_{v,angular} v & K_{v,angular} v
    \end{bmatrix}
    \begin{bmatrix}
      A_1 \\
      A_2 \\
      B_1 \\
      B_2
    \end{bmatrix} \\
  0 &=
    \begin{bmatrix}
      -1 & 1 & -K_{v,angular} & K_{v,angular}
    \end{bmatrix}
    \begin{bmatrix}
      A_1 \\
      A_2 \\
      B_1 \\
      B_2
    \end{bmatrix}
\end{align*}

Let $\mtx{u} =
\begin{bmatrix}
  -K_{v,angular} v - K_{a,angular} a & K_{v,angular} v + K_{a,angular} a
\end{bmatrix}^T$ be the input that rotates the drivetrain in place where each
wheel has an initial speed of $v$ and accelerates by $a$. Therefore, $\mtx{x} =
\begin{bmatrix}
  -v & v
\end{bmatrix}^T$ and $\dot{\mtx{x}} =
\begin{bmatrix}
  -a & a
\end{bmatrix}^T$.
\begin{align*}
  \begin{bmatrix}
    -a \\
    a
  \end{bmatrix} &=
    \begin{bmatrix}
      A_1 & A_2 \\
      A_2 & A_1
    \end{bmatrix}
    \begin{bmatrix}
      -v \\
      v
    \end{bmatrix} +
    \begin{bmatrix}
      B_1 & B_2 \\
      B_2 & B_1
    \end{bmatrix}
    \begin{bmatrix}
      -K_{v,angular} v - K_{a,angular} a \\
      K_{v,angular} v + K_{a,angular} a
    \end{bmatrix} \\
  \begin{bmatrix}
    -a \\
    a
  \end{bmatrix} &=
    \begin{bmatrix}
      -A_1 & A_2 \\
      -A_2 & A_1
    \end{bmatrix}
    \begin{bmatrix}
      v \\
      v
    \end{bmatrix} +
    \begin{bmatrix}
      -B_1 & B_2 \\
      -B_2 & B_1
    \end{bmatrix}
    \begin{bmatrix}
      K_{v,angular} v + K_{a,angular} a \\
      K_{v,angular} v + K_{a,angular} a
    \end{bmatrix} \\
  \begin{bmatrix}
    -a \\
    a
  \end{bmatrix} &=
    \begin{bmatrix}
      -A_1 & A_2 \\
      A1 & -A_2
    \end{bmatrix}
    \begin{bmatrix}
      v \\
      v
    \end{bmatrix} +
    \begin{bmatrix}
      -B_1 & B_2 \\
      B_1 & -B_2
    \end{bmatrix}
    \begin{bmatrix}
      K_{v,angular} v + K_{a,angular} a \\
      K_{v,angular} v + K_{a,angular} a
    \end{bmatrix}
\end{align*}

Since the column vectors contain the same element, the elements in the second
row can be rearranged.
\begin{equation*}
  \begin{bmatrix}
    -a \\
    -a
  \end{bmatrix} =
  \begin{bmatrix}
    -A_1 & A_2 \\
    -A_1 & A_2
  \end{bmatrix}
  \begin{bmatrix}
    v \\
    v
  \end{bmatrix} +
  \begin{bmatrix}
    -B_1 & B_2 \\
    -B_1 & B_2
  \end{bmatrix}
  \begin{bmatrix}
    K_{v,angular} v + K_{a,angular} a \\
    K_{v,angular} v + K_{a,angular} a
  \end{bmatrix}
\end{equation*}

Since the rows are linearly dependent, we can use just one of them.
\begin{align*}
  -a &=
    \begin{bmatrix}
      -A_1 & A_2
    \end{bmatrix} v +
    \begin{bmatrix}
      -B_1 & B_2
    \end{bmatrix}
    \begin{bmatrix}
      K_{v,angular} v + K_{a,angular} a
    \end{bmatrix} \\
  -a &= -v A_1 + v A_2 - (K_{v,angular} v + K_{a,angular} a) B_1 + K_{v,angular} v + K_{a,angular} a) B_2 \\
  -a &=
    \begin{bmatrix}
      -v & v & -(K_{v,angular} v + K_{a,angular} a) & K_{v,angular} v+ K_{a,angular} a
    \end{bmatrix}
    \begin{bmatrix}
      A_1 \\
      A_2 \\
      B_1 \\
      B_2
    \end{bmatrix} \\
  a &=
    \begin{bmatrix}
      v & -v & K_{v,angular} v + K_{a,angular} a & -(K_{v,angular} v + K_{a,angular} a)
    \end{bmatrix}
    \begin{bmatrix}
      A_1 \\
      A_2 \\
      B_1 \\
      B_2
    \end{bmatrix}
\end{align*}

Now stack the rows.
\begin{equation*}
  \begin{bmatrix}
    0 \\
    a \\
    0 \\
    a
  \end{bmatrix} =
  \begin{bmatrix}
    1 & 1 & K_{v,linear} & K_{v,linear} \\
    v & v & K_{v,linear} v + K_{a,linear} a & K_{v,linear} v + K_{a,linear} a \\
    -1 & 1 & -K_{v,angular} & K_{v,angular} \\
    v & -v & K_{v,angular} v + K_{a,angular} a & -(K_{v,angular} v + K_{a,angular} a)
  \end{bmatrix}
  \begin{bmatrix}
    A_1 \\
    A_2 \\
    B_1 \\
    B_2
  \end{bmatrix}
\end{equation*}

Solve for matrix elements with Wolfram Alpha. Let
$b = K_{v,linear}$, $c = K_{a,linear}$, $d = K_{v,angular}$, and $f = K_{a,angular}$.
\begin{verbatim}
inverse of {{1, 1, b, b}, {v, v, b v + c a, b v + c a},
  {-1, 1, -d, d}, {v, -v, d v + f a, -(d v + f a)}} * {{0}, {a}, {0}, {a}}
\end{verbatim}
\begin{align*}
  \begin{bmatrix}
    A_1 \\
    A_2 \\
    B_1 \\
    B_2
  \end{bmatrix} &= \frac{1}{2cf}
  \begin{bmatrix}
    -cd - bf \\
    cd - bf \\
    c + f \\
    f - c
  \end{bmatrix} \\
  \begin{bmatrix}
    A_1 \\
    A_2 \\
    B_1 \\
    B_2
  \end{bmatrix} &= \frac{1}{2 K_{a,linear} K_{a,angular}}
  \begin{bmatrix}
    -K_{a,linear} K_{v,angular} - K_{v,linear} K_{a,angular} \\
    K_{a,linear} K_{v,angular} - K_{v,linear} K_{a,angular} \\
    K_{a,linear} + K_{a,angular} \\
    K_{a,angular} - K_{a,linear}
  \end{bmatrix}
\end{align*}

To summarize,
\begin{theorem}[Drivetrain velocity model]
  \begin{equation*}
    \dot{\mtx{x}} = \mtx{A}\mtx{x} + \mtx{B}\mtx{u}
  \end{equation*}
  \begin{equation*}
    \begin{array}{cc}
    \mtx{x} =
      \begin{bmatrix}
        \text{left velocity} \\
        \text{right velocity}
      \end{bmatrix} &
    \mtx{u} =
      \begin{bmatrix}
        \text{left voltage} \\
        \text{right voltage}
      \end{bmatrix}
    \end{array}
  \end{equation*}
  \begin{equation*}
    \dot{\mtx{x}} =
      \begin{bmatrix}
        A_1 & A_2 \\
        A_2 & A_1
      \end{bmatrix} \mtx{x} +
      \begin{bmatrix}
        B_1 & B_2 \\
        B_2 & B_1
      \end{bmatrix} \mtx{u}
  \end{equation*}
  \begin{equation}
    \begin{bmatrix}
      A_1 \\
      A_2 \\
      B_1 \\
      B_2
    \end{bmatrix} = \frac{1}{2 K_{a,linear} K_{a,angular}}
      \begin{bmatrix}
        -K_{a,linear} K_{v,angular} - K_{v,linear} K_{a,angular} \\
        K_{a,linear} K_{v,angular} - K_{v,linear} K_{a,angular} \\
        K_{a,linear} + K_{a,angular} \\
        K_{a,angular} - K_{a,linear}
      \end{bmatrix}
  \end{equation}
\end{theorem}

If $K_v$ and $K_a$ are the same for both the linear and angular cases, it devolves to the one-dimensional case. This means the left and right sides are decoupled.
\begin{align*}
  \begin{bmatrix}
    A_1 \\
    A_2 \\
    B_1 \\
    B_2
  \end{bmatrix} &= \frac{1}{2 K_a K_a}
    \begin{bmatrix}
      -K_a K_v - K_v K_a \\
      K_a K_v - K_v K_a \\
      K_a + K_a \\
      K_a - K_a
    \end{bmatrix} \\
  \begin{bmatrix}
    A_1 \\
    A_2 \\
    B_1 \\
    B_2
  \end{bmatrix} &= \frac{1}{2 K_a K_a}
    \begin{bmatrix}
      -2K_v K_a \\
      0 \\
      2 K_a \\
      0
    \end{bmatrix} \\
  \begin{bmatrix}
    A_1 \\
    A_2 \\
    B_1 \\
    B_2
  \end{bmatrix} &=
    \begin{bmatrix}
      -\frac{K_v}{K_a} \\
      0 \\
      \frac{1}{K_a} \\
      0
    \end{bmatrix}
\end{align*}
