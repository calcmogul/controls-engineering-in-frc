\section{1-DOF mechanism state-space model}
\begin{equation*}
  \dot{x} = \alpha x + \beta u + \gamma \sgn(x)
\end{equation*}
\begin{equation*}
  \dot{x} = \text{acceleration}
  \quad
  x = \text{velocity}
  \quad
  u = \text{voltage}
\end{equation*}

We can determine $\alpha$, $\beta$, and $\gamma$ by applying ordinary least
squares (OLS) to vectors of recorded input voltage, velocity, and acceleration
data from quasistatic velocity tests and acceleration tests, where acceleration
is the dependent variable. If acceleration isn't directly available, it can be
computed numerically from the velocity data and filtered.

Solving this model for $u$ gives the following feedforward model.
\begin{align}
  u &= -\frac{\gamma}{\beta} \sgn(x) - \frac{\alpha}{\beta}x +
    \frac{1}{\beta}\dot{x} \nonumber
  \intertext{Let $K_s = -\frac{\gamma}{\beta}$, $K_v = -\frac{\alpha}{\beta}$,
    and $K_a = \frac{1}{\beta}$.}
  u &= K_s \sgn(x) + K_v x + K_a \dot{x} \label{eq:ols_u_ff}
\end{align}

$K_s$ is a constant that describes how much voltage is required to overcome
friction and start moving.

$K_v$ is a proportional constant that describes how much voltage is required to
maintain a given constant velocity by offsetting the electromagnetic resistance
of the motor and any friction that increases linearly with speed (viscous drag).
The relationship between speed and voltage (for a given initial acceleration) is
linear for permanent-magnet DC motors in the FRC operating regime.

$K_a$ is a proportional constant that describes how much voltage is required to
induce a given acceleration in the motor shaft. As with $K_v$, the relationship
between voltage and acceleration (for a given initial velocity) is linear.

Convert equation (\ref{eq:ols_u_ff}) to state-space form by solving for
$\dot{x}$.
\begin{equation*}
  \dot{x} = -\frac{K_v}{K_a}x + \frac{1}{K_a}u - \frac{K_s}{K_a}\sgn(x)
\end{equation*}

By inspection, $\mat{A} = -\frac{K_v}{K_a}$, $\mat{B} = \frac{1}{K_a}$, and
$\mat{c} = -\frac{K_s}{K_a}\sgn(\mat{x})$ in the state-space model
$\dot{\mat{x}} = \mat{A}\mat{x} + \mat{B}\mat{u} + \mat{c}$. A model with
position and velocity states would be
\begin{theorem}[1-DOF mechanism position model]
  \label{thm:1-dof_position_model}
  \begin{equation*}
    \dot{\mat{x}} = \mat{A}\mat{x} + \mat{B}\mat{u} + \mat{c}
  \end{equation*}
  \begin{equation*}
    \mat{x} =
    \begin{bmatrix}
      \text{position} \\
      \text{velocity}
    \end{bmatrix}
    \quad
    \mat{u} =
    \begin{bmatrix}
      \text{voltage}
    \end{bmatrix}
  \end{equation*}
  \begin{equation}
    \dot{\mat{x}} =
    \begin{bmatrix}
      0 & 1 \\
      0 & -\frac{K_v}{K_a}
    \end{bmatrix}
    \mat{x} +
    \begin{bmatrix}
      0 \\
      \frac{1}{K_a}
    \end{bmatrix}
    \mat{u} +
    \begin{bmatrix}
      0 \\
      -\frac{K_s}{K_a}\sgn(\text{velocity})
    \end{bmatrix}
  \end{equation}
\end{theorem}

The model in theorem \ref{thm:1-dof_position_model} is undefined when $K_a = 0$.
To design an LQR for such a system, use the model in theorem
\ref{thm:1-dof_position_model_ka=0} instead. As $K_a$ tends to zero,
acceleration requires no effort and velocity becomes an input for position
control.
\begin{theorem}[1-DOF mechanism position model ($K_a = 0$)]
  \label{thm:1-dof_position_model_ka=0}
  \begin{equation*}
    \dot{\mat{x}} = \mat{A}\mat{x} + \mat{B}\mat{u}
  \end{equation*}
  \begin{equation*}
    \mat{x} =
    \begin{bmatrix}
      \text{position}
    \end{bmatrix}
    \quad
    \mat{u} =
    \begin{bmatrix}
      \text{velocity}
    \end{bmatrix}
  \end{equation*}
  \begin{equation}
    \dot{\mat{x}} =
    \begin{bmatrix}
      0
    \end{bmatrix}
    \mat{x} +
    \begin{bmatrix}
      1
    \end{bmatrix}
    \mat{u}
  \end{equation}
\end{theorem}
