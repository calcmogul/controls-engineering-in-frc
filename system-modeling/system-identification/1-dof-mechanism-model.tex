\section{1-DOF mechanism model}
\begin{equation*}
  \dot{\mat{x}} = \mat{A}\mat{x} + \mat{B}\mat{u}
\end{equation*}
\begin{equation*}
  \begin{array}{ccc}
    \mat{x} = \text{velocity} & \dot{\mat{x}} = \text{acceleration} &
      \mat{u} = \text{voltage}
  \end{array}
\end{equation*}

We want to derive what $\mat{A}$ and $\mat{B}$ are from the following
feedforward model
\begin{equation*}
  \mat{u} = K_v v + K_a a
\end{equation*}

where $K_v$ and $K_a$ are constants that should be measured from empirical
quasistatic velocity tests and acceleration tests.

$K_v$ is a proportional constant that describes how much voltage is required to
maintain a given constant velocity by offsetting the electromagnetic resistance
of the motor and any friction that increases linearly with speed (viscous drag).
The relationship between speed and voltage (at constant acceleration) is linear
for permanent-magnet DC motors in the FRC operating regime.

$K_a$ is a proportional constant that describes how much voltage is required to
induce a given acceleration in the motor shaft. As with $K_v$, the relationship
between voltage and acceleration (at constant velocity) is linear.

Let $\mat{u} = K_v v$ be the input that makes the system move at a constant
velocity $v$. Therefore, $\mat{x} = v$ and $\dot{\mat{x}} = \mat{0}$. Substitute
these into the state-space model.
\begin{equation}
  0 = \mat{A}v + \mat{B}(K_v v) \label{eq:sysid-eq1}
\end{equation}

Let $\mat{u} = K_v v + K_a a$ be the input that accelerates the system by $a$
from an initial velocity of $v$. Therefore, $\mat{x} = v$ and
$\dot{\mat{x}} = a$. Substitute these into the state-space model.
\begin{equation}
  a = \mat{A}v + \mat{B}(K_v v + K_a a) \label{eq:sysid-eq2}
\end{equation}

Subtract equation \eqref{eq:sysid-eq1} from equation \eqref{eq:sysid-eq2}.
\begin{align*}
  a &= \mat{B} (K_a a) \\
  1 &= \mat{B} K_a \\
  \mat{B} &= \frac{1}{K_a}
\end{align*}

Substitute $\mat{B}$ back into \eqref{eq:sysid-eq1} to obtain $\mat{A}$.
\begin{align*}
  \mat{0} &= \mat{A}v + \left(\frac{1}{K_a}\right)(K_v v) \\
  \mat{0} &= \mat{A}v + \frac{K_v}{K_a} v \\
  -\frac{K_v}{K_a} v &= \mat{A}v \\
  \mat{A} &= -\frac{K_v}{K_a}
\end{align*}

A model with position and velocity states would be
\begin{theorem}[1-DOF mechanism position model]
  \begin{equation*}
    \dot{\mat{x}} = \mat{A}\mat{x} + \mat{B}\mat{u}
  \end{equation*}
  \begin{equation*}
    \begin{array}{cc}
    \mat{x} =
      \begin{bmatrix}
        \text{position} \\
        \text{velocity}
      \end{bmatrix} &
    \mat{u} =
      \begin{bmatrix}
        \text{voltage}
      \end{bmatrix}
    \end{array}
  \end{equation*}
  \begin{equation}
    \dot{\mat{x}} =
      \begin{bmatrix}
        0 & 1 \\
        0 & -\frac{K_v}{K_a}
      \end{bmatrix}
      \mat{x} +
      \begin{bmatrix}
        0 \\
        \frac{1}{K_a}
      \end{bmatrix}
      \mat{u}
  \end{equation}
\end{theorem}
