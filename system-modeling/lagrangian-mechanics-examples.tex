\chapterimage{system-modeling.jpg}{Hills by northbound freeway between Santa Maria and Ventura}

\chapter{Lagrangian mechanics examples}

A \gls{model} is a set of differential equations describing how the \gls{system}
behaves over time. There are two common approaches for developing them.
\begin{enumerate}
  \item Collecting data on the physical system's behavior and performing
    \gls{system} identification with it.
  \item Using physics to derive the \gls{system}'s model from first principles.
\end{enumerate}

This chapter covers the second approach using Lagrangian mechanics.

We suggest reading \textit{An introduction to Lagrangian and Hamiltonian
Mechanics} by Simon J.A. Malham for the basics
\cite{bib:an_intro_to_lagrangian_and_hamiltonian_mechanics}.

\renewcommand*{\chapterpath}{\partpath/lagrangian-mechanics-examples}
\section{Pendulum}

See \url{https://underactuated.mit.edu/multibody.html#section1} for a derivation
and \url{https://underactuated.mit.edu/pend.html} for extended analysis.

\section{Cart-pole}

See \url{https://underactuated.mit.edu/acrobot.html#cart_pole} for a derivation
of the kinematics and dynamics of a cart-pole via Lagrangian mechanics.

