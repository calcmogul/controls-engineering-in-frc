\chapterimage{system-modeling.jpg}{Hills by northbound freeway between Santa Maria and Ventura}

\chapter{Lagrangian mechanics examples}

A \gls{model} is a set of differential equations describing how the \gls{system}
behaves over time. There are two common approaches for developing them.
\begin{enumerate}
  \item Collecting data on the physical system's behavior and performing
    \gls{system} identification with it.
  \item Using physics to derive the \gls{system}'s model from first principles.
\end{enumerate}

This chapter covers the second approach using Lagrangian mechanics.

We suggest reading \textit{An introduction to Lagrangian and Hamiltonian
Mechanics} by Simon J.A. Malham for the basics
\cite{bib:an_intro_to_lagrangian_and_hamiltonian_mechanics}. Then, we'll apply
Lagrangian mechanics to FRC mechanisms in particular.

The \glspl{model} derived here should cover most types of motion seen on an FRC
robot. Furthermore, they can be easily tweaked to describe many types of
mechanisms just by pattern-matching. There's only so many ways to hook up a mass
to a motor in FRC. The flywheel \gls{model} can be used for spinning mechanisms,
the elevator \gls{model} can be used for spinning mechanisms transformed to
linear motion, and the single-jointed arm \gls{model} can be used for rotating
servo mechanisms (it's just the flywheel \gls{model} augmented with a position
\gls{state}).

These \glspl{model} assume all motor controllers driving DC brushed motors are
set to brake mode instead of coast mode. Brake mode behaves the same as coast
mode except where the applied voltage is zero. In brake mode, the motor leads
are shorted together to prevent movement. In coast mode, the motor leads are an
open circuit.

\renewcommand*{\chapterpath}{\partpath/lagrangian-mechanics-examples}
\section{Single-jointed arm}

\subsection{Continuous state-space model}
\index{FRC models!single-jointed arm equations}

The angle and angular rate derivatives of the arm can be written as

\begin{align}
  \dot{\theta}_{arm} &= \omega_{arm} \label{eq:arm_cont_ss_pos} \\
  \dot{\omega}_{arm} &= \dot{\omega}_{arm} \label{eq:arm_cont_ss_vel}
\end{align}

By equation (\ref{eq:dot_omega_arm})

\begin{equation*}
  \dot{\omega}_{arm} = -\frac{G^2 K_t}{K_v RJ} \omega_{arm} + \frac{G K_t}{RJ} V
\end{equation*}

\begin{theorem}[Single-jointed arm state-space model]
  \begin{align*}
    \dot{\mtx{x}} &= \mtx{A} \mtx{x} + \mtx{B} \mtx{u} \\
    \mtx{y} &= \mtx{C} \mtx{x} + \mtx{D} \mtx{u}
  \end{align*}
  \begin{equation*}
    \begin{array}{ccc}
      \mtx{x} =
      \begin{bmatrix}
        \theta_{arm} \\
        \omega_{arm}
      \end{bmatrix} &
      \mtx{y} = \theta_{arm} &
      \mtx{u} = V
    \end{array}
  \end{equation*}
  \begin{equation}
    \begin{array}{cccc}
      \mtx{A} =
      \begin{bmatrix}
        0 & 1 \\
        0 & -\frac{G^2 K_t}{K_v RJ}
      \end{bmatrix} &
      \mtx{B} =
      \begin{bmatrix}
        0 \\
        \frac{G K_t}{RJ}
      \end{bmatrix} &
      \mtx{C} =
      \begin{bmatrix}
        1 & 0
      \end{bmatrix} &
      \mtx{D} = 0
    \end{array}
  \end{equation}
\end{theorem}

\subsection{Model augmentation}

As per subsection \ref{subsec:u_error_estimation}, we will now augment the
\gls{model} so a $u_{error}$ term is added to the \gls{control input}.

The \gls{plant} and \gls{observer} augmentations should be performed before the
\gls{model} is \glslink{discretization}{discretized}. After the \gls{controller}
gain is computed with the unaugmented discrete \gls{model}, the controller may
be augmented. Therefore, the \gls{plant} and \gls{observer} augmentations assume
a continuous \gls{model} and the \gls{controller} augmentation assumes a
discrete \gls{controller}.

\begin{equation*}
  \begin{array}{ccc}
    \mtx{x}_{aug} =
    \begin{bmatrix}
      \mtx{x} \\
      u_{error}
    \end{bmatrix} &
    \mtx{y} = \theta_{arm} &
    \mtx{u} = V
  \end{array}
\end{equation*}

\begin{equation}
  \begin{array}{cccc}
    \mtx{A}_{aug} =
    \begin{bmatrix}
      \mtx{A} & \mtx{B} \\
      \mtx{0}_{1 \times 2} & 0
    \end{bmatrix} &
    \mtx{B}_{aug} =
    \begin{bmatrix}
      \mtx{B} \\
      0
    \end{bmatrix} &
    \mtx{C}_{aug} =
    \begin{bmatrix}
      \mtx{C} & 0
    \end{bmatrix} &
    \mtx{D}_{aug} = \mtx{D}
  \end{array}
\end{equation}

\begin{equation}
  \begin{array}{cc}
    \mtx{K}_{aug} = \begin{bmatrix}
      \mtx{K} & 1
    \end{bmatrix} &
    \mtx{r}_{aug} = \begin{bmatrix}
      \mtx{r} \\
      0
    \end{bmatrix}
  \end{array}
\end{equation}

This will compensate for unmodeled dynamics such as gravity or other external
loading from lifted objects. However, if only gravity compensation is desired,
a feedforward of the form $u_{ff} = V_{gravity} \cos\theta$ is preferred where
$V_{gravity}$ is the voltage required to hold the arm level with the ground and
$\theta$ is the angle of the arm with the ground.

\subsection{Simulation}

Python Control will be used to \glslink{discretization}{discretize} the
\gls{model} and simulate it. One of the frccontrol
examples\footnote{\url{https://github.com/calcmogul/frccontrol/blob/master/examples/single_jointed_arm.py}}
creates and tests a controller for it.

\begin{remark}
  Python Control currently doesn't support finding the transmission zeroes of
  MIMO \glspl{system} with a different number of \glspl{input} than
  \glspl{output}, so \texttt{control.pzmap()} and
  \texttt{frccontrol.System.plot\_pzmaps()} fail with an error if Slycot isn't
  installed.
\end{remark}

Figure \ref{fig:single_jointed_arm_pzmaps} shows the pole-zero maps for the
open-loop \gls{system}, closed-loop \gls{system}, and \gls{observer}. Figure
\ref{fig:single_jointed_arm_response} shows the \gls{system} response with them.

\begin{svg}{build/frccontrol/examples/single_jointed_arm_pzmaps}
  \caption{Single-jointed arm pole-zero maps}
  \label{fig:single_jointed_arm_pzmaps}
\end{svg}

\begin{svg}{build/frccontrol/examples/single_jointed_arm_response}
  \caption{Single-jointed arm response}
  \label{fig:single_jointed_arm_response}
\end{svg}

\subsection{Implementation}

The script linked above also generates two files: SingleJointedArmCoeffs.h and
SingleJointedArmCoeffs.cpp. These can be used with the WPILib StateSpacePlant,
StateSpaceController, and StateSpaceObserver classes in C++ and Java. A C++
implementation of this single-jointed arm controller is available
online\footnote{\url{https://github.com/calcmogul/allwpilib/tree/state-space/wpilibcExamples/src/main/cpp/examples/StateSpaceSingleJointedArm}}.

\section{Pendulum}

See \url{https://underactuated.mit.edu/multibody.html#section1} for a derivation
and \url{https://underactuated.mit.edu/pend.html} for extended analysis.

