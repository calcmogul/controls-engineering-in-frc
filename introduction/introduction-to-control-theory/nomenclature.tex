\section{Nomenclature}

Most resources for advanced engineering topics assume a level of knowledge well
above that which is necessary. Part of the problem is the use of jargon. While
it efficiently communicates ideas to those within the field, new people who
aren't familiar with it are lost. See the glossary for a list of words and
phrases commonly used in control theory, their origins, and their meaning. Links
to the glossary are provided for certain words throughout the book and will use
\textcolor{glscolor}{this color}.

The \gls{system} or collection of actuators being controlled by a
\gls{control system} is called the \gls{plant}. \Glspl{controller} which don't
include information measured from the \gls{plant}'s \gls{output} are called
open-loop \glspl{controller}. \Glspl{controller} which incorporate information
fed back from the \gls{plant}'s \gls{output} are called closed-loop
\glspl{controller} or feedback \glspl{controller}.

Table \ref{tab:plant_vs_controller} describes how the terms \gls{input} and
\gls{output} apply to \glspl{plant} versus \glspl{controller} and what letters
are commonly associated with each when working with them. Namely, that the terms
\gls{input} and \gls{output} are defined with respect to the \gls{plant}, not
the \gls{controller}.

\begin{booktable}
  \begin{tabular}{|l|ll|}
    \hline
    \rowcolor{headingbg}
    & \textbf{Plant} & \textbf{Controller} \\
    \hline
    Inputs & $u(t)$ & $r(t)$, $y(t)$ \\
    Outputs & $y(t)$ & $u(t)$ \\
    \hline
  \end{tabular}
  \caption{Plant versus controller nomenclature}
  \label{tab:plant_vs_controller}
\end{booktable}
