\section{Ethos of this book}

This book is intended as both a tutorial for new students and as a reference
manual for more experienced readers who need to review a thing or two. While it
isn't comprehensive, the reader will hopefully learn enough to either implement
the concepts presented themselves or know where to look for more information.

Some parts are mathematically rigorous, but I believe in giving students a solid
theoretical foundation with emphasis on intuition so they can apply it to new
problems. To achieve deep understanding of the topics in this book, math is
unavoidable. With that said, I try to provide practical and intuitive
explanations whenever possible.

\subsection{Role of classical control theory}

The sections on classical control theory are only included to develop geometric
intuition for the mathematical machinery of modern control theory. Many tools
exclusive to classical control theory (root locus, Bode plots, Nyquist plots,
etc.) aren't useful for or relevant to the examples presented, so they serve
only to complicate the learning process.

Classical control theory is interesting in that one can perform stability and
robustness analyses and design reasonable controllers for systems on the back of
a napkin. It's also useful for controlling systems which don't have a model. One
can generate a Bode plot of a system by feeding in sine waves of increasing
frequency and recording the amplitude of the output oscillations. This data can
be used to create a transfer function or lead and lag compensators can be
applied directly based on the Bode plot. However, computing power is much more
plentiful nowadays; we should take advantage of this in the design phase and use
the more modern tools that enables when it makes sense.

This book uses LQR and modern control over, say, loop shaping with Bode and
Nyquist plots because we have accurate dynamical models to leverage, and LQR
allows directly expressing what the author is concerned with optimizing: state
excursion relative to control effort. Applying lead and lag compensators, while
effective for robust controller design, doesn't provide the same expressive
power.

\subsection{An integrated approach to nonlinear control theory}

Most teaching resources separate linear and nonlinear control with the latter
being reserved for a different course. Here, they are introduced together
because the concepts of nonlinear control apply often, and it isn't that much of
a leap (if Lyapunov stability isn't included). The control and estimation
chapters cover relevant tools for dealing with nonlinearities like linearization
when appropriate.
