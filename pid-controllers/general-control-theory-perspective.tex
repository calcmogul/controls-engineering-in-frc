\section{General control theory perspective}

PID control defines \gls{setpoint} as the desired position and
\gls{process variable} as the measured position. Control theory has more general
terms for these: \gls{reference} and \gls{output} respectively.

The derivative term is commonly used to ``slow down" the \gls{system} if it's
already heading toward the \gls{reference}. We will explore what $K_p$ and $K_d$
are really doing for a two-state \gls{system} (position and velocity) and why
$K_d$ acts that way.

First, we will rearrange the equation for a PD controller.

\begin{equation*}
  u_k = K_p e_k + K_d \frac{e_k - e_{k-1}}{dt}
\end{equation*}

where $u_k$ is the \gls{control input} at timestep $k$ and $e_k$ is the
\gls{error} at timestep $k$. $e_k$ is defined as $e_k = r_k - x_k$ where $r_k$
is the \gls{reference} and $x_k$ is the current \gls{state} at timestep $k$.

\begin{align*}
  u_k &= K_p (r_k - x_k) + K_d \frac{(r_k - x_k) - (r_{k-1} - x_{k-1})}{dt} \\
  u_k &= K_p (r_k - x_k) + K_d \frac{r_k - x_k - r_{k-1} + x_{k-1}}{dt} \\
  u_k &= K_p (r_k - x_k) + K_d \frac{r_k - r_{k-1} - x_k + x_{k-1}}{dt} \\
  u_k &= K_p (r_k - x_k) + K_d \frac{(r_k - r_{k-1}) - (x_k - x_{k-1})}{dt} \\
  u_k &= K_p (r_k - x_k) + K_d \left(\frac{r_k - r_{k-1}}{dt} -
    \frac{x_k - x_{k-1}}{dt}\right)
\end{align*}

Notice how $\frac{r_k - r_{k-1}}{dt}$ is the velocity of the \gls{reference}. By
the same reasoning, $\frac{x_k - x_{k-1}}{dt}$ is the \gls{system}'s velocity at
a given timestep. That means the $K_d$ term of the PD controller is driving the
estimated velocity to the \gls{reference} velocity. If the \gls{reference} is
constant, that means the $K_d$ term is trying to drive the velocity of the
\gls{system} to zero.

However, $K_p$ and $K_d$ are controlling the same actuator, and their effects
conflict with each other; $K_p$ is trying to make the \gls{system} move while
$K_d$ is trying to stop it. If $K_p$ is larger than $K_d$, one is in effect
slowing down the response of the controller during transients with the hope of
decreasing \gls{overshoot} and \gls{settling time}. If one makes $K_d$ much
larger than $K_p$, $K_d$ overpowers $K_p$ to bring the \gls{system} to a stop.
However, when the velocity is low enough, $K_p$ is stronger and starts
accelerating the \gls{system} again. This oscillatory behavior in the velocity
repeats as the \gls{system} moves toward the \gls{reference}.
