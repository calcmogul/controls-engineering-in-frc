\chapterimage{ss-model-examples.jpg}{Fake plant in hotel lobby in Ventura, CA}

\chapter{State-space model examples}

Up to now, we've just been teaching what tools are available. Now, we'll go into
specifics on how to apply them and provide advice on certain applications.

The \gls{model} examples are intended to walk the reader through the
implementation steps described in section \ref{sec:implementation_steps}. The
code shown in each example can be obtained from frccontrol's Git repository at
\url{https://github.com/calcmogul/frccontrol/tree/master/examples}. See appendix
\ref{ch:installing_python_packages} for setup instructions.

The \glspl{model} derived here should cover most types of motion seen on an FRC
robot. Furthermore, they can be easily tweaked to describe many types of
mechanisms just by pattern-matching. There's only so many ways to hook up a mass
to a motor in FRC. The flywheel \gls{model} can be used for spinning mechanisms,
the elevator \gls{model} can be used for spinning mechanisms transformed to
linear motion, and the single-jointed arm \gls{model} can be used for rotating
servo mechanisms (it's just the flywheel \gls{model} augmented with a position
\gls{state}).

These \glspl{model} assume all motor controllers driving DC brushed motors are
set to brake mode instead of coast mode. Brake mode behaves the same as coast
mode except where the applied voltage is zero. In brake mode, the motor leads
are shorted together to prevent movement. In coast mode, the motor leads are an
open circuit.

\renewcommand*{\chapterpath}{\partpath/ss-model-examples}
\chapterimage{implementation-steps.jpg}{Trees by Interdisciplinary Sciences building at UCSC}

\chapter{Implementation steps} \label{ch:implementation-steps}

\section{Derive physical model}

A \gls{model} is a set of differential equations describing how the system
behaves over time. There are two common approaches for developing them.

\begin{enumerate}
  \item Collecting data on the physical system's behavior and performing system
  identification with it.
  \item Using physics to derive the system's model from first principles.
\end{enumerate}

We'll use the second approach in this book.

\subsection{Kinematics and Dynamics}

Dynamics is a rather large topic, so for now, we'll just focus on the basics
required for working with the models in this book. We'll derive the same model,
a pendulum, using three approaches: sum of forces, sum of torques, and
conservation of energy.

\begin{figure}[H]
  \centering
  \begin{subfigure}{0.5\textwidth}
    \centering
    \begin{tikzpicture}
      % Save length of g-vector and theta to macros
      \pgfmathsetmacro{\Gvec}{1.5}
      \pgfmathsetmacro{\myAngle}{30}
      % Calculate lengths of vector components
      \pgfmathsetmacro{\Gcos}{\Gvec*cos(\myAngle)}
      \pgfmathsetmacro{\Gsin}{\Gvec*sin(\myAngle)}

      \coordinate (centro) at (0,0);
      \draw[dashed,gray,-] (centro) -- ++ (0,-3.5)
        node (mary) [black,below] {$ $};
      \draw[thick] (centro) -- ++(270+\myAngle:3) coordinate (bob);
      \path pic [draw,->,"$\theta$",angle eccentricity=1.5]
        {angle=mary--centro--bob};
      \draw [draw=violet,-stealth] (bob) -- ($(bob)!-\Gcos cm!(centro)$)
        coordinate (gcos)
        node[midway,above right] {$mg\cos\theta$};
      \draw [dashed,draw=red,-stealth] (bob) -- ($(bob)!2*\Gsin cm!90:(centro)$)
        coordinate node[midway,above right] {};
      \draw [draw=violet,-stealth] (bob) -- ($(bob)!\Gsin cm!90:(centro)$)
        coordinate (gsin)
        node[midway,above left] {$mg\sin\theta$};
      \draw [draw=blue,-stealth] (bob) -- ++(0,-\Gvec)
        coordinate (g)
        node[near end,left] {$mg$};
      \pic [draw,->,"$\theta$",angle eccentricity=1.5] {angle=g--bob--gcos};
      \filldraw [fill=black!40,draw=black] (bob) circle[radius=0.2];
    \end{tikzpicture}
    \caption{Force diagram of a pendulum.}
    \label{subfig:force_pendulum}
  \end{subfigure}%
  \begin{subfigure}{0.5\textwidth}
    \centering
    \begin{tikzpicture}
      % Save length of g-vector and theta to macros
      \pgfmathsetmacro{\Gvec}{1.5}
      \pgfmathsetmacro{\myAngle}{30}
      % Calculate lengths of vector components
      \pgfmathsetmacro{\Gcos}{\Gvec*cos(\myAngle)}
      \pgfmathsetmacro{\Gsin}{\Gvec*sin(\myAngle)}

      \coordinate (centro) at (0,0);
      \coordinate (heightmes_lo) at (-1,0);
      \coordinate (heightmes_hi) at (-0.25,0);
      \coordinate (h) at (\Gcos/2,\Gsin);

      \draw[thick] (centro) -- ++(270+\myAngle:3) coordinate (bob_lo);
      \draw[dashed,gray,-] (centro) -- ++ (0,0 |- bob_lo)
        node (mary) [black,below]{$ $};
      \draw[dashed,gray,-] (heightmes_lo |- bob_lo) -- (bob_lo)
        node [black,below]{$ $};
      \draw[<->] (heightmes_lo) -- ++ (0,0 |- bob_lo)
        node [black,pos=0.5,left]{$y_1$};
      \pic [draw,->, "$\theta$",angle eccentricity=1.5]
        {angle=mary--centro--bob_lo};

      % Save length of g-vector and theta to macros
      \pgfmathsetmacro{\Gvec}{1.5cm}
      \pgfmathsetmacro{\myAngle}{45}
      % Calculate lengths of vector components
      \pgfmathsetmacro{\Gcos}{\Gvec*cos(\myAngle)}
      \pgfmathsetmacro{\Gsin}{\Gvec*sin(\myAngle)}

      \draw[gray,thick] (centro) -- ++(270+\myAngle:3) coordinate (bob_hi);
      \pic [draw,->,"$\theta_0$",angle eccentricity=1.5,angle radius=1cm]
        {angle=mary--centro--bob_hi};

      \draw[dashed,gray,-] (0,0 |- bob_hi) -- (bob_hi)
        node (mary) [black,below]{$ $};
      \draw[<->] (heightmes_hi) -- ++ (0,0 |- bob_hi)
        node (mary) [black,pos=0.5,left]{$y_0$};
      \draw[<->] (h |- bob_hi) -- (h |- bob_lo)
        node [black,pos=0.5,left]{$h$};

      % Path of pendulum
      \pic [draw,dashed,gray,<-,angle eccentricity=1.5,angle radius=2*\Gvec]
        {angle=mary--centro--bob_hi};

      % Pendulum balls
      \filldraw [fill=black!40,draw=black] (bob_lo) circle[radius=0.2];
      \filldraw [fill=black!20,draw=gray] (bob_hi) circle[radius=0.2];
    \end{tikzpicture}
    \caption{Trigonometry of a pendulum.}
    \label{subfig:trig_pendulum}
  \end{subfigure}
  \caption{Pendulum force diagrams.}
\end{figure}

\subsubsection{Force derivation}

Consider figure \ref{subfig:force_pendulum}, which shows the forces acting on a
pendulum.

Note that the path of the pendulum sweeps out an arc of a circle. The angle
$\theta$ is measured in radians. The blue arrow is the gravitational force
acting on the bob, and the violet arrows are that same force resolved into
components parallel and perpendicular to the bob's instantaneous motion. The
direction of the bob's instantaneous velocity always points along the red axis,
which is considered the tangential axis because its direction is always tangent
to the circle. Consider Newton's second law

\begin{equation*}
  F = ma
\end{equation*}

where $F$ is the sum of forces on the object, $m$ is mass, and $a$ is the
acceleration. Because we are only concerned with changes in speed, and because
the bob is forced to stay in a circular path, we apply Newton's equation to the
tangential axis only. The short violet arrow represents the component of the
gravitational force in the tangential axis, and trigonometry can be used to
determine its magnitude. Therefore

\begin{align*}
  F &= -mg\sin\theta = ma \\
  a &= -g\sin\theta
\end{align*}

where $g$ is the acceleration due to gravity near the surface of the earth. The
negative sign on the right hand side implies that $\theta$ and a always point in
opposite directions. This makes sense because when a pendulum swings further to
the left, we would expect it to accelerate back toward the right.

This linear acceleration $a$ along the red axis can be related to the change in
angle $\theta$ by the arc length formulas; $s$ is arc length and $l$ is the
length of the pendulum.

\begin{align}
  s &= l\theta \label{eq:arc_length} \\
  v &= \frac{ds}{dt} = l\frac{d\theta}{dt} \nonumber \\
  a &= \frac{d^2s}{dt^2} = l\frac{d^2\theta}{dt^2} \nonumber
\end{align}

Therefore

\begin{align*}
  l\frac{d^2\theta}{dt^2} &= -g\sin\theta \\
  \frac{d^2\theta}{dt^2} &= -\frac{g}{l}\sin\theta \\
  \ddot{\theta} &= -\frac{g}{l}\sin\theta
\end{align*}

\subsubsection{Torque derivation}

The equation can be obtained using two definitions for torque.

\begin{equation*}
  \mtx{\tau} = \mtx{r} \times \mtx{F}
\end{equation*}

First start by defining the torque on the pendulum bob using the force due to
gravity.

\begin{equation*}
  \mtx{\tau} = \mtx{l} \times \mtx{F}_g
\end{equation*}

where $\mtx{l}$ is the length vector of the pendulum and $\mtx{F}_g$ is the
force due to gravity.

For now just consider the magnitude of the torque on the pendulum.

\begin{equation*}
  \lvert\tau\rvert = -mgl\sin\theta
\end{equation*}

where $m$ is the mass of the pendulum, $g$ is the acceleration due to gravity,
$l$ is the length of the pendulum and $\theta$ is the angle between the length
vector and the force due to gravity.

Next rewrite the angular momentum.

\begin{equation*}
  \mtx{L} = \mtx{r} \times \mtx{p} =
    m\mtx{r} \times (\mtx{\omega} \times \mtx{r})
\end{equation*}

Again just consider the magnitude of the angular momentum.

\begin{align*}
  \lvert\mtx{L}\rvert &= mr^2\omega \\
  \lvert\mtx{L}\rvert &= ml^2 \frac{d\theta}{dt} \\
  \frac{d}{dt}\lvert\mtx{L}\rvert &= ml^2 \frac{d^2\theta}{dt^2}
\end{align*}

According to $\tau = \frac{d\mtx{L}}{dt}$, we can just compare the magnitudes.

\begin{align*}
  -mgl\sin\theta &= ml^2\frac{d^2\theta}{dt^2} \\
  -\frac{g}{l}\sin\theta &= \frac{d^2\theta}{dt^2} \\
  \ddot{\theta} &= -\frac{g}{l}\sin\theta
\end{align*}

which is the same result from force analysis.

\subsubsection{Energy derivation}

The equation can also be obtained via the conservation of mechanical energy
principle: any object falling a vertical distance $h$ would acquire kinetic
energy equal to that which it lost to the fall. In other words, gravitational
potential energy is converted into kinetic energy. Change in potential energy is
given by

\begin{equation*}
  \Delta U = mgh
\end{equation*}

The change in kinetic energy (body started from rest) is given by

\begin{equation*}
  \Delta K = \frac{1}{1}mv^2
\end{equation*}

Since no energy is lost, the gain in one must be equal to the loss in the other

\begin{equation*}
  \frac{1}{2}mv^2 = mgh
\end{equation*}

The change in velocity for a given change in height can be expressed as

\begin{equation*}
  v = \sqrt{2gh}
\end{equation*}

Using equation (\ref{eq:arc_length}), this equation can be rewritten in terms of
$\frac{d\theta}{dt}$.

\begin{align}
  v = l\frac{d\theta}{dt} &= \sqrt{2gh} \nonumber \\
  \frac{d\theta}{dt} &= \frac{2gh}{l} \label{eq:energy_dtheta}
\end{align}

where $h$ is the vertical distance the pendulum fell. Look at figure \ref{subfig:trig_pendulum}, which presents the trigonometry of a pendulum. If the pendulum
starts its swing from some initial angle $\theta_0$, then $y_0$, the vertical
distance from the pivot point, is given by

\begin{equation*}
  y_0 = l\cos\theta_0
\end{equation*}

Similarly for $y_1$, we have

\begin{equation*}
  y_1 = l\cos\theta
\end{equation*}

Then $h$ is the difference of the two

\begin{equation*}
  h = l(\cos\theta - \cos\theta_0)
\end{equation*}

Substituting this into equation (\ref{eq:energy_dtheta}) gives

\begin{equation*}
  \frac{d\theta}{dt} = \sqrt{\frac{2g}{l}(\cos\theta - \cos\theta_0)}
\end{equation*}

This equation is known as the first integral of motion. It gives the velocity in
terms of the location and includes an integration constant related to the
initial displacement ($\theta_0$). We can differentiate by applying the chain
rule with respect to time. Doing so gives the acceleration.

\begin{align*}
  \frac{d}{dt}\frac{d\theta}{dt} &=
    \frac{d}{dt}\sqrt{\frac{2g}{l}(\cos\theta - \cos\theta_0)} \\
  \frac{d^2\theta}{dt^2} &= \frac{1}{2}\frac
    {-\frac{2g}{l}\sin\theta}
    {\sqrt{\frac{2g}{l}(\cos\theta - \cos\theta_0)}}\frac{d\theta}{dt} \\
  \frac{d^2\theta}{dt^2} &= \frac{1}{2}\frac
    {-\frac{2g}{l}\sin\theta}
    {\sqrt{\frac{2g}{l}(\cos\theta - \cos\theta_0)}}
    \sqrt{\frac{2g}{l}(\cos\theta - \cos\theta_0)} \\
  \frac{d^2\theta}{dt^2} &= -\frac{g}{l}\sin\theta \\
  \ddot{\theta} &= -\frac{g}{l}\sin\theta
\end{align*}

which is the same result from force analysis.

\section{Write model in state-space representation}

Below is the \gls{model} for a pendulum

\begin{equation*}
  \ddot{\theta} = -\frac{g}{l}\sin\theta
\end{equation*}

where $\theta$ is the angle of the pendulum and $l$ is the length of the
pendulum.

Since state-space representation requires that only single derivatives be used,
they should be broken up as separate states. We'll reassign $\theta$ to be $x_1$
and $\dot{\theta}$ to be $x_2$ so the derivatives are easier to keep straight
for state-space representation.

\begin{equation*}
  \ddot{x}_1 = -\frac{g}{l}\sin x_1
\end{equation*}

Now separate the states.

\begin{align*}
  \dot{x}_1 &= x_2 \\
  \dot{x}_2 &= -\frac{g}{l} \sin x_1
\end{align*}

Since this \gls{model} is nonlinear, we should linearize it. We will use the
small angle approximation ($\sin\theta = \theta$ for small values of $\theta$).

\begin{align*}
  \dot{x}_1 &= x_2 \\
  \dot{x}_2 &= -\frac{g}{l} x_1
\end{align*}

Now write the model in state-space representation.

\begin{align}
  \dot{
  \begin{bmatrix}
    x_1 \\
    x_2
  \end{bmatrix}} =
  \begin{bmatrix}
    0 & 1 \\
    -\frac{g}{l} & 0
  \end{bmatrix}
  \begin{bmatrix}
    x_1 \\
    x_2
  \end{bmatrix}
\end{align}

\section{Add estimator for unmeasured states}

For full state feedback, knowledge of all states is required. If not all states
are measured directly, an estimator can be used to supplement them.

For example, we may only be measuring $\theta$ in the pendulum example, not
$\dot{\theta}$, so we'll need to estimate the latter.

\section{Implement controller}

Use Bryson's rule when making the performance vs actuation effort trade-off.
Optimizing for performance will get you to the reference as fast as possible
while optimizing actuation effort will get you to the reference in the most
``fuel-efficient" way possible. The latter, for example, would potentially avoid
voltage drops from motor usage on robots with a limited power supply, but the
result would be slower to reach the reference.

\section{Simulate model/controller}

This can be done in any platform supporting numerical computation. Common
choices are MATLAB, v-REP, or Python. Tweak the LQR gains as necessary.

If you're comfortable enough with it, you can use the controller designed by LQR
as a starting point and tweak the pole locations after that with pole placement
to produce the desired response.

\section{Verify pole locations}

Check the pole locations as a sanity check and to potentially gain an intuition
for the chosen pole locations.

\section{Unit test}

Write unit tests to test the \gls{model} performance and \gls{robustness} under
different initial conditions and command inputs. If you are using C++, we
recommend Google Test.

\section{Test on real system}

Try the controller on a real \gls{system} with low maximum outputs for safety.
The outputs can be increased after verifying the sensors function and mechanisms
move the correct direction.

\section{DC brushed motor}
\label{sec:dc_brushed_motor}

We will be deriving a first-order \gls{model} for a DC brushed motor. A
second-order \gls{model} would include the inductance of the motor windings as
well, but we're assuming the time constant of the inductor is small enough that
its affect on the \gls{model} behavior is negligible for FRC use cases (see
subsection \ref{subsec:root_locus} for a demonstration of this for a real DC
brushed motor).

The first-order \gls{model} will only require numbers from the motor's
datasheet. The second-order \gls{model} would require measuring the motor
inductance as well, which generally isn't in the datasheet. It can be difficult
to measure accurately without the right equipment.

\subsection{Equations of motion}

The circuit for a DC brushed motor is shown in figure
\ref{fig:dc_motor_circuit}.

\begin{bookfigure}
  \begin{tikzpicture}[auto, >=latex', circuit ee IEC,
                      set resistor graphic=var resistor IEC graphic]
    \node [opencircuit] (start) at (0,0) {};
    \node [] (V+) at (-0.5,0) { $+$ };
    \node [opencircuit] (end) at (0,-3.5) {};
    \node [] (V-) at (-0.5,-3.5) { $-$ };
    \node [coordinate] (topright) at (2.5,0) {};
    \node [coordinate] (bottomright) at (2.5,-3.5) {};
    \node [] at (0, -1.75) { $V$ };
    \draw (start) to (topright)
                  to [resistor={near start, info'={ $R$ }},
                      voltage source={near end, direction info'={<-},
                      info={ $V_{emf}=\frac{\omega}{K_v}$ }}] (bottomright)
                  to (end);
  \end{tikzpicture}

  \caption{DC brushed motor circuit}
  \label{fig:dc_motor_circuit}
\end{bookfigure}

$V$ is the voltage applied to the motor, $I$ is the current through the motor in
Amps, $R$ is the resistance across the motor in Ohms, $\omega$ is the angular
velocity of the motor in radians per second, and $K_v$ is the angular velocity
constant in radians per second per Volt. This circuit reflects the following
relation.

\begin{equation}
  V = IR + \frac{\omega}{K_v} \label{eq:motor_V}
\end{equation}

The mechanical relation for a DC brushed motor is

\begin{equation}
  \tau = K_t I
\end{equation}

where $\tau$ is the torque produced by the motor in Newton-meters and $K_t$ is
the torque constant in Newton-meters per Amp. Therefore

\begin{equation*}
  I = \frac{\tau}{K_t}
\end{equation*}

Substitute this into equation \eqref{eq:motor_V}.

\index{FRC models!DC brushed motor equations}
\begin{equation}
  V = \frac{\tau}{K_t} R + \frac{\omega}{K_v} \label{eq:motor_tau_V}
\end{equation}

\subsection{Calculating constants}

A typical motor's datasheet should include graphs of the motor's measured torque
and current for different angular velocities for a given voltage applied to the
motor. Figure \ref{fig:motor_data} is an example. Data for the most common
motors in FRC can be found at \url{https://motors.vex.com}.

\begin{svg}{build/code/motor_data}
  \caption{Example motor datasheet for 775pro}
  \label{fig:motor_data}
\end{svg}

\subsubsection{Torque constant $K_t$}

\begin{align}
  \tau &= K_t I \nonumber \\
  K_t &= \frac{\tau}{I} \nonumber \\
  K_t &= \frac{\tau_{stall}}{I_{stall}}
\end{align}

where $\tau_{stall}$ is the stall torque and $I_{stall}$ is the stall current of
the motor from its datasheet.

\subsubsection{Resistance $R$}

Recall equation \eqref{eq:motor_V}.

\begin{equation*}
  V = IR + \frac{\omega}{K_v}
\end{equation*}

When the motor is stalled, $\omega = 0$.

\begin{align}
  V &= I_{stall} R \nonumber \\
  R &= \frac{V}{I_{stall}}
\end{align}

where $I_{stall}$ is the stall current of the motor and $V$ is the voltage
applied to the motor at stall.

\subsubsection{Angular velocity constant $K_v$}

Recall equation \eqref{eq:motor_V}.

\begin{align*}
  V &= IR + \frac{\omega}{K_v} \\
  V - IR &= \frac{\omega}{K_v} \\
  K_v &= \frac{\omega}{V - IR}
\end{align*}

When the motor is spinning under no load

\begin{align}
  K_v &= \frac{\omega_{free}}{V - I_{free}R}
\end{align}

where $\omega_{free}$ is the angular velocity of the motor under no load (also
known as the free speed), and $V$ is the voltage applied to the motor when it's
spinning at $\omega_{free}$, and $I_{free}$ is the current drawn by the motor
under no load.

If several identical motors are being used in one gearbox for a mechanism,
multiply the stall torque by the number of motors.

\subsection{Current limiting}

Current limiting of a DC brushed motor reduces the maximum input voltage to
avoid exceeding a current threshold. Predictive current limiting uses a
projected estimate of the current, so the voltage is reduced before the current
threshold is exceeded. Reactive current limiting uses an actual current
measurement, so the voltage is reduced after the current threshold is exceeded.

The following pseudocode demonstrates each type of current limiting.

\begin{code}{Python}{snippets/current_limit.py}
  \caption{Limits current of DC motor to $I_{max}$}
\end{code}

\section{Elevator}

\subsection{Equations of motion}

This elevator consists of a DC brushed motor attached to a pulley that drives a
mass up or down.

\begin{bookfigure}
  \begin{tikzpicture}[auto, >=latex', circuit ee IEC,
                      set resistor graphic=var resistor IEC graphic]
    % \draw [help lines] (-1,-3) grid (7,4);

    % Electrical equivalent circuit
    \draw (0,2) to [voltage source={direction info'={->}, info'=$V$}] (0,0);
    \draw (0,2) to [current direction={info=$I$}] (0,3);
    \draw (0,3) -- (0.5,3);
    \draw (0.5,3) to [resistor={info={$R$}}] (2,3);

    \draw (2,3) -- (2.5,3);
    \draw (2.5,3) to [voltage source={direction info'={->}, info'=$V_{emf}$}]
      (2.5,0);
    \draw (0,0) -- (2.5,0);

    % Motor
    \begin{scope}[xshift=2.4cm,yshift=1.05cm]
      \draw[fill=black] (0,0) rectangle (0.2,0.9);
      \draw[fill=white] (0.1,0.45) ellipse (0.3 and 0.3);
    \end{scope}

    % Transmission gear one
    \begin{scope}[xshift=3.75cm,yshift=1.17cm]
      \draw[fill=black!50] (0.2,0.33) ellipse (0.08 and 0.33);
      \draw[fill=black!50, color=black!50] (0,0) rectangle (0.2,0.66);
      \draw[fill=white] (0,0.33) ellipse (0.08 and 0.33);
      \draw (0,0.66) -- (0.2,0.66);
      \draw (0,0) -- (0.2,0) node[pos=0.5,below] {$G$};
    \end{scope}

    % Output shaft of motor
    \begin{scope}[xshift=2.8cm,yshift=1.45cm]
      \draw[fill=black!50] (0,0) rectangle (0.95,0.1);
    \end{scope}

    % Angular velocity arrow of drive -> transmission
    \draw[line width=0.7pt,<-] (3.2,1) arc (-30:30:1) node[above] {$\omega_m$};

    % Transmission gear two
    \begin{scope}[xshift=3.75cm,yshift=1.83cm]
      \draw[fill=black!50] (0.2,0.68) ellipse (0.13 and 0.67);
      \draw[fill=black!50, color=black!50] (0,0) rectangle (0.2,1.35);
      \draw[fill=white] (0,0.68) ellipse (0.13 and 0.67);
      \draw (0,1.35) -- (0.2,1.35);
      \draw (0,0) -- (0.2,0);
    \end{scope}

    % Pulley rear chain
    \begin{scope}[xshift=5.03cm,yshift=0.32cm]
      \draw[fill=black!70, color=black!70] (0.01,2.17) rectangle (0.09,0);
      \draw (0,2.17) -- (0,0);
      \draw (0.1,2.17) -- (0.1,0);
    \end{scope}

    % Upper pulley
    \begin{scope}[xshift=5.05cm,yshift=2.09cm]
      \draw[fill=black!50] (0.2,0.4) ellipse (0.13 and 0.4);
      \draw[fill=black!70] (0.15,0.4) ellipse (0.13 and 0.4);
      \draw[fill=black!50, color=black!50] (0,0) rectangle (0.1,0.8);
      \draw[fill=black!70, color=black!70] (0.1,0) rectangle (0.15,0.8);
      \draw[fill=black!50] (0.05,0.4) ellipse (0.13 and 0.4);
      \draw[fill=black!50, color=black!50] (0,0) rectangle (0.05,0.8);
      \draw[fill=white] (0,0.4) ellipse (0.13 and 0.4);
      \draw (0,0) -- (0.2,0);
      \draw (0,0.8) -- (0.2,0.8);
    \end{scope}

    % Lower pulley
    \begin{scope}[xshift=5.05cm,yshift=-0.05cm]
      \draw[fill=black!50] (0.2,0.4) ellipse (0.13 and 0.4);
      \draw[fill=black!70] (0.15,0.4) ellipse (0.13 and 0.4);
      \draw[fill=black!50, color=black!50] (0,0) rectangle (0.1,0.8);
      \draw[fill=black!70, color=black!70] (0.1,0) rectangle (0.15,0.8);
      \draw[fill=black!50] (0.05,0.4) ellipse (0.13 and 0.4);
      \draw[fill=black!50, color=black!50] (0,0) rectangle (0.05,0.8);
      \draw[fill=white] (0,0.4) ellipse (0.13 and 0.4);
      \draw (0,0) -- (0.2,0);
      \draw (0,0.8) -- (0.2,0.8);
    \end{scope}

    % Transmission shaft from gear two to pulley
    \begin{scope}[xshift=4.09cm,yshift=2.42cm]
      \draw[fill=black!50] (0,0) rectangle (0.96,0.1);
    \end{scope}

    % Angular velocity arrow between transmission and pulley
    \draw[line width=0.7pt,->] (4.54,1.97) arc (-30:30:1) node[above]
      {$\omega_p$};

    % Pulley front chain
    \begin{scope}[xshift=5.23cm,yshift=0.32cm]
      \draw[fill=black!70, color=black!70] (0.01,2.17) rectangle (0.09,0);
      \draw (0,2.17) -- (0,0);
      \draw (0.1,2.17) -- (0.1,0);
    \end{scope}

    % Pulley radius arrow
    \begin{scope}[xshift=5.54cm,yshift=2.49]
      \draw[line width=0.7pt,<->] (0,-0.15) -- node[right] {$r$} (0,0.35);
    \end{scope}

    % Mass
    \begin{scope}[xshift=4.89cm,yshift=0.82cm]
      \fill[fill=white] (0,0.8) -- (0,0.2) -- (0.2,0) -- (0.2,0.2)
        -- (0.98,0.2) -- (0.78,0.8) -- cycle;
      \draw (0,0.8) -- (0.78,0.8);
      \draw (0,0.8) -- (0,0.2);
      \draw (0,0.2) -- (0.2,0);
      \draw (0,0.8) -- (0.2,0.6);
      \draw (0.78,0.8) -- (0.98,0.6);
      \draw[fill=white] (0.2,0.6) rectangle (0.98,0);
    \end{scope}

    % Mass velocity arrow
    \begin{scope}[xshift=6.04cm,yshift=0.95cm]
      \draw[line width=0.7pt,<-] (0,0.4) -- node {$v_m$} (0,0);
    \end{scope}

    % Descriptions inside graphic
    \draw (5.48,1.12) node {$m$};

    % Descriptions of subsystems under graphic
    \begin{scope}[xshift=-0.5cm,yshift=-0.28cm]
      \draw[decorate,decoration={brace,amplitude=10pt}]
        (3.5,0) -- (0,0) node[midway,yshift=-20pt] {circuit};
      \draw[decorate,decoration={brace,amplitude=10pt}]
        (7.05,0) -- (3.75,0) node[midway,yshift=-20pt] {mechanics};
    \end{scope}
  \end{tikzpicture}

  \caption{Elevator system diagram}
  \label{fig:elevator}
\end{bookfigure}

Gear ratios are written as output over input, so $G$ is greater than one in
figure \ref{fig:elevator}.

Based on figure \ref{fig:elevator}

\begin{equation}
  \tau_m G = \tau_p \label{eq:elevator_tau_m_ratio}
\end{equation}

where $G$ is the gear ratio between the motor and the pulley and $\tau_p$ is the
torque produced by the pulley.

\begin{equation}
  rF_m = \tau_p \label{eq:elevator_torque_pulley}
\end{equation}

where $r$ is the radius of the pulley. Substitute equation
(\ref{eq:elevator_tau_m_ratio}) into equation (\ref{eq:motor_tau_V}).

\begin{align*}
  V &= \frac{\frac{\tau_p}{G}}{K_t} R + \frac{\omega_m}{K_v} \\
  V &= \frac{\tau_p}{GK_t} R + \frac{\omega_m}{K_v}
\end{align*}

Substitute in equation (\ref{eq:elevator_torque_pulley}).

\begin{equation}
  V = \frac{rF_m}{GK_t} R + \frac{\omega_m}{K_v} \label{eq:elevator_Vinter1}
\end{equation}

The angular velocity of the motor armature $\omega_m$ is

\begin{equation}
  \omega_m = G \omega_p \label{eq:elevator_omega_m_ratio}
\end{equation}

where $\omega_p$ is the angular velocity of the pulley. The velocity of the mass
(the elevator carriage) is

\begin{equation*}
  v_m = r \omega_p
\end{equation*}

\begin{equation}
  \omega_p = \frac{v_m}{r} \label{eq:elevator_omega_p}
\end{equation}

Substitute equation (\ref{eq:elevator_omega_p}) into equation
(\ref{eq:elevator_omega_m_ratio}).

\begin{equation}
  \omega_m = G \frac{v_m}{r} \label{eq:elevator_omega_m}
\end{equation}

Substitute equation (\ref{eq:elevator_omega_m}) into equation
(\ref{eq:elevator_Vinter1}).

\begin{align*}
  V &= \frac{rF_m}{GK_t} R + \frac{G \frac{v_m}{r}}{K_v} \\
  V &= \frac{RrF_m}{GK_t} + \frac{G}{rK_v} v_m
\end{align*}

Solve for $F_m$.

\begin{align}
  \frac{RrF_m}{GK_t} &= V - \frac{G}{rK_v} v_m \nonumber \\
  F_m &= \left(V - \frac{G}{rK_v} v_m\right) \frac{GK_t}{Rr} \nonumber \\
  F_m &= \frac{GK_t}{Rr} V - \frac{G^2K_t}{Rr^2 K_v} v_m \label{eq:elevator_F_m}
\end{align}

\begin{equation}
  \sum F = ma_m \label{eq:elevator_F_ma}
\end{equation}

where $\sum F$ is the sum of forces applied to the elevator carriage, $m$ is the
mass of the elevator carriage in kilograms, and $a_m$ is the acceleration of the
elevator carriage.

\begin{equation*}
  ma_m = F_m
\end{equation*}

\begin{remark}
  Gravity is not part of the modeled dynamics because it complicates the
  state-space \gls{model} and the controller will behave well enough without it.
\end{remark}

\begin{align}
  ma_m &= \left(\frac{GK_t}{Rr} V - \frac{G^2K_t}{Rr^2 K_v} v_m\right)
    \nonumber \\
  a_m &= \frac{GK_t}{Rrm} V - \frac{G^2K_t}{Rr^2 mK_v} v_m
    \label{eq:elevator_accel}
\end{align}

\subsection{Continuous state-space model}
\index{FRC models!elevator equations}

The position and velocity of the elevator can be written as

\begin{align}
  \dot{x}_m &= v_m \label{eq:elevator_cont_ss_pos} \\
  \dot{v}_m &= a_m \label{eq:elevator_cont_ss_vel}
\end{align}

where by equation (\ref{eq:elevator_accel}),

\begin{equation*}
  a_m = \frac{GK_t}{Rrm} V - \frac{G^2 K_t}{Rr^2 m K_v} v_m
\end{equation*}

Substitute this into equation (\ref{eq:elevator_cont_ss_vel}).

\begin{align}
  \dot{v}_m &= \frac{GK_t}{Rrm} V - \frac{G^2 K_t}{Rr^2 m K_v} v_m \nonumber \\
  \dot{v}_m &= -\frac{G^2 K_t}{Rr^2 m K_v} v_m + \frac{GK_t}{Rrm} V
\end{align}

\begin{theorem}[Elevator state-space model]
  \begin{align*}
    \dot{\mtx{x}} &= \mtx{A} \mtx{x} + \mtx{B} \mtx{u} \\
    \mtx{y} &= \mtx{C} \mtx{x} + \mtx{D} \mtx{u}
  \end{align*}
  \begin{equation*}
    \begin{array}{ccc}
      \mtx{x} =
      \begin{bmatrix}
        x \\
        v_m
      \end{bmatrix} &
      \mtx{y} = x &
      \mtx{u} = V
    \end{array}
  \end{equation*}
  \begin{equation}
    \begin{array}{cccc}
      \mtx{A} =
      \begin{bmatrix}
        0 & 1 \\
        0 & -\frac{G^2 K_t}{Rr^2 mK_v}
      \end{bmatrix} &
      \mtx{B} =
      \begin{bmatrix}
        0 \\
        \frac{GK_t}{Rrm}
      \end{bmatrix} &
      \mtx{C} =
      \begin{bmatrix}
        1 & 0
      \end{bmatrix} &
      \mtx{D} = 0
    \end{array}
  \end{equation}
\end{theorem}

\subsection{Model augmentation}

As per subsection \ref{subsec:u_error_estimation}, we will now augment the
\gls{model} so a $u_{error}$ term is added to the \gls{control input}.

The \gls{plant} and \gls{observer} augmentations should be performed before the
\gls{model} is \glslink{discretization}{discretized}. After the \gls{controller}
gain is computed with the unaugmented discrete \gls{model}, the controller may
be augmented. Therefore, the \gls{plant} and \gls{observer} augmentations assume
a continuous \gls{model} and the \gls{controller} augmentation assumes a
discrete \gls{controller}.

\begin{equation*}
  \begin{array}{ccc}
    \mtx{x}_{aug} =
    \begin{bmatrix}
      x \\
      v_m \\
      u_{error}
    \end{bmatrix} &
    \mtx{y} = x &
    \mtx{u} = V
  \end{array}
\end{equation*}

\begin{equation}
  \begin{array}{cccc}
    \mtx{A}_{aug} =
    \begin{bmatrix}
      \mtx{A} & \mtx{B} \\
      \mtx{0}_{1 \times 2} & 0
    \end{bmatrix} &
    \mtx{B}_{aug} =
    \begin{bmatrix}
      \mtx{B} \\
      0
    \end{bmatrix} &
    \mtx{C}_{aug} = \begin{bmatrix}
      \mtx{C} & 0
    \end{bmatrix} &
    \mtx{D}_{aug} = \mtx{D}
  \end{array}
\end{equation}

\begin{equation}
  \begin{array}{cc}
    \mtx{K}_{aug} = \begin{bmatrix}
      \mtx{K} & 1
    \end{bmatrix} &
    \mtx{r}_{aug} = \begin{bmatrix}
      \mtx{r} \\
      0
    \end{bmatrix}
  \end{array}
\end{equation}

This will compensate for unmodeled dynamics such as gravity. However, using a
constant feedforward to counteract gravity is preferred over this method.

\subsection{Simulation}

Python Control will be used to \glslink{discretization}{discretize} the
\gls{model} and simulate it. One of the frccontrol
examples\footnote{\url{https://github.com/calcmogul/frccontrol/blob/master/examples/elevator.py}}
creates and tests a controller for it.

Figure \ref{fig:elevator_pzmaps} shows the pole-zero maps for the open-loop
\gls{system}, closed-loop \gls{system}, and \gls{observer}. Figure
\ref{fig:elevator_response} shows the \gls{system} response with them.

\begin{svg}{build/elevator_pzmaps}
  \caption{Elevator pole-zero maps}
  \label{fig:elevator_pzmaps}
\end{svg}

\begin{svg}{build/elevator_response}
  \caption{Elevator response}
  \label{fig:elevator_response}
\end{svg}

\subsection{Implementation}

The script linked above also generates two files: ElevatorCoeffs.h and
ElevatorCoeffs.cpp. These can be used with the WPILib StateSpacePlant,
StateSpaceController, and StateSpaceObserver classes in C++ and Java. A C++
implementation of this elevator controller is available online\footnote{
\url{https://github.com/calcmogul/allwpilib/tree/state-space/wpilibcExamples/src/main/cpp/examples/StateSpaceElevator}}.

\begin{remark}
  Instead of implementing $u_{error}$ estimation to compensate for gravity, one
  can apply a constant voltage feedforward since input voltage is proportional
  to force and gravity is a constant force.
\end{remark}

\section{Flywheel}
\label{sec:ss_model_flywheel}

This flywheel consists of a DC brushed motor attached to a spinning mass of
non-negligible moment of inertia.
\begin{bookfigure}
  \begin{tikzpicture}[auto, >=latex', circuit ee IEC,
                    set resistor graphic=var resistor IEC graphic]
  % \draw [help lines] (-1,-3) grid (7,4);

  % Electrical equivalent circuit
  \draw (0,2) to [voltage source={direction info'={->}, info'=$V$}] (0,0);
  \draw (0,2) to [current direction={info=$I$}] (0,3);
  \draw (0,3) -- (0.5,3);
  \draw (0.5,3) to [resistor={info={$R$}}] (2,3);

  \draw (2,3) -- (2.5,3);
  \draw (2.5,3) to [voltage source={direction info'={->}, info'=$V_{emf}$}]
    (2.5,0);
  \draw (0,0) -- (2.5,0);

  % Motor
  \begin{scope}[xshift=2.4cm,yshift=1.05cm]
    \draw[fill=black] (0,0) rectangle (0.2,0.9);
    \draw[fill=white] (0.1,0.45) ellipse (0.3 and 0.3);
  \end{scope}

  % Transmission gear one
  \begin{scope}[xshift=3.75cm,yshift=1.17cm]
    \draw[fill=black!50] (0.2,0.33) ellipse (0.08 and 0.33);
    \draw[fill=black!50, color=black!50] (0,0) rectangle (0.2,0.66);
    \draw[fill=white] (0,0.33) ellipse (0.08 and 0.33);
    \draw (0,0.66) -- (0.2,0.66);
    \draw (0,0) -- (0.2,0) node[pos=0.5,below] {$G$};
  \end{scope}

  % Output shaft of motor
  \begin{scope}[xshift=2.8cm,yshift=1.45cm]
    \draw[fill=black!50] (0,0) rectangle (0.95,0.1);
  \end{scope}

  % Angular velocity arrow of drive -> transmission
  \draw[line width=0.7pt,<-] (3.2,1) arc (-30:30:1) node[above] {$\omega_m$};

  % Transmission gear two
  \begin{scope}[xshift=3.75cm,yshift=1.83cm]
    \draw[fill=black!50] (0.2,0.68) ellipse (0.13 and 0.67);
    \draw[fill=black!50, color=black!50] (0,0) rectangle (0.2,1.35);
    \draw[fill=white] (0,0.68) ellipse (0.13 and 0.67);
    \draw (0,1.35) -- (0.2,1.35);
    \draw (0,0) -- (0.2,0);
  \end{scope}

  % Flywheel
  \begin{scope}[xshift=5.05cm,yshift=2.09cm]
    \draw[fill=white] (0.6,0.4) ellipse (0.13 and 0.4);
    \draw[fill=white,color=white] (0,0.8) rectangle (0.6,0);
    \draw[fill=white] (0,0.4) ellipse (0.13 and 0.4);
    \draw (0,0) -- (0.6,0);
    \draw (0,0.8) -- (0.6,0.8);
  \end{scope}

  % Transmission shaft from gear two to flywheel
  \begin{scope}[xshift=4.09cm,yshift=2.42cm]
    \draw[fill=black!50] (0,0) rectangle (0.96,0.1);
  \end{scope}

  % Angular velocity arrow between transmission and flywheel
  \draw[line width=0.7pt,->] (4.54,1.97) arc (-30:30:1) node[above]
    {$\omega_f$};

  % Descriptions inside graphic
  \draw (5.45,2.49) node {$J$};

  % Descriptions of subsystems under graphic
  \begin{scope}[xshift=-0.5cm,yshift=-0.28cm]
    \draw[decorate,decoration={brace,amplitude=10pt}]
      (3.5,0) -- (0,0) node[midway,yshift=-20pt] {circuit};
    \draw[decorate,decoration={brace,amplitude=10pt}]
      (6.55,0) -- (3.75,0) node[midway,yshift=-20pt] {mechanics};
  \end{scope}
\end{tikzpicture}

  \caption{Flywheel system diagram}
\end{bookfigure}

\subsection{Continuous state-space model}
\index{FRC models!flywheel equations}

By equation \eqref{eq:dot_omega_flywheel}
\begin{align*}
  \dot{\omega} &= -\frac{G^2 K_t}{K_v RJ} \omega + \frac{G K_t}{RJ} V
  \intertext{Factor out $\omega$ and $V$ into column vectors.}
  \dot{\begin{bmatrix}
    \omega
  \end{bmatrix}} &=
  \begin{bmatrix}
    -\frac{G^2 K_t}{K_v RJ}
  \end{bmatrix}
  \begin{bmatrix}
    \omega
  \end{bmatrix} +
  \begin{bmatrix}
    \frac{GK_t}{RJ}
  \end{bmatrix}
  \begin{bmatrix}
    V
  \end{bmatrix}
\end{align*}
\begin{theorem}[Flywheel state-space model]
  \begin{align*}
    \dot{\mat{x}} &= \mat{A} \mat{x} + \mat{B} \mat{u} \\
    \mat{y} &= \mat{C} \mat{x} + \mat{D} \mat{u}
  \end{align*}
  \begin{equation*}
    \mat{x} = \omega
    \quad
    \mat{y} = \omega
    \quad
    \mat{u} = V
  \end{equation*}
  \begin{equation}
    \mat{A} = -\frac{G^2 K_t}{K_v RJ}
    \quad
    \mat{B} = \frac{G K_t}{RJ}
    \quad
    \mat{C} = 1
    \quad
    \mat{D} = 0
  \end{equation}
\end{theorem}

\subsection{Model augmentation}

As per subsection \ref{subsec:input_error_estimation}, we will now augment the
\gls{model} so a $u_{error}$ state is added to the \gls{control input}.

The \gls{plant} and \gls{observer} augmentations should be performed before the
\gls{model} is \glslink{discretization}{discretized}. After the \gls{controller}
gain is computed with the unaugmented discrete \gls{model}, the controller may
be augmented. Therefore, the \gls{plant} and \gls{observer} augmentations assume
a continuous \gls{model} and the \gls{controller} augmentation assumes a
discrete \gls{controller}.
\begin{equation*}
  \mat{x} =
  \begin{bmatrix}
    \omega \\
    u_{error}
  \end{bmatrix}
  \quad
  \mat{y} = \omega
  \quad
  \mat{u} = V
\end{equation*}
\begin{equation}
  \mat{A}_{aug} =
  \begin{bmatrix}
    \mat{A} & \mat{B} \\
    0 & 0
  \end{bmatrix}
  \quad
  \mat{B}_{aug} =
  \begin{bmatrix}
    \mat{B} \\
    0
  \end{bmatrix}
  \quad
  \mat{C}_{aug} = \begin{bmatrix}
    \mat{C} & 0
  \end{bmatrix}
  \quad
  \mat{D}_{aug} = \mat{D}
\end{equation}
\begin{equation}
  \mat{K}_{aug} = \begin{bmatrix}
    \mat{K} & 1
  \end{bmatrix}
  \quad
  \mat{r}_{aug} = \begin{bmatrix}
    \mat{r} \\
    0
  \end{bmatrix}
\end{equation}

This will compensate for unmodeled dynamics such as projectiles slowing down the
flywheel.

\subsection{Simulation}

Python Control will be used to \glslink{discretization}{discretize} the
\gls{model} and simulate it. One of the frccontrol
examples\footnote{\url{https://github.com/calcmogul/frccontrol/blob/main/examples/flywheel.py}}
creates and tests a controller for it. Figure \ref{fig:flywheel_response} shows
the closed-loop \gls{system} response.
\begin{svg}{build/frccontrol/examples/flywheel_response}
  \caption{Flywheel response}
  \label{fig:flywheel_response}
\end{svg}

Notice how the \gls{control effort} when the \gls{reference} is reached is
nonzero. This is a plant inversion feedforward compensating for the \gls{system}
dynamics attempting to slow the flywheel down when no voltage is applied.

\subsection{Implementation}

C++ and Java implementations of this flywheel controller are available
online.\footnote{\url{https://github.com/wpilibsuite/allwpilib/blob/main/wpilibcExamples/src/main/cpp/examples/StateSpaceFlywheel/cpp/Robot.cpp}}
\footnote{\url{https://github.com/wpilibsuite/allwpilib/blob/main/wpilibjExamples/src/main/java/edu/wpi/first/wpilibj/examples/statespaceflywheel/Robot.java}}

\subsection{Flywheel model without encoder}

In the FIRST Robotics Competition, we can get the current drawn for specific
channels on the power distribution panel. We can theoretically use this to
estimate the angular velocity of a DC motor without an encoder. We'll start with
the flywheel model derived earlier as equation \eqref{eq:dot_omega_flywheel}.
\begin{align*}
  \dot{\omega} &= \frac{G K_t}{RJ} V - \frac{G^2 K_t}{K_v RJ} \omega \\
  \dot{\omega} &= -\frac{G^2 K_t}{K_v RJ} \omega + \frac{G K_t}{RJ} V
  \intertext{Next, we'll derive the current $I$ as an output.}
  V &= IR + \frac{\omega}{K_v} \\
  IR &= V - \frac{\omega}{K_v} \\
  I &= -\frac{1}{K_v R} \omega + \frac{1}{R} V
\end{align*}

Therefore,
\begin{theorem}[Flywheel state-space model without encoder]
  \begin{align*}
    \dot{\mat{x}} &= \mat{A} \mat{x} + \mat{B} \mat{u} \\
    \mat{y} &= \mat{C} \mat{x} + \mat{D} \mat{u}
  \end{align*}
  \begin{equation*}
    \mat{x} = \omega
    \quad
    \mat{y} = I
    \quad
    \mat{u} = V
  \end{equation*}
  \begin{equation}
    \mat{A} = -\frac{G^2 K_t}{K_v RJ}
    \quad
    \mat{B} = \frac{G K_t}{RJ}
    \quad
    \mat{C} = -\frac{1}{K_v R}
    \quad
    \mat{D} = \frac{1}{R}
  \end{equation}
\end{theorem}

Notice that in this \gls{model}, the \gls{output} doesn't provide any direct
measurements of the \gls{state}. To estimate the full \gls{state} (also known as
full observability), we only need the \glspl{output} to collectively include
linear combinations of every \gls{state}\footnote{While the flywheel model's
outputs are a linear combination of both the states and inputs, \glspl{input}
don't provide new information about the \glspl{state}. Therefore, they don't
affect whether the system is observable.}. We'll revisit this in chapter
\ref{ch:stochastic_control_theory} with an example that uses range measurements
to estimate an object's orientation.

The effectiveness of this \gls{model}'s \gls{observer} is heavily dependent on
the quality of the current sensor used. If the sensor's noise isn't zero-mean,
the \gls{observer} won't converge to the true \gls{state}.

\subsection{Voltage compensation}

To improve controller \gls{tracking}, one may want to use the voltage
renormalized to the power rail voltage to compensate for voltage drop when
current spikes occur. This can be done as follows.
\begin{equation}
  V = V_{cmd} \frac{V_{nominal}}{V_{rail}}
\end{equation}

where $V$ is the \gls{controller}'s new input voltage, $V_{cmd}$ is the old
input voltage, $V_{nominal}$ is the rail voltage when effects like voltage drop
due to current draw are ignored, and $V_{rail}$ is the real rail voltage.

To drive the \gls{model} with a more accurate voltage that includes voltage
drop, the reciprocal can be used.
\begin{equation}
  V = V_{cmd} \frac{V_{rail}}{V_{nominal}}
\end{equation}

where $V$ is the \gls{model}'s new input voltage. Note that if both the
\gls{controller} compensation and \gls{model} compensation equations are
applied, the original voltage is obtained. The \gls{model} input only drops from
ideal if the compensated \gls{controller} voltage saturates.

\section{Drivetrain}

\subsection{Continuous state-space model}
\index{FRC models!drivetrain equations}

The position and velocity of each drivetrain side can be written as

\begin{align}
  \dot{x}_l &= v_l \label{eq:drivetrain_cont_ss_posl} \\
  \dot{v}_l &= \dot{v}_l \label{eq:drivetrain_cont_ss_vell} \\
  \dot{x}_r &= v_r \label{eq:drivetrain_cont_ss_posr} \\
  \dot{v}_r &= \dot{v}_r \label{eq:drivetrain_cont_ss_velr}
\end{align}

By equations (\ref{eq:drivetrain_model_right}) and
(\ref{eq:drivetrain_model_left})

\begin{align*}
  \dot{v}_l &= \left(\frac{1}{m} + \frac{r_b^2}{J}\right)
    \left(C_1 v_l + C_2 V_l\right) +
    \left(\frac{1}{m} - \frac{r_b^2}{J}\right) \left(C_3 v_r + C_4 V_r\right) \\
  \dot{v}_r &= \left(\frac{1}{m} - \frac{r_b^2}{J}\right)
    \left(C_1 v_l + C_2 V_l\right) +
    \left(\frac{1}{m} + \frac{r_b^2}{J}\right) \left(C_3 v_r + C_4 V_r\right)
\end{align*}

\begin{theorem}[Drivetrain state-space model]
  \begin{align*}
    \dot{\mtx{x}} &= \mtx{A} \mtx{x} + \mtx{B} \mtx{u} \\
    \mtx{y} &= \mtx{C} \mtx{x} + \mtx{D} \mtx{u}
  \end{align*}
  \begin{equation*}
    \begin{array}{ccc}
      \mtx{x} =
      \begin{bmatrix}
        x_l \\
        v_l \\
        x_r \\
        v_r
      \end{bmatrix} &
      \mtx{y} =
      \begin{bmatrix}
        x_l \\
        x_r
      \end{bmatrix} &
      \mtx{u} =
      \begin{bmatrix}
        V_l \\
        V_r
      \end{bmatrix}
    \end{array}
  \end{equation*}
  \begin{equation}
    \label{eq:drivetrain_ss_model}
    \begin{array}{ll}
      \mtx{A} =
      \begin{bmatrix}
        0 & 1 & 0 & 0 \\
        0 & \left(\frac{1}{m} + \frac{r_b^2}{J}\right) C_1 & 0 & \left(\frac{1}{m} - \frac{r_b^2}{J}\right) C_3 \\
        0 & 0 & 0 & 1 \\
        0 & \left(\frac{1}{m} - \frac{r_b^2}{J}\right) C_1 & 0 & \left(\frac{1}{m} + \frac{r_b^2}{J}\right) C_3
      \end{bmatrix} &
      \mtx{B} =
      \begin{bmatrix}
        0 & 0 \\
        \left(\frac{1}{m} + \frac{r_b^2}{J}\right) C_2 & \left(\frac{1}{m} - \frac{r_b^2}{J}\right) C_4 \\
        0 & 0 \\
        \left(\frac{1}{m} - \frac{r_b^2}{J}\right) C_2 & \left(\frac{1}{m} + \frac{r_b^2}{J}\right) C_4
      \end{bmatrix} \\
      \mtx{C} =
      \begin{bmatrix}
        1 & 0 & 0 & 0 \\
        0 & 0 & 1 & 0 \\
      \end{bmatrix} &
      \mtx{D} = \mtx{0}_{2 \times 2}
    \end{array}
  \end{equation}

  where $C_1 = -\frac{G_l^2 K_t}{K_v R r^2}$, $C_2 = \frac{G_l K_t}{Rr}$,
  $C_3 = -\frac{G_r^2 K_t}{K_v R r^2}$, and $C_4 = \frac{G_r K_t}{Rr}$.
\end{theorem}

\subsection{Model augmentation}

As per subsection \ref{subsec:u_error_estimation}, we will now augment the
\gls{model} so $u_{error}$ terms are added to the \glspl{control input}.

The \gls{plant} and \gls{observer} augmentations should be performed before the
\gls{model} is \glslink{discretization}{discretized}. After the \gls{controller}
gain is computed with the unaugmented discrete \gls{model}, the controller may
be augmented. Therefore, the \gls{plant} and \gls{observer} augmentations assume
a continuous \gls{model} and the \gls{controller} augmentation assumes a
discrete \gls{controller}.

For this augmented \gls{model}, the left and right wheel positions are filtered
encoder positions and are not adjusted for heading error. The turning velocity
computed from the left and right velocities is adjusted by the gyroscope angular
velocity. The angular velocity $u_{error}$ term is the angular velocity error
between the wheel speeds and the gyroscope measurement.

\begin{equation*}
  \begin{array}{ccc}
    \mtx{x}_{aug} =
    \begin{bmatrix}
      \mtx{x} \\
      u_{error,l} \\
      u_{error,r} \\
      u_{error,angle}
    \end{bmatrix} &
    \mtx{y}_{aug} =
    \begin{bmatrix}
      \mtx{y} \\
      \omega
    \end{bmatrix} &
    \mtx{u} =
    \begin{bmatrix}
      V_l \\
      V_r
    \end{bmatrix}
  \end{array}
\end{equation*}

where $\omega$ is the angular rate of the robot's center of mass measured by a
gyroscope.

\begin{equation*}
  \begin{array}{cc}
    \mtx{B}_{\omega} =
    \begin{bmatrix}
      1 \\
      0 \\
      -1 \\
      0
    \end{bmatrix} &
    \mtx{C}_{\omega} =
    \begin{bmatrix}
      0 & -\frac{1}{2r_b} & 0 & \frac{1}{2r_b}
    \end{bmatrix}
  \end{array}
\end{equation*}

\begin{equation}
  \begin{array}{ll}
    \mtx{A}_{aug} =
    \begin{bmatrix}
      \mtx{A} & \mtx{B} & \mtx{B}_{\omega} \\
      \mtx{0}_{3 \times 4} & \mtx{0}_{3 \times 2} & \mtx{0}_{3 \times 1}
    \end{bmatrix} &
    \mtx{B}_{aug} =
    \begin{bmatrix}
      \mtx{B} \\
      \mtx{0}_{3 \times 2}
    \end{bmatrix} \\
    \mtx{C}_{aug} =
    \begin{bmatrix}
      \mtx{C} & \mtx{0}_{2 \times 3} \\
      \mtx{C}_{\omega} & \mtx{0}_{1 \times 3}
    \end{bmatrix} &
    \mtx{D}_{aug} =
    \begin{bmatrix}
      \mtx{D} \\
      \mtx{0}_{1 \times 2}
    \end{bmatrix}
  \end{array}
\end{equation}

The augmentation of $\mtx{A}$ with $\mtx{B}$ maps the left and right wheel
velocity $u_{error}$ terms to their respective states. The augmentation of
$\mtx{A}$ with $\mtx{B}_{\omega}$ maps the angular velocity $u_{error}$ term to
corresponding changes in the left and right wheel positions.

$\mtx{C}_{\omega}$ maps the left and right wheel velocities to an angular
velocity estimate that the \gls{observer} can compare against the gyroscope
measurement.

\begin{equation}
  \begin{array}{ccc}
    \mtx{K}_{error} =
    \begin{bmatrix}
      1 & 0 & 0 \\
      0 & 1 & 0
    \end{bmatrix} &
    \mtx{K}_{aug} = \begin{bmatrix}
      \mtx{K} & \mtx{K}_{error}
    \end{bmatrix} &
    \mtx{r}_{aug} = \begin{bmatrix}
      \mtx{r} \\
      0 \\
      0 \\
      0
    \end{bmatrix}
  \end{array}
\end{equation}

This will compensate for unmodeled dynamics such as understeer due to the wheel
friction inherent in skid-steer robots.

\begin{remark}
  The process noise for the angular velocity $u_{error}$ term should be the
  encoder \gls{model} uncertainty. The augmented measurement noise term is
  obviously for the gyroscope measurement.
\end{remark}

\subsection{Simulation}

Python Control will be used to \glslink{discretization}{discretize} the
\gls{model} and simulate it. One of the frccontrol
examples\footnote{\url{https://github.com/calcmogul/frccontrol/blob/master/examples/drivetrain.py}}
creates and tests a controller for it.

\begin{remark}
  Python Control currently doesn't support finding the transmission zeroes of
  MIMO \glspl{system} with a different number of \glspl{input} than
  \glspl{output}, so \texttt{control.pzmap()} and
  \texttt{frccontrol.System.plot\_pzmaps()} fail with an error if Slycot isn't
  installed.
\end{remark}

Figure \ref{fig:drivetrain_pzmaps} shows the pole-zero maps for the open-loop
\gls{system}, closed-loop \gls{system}, and \gls{observer}. Figure
\ref{fig:drivetrain_response} shows the \gls{system} response with them.

\begin{svg}{build/frccontrol/examples/drivetrain_pzmaps}
  \caption{Drivetrain pole-zero maps}
  \label{fig:drivetrain_pzmaps}
\end{svg}

Figure \ref{fig:drivetrain_response} shows the \gls{system} response.

\begin{svg}{build/frccontrol/examples/drivetrain_response}
  \caption{Drivetrain response}
  \label{fig:drivetrain_response}
\end{svg}

Given the high inertia in drivetrains, it's better to drive the \gls{reference}
with a motion profile instead of a \gls{step input} for reproducibility.

\subsection{Implementation}

The script linked above also generates two files: DrivetrainCoeffs.h and
DrivetrainCoeffs.cpp. These can be used with the WPILib StateSpacePlant,
StateSpaceController, and StateSpaceObserver classes in C++ and Java. A C++
implementation of this drivetrain controller is available online\footnote{
\url{https://github.com/calcmogul/allwpilib/tree/state-space/wpilibcExamples/src/main/cpp/examples/StateSpaceDrivetrain}}.

\section{Single-jointed arm}

\subsection{Continuous state-space model}
\index{FRC models!single-jointed arm equations}

The angle and angular rate derivatives of the arm can be written as

\begin{align}
  \dot{\theta}_{arm} &= \omega_{arm} \label{eq:arm_cont_ss_pos} \\
  \dot{\omega}_{arm} &= \dot{\omega}_{arm} \label{eq:arm_cont_ss_vel}
\end{align}

By equation (\ref{eq:dot_omega_arm})

\begin{equation*}
  \dot{\omega}_{arm} = -\frac{G^2 K_t}{K_v RJ} \omega_{arm} + \frac{G K_t}{RJ} V
\end{equation*}

\begin{theorem}[Single-jointed arm state-space model]
  \begin{align*}
    \dot{\mtx{x}} &= \mtx{A} \mtx{x} + \mtx{B} \mtx{u} \\
    \mtx{y} &= \mtx{C} \mtx{x} + \mtx{D} \mtx{u}
  \end{align*}
  \begin{equation*}
    \begin{array}{ccc}
      \mtx{x} =
      \begin{bmatrix}
        \theta_{arm} \\
        \omega_{arm}
      \end{bmatrix} &
      \mtx{y} = \theta_{arm} &
      \mtx{u} = V
    \end{array}
  \end{equation*}
  \begin{equation}
    \begin{array}{cccc}
      \mtx{A} =
      \begin{bmatrix}
        0 & 1 \\
        0 & -\frac{G^2 K_t}{K_v RJ}
      \end{bmatrix} &
      \mtx{B} =
      \begin{bmatrix}
        0 \\
        \frac{G K_t}{RJ}
      \end{bmatrix} &
      \mtx{C} =
      \begin{bmatrix}
        1 & 0
      \end{bmatrix} &
      \mtx{D} = 0
    \end{array}
  \end{equation}
\end{theorem}

\subsection{Model augmentation}

As per subsection \ref{subsec:u_error_estimation}, we will now augment the
\gls{model} so a $u_{error}$ term is added to the \gls{control input}.

The \gls{plant} and \gls{observer} augmentations should be performed before the
\gls{model} is \glslink{discretization}{discretized}. After the \gls{controller}
gain is computed with the unaugmented discrete \gls{model}, the controller may
be augmented. Therefore, the \gls{plant} and \gls{observer} augmentations assume
a continuous \gls{model} and the \gls{controller} augmentation assumes a
discrete \gls{controller}.

\begin{equation*}
  \begin{array}{ccc}
    \mtx{x}_{aug} =
    \begin{bmatrix}
      \mtx{x} \\
      u_{error}
    \end{bmatrix} &
    \mtx{y} = \theta_{arm} &
    \mtx{u} = V
  \end{array}
\end{equation*}

\begin{equation}
  \begin{array}{cccc}
    \mtx{A}_{aug} =
    \begin{bmatrix}
      \mtx{A} & \mtx{B} \\
      \mtx{0}_{1 \times 2} & 0
    \end{bmatrix} &
    \mtx{B}_{aug} =
    \begin{bmatrix}
      \mtx{B} \\
      0
    \end{bmatrix} &
    \mtx{C}_{aug} =
    \begin{bmatrix}
      \mtx{C} & 0
    \end{bmatrix} &
    \mtx{D}_{aug} = \mtx{D}
  \end{array}
\end{equation}

\begin{equation}
  \begin{array}{cc}
    \mtx{K}_{aug} = \begin{bmatrix}
      \mtx{K} & 1
    \end{bmatrix} &
    \mtx{r}_{aug} = \begin{bmatrix}
      \mtx{r} \\
      0
    \end{bmatrix}
  \end{array}
\end{equation}

This will compensate for unmodeled dynamics such as gravity or other external
loading from lifted objects. However, if only gravity compensation is desired,
a feedforward of the form $u_{ff} = V_{gravity} \cos\theta$ is preferred where
$V_{gravity}$ is the voltage required to hold the arm level with the ground and
$\theta$ is the angle of the arm with the ground.

\subsection{Simulation}

Python Control will be used to \glslink{discretization}{discretize} the
\gls{model} and simulate it. One of the frccontrol
examples\footnote{\url{https://github.com/calcmogul/frccontrol/blob/master/examples/single_jointed_arm.py}}
creates and tests a controller for it.

\begin{remark}
  Python Control currently doesn't support finding the transmission zeroes of
  MIMO \glspl{system} with a different number of \glspl{input} than
  \glspl{output}, so \texttt{control.pzmap()} and
  \texttt{frccontrol.System.plot\_pzmaps()} fail with an error if Slycot isn't
  installed.
\end{remark}

Figure \ref{fig:single_jointed_arm_pzmaps} shows the pole-zero maps for the
open-loop \gls{system}, closed-loop \gls{system}, and \gls{observer}. Figure
\ref{fig:single_jointed_arm_response} shows the \gls{system} response with them.

\begin{svg}{build/frccontrol/examples/single_jointed_arm_pzmaps}
  \caption{Single-jointed arm pole-zero maps}
  \label{fig:single_jointed_arm_pzmaps}
\end{svg}

\begin{svg}{build/frccontrol/examples/single_jointed_arm_response}
  \caption{Single-jointed arm response}
  \label{fig:single_jointed_arm_response}
\end{svg}

\subsection{Implementation}

The script linked above also generates two files: SingleJointedArmCoeffs.h and
SingleJointedArmCoeffs.cpp. These can be used with the WPILib StateSpacePlant,
StateSpaceController, and StateSpaceObserver classes in C++ and Java. A C++
implementation of this single-jointed arm controller is available
online\footnote{\url{https://github.com/calcmogul/allwpilib/tree/state-space/wpilibcExamples/src/main/cpp/examples/StateSpaceSingleJointedArm}}.

\section{Rotating claw}

\subsection{Equations of motion}

This claw consists of independent upper and lower jaw pieces each driven by its
own DC brushed motor.

\subsection{Continuous state-space model}
\index{FRC models!rotating claw equations}

\subsection{Simulation}

