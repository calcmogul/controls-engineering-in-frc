\chapterimage{discrete-state-space-control.jpg}{Chaparral by Merril Apartments at UCSC}

\chapter{Discrete state-space control}
\label{ch:discrete_state-space_control}

The complex plane discussed so far deals with continuous \glspl{system}. In
decades past, \glspl{plant} and controllers were implemented using analog
electronics, which are continuous in nature. Nowadays, microprocessors can be
used to achieve cheaper, less complex controller designs. \Gls{discretization}
converts the continuous \gls{model} we've worked with so far from a differential
equation like
\begin{align}
  \dot{x} &= x - 3 \label{eq:differential_equ_example}
  \intertext{to a difference equation like}
  \frac{x_{k+1} - x_k}{\Delta T} &= x_k - 3 \nonumber \\
  x_{k+1} - x_k &= (x_k - 3) \Delta T \nonumber \\
  x_{k+1} &= x_k + (x_k - 3) \Delta T \label{eq:difference_equ_example}
\end{align}

where $x_k$ refers to the value of $x$ at the $k^{th}$ timestep. The difference
equation is run with some update period denoted by $T$, by $\Delta T$, or
sometimes sloppily by $dt$\footnote{The discretization of equation
\eqref{eq:differential_equ_example} to equation
\eqref{eq:difference_equ_example} uses the forward Euler discretization
method.}.

While higher order terms of a differential equation are derivatives of the
\gls{state} variable (e.g., $\ddot{x}$ in relation to equation
\eqref{eq:differential_equ_example}), higher order terms of a difference
equation are delayed copies of the \gls{state} variable (e.g., $x_{k-1}$ with
respect to $x_k$ in equation \eqref{eq:difference_equ_example}).

\renewcommand*{\chapterpath}{\partpath/discrete-state-space-control}
\section{Continuous to discrete pole mapping}

When a continuous system is discretized, its poles in the LHP map to the inside
of a unit circle. Table \ref{tab:c2d_mapping} contains a few common points and
figure \ref{fig:c2d_mapping} shows the mapping visually.
\begin{booktable}
  \begin{tabular}{|cc|}
    \hline
    \rowcolor{headingbg}
    \textbf{Continuous} & \textbf{Discrete} \\
    \hline
    $(0, 0)$ & $(1, 0)$ \\
    imaginary axis & edge of unit circle \\
    $(-\infty, 0)$ & $(0, 0)$ \\
    \hline
  \end{tabular}
  \caption{Mapping from continuous to discrete}
  \label{tab:c2d_mapping}
\end{booktable}
\begin{bookfigure}
  \begin{minisvg}{2}{build/figs/s_plane}
  \end{minisvg}
  \hfill
  \begin{minisvg}{2}{build/figs/z_plane}
  \end{minisvg}
  \caption{Mapping of complex plane from continuous (left) to discrete (right)}
  \label{fig:c2d_mapping}
\end{bookfigure}

\subsection{Discrete system stability}

Eigenvalues of a \gls{system} that are within the unit circle are stable, but
why is that? Let's consider a scalar equation $x_{k + 1} = ax_k$. $a < 1$ makes
$x_{k + 1}$ converge to zero. The same applies to a complex number like
$z = x + yi$ for $x_{k + 1} = zx_k$. If the magnitude of the complex number $z$
is less than one, $x_{k+1}$ will converge to zero. Values with a magnitude of
$1$ oscillate forever because $x_{k+1}$ never decays.

\subsection{Discrete system behavior}

As $\omega$ increases in $s = j\omega$, a pole in the discrete plane moves
around the perimeter of the unit circle. Once it hits $\frac{\omega_s}{2}$ (half
the sampling frequency) at $(-1, 0)$, the pole wraps around. This is due to
poles faster than the sample frequency folding down to below the sample
frequency (that is, higher frequency signals \textit{alias} to lower frequency
ones).

You may notice that poles can be placed at $(0, 0)$ in the discrete plane. This
is known as a deadbeat controller. An $\rm N^{th}$-order deadbeat controller
decays to the \gls{reference} in N timesteps. While this sounds great, there are
other considerations like \gls{control effort}, \gls{robustness}, and
\gls{noise immunity}.

If poles from $(1, 0)$ to $(0, 0)$ on the x-axis approach infinity, then what do
poles from $(-1, 0)$ to $(0, 0)$ represent? Them being faster than infinity
doesn't make sense. Poles in this location exhibit oscillatory behavior similar
to complex conjugate pairs. See figures \ref{fig:continuous_oscillations_1p} and
\ref{fig:discrete_oscillations_2p}. The jaggedness of these signals is due to
the frequency of the \gls{system} dynamics being above the Nyquist frequency
(twice the sample frequency). The \glslink{discretization}{discretized} signal
doesn't have enough samples to reconstruct the continuous \gls{system}'s
dynamics.
\begin{bookfigure}
  \begin{minisvg}{2}{build/\chapterpath/z_oscillations_1p}
    \caption{Single poles in various locations in discrete plane}
    \label{fig:continuous_oscillations_1p}
  \end{minisvg}
  \hfill
  \begin{minisvg}{2}{build/\chapterpath/z_oscillations_2p}
    \caption{Complex conjugate poles in various locations in discrete plane}
    \label{fig:discrete_oscillations_2p}
  \end{minisvg}
\end{bookfigure}

\subsection{Nyquist frequency}
\index{digital signal processing!Nyquist frequency}
\index{digital signal processing!aliasing}

To completely reconstruct a signal, the Nyquist-Shannon sampling theorem states
that it must be sampled at a frequency at least twice the maximum frequency it
contains. The highest frequency a given sample rate can capture is called the
Nyquist frequency, which is half the sample frequency. This is why recorded
audio is sampled at $44.1$ kHz. The maximum frequency a typical human can hear
is about $20$ kHz, so the Nyquist frequency is $20$ kHz and the minimum sampling
frequency is $40$ kHz. ($44.1$ kHz in particular was chosen for unrelated
historical reasons.)

Frequencies above the Nyquist frequency are folded down across it. The higher
frequency and the folded down lower frequency are said to alias each
other.\footnote{The aliases of a frequency $f$ can be expressed as
$f_{alias}(N) \stackrel{def}{=} |f - Nf_s|$. For example, if a $200$ Hz sine
wave is sampled at $150$ Hz, the \gls{observer} will see a $50$ Hz signal
instead of a $200$ Hz one.} Figure \ref{fig:c2d_aliasing} demonstrates
aliasing.
\begin{svg}{build/\chapterpath/aliasing}
  \caption{The original signal is a $1.5$ Hz sine wave, which means its Nyquist
    frequency is $1.5$ Hz. The signal is being sampled at $2$ Hz, so the aliased
    signal is a $0.5$ Hz sine wave.}
    \label{fig:c2d_aliasing}
\end{svg}

The effect of these high-frequency aliases can be reduced with a low-pass filter
(called an anti-aliasing filter in this application).

\section{Discretization methods}

\Gls{discretization} is done using a zero-order hold. That is, the \gls{system}
\gls{state} is only updated at discrete intervals and it's held constant between
samples (see figure \ref{fig:zoh}). The exact method of applying this uses the
matrix exponential, but this can be computationally expensive. Instead,
approximations such as the following are used.
\begin{enumerate}
  \item Forward Euler method. This is defined as
    $y_{n+1} = y_n + f(t_n, y_n) \Delta t$.
    \index{discretization!forward Euler method}
  \item Backward Euler method. This is defined as
    $y_{n+1} = y_n + f(t_{n+1}, y_{n+1}) \Delta t$.
    \index{discretization!backward Euler method}
  \item Bilinear transform. The first-order bilinear approximation is
    $s = \frac{2}{T} \frac{1 - z^{-1}}{1 + z^{-1}}$.
    \index{discretization!bilinear transform}
\end{enumerate}

where the function $f(t_n, y_n)$ is the slope of $y$ at $n$ and $T$ is the
sample period for the discrete \gls{system}. Each of these methods is
essentially finding the area underneath a curve. The forward and backward Euler
methods use rectangles to approximate that area while the bilinear transform
uses trapezoids (see figures \ref{fig:discretization_methods_vel} and
\ref{fig:discretization_methods_pos}). Since these are approximations, there is
distortion between the real discrete \gls{system}'s poles and the approximate
poles. This is in addition to the phase loss introduced by discretizing at a
given sample rate in the first place. For fast-changing \glspl{system}, this
distortion can quickly lead to instability.
\begin{bookfigure}
  \begin{minisvg}{2}{build/\chapterpath/zoh}
      \caption{Zero-order hold of a system response}
      \label{fig:zoh}
  \end{minisvg}
  \hfill
  \begin{minisvg}{2}{build/\chapterpath/discretization_methods_vel}
    \caption{Discretization methods applied to velocity data}
    \label{fig:discretization_methods_vel}
  \end{minisvg}
  \hfill
  \begin{minisvg}{2}{build/\chapterpath/discretization_methods_pos}
    \caption{Position plot of discretization methods applied to velocity data}
    \label{fig:discretization_methods_pos}
  \end{minisvg}
\end{bookfigure}

Figures \ref{fig:sampling_simulation_0.1}, \ref{fig:sampling_simulation_0.05},
and \ref{fig:sampling_simulation_0.01} show simulations of the same controller
for different sampling methods and sample rates, which have varying levels of
fidelity to the real \gls{system}.
\begin{bookfigure}
  \begin{minisvg}{2}{build/\chapterpath/sampling_simulation_010}
    \caption{Sampling methods for system simulation with $T = 0.1$ s}
    \label{fig:sampling_simulation_0.1}
  \end{minisvg}
  \hfill
  \begin{minisvg}{2}{build/\chapterpath/sampling_simulation_005}
    \caption{Sampling methods for system simulation with $T = 0.05$ s}
    \label{fig:sampling_simulation_0.05}
  \end{minisvg}
  \hfill
  \begin{minisvg}{2}{build/\chapterpath/sampling_simulation_004}
    \caption{Sampling methods for system simulation with $T = 0.01$ s}
    \label{fig:sampling_simulation_0.01}
  \end{minisvg}
\end{bookfigure}

Forward Euler is numerically unstable for low sample rates. The bilinear
transform is a significant improvement due to it being a second-order
approximation, but zero-order hold performs best due to the matrix exponential
including much higher orders.

Table \ref{tab:disc_approx_scalar} compares the Taylor series expansions of
several common discretization methods (these are found using polynomial
division). The bilinear transform does best with accuracy trailing off after the
third-order term. Forward Euler has no second-order or higher terms, so it
undershoots. Backward Euler has twice the second-order term and overshoots the
remaining higher order terms as well.
\begin{booktable}
  \begin{tabular}{|cll|}
    \hline
    \rowcolor{headingbg}
    \multicolumn{1}{|c}{\textbf{Method}} &
      \multicolumn{1}{c}{\textbf{Conversion to z}} &
      \multicolumn{1}{c|}{\textbf{Taylor series expansion}} \\
    \hline
    Zero-order hold &
      $e^{Ts}$ &
      $1 + Ts + \frac{1}{2}T^2s^2 + \frac{1}{6}T^3s^3 + \ldots$ \\
    Bilinear &
      $\frac{1 + \frac{1}{2}Ts}{1 - \frac{1}{2}Ts}$ &
      $1 + Ts + \frac{1}{2}T^2s^2 + \frac{1}{4}T^3s^3 + \ldots$ \\
    Forward Euler &
      $1 + Ts$ &
      $1 + Ts$ \\
    Backward Euler &
      $\frac{1}{1 - Ts}$ &
      $1 + Ts + T^2s^2 + T^3s^3 + \ldots$ \\
    \hline
  \end{tabular}
  \caption{Taylor series expansions of discretization methods (scalar case). The
    zero-order hold discretization method is exact.}
  \label{tab:disc_approx_scalar}
\end{booktable}

\section{Sample delay}

Implementing a discrete control system is easier than implementing a continuous
one, but \gls{discretization} has drawbacks. A microcontroller updates the
system input in discrete intervals of duration $T$; it's held constant between
updates. This introduces an average sample delay of $\frac{T}{2}$. Large delays
can make a stable controller in the continuous domain become unstable in the
discrete domain. Here are a few ways to combat this.
\begin{itemize}
  \item Run the controller with a high sample rate.
  \item Designing the controller in the analog domain with enough
    \gls{phase margin} to compensate for any phase loss that occurs as part of
    \gls{discretization}.
  \item Convert the \gls{plant} to the digital domain and design the controller
    completely in the digital domain.
\end{itemize}

\section{Effects of discretization on controller performance}

Running a feedback controller at a faster update rate doesn't always mean better
control. In fact, you may be using more computational resources than you need.
However, here are some reasons for running at a faster update rate.

Firstly, if you have a discrete \gls{model} of the \gls{system}, that
\gls{model} can more accurately approximate the underlying continuous
\gls{system}. Not all controllers use a \gls{model} though.

Secondly, the controller can better handle fast \gls{system} dynamics. If the
\gls{system} can move from its initial state to the desired one in under 250ms,
you obviously want to run the controller with a period less than 250ms. When you
reduce the sample period, you're making the discrete controller more accurately
reflect what the equivalent continuous controller would do (controllers built
from analog circuit components like op-amps are continuous).

Running at a lower sample rate only causes problems if you don't take into
account the response time of your \gls{system}. Some \glspl{system} like heaters
have \glspl{output} that change on the order of minutes. Running a control loop
at 1kHz doesn't make sense for this because the \gls{plant} \gls{input} the
controller computes won't change much, if at all, in 1ms.

Figures \ref{fig:sampling_simulation_0.1}, \ref{fig:sampling_simulation_0.05},
and \ref{fig:sampling_simulation_0.01} show simulations of the same controller
for different sampling methods and sample rates, which have varying levels of
fidelity to the real \gls{system}.
\begin{bookfigure}
  \begin{minisvg}{2}{build/\chapterpath/sampling_simulation_010}
    \caption{Sampling methods for system simulation with $T = 0.1s$}
    \label{fig:sampling_simulation_0.1}
  \end{minisvg}
  \hfill
  \begin{minisvg}{2}{build/\chapterpath/sampling_simulation_005}
    \caption{Sampling methods for system simulation with $T = 0.05s$}
    \label{fig:sampling_simulation_0.05}
  \end{minisvg}
  \hfill
  \begin{minisvg}{2}{build/\chapterpath/sampling_simulation_004}
    \caption{Sampling methods for system simulation with $T = 0.01s$}
    \label{fig:sampling_simulation_0.01}
  \end{minisvg}
\end{bookfigure}

Forward Euler is numerically unstable for low sample rates. The bilinear
transform is a significant improvement due to it being a second-order
approximation, but zero-order hold performs best due to the matrix exponential
including much higher orders (we'll cover the matrix exponential in the next
section).

Table \ref{tab:disc_approx_scalar} compares the Taylor series expansions of the
discretization methods presented so far (these are found using polynomial
division). The bilinear transform does best with accuracy trailing off after the
third-order term. Forward Euler has no second-order or higher terms, so it
undershoots. Backward Euler has twice the second-order term and overshoots the
remaining higher order terms as well.
\begin{booktable}
  \begin{tabular}{|cll|}
    \hline
    \rowcolor{headingbg}
    \multicolumn{1}{|c}{\textbf{Method}} &
      \multicolumn{1}{c}{\textbf{Conversion}} &
      \multicolumn{1}{c|}{\textbf{Taylor series expansion}} \\
    \hline
    Zero-order hold &
      $z = e^{Ts}$ &
      $z = 1 + Ts + \frac{1}{2}T^2s^2 + \frac{1}{6}T^3s^3 + \ldots$ \\
    Bilinear &
      $z = \frac{1 + \frac{1}{2}Ts}{1 - \frac{1}{2}Ts}$ &
      $z = 1 + Ts + \frac{1}{2}T^2s^2 + \frac{1}{4}T^3s^3 + \ldots$ \\
    Forward Euler &
      $z = 1 + Ts$ &
      $z = 1 + Ts$ \\
    Reverse Euler &
      $z = \frac{1}{1 - Ts}$ &
      $z = 1 + Ts + T^2s^2 + T^3s^3 + \ldots$ \\
    \hline
  \end{tabular}
  \caption{Taylor series expansions of discretization methods (scalar case). The
    zero-order hold discretization method is exact.}
  \label{tab:disc_approx_scalar}
\end{booktable}

\input{\chapterpath/taylor-series}
\section{Matrix exponential}
\index{discretization!matrix exponential}

The matrix exponential (and \gls{system} \gls{discretization} in general) is
typically solved with a computer. Python Control's \texttt{StateSpace.sample()}
with the ``zoh" method (the default) does this.
\begin{definition}[Matrix exponential]
  Let $\mat{X}$ be an $n \times n$ matrix. The exponential of $\mat{X}$ denoted
  by $e^{\mat{X}}$ is the $n \times n$ matrix given by the following power
  series.
  \begin{equation}
    e^{\mat{X}} = \sum_{k=0}^\infty \frac{1}{k!} \mat{X}^k \label{eq:mat_exp}
  \end{equation}

  where $\mat{X}^0$ is defined to be the identity matrix $\mat{I}$ with the same
  dimensions as $\mat{X}$.
\end{definition}

To understand why the matrix exponential is used in the \gls{discretization}
process, consider the set of differential equations
$\dot{\mat{x}} = \mat{A}\mat{x}$ we use to describe \glspl{system}
(\glspl{system} also have a $\mat{B}\mat{u}$ term, but we'll ignore it for
clarity). The solution to this type of differential equation uses an
exponential. Since we are using matrices and vectors here, we use the matrix
exponential.
\begin{equation*}
  \mat{x}(t) = e^{\mat{A}t} \mat{x}_0
\end{equation*}

where $\mat{x}_0$ contains the initial conditions. If the initial \gls{state} is
the current system \gls{state}, then we can describe the \gls{system}'s
\gls{state} over time as
\begin{equation*}
  \mat{x}_{k+1} = e^{\mat{A}T} \mat{x}_k
\end{equation*}

where $T$ is the time between samples $\mat{x}_k$ and $\mat{x}_{k+1}$.

\input{\chapterpath/zoh-for-state-space}
\section{Discrete state-space notation}

Below is the discrete version of state-space notation.
\begin{definition}[Discrete state-space notation]%
  \index{state-space controllers!discrete open-loop}
  \begin{align}
    \mat{x}_{k+1} &= \mat{A}\mat{x}_k + \mat{B}\mat{u}_k \\
    \mat{y}_k &= \mat{C}\mat{x}_k + \mat{D}\mat{u}_k
  \end{align}
  \begin{figurekey}
    \begin{tabular}{llll}
      $\mat{A}$ & system matrix      & $\mat{x}$ & state vector \\
      $\mat{B}$ & input matrix       & $\mat{u}$ & input vector \\
      $\mat{C}$ & output matrix      & $\mat{y}$ & output vector \\
      $\mat{D}$ & feedthrough matrix &  &  \\
    \end{tabular}
  \end{figurekey}
\end{definition}

\section{Closed-loop controller}
\index{state-space controllers!discrete closed-loop}

With the \gls{control law} $\mat{u}_k = \mat{K}(\mat{r}_k - \mat{x}_k)$, we can
derive the closed-loop state-space equations. We'll discuss where this
\gls{control law} comes from in subsection \ref{sec:lqr}.

First is the \gls{state} update equation. Substitute the \gls{control law} into
equation \eqref{eq:disc_ss_x}.
\begin{align}
  \mat{x}_{k+1} &= \mat{A}\mat{x}_k + \mat{B}\mat{K}(\mat{r}_k - \mat{x}_k)
    \nonumber \\
  \mat{x}_{k+1} &= \mat{A}\mat{x}_k + \mat{B}\mat{K}\mat{r}_k -
    \mat{B}\mat{K}\mat{x}_k \nonumber \\
  \mat{x}_{k+1} &= (\mat{A} - \mat{B}\mat{K})\mat{x}_k + \mat{B}\mat{K}\mat{r}_k
    \label{eq:disc_ss_ctrl_x}
  \intertext{Now for the \gls{output} equation. Substitute the \gls{control law}
    into equation \eqref{eq:disc_ss_y}.}
  \mat{y}_k &= \mat{C}\mat{x}_k + \mat{D}(\mat{K}(\mat{r}_k - \mat{x}_k))
    \nonumber \\
  \mat{y}_k &= \mat{C}\mat{x}_k + \mat{D}\mat{K}\mat{r}_k -
    \mat{D}\mat{K}\mat{x}_k \nonumber \\
  \mat{y}_k &= (\mat{C} - \mat{D}\mat{K})\mat{x}_k + \mat{D}\mat{K}\mat{r}_k
\end{align}

\index{stability!eigenvalues}
Instead of commanding the \gls{system} to a \gls{state} using the vector
$\mat{u}_k$ directly, we can now specify a vector of desired \glspl{state}
through $\mat{r}_k$ and the \gls{controller} will choose values of $\mat{u}_k$
for us over time to make the \gls{system} converge to the \gls{reference}.

The eigenvalues of $\mat{A} - \mat{B}\mat{K}$ are the poles of the closed-loop
\gls{system}. Therefore, the rate of convergence and stability of the
closed-loop \gls{system} can be changed by moving the poles via the eigenvalues
of $\mat{A} - \mat{B}\mat{K}$. $\mat{A}$ and $\mat{B}$ are inherent to the
\gls{system}, but $\mat{K}$ can be chosen arbitrarily by the controller
designer. For equation \eqref{eq:disc_ss_ctrl_x} to reach steady-state, the
eigenvalues of $\mat{A} - \mat{B}\mat{K}$ must be in the left-half plane.
\begin{booktable}
  \begin{tabular}{|lll|}
    \hline
    \rowcolor{headingbg}
    \textbf{Symbol} & \textbf{Name} & \textbf{Rows $\times$ Columns} \\
    \hline
    $\mat{A}$ & system matrix & states $\times$ states \\
    $\mat{B}$ & input matrix & states $\times$ inputs \\
    $\mat{C}$ & output matrix & outputs $\times$ states \\
    $\mat{D}$ & feedthrough matrix & outputs $\times$ inputs \\
    $\mat{K}$ & controller gain matrix & inputs $\times$ states \\
    $\mat{r}$ & \gls{reference} vector & states $\times$ 1 \\
    $\mat{x}$ & state vector & states $\times$ 1 \\
    $\mat{u}$ & input vector & inputs $\times$ 1 \\
    $\mat{y}$ & output vector & outputs $\times$ 1 \\
    \hline
  \end{tabular}
  \caption{Controller matrix dimensions}
\end{booktable}

\section{Pole placement}
\index{controller design!pole placement}

This is the practice of placing the poles of a closed-loop \gls{system} directly
to produce a desired response. Python Control offers several pole placement
algorithms for generating controller or observer gains from a set of poles.

Since all our applications will be discrete \glspl{system}, we'll place poles in
the discrete domain (the z-plane). The s-plane's LHP maps to the inside of a
unit circle (see figure \ref{fig:s2z_mapping_pp}).
\begin{bookfigure}
  \begin{minisvg}{2}{build/modern-control-theory/discrete-state-space-control/s_plane}
  \end{minisvg}
  \hfill
  \begin{minisvg}{2}{build/modern-control-theory/discrete-state-space-control/z_plane}
  \end{minisvg}
  \caption{Mapping of axes from s-plane (left) to z-plane (right)}
  \label{fig:s2z_mapping_pp}
\end{bookfigure}

Pole placement should only be used if you know what you're doing. It's much
easier to let LQR place the poles for you, which we'll discuss next.

\chapterimage{appendices.jpg}{Sunset in an airplane over New Mexico}

\chapter{Linear-quadratic regulator}
\label{ch:deriv_lqr}

This appendix will go into more detail on the linear-quadratic regulator's
derivation and interesting applications.

\renewcommand*{\chapterpath}{\partpath/linear-quadratic-regulator}
\section{Derivation}

Let there be a discrete time linear \gls{system} defined as
\begin{equation}
  \mat{x}_{k+1} = \mat{A}\mat{x}_k + \mat{B}\mat{u}_k
\end{equation}

with the cost functional
\begin{equation*}
  J = \sum_{k=0}^\infty
    \begin{bmatrix}
      \mat{x}_k \\
      \mat{u}_k
    \end{bmatrix}\T
    \begin{bmatrix}
      \mat{Q} & \mat{N} \\
      \mat{N}\T & \mat{R}
    \end{bmatrix}
    \begin{bmatrix}
      \mat{x}_k \\
      \mat{u}_k
    \end{bmatrix}
\end{equation*}

where $J$ represents a trade-off between \gls{state} excursion and
\gls{control effort} with the weighting factors $\mat{Q}$, $\mat{R}$, and
$\mat{N}$. $\mat{Q}$ is the weight matrix for \gls{error}, $\mat{R}$ is the
weight matrix for \gls{control effort}, and $\mat{N}$ is a cross weight matrix
between \gls{error} and \gls{control effort}. $\mat{N}$ is commonly utilized
when penalizing the output in addition to the state and input.
\begin{align*}
  J &= \sum_{k=0}^\infty
    \begin{bmatrix}
      \mat{x}_k \\
      \mat{u}_k
    \end{bmatrix}\T
    \begin{bmatrix}
      \mat{Q}\mat{x}_k + \mat{N}\mat{u}_k \\
      \mat{N}\T\mat{x}_k + \mat{R}\mat{u}_k
    \end{bmatrix} \\
  J &= \sum_{k=0}^\infty
    \begin{bmatrix}
      \mat{x}_k\T & \mat{u}_k\T
    \end{bmatrix}
    \begin{bmatrix}
      \mat{Q}\mat{x}_k + \mat{N}\mat{u}_k \\
      \mat{N}\T\mat{x}_k + \mat{R}\mat{u}_k
    \end{bmatrix} \\
  J &= \sum_{k=0}^\infty
    (\mat{x}_k\T (\mat{Q}\mat{x}_k + \mat{N}\mat{u}_k) +
      \mat{u}_k\T (\mat{N}\T\mat{x}_k + \mat{R}\mat{u}_k)) \\
  J &= \sum_{k=0}^\infty
    (\mat{x}_k\T\mat{Q}\mat{x}_k + \mat{x}_k\T\mat{N}\mat{u}_k +
      \mat{u}_k\T\mat{N}\T\mat{x}_k + \mat{u}_k\T\mat{R}\mat{u}_k) \\
  J &= \sum_{k=0}^\infty
    (\mat{x}_k\T\mat{Q}\mat{x}_k + \mat{x}_k\T\mat{N}\mat{u}_k +
      \mat{x}_k\T\mat{N}\mat{u}_k\T + \mat{u_k}\T\mat{R}\mat{u}_k) \\
  J &= \sum_{k=0}^\infty
    (\mat{x}_k\T\mat{Q}\mat{x}_k + 2\mat{x}_k\T\mat{N}\mat{u}_k +
      \mat{u}_k\T\mat{R}\mat{u}_k) \\
  J &= \sum_{k=0}^\infty
    (\mat{x}_k\T\mat{Q}\mat{x}_k + \mat{u}_k\T\mat{R}\mat{u}_k +
      2\mat{x}_k\T\mat{N}\mat{u}_k)
\end{align*}

The feedback \gls{control law} which minimizes $J$ subject to the constraint
$\mat{x}_{k+1} = \mat{A}\mat{x}_k + \mat{B}\mat{u}_k$ is
\begin{equation*}
  \mat{u}_k = -\mat{K}\mat{x}_k
\end{equation*}

where $\mat{K}$ is given by
\begin{equation*}
  \mat{K} = (\mat{R} + \mat{B}\T\mat{S}\mat{B})^{-1}
    (\mat{B}\T\mat{S}\mat{A} + \mat{N}\T)
\end{equation*}

and $\mat{S}$ is found by solving the discrete time algebraic Riccati equation
defined as
\begin{equation*}
  \mat{A}\T\mat{S}\mat{A} - (\mat{A}\T\mat{S}\mat{B} + \mat{N})
    (\mat{R} + \mat{B}\T\mat{S}\mat{B})^{-1}
    (\mat{B}\T\mat{S}\mat{A} + \mat{N}\T) + \mat{Q} = 0
\end{equation*}

or alternatively
\begin{equation*}
  \mathcal{A}\T\mat{S}\mathcal{A} - \mathcal{A}\T\mat{S}\mat{B}
    (\mat{R} + \mat{B}\T\mat{S}\mat{B})^{-1} \mat{B}\T\mat{S}\mathcal{A} +
    \mat{Q} = 0
\end{equation*}

with
\begin{align*}
  \mathcal{A} &= \mat{A} - \mat{B}\mat{R}^{-1}\mat{N}\T \\
  \mathcal{Q} &= \mat{Q} - \mat{N}\mat{R}^{-1}\mat{N}\T
\end{align*}

If there is no cross-correlation between \gls{error} and \gls{control effort},
$\mat{N}$ is a zero matrix and the cost functional simplifies to
\begin{equation*}
  J = \sum_{k=0}^\infty (\mat{x}_k\T\mat{Q}\mat{x}_k +
    \mat{u}_k\T\mat{R}\mat{u}_k)
\end{equation*}

The feedback \gls{control law} which minimizes this $J$ subject to
$\mat{x}_{k+1} = \mat{A}\mat{x}_k + \mat{B}\mat{u}_k$ is
\begin{equation*}
  \mat{u}_k = -\mat{K}\mat{x}_k
\end{equation*}

where $\mat{K}$ is given by
\begin{equation*}
  \mat{K} = (\mat{R} + \mat{B}\T\mat{S}\mat{B})^{-1} \mat{B}\T\mat{S}\mat{A}
\end{equation*}

and $\mat{S}$ is found by solving the discrete time algebraic Riccati equation
defined as
\begin{equation*}
  \mat{A}\T\mat{S}\mat{A} - \mat{A}\T\mat{S}\mat{B}
    (\mat{R} + \mat{B}\T\mat{S}\mat{B})^{-1} \mat{B}\T\mat{S}\mat{A} +
    \mat{Q} = 0
\end{equation*}

Snippet \ref{lst:lqr} computes the infinite horizon, discrete time LQR.
\begin{coderemote}{Python}{snippets/lqr.py}
  \caption{Infinite horizon, discrete time LQR computation in Python}
  \label{lst:lqr}
\end{coderemote}

Other formulations of LQR for finite horizon and discrete time can be seen on
Wikipedia \cite{bib:wiki_lqr}.

MIT OpenCourseWare has a rigorous proof of the results shown above
\cite{bib:lqr_derivs}.

\section{State feedback with output cost}
\index{controller design!linear-quadratic regulator!state feedback with output
  cost}

LQR is normally used for state feedback on
\begin{align*}
  \mat{x}_{k+1} &= \mat{A}\mat{x}_k + \mat{B}\mat{u}_k \\
  \mat{y}_k &= \mat{C}\mat{x}_k + \mat{D}\mat{u}_k
\end{align*}

with the cost functional
\begin{equation*}
  J = \sum_{k=0}^\infty (\mat{x}_k\T\mat{Q}\mat{x}_k +
    \mat{u}_k\T\mat{R}\mat{u}_k)
\end{equation*}

However, we may not know how to select costs for some of the states, or we don't
care what certain internal states are doing. We can address this by writing the
cost functional in terms of the output vector instead of the state vector. Not
only can we make our output contain a subset of states, but we can use any other
cost metric we can think of as long as it's representable as a linear
combination of the states and inputs.\footnote{We'll see this later on in
section \ref{sec:implicit_model_following} when we define the cost metric as
deviation from the behavior of another model.}

For state feedback with an output cost, we want to minimize the following cost
functional.
\begin{align*}
  J &= \sum_{k=0}^\infty (\mat{y}_k\T\mat{Q}\mat{y}_k +
    \mat{u}_k\T\mat{R}\mat{u}_k)
  \intertext{Substitute in the expression for $\mat{y}_k$.}
  J &= \sum_{k=0}^\infty ((\mat{C}\mat{x}_k + \mat{D}\mat{u}_k)\T\mat{Q}
    (\mat{C}\mat{x}_k + \mat{D}\mat{u}_k) + \mat{u}_k\T\mat{R}\mat{u}_k)
  \intertext{Apply the transpose to the left-hand side of the $\mat{Q}$ term.}
  J &= \sum_{k=0}^\infty ((\mat{x}_k\T\mat{C}\T + \mat{u}_k\T\mat{D}\T)\mat{Q}
    (\mat{C}\mat{x}_k + \mat{D}\mat{u}_k) + \mat{u}_k\T\mat{R}\mat{u}_k)
  \intertext{Factor out $\begin{bmatrix}\mat{x}_k \\ \mat{u}_k\end{bmatrix}\T$
    from the left side and $\begin{bmatrix}\mat{x}_k \\ \mat{u}_k\end{bmatrix}$
    from the right side of each term.}
  J &= \sum_{k=0}^\infty \left(
    \begin{bmatrix}
      \mat{x}_k \\
      \mat{u}_k
    \end{bmatrix}\T
    \begin{bmatrix}
      \mat{C}\T \\
      \mat{D}\T
    \end{bmatrix}
    \mat{Q}
    \begin{bmatrix}
      \mat{C} &
      \mat{D}
    \end{bmatrix}
    \begin{bmatrix}
      \mat{x}_k \\
      \mat{u}_k
    \end{bmatrix} +
    \begin{bmatrix}
      \mat{x}_k \\
      \mat{u}_k
    \end{bmatrix}\T
    \begin{bmatrix}
      \mat{0} & \mat{0} \\
      \mat{0} & \mat{R}
    \end{bmatrix}
    \begin{bmatrix}
      \mat{x}_k \\
      \mat{u}_k
    \end{bmatrix}
    \right) \\
  J &= \sum_{k=0}^\infty \left(
    \begin{bmatrix}
      \mat{x}_k \\
      \mat{u}_k
    \end{bmatrix}\T
    \left(
    \begin{bmatrix}
      \mat{C}\T \\
      \mat{D}\T
    \end{bmatrix}
    \mat{Q}
    \begin{bmatrix}
      \mat{C} &
      \mat{D}
    \end{bmatrix} +
    \begin{bmatrix}
      \mat{0} & \mat{0} \\
      \mat{0} & \mat{R}
    \end{bmatrix}
    \right)
    \begin{bmatrix}
      \mat{x}_k \\
      \mat{u}_k
    \end{bmatrix}
    \right)
  \intertext{Multiply in $\mat{Q}$.}
  J &= \sum_{k=0}^\infty \left(
    \begin{bmatrix}
      \mat{x}_k \\
      \mat{u}_k
    \end{bmatrix}\T
    \left(
    \begin{bmatrix}
      \mat{C}\T\mat{Q} \\
      \mat{D}\T\mat{Q}
    \end{bmatrix}
    \begin{bmatrix}
      \mat{C} &
      \mat{D}
    \end{bmatrix} +
    \begin{bmatrix}
      \mat{0} & \mat{0} \\
      \mat{0} & \mat{R}
    \end{bmatrix}
    \right)
    \begin{bmatrix}
      \mat{x}_k \\
      \mat{u}_k
    \end{bmatrix}
    \right)
  \intertext{Multiply matrices in the left term together.}
  J &= \sum_{k=0}^\infty \left(
    \begin{bmatrix}
      \mat{x}_k \\
      \mat{u}_k
    \end{bmatrix}\T
    \left(
    \begin{bmatrix}
      \mat{C}\T\mat{Q}\mat{C} & \mat{C}\T\mat{Q}\mat{D} \\
      \mat{D}\T\mat{Q}\mat{C} & \mat{D}\T\mat{Q}\mat{D}
    \end{bmatrix} +
    \begin{bmatrix}
      \mat{0} & \mat{0} \\
      \mat{0} & \mat{R}
    \end{bmatrix}
    \right)
    \begin{bmatrix}
      \mat{x}_k \\
      \mat{u}_k
    \end{bmatrix}
    \right)
\end{align*}

Add the terms together.
\begin{equation}
  J = \sum_{k=0}^\infty
  \begin{bmatrix}
    \mat{x}_k \\
    \mat{u}_k
  \end{bmatrix}\T
  \begin{bmatrix}
    \underbrace{\mat{C}\T\mat{Q}\mat{C}}_{\mat{Q}} &
    \underbrace{\mat{C}\T\mat{Q}\mat{D}}_{\mat{N}} \\
    \underbrace{\mat{D}\T\mat{Q}\mat{C}}_{\mat{N}\T} &
    \underbrace{\mat{D}\T\mat{Q}\mat{D} + \mat{R}}_{\mat{R}}
  \end{bmatrix}
  \begin{bmatrix}
    \mat{x}_k \\
    \mat{u}_k
  \end{bmatrix}
\end{equation}

Thus, state feedback with an output cost can be defined as the following
optimization problem.
\begin{theorem}[Linear-quadratic regulator with output cost]
  \begin{align}
    \min_{\mat{u}_k} &\sum_{k=0}^\infty
    \begin{bmatrix}
      \mat{x}_k \\
      \mat{u}_k
    \end{bmatrix}\T
    \begin{bmatrix}
      \underbrace{\mat{C}\T\mat{Q}\mat{C}}_{\mat{Q}} &
      \underbrace{\mat{C}\T\mat{Q}\mat{D}}_{\mat{N}} \\
      \underbrace{\mat{D}\T\mat{Q}\mat{C}}_{\mat{N}\T} &
      \underbrace{\mat{D}\T\mat{Q}\mat{D} + \mat{R}}_{\mat{R}}
    \end{bmatrix}
    \begin{bmatrix}
      \mat{x}_k \\
      \mat{u}_k
    \end{bmatrix}
    \nonumber \\
    \text{subject to } &\mat{x}_{k+1} = \mat{A}\mat{x}_k + \mat{B}\mat{u}_k
  \end{align}

  The optimal control policy $\mat{u}_k^*$ is $\mat{K}(\mat{r}_k - \mat{x}_k)$
  where $\mat{r}_k$ is the desired state. Note that the $\mat{Q}$ in
  $\mat{C}\T\mat{Q}\mat{C}$ is outputs $\times$ outputs instead of states
  $\times$ states. $\mat{K}$ can be computed via the typical LQR equations based
  on the algebraic Ricatti equation.
\end{theorem}

If the output is just the state vector, then $\mat{C} = \mat{I}$,
$\mat{D} = \mat{0}$, and the cost functional simplifies to that of LQR with a
state cost.
\begin{equation*}
  J = \sum_{k=0}^\infty
  \begin{bmatrix}
    \mat{x}_k \\
    \mat{u}_k
  \end{bmatrix}\T
  \begin{bmatrix}
    \mat{Q} & \mat{0} \\
    \mat{0} & \mat{R}
  \end{bmatrix}
  \begin{bmatrix}
    \mat{x}_k \\
    \mat{u}_k
  \end{bmatrix}
\end{equation*}

\section{Implicit model following}
\label{sec:implicit_model_following}
\index{controller design!linear-quadratic regulator!implicit model following}

If we want to design a feedback controller that erases the dynamics of our
system and makes it behave like some other system, we can use \textit{implicit
model following}. This is used on the Blackhawk helicopter at NASA Ames research
center when they want to make it fly like experimental aircraft (within the
limits of the helicopter's actuators, of course).

\subsection{Following reference system matrix}

Let the original system dynamics be
\begin{align*}
  \mat{x}_{k+1} &= \mat{A}\mat{x}_k + \mat{B}\mat{u}_k \\
  \mat{y}_k &= \mat{C}\mat{x}_k
\end{align*}

and the desired system dynamics be $\mat{z}_{k+1} = \mat{A}_{ref}\mat{z}_k$.
\begin{align*}
  \mat{y}_{k+1} &= \mat{C}\mat{x}_{k+1} \\
  \mat{y}_{k+1} &= \mat{C}(\mat{A}\mat{x}_k + \mat{B}\mat{u}_k) \\
  \mat{y}_{k+1} &= \mat{C}\mat{A}\mat{x}_k + \mat{C}\mat{B}\mat{u}_k
\end{align*}

We want to minimize the following cost functional.
\begin{equation*}
  J = \sum_{k=0}^\infty \left((\mat{y}_{k+1} - \mat{z}_{k+1})\T \mat{Q}
    (\mat{y}_{k+1} - \mat{z}_{k+1}) + \mat{u}_k\T\mat{R}\mat{u}_k\right)
\end{equation*}

We'll be measuring the desired system's state, so let $\mat{y} = \mat{z}$.
\begin{align*}
  \mat{z}_{k+1} &= \mat{A}_{ref}\mat{y}_k \\
  \mat{z}_{k+1} &= \mat{A}_{ref}\mat{C}\mat{x}_k
\end{align*}

Therefore,
\begin{align*}
  \mat{y}_{k+1} - \mat{z}_{k+1} &=
    \mat{C}\mat{A}\mat{x}_k + \mat{C}\mat{B}\mat{u}_k -
    (\mat{A}_{ref}\mat{C}\mat{x}_k) \\
  \mat{y}_{k+1} - \mat{z}_{k+1} &=
    (\mat{C}\mat{A} - \mat{A}_{ref}\mat{C})\mat{x}_k + \mat{C}\mat{B}\mat{u}_k
\end{align*}

Substitute this into the cost functional.
\begin{align*}
  J &= \sum_{k=0}^\infty \left((\mat{y}_{k+1} - \mat{z}_{k+1})\T \mat{Q}
    (\mat{y}_{k+1} - \mat{z}_{k+1}) + \mat{u}_k\T\mat{R}\mat{u}_k\right) \\
  J &= \sum_{k=0}^\infty \left(
    ((\mat{C}\mat{A} - \mat{A}_{ref}\mat{C})\mat{x}_k + \mat{C}\mat{B}\mat{u}_k)\T
    \mat{Q}
    ((\mat{C}\mat{A} - \mat{A}_{ref}\mat{C})\mat{x}_k + \mat{C}\mat{B}\mat{u}_k) \right. \\
  &\qquad \left. + \mat{u}_k\T\mat{R}\mat{u}_k\right)
\end{align*}

Apply the transpose to the left-hand side of the $\mat{Q}$ term.
\begin{align*}
  J &= \sum_{k=0}^\infty \left(
    (\mat{x}_k\T(\mat{C}\mat{A} - \mat{A}_{ref}\mat{C})\T + \mat{u}_k\T(\mat{C}\mat{B})\T)
    \mat{Q}
    ((\mat{C}\mat{A} - \mat{A}_{ref}\mat{C})\mat{x}_k + \mat{C}\mat{B}\mat{u}_k) \right. \\
  &\qquad \left. + \mat{u}_k\T\mat{R}\mat{u}_k\right)
\end{align*}

Factor out $\begin{bmatrix}\mat{x}_k \\ \mat{u}_k\end{bmatrix}\T$ from the left
side and $\begin{bmatrix}\mat{x}_k \\ \mat{u}_k\end{bmatrix}$ from the right
side of each term.
\begin{align*}
  J &= \sum_{k=0}^\infty \left(
    \begin{bmatrix}
      \mat{x}_k \\
      \mat{u}_k
    \end{bmatrix}\T
    \begin{bmatrix}
      (\mat{C}\mat{A} - \mat{A}_{ref}\mat{C})\T \\
      (\mat{C}\mat{B})\T
    \end{bmatrix}
    \mat{Q}
    \begin{bmatrix}
      \mat{C}\mat{A} - \mat{A}_{ref}\mat{C} &
      \mat{C}\mat{B}
    \end{bmatrix}
    \begin{bmatrix}
      \mat{x}_k \\
      \mat{u}_k
    \end{bmatrix} \right. \\
  &\qquad \left. +
    \begin{bmatrix}
      \mat{x}_k \\
      \mat{u}_k
    \end{bmatrix}\T
    \begin{bmatrix}
      \mat{0} & \mat{0} \\
      \mat{0} & \mat{R}
    \end{bmatrix}
    \begin{bmatrix}
      \mat{x}_k \\
      \mat{u}_k
    \end{bmatrix}
    \right) \\
  J &= \sum_{k=0}^\infty \left(
    \begin{bmatrix}
      \mat{x}_k \\
      \mat{u}_k
    \end{bmatrix}\T
    \left(
    \begin{bmatrix}
      (\mat{C}\mat{A} - \mat{A}_{ref}\mat{C})\T \\
      (\mat{C}\mat{B})\T
    \end{bmatrix}
    \mat{Q}
    \begin{bmatrix}
      \mat{C}\mat{A} - \mat{A}_{ref}\mat{C} &
      \mat{C}\mat{B}
    \end{bmatrix}
    \right.\right. \\
  &\qquad \left.\left. +
    \begin{bmatrix}
      \mat{0} & \mat{0} \\
      \mat{0} & \mat{R}
    \end{bmatrix}
    \right)
    \begin{bmatrix}
      \mat{x}_k \\
      \mat{u}_k
    \end{bmatrix}
    \right)
\end{align*}

Multiply in $\mat{Q}$.
\begin{align*}
  J &= \sum_{k=0}^\infty \left(
    \begin{bmatrix}
      \mat{x}_k \\
      \mat{u}_k
    \end{bmatrix}\T
    \left(
    \begin{bmatrix}
      (\mat{C}\mat{A} - \mat{A}_{ref}\mat{C})\T\mat{Q} \\
      (\mat{C}\mat{B})\T\mat{Q}
    \end{bmatrix}
    \begin{bmatrix}
      \mat{C}\mat{A} - \mat{A}_{ref}\mat{C} &
      \mat{C}\mat{B}
    \end{bmatrix} \right.\right. \\
  &\qquad \left.\left. +
    \begin{bmatrix}
      \mat{0} & \mat{0} \\
      \mat{0} & \mat{R}
    \end{bmatrix}
    \right)
    \begin{bmatrix}
      \mat{x}_k \\
      \mat{u}_k
    \end{bmatrix}
    \right)
\end{align*}

Multiply matrices in the left term together.
\begin{align*}
  J &= \sum_{k=0}^\infty \\
  &\left(
    \begin{bmatrix}
      \mat{x}_k \\
      \mat{u}_k
    \end{bmatrix}\T
    \left(
    \begin{bmatrix}
      (\mat{C}\mat{A} - \mat{A}_{ref}\mat{C})\T\mat{Q}(\mat{C}\mat{A} - \mat{A}_{ref}\mat{C}) &
      (\mat{C}\mat{A} - \mat{A}_{ref}\mat{C})\T\mat{Q}(\mat{C}\mat{B}) \\
      (\mat{C}\mat{B})\T\mat{Q}(\mat{C}\mat{A} - \mat{A}_{ref}\mat{C}) &
      (\mat{C}\mat{B})\T\mat{Q}(\mat{C}\mat{B})
    \end{bmatrix} \right.\right. \\
  &\qquad \left.\left. +
    \begin{bmatrix}
      \mat{0} & \mat{0} \\
      \mat{0} & \mat{R}
    \end{bmatrix}
    \right)
    \begin{bmatrix}
      \mat{x}_k \\
      \mat{u}_k
    \end{bmatrix}
    \right)
\end{align*}

Add the terms together.
\begin{align}
  J &= \sum_{k=0}^\infty \nonumber \\
  &\begin{bmatrix}
    \mat{x}_k \\
    \mat{u}_k
  \end{bmatrix}\T
  \begin{bsmallmatrix}
    \underbrace{(\mat{C}\mat{A} - \mat{A}_{ref}\mat{C})\T\mat{Q}
      (\mat{C}\mat{A} - \mat{A}_{ref}\mat{C})}_{\mat{Q}} &
    \underbrace{(\mat{C}\mat{A} - \mat{A}_{ref}\mat{C})\T\mat{Q}
      (\mat{C}\mat{B})}_{\mat{N}} \\
    \underbrace{(\mat{C}\mat{B})\T\mat{Q}
      (\mat{C}\mat{A} - \mat{A}_{ref}\mat{C})}_{\mat{N}\T} &
    \underbrace{(\mat{C}\mat{B})\T\mat{Q}(\mat{C}\mat{B}) + \mat{R}}_{\mat{R}}
  \end{bsmallmatrix}
  \begin{bmatrix}
    \mat{x}_k \\
    \mat{u}_k
  \end{bmatrix}
\end{align}

Thus, implicit model following can be defined as the following optimization
problem.
\begin{theorem}[Implicit model following]
  \begin{align}
    \mat{u}_k^* = \argmin_{\mat{u}_k} &\sum_{k=0}^\infty
    \begin{bmatrix}
      \mat{x}_k \\
      \mat{u}_k
    \end{bmatrix}\T
    \begin{bmatrix}
      \mat{Q}_{imf} & \mat{N}_{imf} \\
      \mat{N}_{imf}\T & \mat{R}_{imf}
    \end{bmatrix}
    \begin{bmatrix}
      \mat{x}_k \\
      \mat{u}_k
    \end{bmatrix}
    \nonumber \\
    \text{subject to } &\mat{x}_{k+1} = \mat{A}\mat{x}_k + \mat{B}\mat{u}_k
  \end{align}
  where
  \begin{align*}
    \mat{Q}_{imf} &= (\mat{C}\mat{A} - \mat{A}_{ref}\mat{C})\T\mat{Q}
      (\mat{C}\mat{A} - \mat{A}_{ref}\mat{C}) \\
    \mat{N}_{imf} &= (\mat{C}\mat{A} - \mat{A}_{ref}\mat{C})\T\mat{Q}
      (\mat{C}\mat{B}) \\
    \mat{R}_{imf} &= (\mat{C}\mat{B})\T\mat{Q}(\mat{C}\mat{B}) + \mat{R}
  \end{align*}

  The optimal control policy $\mat{u}_k^*$ is $-\mat{K}\mat{x}_k$. $\mat{K}$ can
  be computed via the typical LQR equations based on the algebraic Ricatti
  equation.
\end{theorem}

The control law $\mat{u}_{imf,k} = -\mat{K}\mat{x}_k$ makes
$\mat{x}_{k+1} = \mat{A}\mat{x}_k + \mat{B}\mat{u}_{imf,k}$ match the open-loop
response of $\mat{z}_{k+1} = \mat{A}_{ref}\mat{z}_k$.

If the original and desired system have the same states, then
$\mat{C} = \mat{I}$ and the cost functional simplifies to
\begin{equation}
  J = \sum_{k=0}^\infty
  \begin{bmatrix}
    \mat{x}_k \\
    \mat{u}_k
  \end{bmatrix}\T
  \begin{bmatrix}
    \underbrace{(\mat{A} - \mat{A}_{ref})\T\mat{Q}
      (\mat{A} - \mat{A}_{ref})}_{\mat{Q}} &
    \underbrace{(\mat{A} - \mat{A}_{ref})\T\mat{Q}\mat{B}}_{\mat{N}} \\
    \underbrace{\mat{B}\T\mat{Q}(\mat{A} - \mat{A}_{ref})}_{\mat{N}\T} &
    \underbrace{\mat{B}\T\mat{Q}\mat{B} + \mat{R}}_{\mat{R}}
  \end{bmatrix}
  \begin{bmatrix}
    \mat{x}_k \\
    \mat{u}_k
  \end{bmatrix}
\end{equation}

\subsection{Following reference input matrix}

The feedback control law above makes the open-loop system behave like
$\mat{A}_{ref}$, but the input dynamics are still that of the original system.
Here's how to make the input dynamics behave like $\mat{B}_{ref}$. We want to
find the $\mat{u}_{imf,k}$ that makes the real model follow the reference model.
\begin{align*}
  \mat{x}_{k+1} &= \mat{A}\mat{x}_k + \mat{B}\mat{u}_{imf,k} \\
  \mat{z}_{k+1} &= \mat{A}_{ref}\mat{z}_k + \mat{B}_{ref}\mat{u}_k
\end{align*}

Let $\mat{x} = \mat{z}$.
\begin{align*}
  \mat{x}_{k+1} &= \mat{z}_{k+1} \\
  \mat{A}\mat{x}_k + \mat{B}\mat{u}_{imf,k} &= \mat{A}_{ref}\mat{x}_k +
    \mat{B}_{ref}\mat{u}_k \\
  \mat{B}\mat{u}_{imf,k} &= \mat{A}_{ref}\mat{x}_k - \mat{A}\mat{x}_k +
    \mat{B}_{ref}\mat{u}_k \\
  \mat{B}\mat{u}_{imf,k} &= (\mat{A}_{ref} - \mat{A})\mat{x}_k +
    \mat{B}_{ref}\mat{u}_k \\
  \mat{u}_{imf,k} &= \mat{B}^+ ((\mat{A}_{ref} - \mat{A})\mat{x}_k +
    \mat{B}_{ref}\mat{u}_k) \\
  \mat{u}_{imf,k} &= -\mat{B}^+ (\mat{A} - \mat{A}_{ref})\mat{x}_k +
    \mat{B}^+ \mat{B}_{ref}\mat{u}_k
\end{align*}

The first term makes the open-loop poles match that of the reference model, and
the second term makes the input behave like that of the reference model.

\section{Time delay compensation}
\index{controller design!linear-quadratic regulator!time delay compensation}

Linear-Quadratic regulator controller gains tend to be aggressive, and in the
presence of sensor delay, they may be unstable (see figure
\ref{fig:elevator_time_delay_no_comp}). However, if we know the amount of delay,
we can design the controller for a projected version of the undelayed model.
That is, we can compute the control based on where the system will be after the
time delay (see figure \ref{fig:elevator_time_delay_comp}).
\begin{bookfigure}
  \begin{minisvg}{2}{build/\chapterpath/elevator_time_delay_no_comp}
    \caption{Elevator response at 5ms sample period with 50ms of output lag
      (uncompensated controller gains)}
    \label{fig:elevator_time_delay_no_comp}
  \end{minisvg}
  \hfill
  \begin{minisvg}{2}{build/\chapterpath/elevator_time_delay_comp}
    \caption{Elevator response at 5ms sample period with 50ms of output lag
      (compensated controller gains)}
    \label{fig:elevator_time_delay_comp}
  \end{minisvg}
\end{bookfigure}

This method of delay compensation seems to work best for second-order systems.
Figure \ref{fig:drivetrain_time_delay_no_comp} shows time delay for a drivetrain
velocity system. Figure \ref{fig:drivetrain_time_delay_comp} shows that
compensating the controller gain significantly reduces the feedback gain. Mainly
feedforward is acting for a predominantly open-loop response, which has poor
disturbance rejection.
\begin{bookfigure}
  \begin{minisvg}{2}{build/\chapterpath/drivetrain_time_delay_no_comp}
    \caption{Drivetrain response at 1ms sample period with 40ms of output lag
      (uncompensated controller gain)}
    \label{fig:drivetrain_time_delay_no_comp}
  \end{minisvg}
  \hfill
  \begin{minisvg}{2}{build/\chapterpath/drivetrain_time_delay_comp}
    \caption{Drivetrain response at 1ms sample period with 40ms of output lag
      (compensated controller gain)}
    \label{fig:drivetrain_time_delay_comp}
  \end{minisvg}
\end{bookfigure}

For this controller to be optimal, we need to have a time-varying control gain
to give the system an initial kick in the right direction. Since we are
effectively controlling the state some time in the future, the state
exponentially converges to zero and so does the control gain. However, the
inputs we use to project into the future start at zero since there's no history.
This is reflected by equation (10) in the original paper
\cite{bib:lqr_time_delay}. As the input buffer fills up, the controller gain
converges to the steady-state one we're using. Fixing the source of the time
delay is always preferred in general, but especially in cases that require a
constant controller gain.

We'll show how to derive this controller gain compensation for continuous and
discrete systems.

\subsection{Continuous case}

The continuous linear system is defined as
\begin{equation*}
  \dot{\mtx{x}} = \mtx{A}\mtx{x}(t) + \mtx{B}\mtx{u}(t)
\end{equation*}

Let the controller for this system be
\begin{equation*}
  \mtx{u}(t) = -\mtx{K}\mtx{x}(t)
\end{equation*}

Substitute this into the continuous model.
\begin{align*}
  \dot{\mtx{x}} &= \mtx{A}\mtx{x}(t) + \mtx{B}\mtx{u}(t) \\
  \dot{\mtx{x}} &= \mtx{A}\mtx{x}(t) + \mtx{B}(-\mtx{K}\mtx{x}(t)) \\
  \dot{\mtx{x}} &= \mtx{A}\mtx{x}(t) - \mtx{B}\mtx{K}\mtx{x}(t) \\
  \dot{\mtx{x}} &= (\mtx{A} - \mtx{B}\mtx{K}) \mtx{x}(t)
\end{align*}

Let $L$ be the amount of time delay in seconds. Take the matrix exponential from
the current time $t$ to $L$ in the future.
\begin{equation}
  \mtx{x}(t + L) = e^{(\mtx{A} - \mtx{B}\mtx{K})L} \mtx{x}(t)
    \label{eq:continuous_advance_state_by_delay}
\end{equation}

We can avoid the time delay if we compute the control based on the plant $L$
seconds in the future using equation
\eqref{eq:continuous_advance_state_by_delay}. Therefore, the latency-compensated
controller is
\begin{align}
  \mtx{u}(t) &= -\mtx{K}\mtx{x}(t + L) \nonumber \\
  \mtx{u}(t) &= -\mtx{K} e^{(\mtx{A} - \mtx{B}\mtx{K})L} \mtx{x}(t)
\end{align}

\subsection{Discrete case}

The discrete linear system is defined as
\begin{equation*}
  \mtx{x}_{k+1} = \mtx{A}\mtx{x}_k + \mtx{B}\mtx{u}_k
\end{equation*}

Let the controller for this system be
\begin{equation*}
  \mtx{u}_k = -\mtx{K}\mtx{x}_k
\end{equation*}

Substitute this into the discrete model.
\begin{align*}
  \mtx{x}_{k+1} &= \mtx{A}\mtx{x}_k + \mtx{B}\mtx{u}_k \\
  \mtx{x}_{k+1} &= \mtx{A}\mtx{x}_k + \mtx{B}(-\mtx{K}\mtx{x}_k) \\
  \mtx{x}_{k+1} &= \mtx{A}\mtx{x}_k - \mtx{B}\mtx{K}\mtx{x}_k \\
  \mtx{x}_{k+1} &= (\mtx{A} - \mtx{B}\mtx{K}) \mtx{x}_k
\end{align*}

Let $T$ be the duration between timesteps in seconds and $L$ be the amount of
time delay in seconds. $\frac{L}{T}$ gives the number of timesteps represented
by $L$.
\begin{equation}
  \mtx{x}_{k+L} = (\mtx{A} - \mtx{B}\mtx{K})^\frac{L}{T} \mtx{x}_k
    \label{eq:discrete_advance_state_by_delay}
\end{equation}

We can avoid the time delay if we compute the control based on the plant $L$
seconds in the future using equation \eqref{eq:discrete_advance_state_by_delay}.
Therefore, the latency-compensated controller is
\begin{align}
  \mtx{u}_k &= -\mtx{K}\mtx{x}_{k+L} \nonumber \\
  \mtx{u}_k &= -\mtx{K} (\mtx{A} - \mtx{B}\mtx{K})^\frac{L}{T} \mtx{x}_k
    \label{eq:discrete_delay_comp_control_law}
\end{align}

If the delay $L$ isn't a multiple of the sample period $T$ in equation
\eqref{eq:discrete_delay_comp_control_law}, we have to evaluate a fractional
matrix power, which can be tricky. If $\mtx{A} - \mtx{B}\mtx{K}$ is
diagonalizable, we can obtain an exact answer for
$(\mtx{A} - \mtx{B}\mtx{K})^\frac{L}{T}$ by decomposing
$\mtx{A} - \mtx{B}\mtx{K}$ into $\mtx{P}\mtx{D}\mtx{P}^{-1}$ where $\mtx{D}$ is
a diagonal matrix, computing $\mtx{D}^\frac{L}{T}$ as each diagonal element
raised to $\frac{L}{T}$, then recomposing
$\mtx{P}\mtx{D}^\frac{L}{T}\mtx{P}^{-1}$. If $\mtx{A} - \mtx{B}\mtx{K}$ isn't
diagonalizable, we'll have to approximate the matrix power by rounding
$\frac{L}{T}$ to the nearest integer. This approximation gets worse as
$L \bmod T$ approaches $\frac{T}{2}$.


\section{Feedforward}

Feedback control can be effective for \gls{reference} \gls{tracking} (making a
\gls{system}'s output follow a desired \gls{reference} signal), but it's a
reactionary measure; the \gls{system} won't start applying \gls{control effort}
until the \gls{system} is already behind. If we could tell the \gls{controller}
about the desired movement and required input beforehand, the \gls{system} could
react quicker and the feedback \gls{controller} could do less work. A
\gls{controller} that feeds information forward into the \gls{plant} like this
is called a \gls{feedforward controller}.

A \gls{feedforward controller} injects information about the \gls{system}'s
dynamics (like a \gls{model} does) or the desired movement. The feedforward
handles parts of the control actions we already know must be applied to make a
\gls{system} track a \gls{reference}, then feedback compensates for what we do
not or cannot know about the \gls{system}'s behavior at runtime.

There are two types of feedforwards: model-based feedforward and feedforward for
unmodeled dynamics. The first solves a mathematical model of the system for the
inputs required to meet desired velocities and accelerations. The second
compensates for unmodeled forces or behaviors directly so the feedback
controller doesn't have to. Both types can facilitate simpler feedback
controllers; we'll cover examples of each in later chapters.

\section{Numerical integration methods}

Most systems don't have linear dynamics and their differential equations can't
be solved analytically. Instead, we'll have to approximate their solutions with
numerical integration.

\subsection{Butcher tableaus}

Butcher tableaus are a more succinct representation for explicit and implicit
Runge-Kutta numerical integration methods. Here's the general structure for
explicit methods.
\begin{equation*}
  \renewcommand\arraystretch{1.2}
  \begin{array}{c|cccc}
    0 \\
    c_2    & a_{2,1} \\
    \vdots & \vdots & \ddots \\
    c_s    & a_{s,1} & \hdots & a_{s,s-1} \\
    \hline
           & b_1    & \hdots & \hdots    & b_s
  \end{array}
\end{equation*}

where $s$ is the number of stages in the method, the matrix $[a_{ij}]$ is the
Runge-Kutta matrix, $b_1, \ldots, b_s$ are the weights, and $c_1, \ldots, c_s$
are the nodes. The top-left quadrant contains the sums of the rows in the
top-right quadrant. Each column in the right half corresponds to a $\mat{k}$
coefficient from $\mat{k}_1$ to $\mat{k}_s$.

The family of solutions to $\dot{\mat{x}} = f(t, \mat{x})$ is given by
\begin{align*}
  \mat{k}_1 &= f(t_k, \mat{x}_k) \\
  \mat{k}_2 &= f(t_k + c_2 h, \mat{x}_k + h (a_{2,1} \mat{k}_1)) \\
  &\ \ \vdots \\
  \mat{k}_s &= f(t_k + c_s h, \mat{x}_k +
    h (a_{s,1} \mat{k}_1 + \ldots + a_{s,s-1} \mat{k}_{s-1})) \\
  \mat{x}_{k+1} &= \mat{x}_k + h \sum_{i=1}^s b_i \mat{k}_i
\end{align*}

where $h$ is the timestep duration.

\subsection{Forward Euler method}
\index{numerical integration!Forward Euler}

The simplest explicit Runge-Kutta integration method is forward Euler
integration. We don't recommend using it because it suffers from numerical
stability issues. We'll demonstrate how to translate its Butcher tableau into
equations that integrate $\dot{\mat{x}} = f(t, \mat{x})$ from $0$ to $h$.
\begin{center}
  \begin{minipage}{0.35\linewidth}
    \centering
    \begin{alignat*}{7}
      \mat{k}_1 &= f(t +
        && {\color{blue}0} h,
        && \mat{x}_k)
        && \\
      \mat{x}_{k+1} &=
        &&
        && \mat{x}_k + h (
        && {\color{deepgreen}1} \mat{k}_1)
    \end{alignat*}
  \end{minipage}
  \quad
  \begin{minipage}{0.35\linewidth}
    \centering
    \begin{equation*}
      \renewcommand\arraystretch{1.2}
      \begin{array}{c|c}
        {\color{blue}0} \\
        \hline
        & {\color{deepgreen}1}
      \end{array}
    \end{equation*}
  \end{minipage}
\end{center}

Remove zeroed out terms.
\begin{align*}
  \mat{k}_1 &= f(t, \mat{x}_k) \\
  \mat{x}_{k+1} &= \mat{x}_k + h \mat{k}_1
  \intertext{Simplify.}
  \mat{x}_{k+1} &= \mat{x}_k + h f(t, \mat{x}_k)
\end{align*}

In FRC, our differential equations are of the form
$\dot{\mat{x}} = f(\mat{x}, \mat{u})$ where $\mat{u}$ is held constant between
timesteps. Since it's time-invariant, we can ignore the time argument of the
integration method. This gives theorem \ref{thm:forward_euler}.
\begin{theorem}[Forward Euler integration]
  \label{thm:forward_euler}

  Given the differential equation $\dot{\mat{x}} = f(\mat{x}_k, \mat{u}_k)$,
  this method solves for $\mat{x}_{k+1}$ at $h$ seconds in the future.
  $\mat{u}$ is assumed to be held constant between timesteps.
  \begin{center}
    \begin{minipage}{0.35\linewidth}
      \centering
      \begin{equation*}
        \mat{x}_{k+1} = \mat{x}_k + h f(\mat{x}_k, \mat{u}_k)
      \end{equation*}
    \end{minipage}
    \quad
    \begin{minipage}{0.35\linewidth}
      \centering
      \begin{equation*}
        \renewcommand\arraystretch{1.2}
        \begin{array}{c|c}
          0 \\
          \hline
          & 1
        \end{array}
      \end{equation*}
    \end{minipage}
  \end{center}
\end{theorem}

\subsection{Runge-Kutta fourth-order method}
\index{numerical integration!Runge-Kutta fourth-order}

The most common method we'll cover is Runge-Kutta fourth-order (RK4). It's
simple and accurate for most systems we'll see in FRC. We'll demonstrate how to
translate its Butcher tableau into equations that integrate
$\dot{\mat{x}} = f(t, \mat{x})$ from $0$ to $h$.
\begin{center}
  \begin{minipage}{0.35\linewidth}
    \centering
    \begin{alignat*}{7}
      \mat{k}_1 &= f(t +
        && {\color{blue}0} h,
        && \mat{x}_k)
        &&
        &&
        &&
        && \\
      \mat{k}_2 &= f(t +
        && {\color{blue}\frac{1}{2}} h,
        && \mat{x}_k + h (
        && {\color{deeporange}\frac{1}{2}} \mat{k}_1))
        &&
        &&
        && \\
      \mat{k}_3 &= f(t +
        && {\color{blue}\frac{1}{2}} h,
        && \mat{x}_k + h (
        && {\color{deeporange}0} \mat{k}_1 +
        && {\color{deeporange}\frac{1}{2}} \mat{k}_2))
        &&
        && \\
      \mat{k}_4 &= f(t +
        && {\color{blue}1} h,
        && \mat{x}_k + h (
        && {\color{deeporange}0} \mat{k}_1 +
        && {\color{deeporange}0} \mat{k}_2 +
        && {\color{deeporange}1} \mat{k}_3))
        && \\
      \mat{x}_{k+1} &=
        &&
        && \mat{x}_k + h (
        && {\color{deepgreen}\frac{1}{6}} \mat{k}_1 +
        && {\color{deepgreen}\frac{1}{3}} \mat{k}_2 +
        && {\color{deepgreen}\frac{1}{3}} \mat{k}_3 +
        && {\color{deepgreen}\frac{1}{6}} \mat{k}_4)
    \end{alignat*}
  \end{minipage}
  \quad
  \begin{minipage}{0.35\linewidth}
    \centering
    \begin{equation*}
      \renewcommand\arraystretch{1.2}
      \begin{array}{c|cccc}
        {\color{blue}0} \\
        {\color{blue}\frac{1}{2}} & {\color{deeporange}\frac{1}{2}} \\
        {\color{blue}\frac{1}{2}} & {\color{deeporange}0}           & {\color{deeporange}\frac{1}{2}} \\
        {\color{blue}1}           & {\color{deeporange}0}           & {\color{deeporange}0}           & {\color{deeporange}1} \\
        \hline
                                  & {\color{deepgreen}\frac{1}{6}}  & {\color{deepgreen}\frac{1}{3}}  & {\color{deepgreen}\frac{1}{3}} & {\color{deepgreen}\frac{1}{6}}
      \end{array}
    \end{equation*}
  \end{minipage}
\end{center}

Remove zeroed out terms.
\begin{align*}
  \mat{k}_1 &= f(t, \mat{x}_k) \\
  \mat{k}_2 &= f(t + \frac{1}{2} h, \mat{x}_k + h \frac{1}{2} \mat{k}_1) \\
  \mat{k}_3 &= f(t + \frac{1}{2} h, \mat{x}_k + h \frac{1}{2} \mat{k}_2) \\
  \mat{k}_4 &= f(t + h, \mat{x}_k + h \mat{k}_3) \\
  \mat{x}_{k+1} &= \mat{x}_k + h \left(
    \frac{1}{6} \mat{k}_1 +
    \frac{1}{3} \mat{k}_2 +
    \frac{1}{3} \mat{k}_3 +
    \frac{1}{6} \mat{k}_4\right)
  \intertext{$\frac{1}{6}$ is usually factored out of the last equation to
    reduce the number of floating point operations.}
  \mat{x}_{k+1} &= \mat{x}_k + h \frac{1}{6} (
    \mat{k}_1 + 2\mat{k}_2 + 2\mat{k}_3 + \mat{k}_4)
\end{align*}

In FRC, our differential equations are of the form
$\dot{\mat{x}} = f(\mat{x}, \mat{u})$ where $\mat{u}$ is held constant between
timesteps. Since it's time-invariant, we can ignore the time argument of the
integration method. This gives theorem \ref{thm:rk4}.
\begin{theorem}[Runge-Kutta fourth-order integration]
  \label{thm:rk4}

  Given the differential equation $\dot{\mat{x}} = f(\mat{x}_k, \mat{u}_k)$,
  this method solves for $\mat{x}_{k+1}$ at $h$ seconds in the future.
  $\mat{u}$ is assumed to be held constant between timesteps.
  \begin{center}
    \begin{minipage}{0.35\linewidth}
      \centering
      \begin{align*}
        \mat{k}_1 &= f(\mat{x}_k, \mat{u}_k) \\
        \mat{k}_2 &= f(\mat{x}_k + h \frac{1}{2}\mat{k}_1, \mat{u}_k) \\
        \mat{k}_3 &= f(\mat{x}_k + h \frac{1}{2}\mat{k}_2, \mat{u}_k) \\
        \mat{k}_4 &= f(\mat{x}_k + h \mat{k}_3, \mat{u}_k) \\
        \mat{x}_{k+1} &= \mat{x}_k + h \frac{1}{6} (\mat{k}_1 + 2\mat{k}_2 +
          2\mat{k}_3 + \mat{k}_4)
      \end{align*}
    \end{minipage}
    \quad
    \begin{minipage}{0.35\linewidth}
      \centering
      \begin{equation*}
        \renewcommand\arraystretch{1.2}
        \begin{array}{c|cccc}
          0 \\
          \frac{1}{2} & \frac{1}{2} \\
          \frac{1}{2} & 0 & \frac{1}{2} \\
          1 & 0 & 0 & 1 \\
          \hline
          & \frac{1}{6} & \frac{1}{3} & \frac{1}{3} & \frac{1}{6}
        \end{array}
      \end{equation*}
    \end{minipage}
  \end{center}
\end{theorem}

Here's a reference implementation.
\begin{coderemote}{cpp}{snippets/RK4.cpp}
  \caption{RK4 implementation in C++}
\end{coderemote}

Other methods of Runge-Kutta integration exist with various properties
\cite{bib:wiki_rk4}, but the one presented here is popular for its high accuracy
relative to the amount of floating point operations (FLOPs) it requires.

\subsection{Dormand-Prince method}
\index{numerical integration!Dormand-Prince}

Dormand-Prince (RKDP) is a fourth-order method with fifth-order error checking.
It uses an adaptive stepsize to enforce an upper bound on the integration error.
\begin{theorem}[Dormand-Prince integration]
  \label{thm:rkdp}

  Given the differential equation $\dot{\mat{x}} = f(\mat{x}_k, \mat{u}_k)$,
  this method solves for $\mat{x}_{k+1}$ at $h$ seconds in the future.
  $\mat{u}$ is assumed to be held constant between timesteps. It has the
  following Butcher tableau.
  \begin{equation*}
    \renewcommand\arraystretch{1.2}
    \begin{array}{c|ccccccc}
      0 \\
      \frac{1}{5} & \frac{1}{5} \\
      \frac{3}{10} & \frac{3}{40} & \frac{9}{40} \\
      \frac{4}{5} & \frac{44}{45} & -\frac{56}{15} & \frac{32}{9} \\
      \frac{8}{9} & \frac{19372}{6561} & -\frac{25360}{2187} &
        \frac{64448}{6561} & -\frac{212}{729} \\
      1 & \frac{9017}{3168} & -\frac{355}{33} & \frac{46732}{5247} &
        \frac{49}{176} & -\frac{5103}{18656} \\
      1 & \frac{35}{384} & 0 & \frac{500}{1113} & \frac{125}{192} &
        -\frac{2187}{6784} & \frac{11}{84} \\
      \hline
      & \frac{35}{384} & 0 & \frac{500}{1113} & \frac{125}{192} &
        -\frac{2187}{6784} & \frac{11}{84} & 0 \\
      & \frac{5179}{57600} & 0 & \frac{7571}{16695} & \frac{393}{640} &
        -\frac{92097}{339200} & \frac{187}{2100} & \frac{1}{40}
    \end{array}
  \end{equation*}
  The first row of coefficients below the table divider gives the fifth-order
  accurate solution. The second row gives an alternative solution which,
  when subtracted from the first solution, gives the error estimate.
\end{theorem}

Here's a reference implementation.
\begin{coderemote}{cpp}{snippets/RKDP.cpp}
  \caption{RKDP implementation in C++}
\end{coderemote}

