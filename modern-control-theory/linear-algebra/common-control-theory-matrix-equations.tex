\section{Common control theory matrix equations}

Here's some common matrix equations from control theory we'll use later on.
Solvers for them exist in \href{https://github.com/RobotLocomotion/drake}{Drake}
(C++) and \href{https://github.com/scipy/scipy}{SciPy} (Python).

\subsection{Continuous algebraic Riccati equation (CARE)}
\index{algebraic Riccati equation!continuous}

The continuous algebraic Riccati equation (CARE) appears in the solution to the
continuous time LQ problem.
\begin{equation}
  \mat{A}\mat{X} + \mat{X}\mat{A} - \mat{X}\mat{B}\mat{R}^{-1}\mat{B}\T\mat{X} +
    \mat{Q} = 0
\end{equation}

\subsection{Discrete algebraic Riccati equation (DARE)}
\label{subsec:dare}
\index{algebraic Riccati equation!discrete}

The discrete algebraic Riccati equation (DARE) appears in the solution to the
discrete time LQ problem.
\begin{equation}
  \mat{X} = \mat{A}\T\mat{X}\mat{A} - (\mat{A}\T\mat{X}\mat{B})(\mat{R} +
    \mat{B}\T\mat{X}\mat{B})^{-1} \mat{B}\T\mat{X}\mat{A} + \mat{Q}
\end{equation}

Snippets \ref{lst:dare_dense} and \ref{lst:dare_sparse} compute the unique
stabilizing solution to the discrete algebraic Riccati equation. A robust
implementation should also enforce the following preconditions:
\begin{enumerate}
  \item $\mat{Q} = \mat{Q}\T \geq \mat{0}$ and $\mat{R} = \mat{R}\T > \mat{0}$.
  \item $(\mat{A}, \mat{B})$ is a stabilizable pair (see subsection
    \ref{subsec:stabilizability}).
  \item $(\mat{A}, \mat{C})$ is a detectable pair where
    $\mat{Q} = \mat{C}\T\mat{C}$ (see section \ref{subsec:detectability}).
\end{enumerate}
\begin{coderemote}{cpp}{snippets/DARE_dense.cpp}
  \caption{Dense discrete algebraic Riccati equation solver in C++}
  \label{lst:dare_dense}
\end{coderemote}
\begin{coderemote}{cpp}{snippets/DARE_sparse.cpp}
  \caption{Sparse discrete algebraic Riccati equation solver in C++}
  \label{lst:dare_sparse}
\end{coderemote}

\subsection{Continuous Lyapunov equation}
\index{Lyapunov equation!continuous}

The continuous Lyapunov equation appears in controllability/observability
analysis of continuous time systems.
\begin{equation}
  \mat{A}\mat{X} + \mat{X}\mat{A}\T + \mat{Q} = 0
\end{equation}

\subsection{Discrete Lyapunov equation}
\index{Lyapunov equation!discrete}

The discrete Lyapunov equation appears in controllability/observability analysis
of discrete time systems.
\begin{equation}
  \mat{A}\mat{X}\mat{A}\T - \mat{X} + \mat{Q} = 0
\end{equation}
