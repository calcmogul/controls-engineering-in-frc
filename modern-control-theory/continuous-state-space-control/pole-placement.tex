\section{Pole placement}
\index{controller design!pole placement}

This is the practice of placing the poles of a closed-loop \gls{system} directly
to produce a desired response. Python Control offers several pole placement
algorithms for generating controller or observer gains from a set of poles.

Since all our applications will be discrete \glspl{system}, we'll place poles in
the discrete domain (the z-plane). The s-plane's LHP maps to the inside of a
unit circle (see figure \ref{fig:s2z_mapping_pp}).
\begin{bookfigure}
  \begin{minisvg}{2}{build/modern-control-theory/discrete-state-space-control/s_plane}
  \end{minisvg}
  \hfill
  \begin{minisvg}{2}{build/modern-control-theory/discrete-state-space-control/z_plane}
  \end{minisvg}
  \caption{Mapping of axes from s-plane (left) to z-plane (right)}
  \label{fig:s2z_mapping_pp}
\end{bookfigure}

Pole placement should only be used if you know what you're doing. It's much
easier to let LQR place the poles for you, which we'll discuss next.
