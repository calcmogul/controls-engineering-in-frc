\section{Matrix exponential}
\index{discretization!matrix exponential}

The matrix exponential (and \gls{system} \gls{discretization} in general) is
typically solved with a computer. Python Control's \texttt{StateSpace.sample()}
with the ``zoh" method (the default) does this.
\begin{definition}[Matrix exponential]
  Let $\mat{X}$ be an $n \times n$ matrix. The exponential of $\mat{X}$ denoted
  by $e^{\mat{X}}$ is the $n \times n$ matrix given by the following power
  series.
  \begin{equation}
    e^{\mat{X}} = \sum_{k=0}^\infty \frac{1}{k!} \mat{X}^k \label{eq:mat_exp}
  \end{equation}

  where $\mat{X}^0$ is defined to be the identity matrix $\mat{I}$ with the same
  dimensions as $\mat{X}$.
\end{definition}

To understand why the matrix exponential is used in the \gls{discretization}
process, consider the set of differential equations
$\dot{\mat{x}} = \mat{A}\mat{x}$ we use to describe \glspl{system}
(\glspl{system} also have a $\mat{B}\mat{u}$ term, but we'll ignore it for
clarity). The solution to this type of differential equation uses an
exponential. Since we are using matrices and vectors here, we use the matrix
exponential.
\begin{equation*}
  \mat{x}(t) = e^{\mat{A}t} \mat{x}_0
\end{equation*}

where $\mat{x}_0$ contains the initial conditions. If the initial \gls{state} is
the current system \gls{state}, then we can describe the \gls{system}'s
\gls{state} over time as
\begin{equation*}
  \mat{x}_{k+1} = e^{\mat{A}T} \mat{x}_k
\end{equation*}

where $T$ is the time between samples $\mat{x}_k$ and $\mat{x}_{k+1}$.
