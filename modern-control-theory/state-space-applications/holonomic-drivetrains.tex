\section{Holonomic drivetrains}
\index{FRC models!holonomic drivetrains}

\subsection{Model}

Holonomic drivetrains have three degrees of freedom: $x$, $y$, and heading. They
are described by the following kinematics.
\begin{equation*}
  \dot{\begin{bmatrix}
    x \\
    y \\
    \theta
  \end{bmatrix}} =
  \begin{bmatrix}
    \cos\theta & -\sin\theta & 0 \\
    \sin\theta & \cos\theta & 0 \\
    0 & 0 & 1
  \end{bmatrix}
  \begin{bmatrix}
    v_{x,chassis} \\
    v_{y,chassis} \\
    \omega_{chassis}
  \end{bmatrix}
\end{equation*}

where $v_{x,chassis}$ is the velocity ahead in the chassis frame,
$v_{y,chassis}$ is the velocity to the left in the chassis frame, and
$\omega_{chassis}$ is the angular velocity in the chassis frame. This can be
written in state-space notation as
\begin{equation*}
  \dot{\begin{bmatrix}
    x \\
    y \\
    \theta
  \end{bmatrix}} =
  \begin{bmatrix}
    0 & 0 & 0 \\
    0 & 0 & 0 \\
    0 & 0 & 0
  \end{bmatrix}
  \begin{bmatrix}
    x \\
    y \\
    \theta
  \end{bmatrix} +
  \begin{bmatrix}
    \cos\theta & -\sin\theta & 0 \\
    \sin\theta & \cos\theta & 0 \\
    0 & 0 & 1
  \end{bmatrix}
  \begin{bmatrix}
    v_{x,chassis} \\
    v_{y,chassis} \\
    \omega_{chassis}
  \end{bmatrix}
\end{equation*}

\subsection{Control}

This control-affine model is fully actuated but nonlinear in the chassis frame.
However, we can apply linear control theory to the error dynamics in the global
frame instead. Note how equation \eqref{eq:holonomic_eq1}'s state vector
contains the global pose and its input vector contains the global linear and
angular velocities.
\begin{align}
  &\dot{\begin{bmatrix}
    x \\
    y \\
    \theta
  \end{bmatrix}} =
  \begin{bmatrix}
    0 & 0 & 0 \\
    0 & 0 & 0 \\
    0 & 0 & 0
  \end{bmatrix}
  \begin{bmatrix}
    x \\
    y \\
    \theta
  \end{bmatrix} +
  \begin{bmatrix}
    1 & 0 & 0 \\
    0 & 1 & 0 \\
    0 & 0 & 1
  \end{bmatrix}
  \begin{bmatrix}
    v_{x,global} \\
    v_{y,global} \\
    \omega_{global}
  \end{bmatrix}
  \label{eq:holonomic_eq1} \\
  \text{where}
  &\begin{bmatrix}
    v_{x,global} \\
    v_{y,global} \\
    \omega_{global}
  \end{bmatrix} =
  \begin{bmatrix}
    \cos\theta & -\sin\theta & 0 \\
    \sin\theta & \cos\theta & 0 \\
    0 & 0 & 1
  \end{bmatrix}
  \begin{bmatrix}
    v_{x,chassis} \\
    v_{y,chassis} \\
    \omega_{chassis}
  \end{bmatrix}
  \label{eq:holonomic_eq2}
\end{align}

We can control the model in equation \eqref{eq:holonomic_eq1} with an LQR, which
will have three independent proportional controllers. Then, we can convert the
global velocity commands to chassis velocity commands with equation
\eqref{eq:holonomic_eq2} and convert the chassis velocity commands to wheel
speed commands with inverse kinematics. LQRs on each wheel can track the wheel
speed commands.

Note that the full control law is nonlinear because the kinematics contain a
rotation matrix for transforming from the chassis frame to the global frame.
However, the nonlinear part has been abstracted away from the tunable linear
control laws.

\subsection{Holonomic vs nonholonomic control}

Drivetrains that are unable to exercise all possible degrees of freedom (e.g.,
moving sideways with respect to the chassis) are nonholonomic. An LQR on each
degree of freedom is ideal for holonomic drivetrains, but not for nonholonomic.
Section \ref{sec:ss_model_differential_drive} will use the differential drive as
a motivating example for various nonholonomic controllers.
\begin{remark}
  Nonholonomic controllers should not be used for holonomic drivetrains. They
  make different assumptions about the drivetrain dynamics that yield suboptimal
  results compared to holonomic controllers.
\end{remark}
