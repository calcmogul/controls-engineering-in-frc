\section{Linear time-varying unicycle controller (cascaded)}

One can also create a linear time-varying controller with a cascaded control
architecture like Ramsete. This section will derive a locally optimal
replacement for Ramsete.

The change in global pose for a unicycle is defined by the following three
equations.

\begin{align*}
  \dot{x} &= v\cos\theta \\
  \dot{y} &= v\sin\theta \\
  \dot{\theta} &= \omega
\end{align*}

Here's the model as a vector function where
$\mtx{x} = \begin{bmatrix} x & y & \theta \end{bmatrix}^T$ and
$\mtx{u} = \begin{bmatrix} v & \omega \end{bmatrix}^T$.

\begin{equation}
  f(\mtx{x}, \mtx{u}) =
  \begin{bmatrix}
    v\cos\theta \\
    v\sin\theta \\
    \omega
  \end{bmatrix}
\end{equation}

To create an LQR, we need to linearize this.

\begin{align*}
  \frac{\partial f(\mtx{x}, \mtx{u})}{\partial\mtx{x}} &=
  \begin{bmatrix}
    0 & 0 & -v\sin\theta \\
    0 & 0 & v\cos\theta \\
    0 & 0 & 0
  \end{bmatrix} \\
  \frac{\partial f(\mtx{x}, \mtx{u})}{\partial\mtx{u}} &=
  \begin{bmatrix}
    \cos\theta & 0 \\
    \sin\theta & 0 \\
    0 & 1
  \end{bmatrix}
\end{align*}

Therefore,

\begin{theorem}[Linear time-varying unicycle state-space model]
  \label{thm:ltv_unicycle_model}
  \begin{equation*}
    \begin{array}{ccc}
      \mtx{x} =
      \begin{bmatrix}
        x \\
        y \\
        \theta \\
      \end{bmatrix} &
      \mtx{y} =
      \begin{bmatrix}
        \theta
      \end{bmatrix} &
      \mtx{u} =
      \begin{bmatrix}
        v \\
        \omega
      \end{bmatrix}
    \end{array}
  \end{equation*}

  \begin{equation}
    \begin{array}{ll}
      \mtx{A} =
      \begin{bmatrix}
        0 & 0 & -v\sin\theta \\
        0 & 0 & v\cos\theta \\
        0 & 0 & 0
      \end{bmatrix} &
      \mtx{B} =
      \begin{bmatrix}
        \cos\theta & 0 \\
        \sin\theta & 0 \\
        0 & 1
      \end{bmatrix} \\
      \mtx{C} =
      \begin{bmatrix}
        0 & 0 & 1
      \end{bmatrix} &
      \mtx{D} = \mtx{0}_{1 \times 2}
    \end{array}
  \end{equation}

  The LQR gain should be recomputed around the current operating point regularly
  due to the high nonlinearity of this system. A less computationally expensive
  controller will be developed next.
\end{theorem}

\subsection{Explicit time-varying control law}

As with the LTV differential drive, we will fit a vector function of the states
to the LTV unicycle controller gains. Fortunately, the formulation of the
controller dealing with cross-track error only requires linearization around
different values of $v$ rather than $v$ and $\theta$; only a two-dimensional
function is needed rather than a three-dimensional one. See figures
\ref{fig:ltv_unicycle_cascaded_lqr_0} through
\ref{fig:ltv_unicycle_cascaded_lqr_2} for plots of each controller gain for a
range of velocities.

\begin{bookfigure}
  \begin{minisvg}{2}{build/code/ltv_unicycle_cascaded_lqr_0}
    \caption{Linear time-varying unicycle controller cascaded LQR gain
      regression ($x$)}
    \label{fig:ltv_unicycle_cascaded_lqr_0}
  \end{minisvg}
  \hfill
  \begin{minisvg}{2}{build/code/ltv_unicycle_cascaded_lqr_1}
    \caption{Linear time-varying unicycle controller cascaded LQR gain
      regression ($y$)}
    \label{fig:ltv_unicycle_cascaded_lqr_1}
  \end{minisvg}
  \hfill
  \begin{minisvg}{2}{build/code/ltv_unicycle_cascaded_lqr_2}
    \caption{Linear time-varying unicycle controller cascaded LQR gain
      regression ($\theta$)}
    \label{fig:ltv_unicycle_cascaded_lqr_2}
  \end{minisvg}
\end{bookfigure}

With the exception of the $x$ gain plot, all functions are a variation of a
square root. $v = 0$ and $v = 1$ were used to scale each function to produce a
close approximation.

\begin{theorem}[Linear time-varying unicycle controller]
  The following $\mtx{A}$ and $\mtx{B}$ matrices are used to compute the LQR.

  \begin{equation}
    \begin{array}{ll}
      \mtx{A} =
      \begin{bmatrix}
        0 & 0 & 0 \\
        0 & 0 & v \\
        0 & 0 & 0
      \end{bmatrix} &
      \mtx{x} =
      \begin{bmatrix}
        x \\
        y \\
        \theta
      \end{bmatrix} \\
      \mtx{B} =
      \begin{bmatrix}
        1 & 0 \\
        0 & 0 \\
        0 & 1
      \end{bmatrix} &
      \mtx{u} =
      \begin{bmatrix}
        v \\
        \omega
      \end{bmatrix}
    \end{array}
  \end{equation}

  The locally optimal controller for this model is
  $\mtx{u} = \mtx{K}(v) (\mtx{r} - \mtx{x})$ where

  \begin{align}
    \mtx{K}(v) &= \begin{bmatrix}
      k_x & 0 & 0 \\
      0 & k_y(v) \sign(v) & k_{\theta} \sqrt{|v|}
    \end{bmatrix} \\
    k_y(v) &= k_{y, 0} + (k_{y, 1} - k_{y, 0}) \sqrt{|v|} \\
    \sign(v) &= \begin{cases}
      -1 & v < 0 \\
      0 & v = 0 \\
      1 & v > 0
    \end{cases}
  \end{align}

  Using $\mtx{K}$ computed via LQR at $v = 0$
  \begin{figurekey}
    \begin{tabular}{lllll}
      $k_x = \mtx{K}_{1, 1}$ & $k_{y, 0} = \mtx{K}_{2, 2}$
    \end{tabular}
  \end{figurekey}

  Using $\mtx{K}$ computed via LQR at $v = 1$
  \begin{figurekey}
    \begin{tabular}{lll}
        $k_{y, 1} = \mtx{K}_{2, 2}$ & $k_{\theta} = \mtx{K}_{2, 3}$
    \end{tabular}
  \end{figurekey}
\end{theorem}
