\section{Effects of discretization on controller performance}

Running a feedback controller at a faster update rate doesn't always mean better
control. In fact, you may be using more computational resources than you need.
However, here are some reasons for running at a faster update rate.

Firstly, if you have a discrete \gls{model} of the \gls{system}, that
\gls{model} can more accurately approximate the underlying continuous
\gls{system}. Not all controllers use a \gls{model} though.

Secondly, the controller can better handle fast \gls{system} dynamics. If the
\gls{system} can move from its initial state to the desired one in under $250$
ms, you obviously want to run the controller with a period less than $250$ ms.
When you reduce the sample period, you're making the discrete controller more
accurately reflect what the equivalent continuous controller would do
(controllers built from analog circuit components like op-amps are continuous).

Running at a lower sample rate only causes problems if you don't take into
account the response time of your \gls{system}. Some \glspl{system} like heaters
have \glspl{output} that change on the order of minutes. Running a control loop
at $1$ kHz doesn't make sense for this because the \gls{plant} \gls{input} the
controller computes won't change much, if at all, in $1$ ms.

Figures \ref{fig:sampling_simulation_0.1}, \ref{fig:sampling_simulation_0.05},
and \ref{fig:sampling_simulation_0.01} show simulations of the same controller
for different sampling methods and sample rates, which have varying levels of
fidelity to the real \gls{system}.
\begin{bookfigure}
  \begin{minisvg}{2}{build/\chapterpath/sampling_simulation_010}
    \caption{Sampling methods for system simulation with $T = 0.1$ s}
    \label{fig:sampling_simulation_0.1}
  \end{minisvg}
  \hfill
  \begin{minisvg}{2}{build/\chapterpath/sampling_simulation_005}
    \caption{Sampling methods for system simulation with $T = 0.05$ s}
    \label{fig:sampling_simulation_0.05}
  \end{minisvg}
  \hfill
  \begin{minisvg}{2}{build/\chapterpath/sampling_simulation_004}
    \caption{Sampling methods for system simulation with $T = 0.01$ s}
    \label{fig:sampling_simulation_0.01}
  \end{minisvg}
\end{bookfigure}

Forward Euler is numerically unstable for low sample rates. The bilinear
transform is a significant improvement due to it being a second-order
approximation, but zero-order hold performs best due to the matrix exponential
including much higher orders (we'll cover the matrix exponential in the next
section).
