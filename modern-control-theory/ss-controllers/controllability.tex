\section{Controllability}
\index{controller design!controllability}

A \gls{system} is controllable if it can be steered from any \gls{state} to any
\gls{state} by a finite sequence of admissible \glspl{input}.
\begin{theorem}[Controllability]
  A continuous \gls{time-invariant} linear state-space \gls{model} is
  controllable if and only if
  \begin{equation}
    \rank\left(
    \begin{bmatrix}
      \mat{B} & \mat{A}\mat{B} & \cdots & \mat{A}^{n-1}\mat{B}
    \end{bmatrix}
    \right) = n
    \label{eq:ctrl_rank}
  \end{equation}

  where rank is the number of linearly independent rows in a matrix and $n$ is
  the number of \gls{state} variables.
\end{theorem}

The matrix in equation \eqref{eq:ctrl_rank} being rank-deficient means the
\glspl{input} cannot apply transforms along all axes in the state-space; the
transformation the matrix represents is collapsed into a lower dimension.

The condition number of the controllability matrix $\mathcal{C}$ is defined as
$\frac{\sigma_{max}(\mathcal{C})}{\sigma_{min}(\mathcal{C})}$ where
$\sigma_{max}$ is the maximum singular
value\footnote{\label{footn:singular_val}Singular values are a generalization of
eigenvalues for nonsquare matrices.} and $\sigma_{min}$ is the minimum singular
value. As this number approaches infinity, one or more of the \glspl{state}
becomes uncontrollable. This number can also be used to tell us which actuators
are better than others for the given \gls{system}; a lower condition number
means that the actuators have more control authority.
