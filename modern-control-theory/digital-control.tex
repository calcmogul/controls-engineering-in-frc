\chapterimage{digital-control.jpg}{Chaparral by Merril Apartments at UCSC}

\chapter{Digital control} \label{ch:digital_control}

The complex plane discussed so far deals with continuous \glspl{system}. In
decades past, \glspl{plant} and controllers were implemented using analog
electronics, which are continuous in nature. Nowadays, microprocessors can be
used to achieve cheaper, less complex controller designs. \Gls{discretization}
converts the continuous \gls{model} we've worked with so far from a differential
equation like
\begin{equation}
  \dot{x} = x - 3 \label{eq:differential_equ_example}
\end{equation}

to a difference equation like
\begin{align}
  \frac{x_{k+1} - x_k}{\Delta T} = x_k - 3 \nonumber \\
  x_{k+1} - x_k = (x_k - 3) \Delta T \nonumber \\
  x_{k+1} = x_k + (x_k - 3) \Delta T \label{eq:difference_equ_example}
\end{align}

where $x_k$ refers to the value of $x$ at the $k^{th}$ timestep. The difference
equation is run with some update period denoted by $T$, by $\Delta T$, or
sometimes sloppily by $dt$\footnote{The discretization of equation
\eqref{eq:differential_equ_example} to equation
\eqref{eq:difference_equ_example} uses the forward Euler discretization
method.}.

While higher order terms of a differential equation are derivatives of the
\gls{state} variable (e.g., $\ddot{x}$ in relation to equation
\eqref{eq:differential_equ_example}), higher order terms of a difference
equation are delayed copies of the \gls{state} variable (e.g., $x_{k-1}$ with
respect to $x_k$ in equation \eqref{eq:difference_equ_example}).

\renewcommand*{\chapterpath}{\partpath/digital-control}
\section{Phase loss}

However, \gls{discretization} has drawbacks. Since a microcontroller performs
discrete steps, there is a sample delay that introduces phase loss in the
controller. Phase loss is the reduction of \gls{phase margin}\footnote{See
section \ref{sec:gain_phase_margin} for an explanation of phase margin.} that
occurs in digital implementations of feedback controllers from sampling the
continuous \gls{system} at discrete time intervals. As the sample rate of the
controller decreases, the \gls{phase margin} decreases according to
$-\frac{T}{2}\omega$ where $T$ is the sample period and $\omega$ is the
frequency of the \gls{system} dynamics. Instability occurs if the
\gls{phase margin} of the \gls{system} reaches zero. Large amounts of phase loss
can make a stable controller in the continuous domain become unstable in the
discrete domain. Here are a few ways to combat this.

\begin{itemize}
  \item Run the controller with a high sample rate.
  \item Designing the controller in the analog domain with enough
    \gls{phase margin} to compensate for any phase loss that occurs as part of
    \gls{discretization}.
  \item Convert the \gls{plant} to the digital domain and design the controller
    completely in the digital domain.
\end{itemize}

\section{s-plane to z-plane}

Transfer functions are converted to impulse responses using the Z-transform. The
s-plane's LHP maps to the inside of a unit circle in the z-plane. Table
\ref{tab:s2z_mapping} contains a few common points and figure
\ref{fig:s2z_mapping} shows the mapping visually.
\begin{booktable}
  \begin{tabular}{|cc|}
    \hline
    \rowcolor{headingbg}
    \textbf{s-plane} & \textbf{z-plane} \\
    \hline
    $(0, 0)$ & $(1, 0)$ \\
    imaginary axis & edge of unit circle \\
    $(-\infty, 0)$ & $(0, 0)$ \\
    \hline
  \end{tabular}
  \caption{Mapping from s-plane to z-plane}
  \label{tab:s2z_mapping}
\end{booktable}
\begin{bookfigure}
  \begin{minisvg}{2}{build/figs/s_plane}
  \end{minisvg}
  \hfill
  \begin{minisvg}{2}{build/figs/z_plane}
  \end{minisvg}
  \caption{Mapping of complex plane from s-plane (left) to z-plane (right)}
  \label{fig:s2z_mapping}
\end{bookfigure}

\subsection{Discrete system stability}

Eigenvalues of a \gls{system} that are within the unit circle are stable. To
demonstrate this, consider the discrete system $x_{k + 1} = ax_k$ where $a$ is a
complex number. $|a| < 1$ will make $x_{k + 1}$ converge to zero.

\subsection{Discrete system behavior}

Figure \ref{fig:disc_impulse_response_poles} shows the \glspl{impulse response}
in the time domain for \glspl{system} with various pole locations in the complex
plane (real numbers on the x-axis and imaginary numbers on the y-axis). Each
response has an initial condition of $1$.
\begin{bookfigure}
  \begin{tikzpicture}[auto, >=latex']
  % \draw [help lines] (-4,-4) grid (4,4);

  % Draw main axes
  \draw[->] (-4,0) -- (4,0) node[below] {\small Re};
  \draw[->] (0,-4) -- (0,4) node[right] {\small Im};

  % Unit circle
  \draw[black] (0,0) circle (3cm);

  % Exponent for the given x plot coordinate:
  %
  %   exp(at) = x
  %   at = ln(x)
  %   a = ln(x)/t
  %
  % t = 50 ms

  % LHP integrator
  \drawdiscretecoordplot{-3cm - 0.166cm}{-0.75cm}{0.125cm}{0.44375cm}
    {(0,1) (0.05,-1) (0.1,1) (0.15,-1) (0.2,1) (0.25,-1) (0.3,1) (0.35,-1)
     (0.4,1) (0.45,-1) (0.5,1)}
  \drawpole{-3cm}{0cm}

  % LHP stable: xₖ₊₁ = −2/3xₖ
  \drawdiscretecoordplot{-2cm}{0.75cm}{0.125cm}{0.44375cm}
    {(0,1) (0.05,-0.667) (0.1,0.444) (0.15,-0.296) (0.2,0.198) (0.25,-0.132)
     (0.3,0.088) (0.35,-0.059) (0.4,0.039) (0.45,-0.026) (0.5,0.017)}
  \drawpole{-2cm}{0cm}

  % LHP stable: xₖ₊₁ = −1/3xₖ
  \drawdiscretecoordplot{-1cm + 0.166cm}{-0.75cm}{0.125cm}{0.44375cm}
    {(0,1) (0.05,-0.333) (0.1,0.111) (0.15,-0.037) (0.2,0.012) (0.25,-0.004)
     (0.3,0.001) (0.35,0) (0.4,0) (0.45,0) (0.5,0)}
  \drawpole{-1cm}{0cm}

  % LHP stable: xₖ₊₁ = (0.333 + 0.5i)xₖ
  \drawdiscretecoordplot{-1cm}{2.25cm}{0.125cm}{0.44375cm}
    {(0,1) (0.05,-0.333) (0.1,-0.139) (0.15,0.213) (0.2,-0.092) (0.25,-0.016)
     (0.3,0.044) (0.35,-0.023) (0.4,0) (0.45,0.009) (0.5,-0.006)}
  \drawpole{-1cm}{1.5cm}

  % LHP unstable: xₖ₊₁ = (−1 − 0.75i)xₖ
  \drawdiscretecoordplot{-2.25cm}{-2.25cm}{0.125cm}{0.44375cm}
    {(0,1) (0.05,-1) (0.1,0.4375) (0.15,0.6875) (0.2,-2.059) (0.25,3.043)
     (0.3,-2.869) (0.35,0.984) (0.4,2.515) (0.45,-6.568) (0.5,9.206)}
  \drawpole{-3cm}{-2.25cm}

  % RHP stable: exp(-10.192t) cos(19.665wt) (40/3*w for readability)
  %
  % a = ln(0.333 + 0.5i)/0.05 = -10.192 + 19.665i
  \drawdiscretetimeplot{1cm}{2.25cm}{0.125cm}{0.44375cm}{
    exp(-10.192 * \x) * cos(40/3 * 19.665 * deg(\x))}
  \drawpole{1cm}{1.5cm}

  % RHP stable: exp(-21.97t)
  %
  % a = ln(1/3)/0.05 = -21.97
  \drawdiscretetimeplot{1cm - 0.166cm}{-0.75cm}{0.125cm}{0.125cm}{exp(-21.97 * \x)}
  \drawpole{1cm}{0cm}

  % RHP stable: exp(-8.109t)
  %
  % a = ln(2/3)/0.05 = -8.109
  \drawdiscretetimeplot{2cm}{0.75cm}{0.125cm}{0.125cm}{exp(-8.109 * \x)}
  \drawpole{2cm}{0cm}

  % RHP marginally stable: cos(15.707wt)
  %
  % a = ln(√2 + √2i)/0.05 = 15.707i
  \drawdiscretetimeplot{3 * (0.707cm + 0.25cm)}{3 * 0.707cm + 0.375cm}{0.125cm}{0.44375cm}{cos(15.707 * deg(\x))}
  \drawpole{3 * 0.707cm}{3 * 0.707cm}

  % RHP integrator
  \drawdiscretetimeplot{3cm + 0.166cm}{-0.75cm}{0.125cm}{0.125cm}{exp(0 * \x)}
  \drawpole{3cm}{0cm}

  % RHP unstable: exp(4.463t) cos(-12.87wt)
  %
  % a = ln(1 - 0.75i)/0.05 = 4.463 - 12.87i
  \drawdiscretetimeplot{2.25cm}{-2.25cm}{0.125cm}{0.44375cm}{exp(4.463 * \x) * cos(-12.87 * deg(\x))}
  \drawpole{3cm}{-2.25cm}

  % LHP and RHP labels
  \draw (-3.5,1.5) node {LHP};
  \draw (3.5,1.5) node {RHP};

  % Stable and unstable labels
  \fill[white] (-0.5,-2.25) rectangle (0.5,-1.75);
  \draw (0,-2) node {\small Stable};
  \draw (2.5,3.5) node {\small Unstable};
  \draw (-2.5,3.5) node {\small Unstable};
  \draw (-2.5,-3.5) node {\small Unstable};
  \draw (2.5,-3.5) node {\small Unstable};
\end{tikzpicture}

  \caption{Discrete impulse response vs pole location}
  \label{fig:disc_impulse_response_poles}
\end{bookfigure}

As $\omega$ increases in $s = j\omega$, a pole in the z-plane moves around the
perimeter of the unit circle. Once it hits $\frac{\omega_s}{2}$ (half the
sampling frequency) at $(-1, 0)$, the pole wraps around. This is due to poles
faster than the sample frequency folding down to below the sample frequency
(that is, higher frequency signals \textit{alias} to lower frequency ones).

Placing the poles at $(0, 0)$ produces a \textit{deadbeat controller}. An
$\rm N^{th}$-order deadbeat controller decays to the \gls{reference} in N
timesteps. While this sounds great, there are other considerations like
\gls{control effort}, \gls{robustness}, and \gls{noise immunity}.

Poles in the left half-plane cause jagged outputs because the frequency of the
\gls{system} dynamics is above the Nyquist frequency (twice the sample
frequency). The \glslink{discretization}{discretized} signal doesn't have enough
samples to reconstruct the continuous \gls{system}'s dynamics. See figures
\ref{fig:z_oscillations_1p} and \ref{fig:z_oscillations_2p} for examples.
\begin{bookfigure}
  \begin{minisvg}{2}{build/\chapterpath/z_oscillations_1p}
    \caption{Single poles in various locations in z-plane}
    \label{fig:z_oscillations_1p}
  \end{minisvg}
  \hfill
  \begin{minisvg}{2}{build/\chapterpath/z_oscillations_2p}
    \caption{Complex conjugate poles in various locations in z-plane}
    \label{fig:z_oscillations_2p}
  \end{minisvg}
\end{bookfigure}

\section{Discretization methods}

\Gls{discretization} is done using a zero-order hold. That is, the \gls{system}
\gls{state} is only updated at discrete intervals and it's held constant between
samples (see figure \ref{fig:zoh}). The exact method of applying this uses the
matrix exponential, but this can be computationally expensive. Instead,
approximations such as the following are used.
\begin{enumerate}
  \item Forward Euler method. This is defined as
    $y_{n+1} = y_n + f(t_n, y_n) \Delta t$.
    \index{discretization!forward Euler method}
  \item Backward Euler method. This is defined as
    $y_{n+1} = y_n + f(t_{n+1}, y_{n+1}) \Delta t$.
    \index{discretization!backward Euler method}
  \item Bilinear transform. The first-order bilinear approximation is
    $s = \frac{2}{T} \frac{1 - z^{-1}}{1 + z^{-1}}$.
    \index{discretization!bilinear transform}
\end{enumerate}

where the function $f(t_n, y_n)$ is the slope of $y$ at $n$ and $T$ is the
sample period for the discrete \gls{system}. Each of these methods is
essentially finding the area underneath a curve. The forward and backward Euler
methods use rectangles to approximate that area while the bilinear transform
uses trapezoids (see figures \ref{fig:discretization_methods_vel} and
\ref{fig:discretization_methods_pos}). Since these are approximations, there is
distortion between the real discrete \gls{system}'s poles and the approximate
poles. This is in addition to the phase loss introduced by discretizing at a
given sample rate in the first place. For fast-changing \glspl{system}, this
distortion can quickly lead to instability.
\begin{bookfigure}
  \begin{minisvg}{2}{build/\chapterpath/zoh}
      \caption{Zero-order hold of a system response}
      \label{fig:zoh}
  \end{minisvg}
  \hfill
  \begin{minisvg}{2}{build/\chapterpath/discretization_methods_vel}
    \caption{Discretization methods applied to velocity data}
    \label{fig:discretization_methods_vel}
  \end{minisvg}
  \hfill
  \begin{minisvg}{2}{build/\chapterpath/discretization_methods_pos}
    \caption{Position plot of discretization methods applied to velocity data}
    \label{fig:discretization_methods_pos}
  \end{minisvg}
\end{bookfigure}

Figures \ref{fig:sampling_simulation_0.1}, \ref{fig:sampling_simulation_0.05},
and \ref{fig:sampling_simulation_0.01} show simulations of the same controller
for different sampling methods and sample rates, which have varying levels of
fidelity to the real \gls{system}.
\begin{bookfigure}
  \begin{minisvg}{2}{build/\chapterpath/sampling_simulation_010}
    \caption{Sampling methods for system simulation with $T = 0.1$ s}
    \label{fig:sampling_simulation_0.1}
  \end{minisvg}
  \hfill
  \begin{minisvg}{2}{build/\chapterpath/sampling_simulation_005}
    \caption{Sampling methods for system simulation with $T = 0.05$ s}
    \label{fig:sampling_simulation_0.05}
  \end{minisvg}
  \hfill
  \begin{minisvg}{2}{build/\chapterpath/sampling_simulation_004}
    \caption{Sampling methods for system simulation with $T = 0.01$ s}
    \label{fig:sampling_simulation_0.01}
  \end{minisvg}
\end{bookfigure}

Forward Euler is numerically unstable for low sample rates. The bilinear
transform is a significant improvement due to it being a second-order
approximation, but zero-order hold performs best due to the matrix exponential
including much higher orders.

Table \ref{tab:disc_approx_scalar} compares the Taylor series expansions of
several common discretization methods (these are found using polynomial
division). The bilinear transform does best with accuracy trailing off after the
third-order term. Forward Euler has no second-order or higher terms, so it
undershoots. Backward Euler has twice the second-order term and overshoots the
remaining higher order terms as well.
\begin{booktable}
  \begin{tabular}{|cll|}
    \hline
    \rowcolor{headingbg}
    \multicolumn{1}{|c}{\textbf{Method}} &
      \multicolumn{1}{c}{\textbf{Conversion to z}} &
      \multicolumn{1}{c|}{\textbf{Taylor series expansion}} \\
    \hline
    Zero-order hold &
      $e^{Ts}$ &
      $1 + Ts + \frac{1}{2}T^2s^2 + \frac{1}{6}T^3s^3 + \ldots$ \\
    Bilinear &
      $\frac{1 + \frac{1}{2}Ts}{1 - \frac{1}{2}Ts}$ &
      $1 + Ts + \frac{1}{2}T^2s^2 + \frac{1}{4}T^3s^3 + \ldots$ \\
    Forward Euler &
      $1 + Ts$ &
      $1 + Ts$ \\
    Backward Euler &
      $\frac{1}{1 - Ts}$ &
      $1 + Ts + T^2s^2 + T^3s^3 + \ldots$ \\
    \hline
  \end{tabular}
  \caption{Taylor series expansions of discretization methods (scalar case). The
    zero-order hold discretization method is exact.}
  \label{tab:disc_approx_scalar}
\end{booktable}

\section{Effects of discretization on controller performance}

Running a feedback controller at a faster update rate doesn't always mean better
control. In fact, you may be using more computational resources than you need.
However, here are some reasons for running at a faster update rate.

Firstly, if you have a discrete \gls{model} of the \gls{system}, that
\gls{model} can more accurately approximate the underlying continuous
\gls{system}. Not all controllers use a \gls{model} though.

Secondly, the controller can better handle fast \gls{system} dynamics. If the
\gls{system} can move from its initial state to the desired one in under 250ms,
you obviously want to run the controller with a period less than 250ms. When you
reduce the sample period, you're making the discrete controller more accurately
reflect what the equivalent continuous controller would do (controllers built
from analog circuit components like op-amps are continuous).

Running at a lower sample rate only causes problems if you don't take into
account the response time of your \gls{system}. Some \glspl{system} like heaters
have \glspl{output} that change on the order of minutes. Running a control loop
at 1kHz doesn't make sense for this because the \gls{plant} \gls{input} the
controller computes won't change much, if at all, in 1ms.

Figures \ref{fig:sampling_simulation_0.1}, \ref{fig:sampling_simulation_0.05},
and \ref{fig:sampling_simulation_0.01} show simulations of the same controller
for different sampling methods and sample rates, which have varying levels of
fidelity to the real \gls{system}.
\begin{bookfigure}
  \begin{minisvg}{2}{build/\chapterpath/sampling_simulation_010}
    \caption{Sampling methods for system simulation with $T = 0.1s$}
    \label{fig:sampling_simulation_0.1}
  \end{minisvg}
  \hfill
  \begin{minisvg}{2}{build/\chapterpath/sampling_simulation_005}
    \caption{Sampling methods for system simulation with $T = 0.05s$}
    \label{fig:sampling_simulation_0.05}
  \end{minisvg}
  \hfill
  \begin{minisvg}{2}{build/\chapterpath/sampling_simulation_004}
    \caption{Sampling methods for system simulation with $T = 0.01s$}
    \label{fig:sampling_simulation_0.01}
  \end{minisvg}
\end{bookfigure}

Forward Euler is numerically unstable for low sample rates. The bilinear
transform is a significant improvement due to it being a second-order
approximation, but zero-order hold performs best due to the matrix exponential
including much higher orders (we'll cover the matrix exponential in the next
section).

Table \ref{tab:disc_approx_scalar} compares the Taylor series expansions of the
discretization methods presented so far (these are found using polynomial
division). The bilinear transform does best with accuracy trailing off after the
third-order term. Forward Euler has no second-order or higher terms, so it
undershoots. Backward Euler has twice the second-order term and overshoots the
remaining higher order terms as well.
\begin{booktable}
  \begin{tabular}{|cll|}
    \hline
    \rowcolor{headingbg}
    \multicolumn{1}{|c}{\textbf{Method}} &
      \multicolumn{1}{c}{\textbf{Conversion}} &
      \multicolumn{1}{c|}{\textbf{Taylor series expansion}} \\
    \hline
    Zero-order hold &
      $z = e^{Ts}$ &
      $z = 1 + Ts + \frac{1}{2}T^2s^2 + \frac{1}{6}T^3s^3 + \ldots$ \\
    Bilinear &
      $z = \frac{1 + \frac{1}{2}Ts}{1 - \frac{1}{2}Ts}$ &
      $z = 1 + Ts + \frac{1}{2}T^2s^2 + \frac{1}{4}T^3s^3 + \ldots$ \\
    Forward Euler &
      $z = 1 + Ts$ &
      $z = 1 + Ts$ \\
    Reverse Euler &
      $z = \frac{1}{1 - Ts}$ &
      $z = 1 + Ts + T^2s^2 + T^3s^3 + \ldots$ \\
    \hline
  \end{tabular}
  \caption{Taylor series expansions of discretization methods (scalar case). The
    zero-order hold discretization method is exact.}
  \label{tab:disc_approx_scalar}
\end{booktable}

\section{Matrix exponential}
\index{discretization!matrix exponential}

The matrix exponential (and \gls{system} \gls{discretization} in general) is
typically solved with a computer. Python Control's \texttt{StateSpace.sample()}
with the ``zoh" method (the default) does this.
\begin{definition}[Matrix exponential]
  Let $\mat{X}$ be an $n \times n$ matrix. The exponential of $\mat{X}$ denoted
  by $e^{\mat{X}}$ is the $n \times n$ matrix given by the following power
  series.
  \begin{equation}
    e^{\mat{X}} = \sum_{k=0}^\infty \frac{1}{k!} \mat{X}^k \label{eq:mat_exp}
  \end{equation}

  where $\mat{X}^0$ is defined to be the identity matrix $\mat{I}$ with the same
  dimensions as $\mat{X}$.
\end{definition}

To understand why the matrix exponential is used in the \gls{discretization}
process, consider the set of differential equations
$\dot{\mat{x}} = \mat{A}\mat{x}$ we use to describe \glspl{system}
(\glspl{system} also have a $\mat{B}\mat{u}$ term, but we'll ignore it for
clarity). The solution to this type of differential equation uses an
exponential. Since we are using matrices and vectors here, we use the matrix
exponential.
\begin{equation*}
  \mat{x}(t) = e^{\mat{A}t} \mat{x}_0
\end{equation*}

where $\mat{x}_0$ contains the initial conditions. If the initial \gls{state} is
the current system \gls{state}, then we can describe the \gls{system}'s
\gls{state} over time as
\begin{equation*}
  \mat{x}_{k+1} = e^{\mat{A}T} \mat{x}_k
\end{equation*}

where $T$ is the time between samples $\mat{x}_k$ and $\mat{x}_{k+1}$.

\input{\chapterpath/taylor-series}
\section{Zero-order hold for state-space}
\index{discretization!zero-order hold}

Given the following continuous time state space model
\begin{align*}
  \dot{\mat{x}} &= \mat{A}_c\mat{x} + \mat{B}_c\mat{u} + \mat{w} \\
  \mat{y} &= \mat{C}_c\mat{x} + \mat{D}_c\mat{u} + \mat{v}
\end{align*}

where $\mat{w}$ is the process noise, $\mat{v}$ is the measurement noise, and
both are zero-mean white noise sources with covariances of $\mat{Q}_c$ and
$\mat{R}_c$ respectively. $\mat{w}$ and $\mat{v}$ are defined as normally
distributed random variables.
\begin{align*}
  \mat{w} &\sim N(0, \mat{Q}_c) \\
  \mat{v} &\sim N(0, \mat{R}_c)
\end{align*}

The model can be \glslink{discretization}{discretized} as follows
\begin{align*}
  \mat{x}_{k+1} &= \mat{A}_d \mat{x}_k + \mat{B}_d \mat{u}_k + \mat{w}_k \\
   \mat{y}_k &= \mat{C}_d \mat{x}_k + \mat{D}_d \mat{u}_k + \mat{v}_k
\end{align*}

with covariances
\begin{align*}
  \mat{w}_k &\sim N(0, \mat{Q}_d) \\
  \mat{v}_k &\sim N(0, \mat{R}_d)
\end{align*}
\begin{theorem}[Zero-order hold for state-space]
  \label{thm:zoh_ss}
  \begin{align}
    \mat{A}_d &= e^{\mat{A}_c T} \\
    \mat{B}_d &= \int_0^T e^{\mat{A}_c \tau} d\tau \mat{B}_c =
      \mat{A}_c^{-1} (\mat{A}_d - \mat{I}) \mat{B}_c \\
    \mat{C}_d &= \mat{C}_c \\
    \mat{D}_d &= \mat{D}_c \\
    \mat{Q}_d &= \int_{\tau = 0}^{T} e^{\mat{A}_c\tau} \mat{Q}_c
      e^{\mat{A}_c^T\tau} d\tau \\
    \mat{R}_d &= \frac{1}{T}\mat{R}_c
  \end{align}

  where subscripts $c$ and $d$ denote matrices for the continuous or discrete
  \glspl{system} respectively, $T$ is the sample period of the discrete
  \gls{system}, and $e^{\mat{A}_c T}$ is the matrix exponential of
  $\mat{A}_c T$.

  $\mat{A}_d$ and $\mat{B}_d$ can be computed in one step as
  \begin{equation*}
    e^{
    \begin{bmatrix}
      \mat{A}_c & \mat{B}_c \\
      \mat{0} & \mat{0}
    \end{bmatrix}T} =
    \begin{bmatrix}
      \mat{A}_d & \mat{B}_d \\
      \mat{0} & \mat{I}
    \end{bmatrix}
  \end{equation*}

  and $\mat{Q}_d$ can be computed as
  \begin{equation*}
    \Phi = e^{
    \begin{bmatrix}
      -\mat{A}_c & \mat{Q}_c \\
      \mat{0} & \mat{A}_c^T
    \end{bmatrix}T} =
    \begin{bmatrix}
      -\mat{A}_d & \mat{A}_d^{-1} \mat{Q}_d \\
      \mat{0} & \mat{A}_d^T
    \end{bmatrix}
  \end{equation*}

  where $\mat{Q}_d = \Phi_{2,2}^T \Phi_{1,2}$ \cite{bib:integral_matrix_exp}.
\end{theorem}

See appendix \ref{sec:deriv_zoh_ss} for derivations.

To see why $\mat{R}_c$ is being divided by $T$, consider the discrete white
noise sequence $\mat{v}_k$ and the (non-physically realizable) continuous white
noise process $\mat{v}$. Whereas $\mat{R}_{d,k} = E[\mat{v}_k \mat{v}_k^T]$ is a
covariance matrix, $\mat{R}_c(t)$ defined by
$E[\mat{v}(t) \mat{v}^T(\tau)] = \mat{R}_c(t)\delta(t - \tau)$ is a spectral
density matrix (the Dirac function $\delta(t - \tau)$ has units of
$1/\text{sec}$). The covariance matrix $\mat{R}_c(t)\delta(t - \tau)$ has
infinite-valued elements. The discrete white noise sequence can be made to
approximate the continuous white noise process by shrinking the pulse lengths
($T$) and increasing their amplitude, such that
$\mat{R}_d \rightarrow \frac{1}{T}\mat{R}_c$.

That is, in the limit as $T \rightarrow 0$, the discrete noise sequence tends to
one of infinite-valued pulses of zero duration such that the area under the
``impulse" autocorrelation function is $\mat{R}_d T$. This is equal to the area
$\mat{R}_c$ under the continuous white noise impulse autocorrelation function.

\section{Numerical integration methods}

Most systems don't have linear dynamics and their differential equations can't
be solved analytically. Instead, we'll have to approximate their solutions with
numerical integration.

\subsection{Butcher tableaus}

Butcher tableaus are a more succinct representation for explicit and implicit
Runge-Kutta numerical integration methods. Here's the general structure for
explicit methods.
\begin{equation*}
  \renewcommand\arraystretch{1.2}
  \begin{array}{c|cccc}
    0 \\
    c_2    & a_{2,1} \\
    \vdots & \vdots & \ddots \\
    c_s    & a_{s,1} & \hdots & a_{s,s-1} \\
    \hline
           & b_1    & \hdots & \hdots    & b_s
  \end{array}
\end{equation*}

where $s$ is the number of stages in the method, the matrix $[a_{ij}]$ is the
Runge-Kutta matrix, $b_1, \ldots, b_s$ are the weights, and $c_1, \ldots, c_s$
are the nodes. The top-left quadrant contains the sums of the rows in the
top-right quadrant. Each column in the right half corresponds to a $\mat{k}$
coefficient from $\mat{k}_1$ to $\mat{k}_s$.

The family of solutions to $\dot{\mat{x}} = f(t, \mat{x})$ is given by
\begin{align*}
  \mat{k}_1 &= f(t_k, \mat{x}_k) \\
  \mat{k}_2 &= f(t_k + c_2 h, \mat{x}_k + h (a_{2,1} \mat{k}_1)) \\
  &\ \ \vdots \\
  \mat{k}_s &= f(t_k + c_s h, \mat{x}_k +
    h (a_{s,1} \mat{k}_1 + \ldots + a_{s,s-1} \mat{k}_{s-1})) \\
  \mat{x}_{k+1} &= \mat{x}_k + h \sum_{i=1}^s b_i \mat{k}_i
\end{align*}

where $h$ is the timestep duration.

\subsection{Forward Euler method}
\index{numerical integration!Forward Euler}

The simplest explicit Runge-Kutta integration method is forward Euler
integration. We don't recommend using it because it suffers from numerical
stability issues. We'll demonstrate how to translate its Butcher tableau into
equations that integrate $\dot{\mat{x}} = f(t, \mat{x})$ from $0$ to $h$.
\begin{center}
  \begin{minipage}{0.35\linewidth}
    \centering
    \begin{alignat*}{7}
      \mat{k}_1 &= f(t +
        && {\color{blue}0} h,
        && \mat{x}_k)
        && \\
      \mat{x}_{k+1} &=
        &&
        && \mat{x}_k + h (
        && {\color{deepgreen}1} \mat{k}_1)
    \end{alignat*}
  \end{minipage}
  \quad
  \begin{minipage}{0.35\linewidth}
    \centering
    \begin{equation*}
      \renewcommand\arraystretch{1.2}
      \begin{array}{c|c}
        {\color{blue}0} \\
        \hline
        & {\color{deepgreen}1}
      \end{array}
    \end{equation*}
  \end{minipage}
\end{center}

Remove zeroed out terms.
\begin{align*}
  \mat{k}_1 &= f(t, \mat{x}_k) \\
  \mat{x}_{k+1} &= \mat{x}_k + h \mat{k}_1
  \intertext{Simplify.}
  \mat{x}_{k+1} &= \mat{x}_k + h f(t, \mat{x}_k)
\end{align*}

In FRC, our differential equations are of the form
$\dot{\mat{x}} = f(\mat{x}, \mat{u})$ where $\mat{u}$ is held constant between
timesteps. Since it's time-invariant, we can ignore the time argument of the
integration method. This gives theorem \ref{thm:forward_euler}.
\begin{theorem}[Forward Euler integration]
  \label{thm:forward_euler}

  Given the differential equation $\dot{\mat{x}} = f(\mat{x}_k, \mat{u}_k)$,
  this method solves for $\mat{x}_{k+1}$ at $h$ seconds in the future.
  $\mat{u}$ is assumed to be held constant between timesteps.
  \begin{center}
    \begin{minipage}{0.35\linewidth}
      \centering
      \begin{equation*}
        \mat{x}_{k+1} = \mat{x}_k + h f(\mat{x}_k, \mat{u}_k)
      \end{equation*}
    \end{minipage}
    \quad
    \begin{minipage}{0.35\linewidth}
      \centering
      \begin{equation*}
        \renewcommand\arraystretch{1.2}
        \begin{array}{c|c}
          0 \\
          \hline
          & 1
        \end{array}
      \end{equation*}
    \end{minipage}
  \end{center}
\end{theorem}

\subsection{Runge-Kutta fourth-order method}
\index{numerical integration!Runge-Kutta fourth-order}

The most common method we'll cover is Runge-Kutta fourth-order (RK4). It's
simple and accurate for most systems we'll see in FRC. We'll demonstrate how to
translate its Butcher tableau into equations that integrate
$\dot{\mat{x}} = f(t, \mat{x})$ from $0$ to $h$.
\begin{center}
  \begin{minipage}{0.35\linewidth}
    \centering
    \begin{alignat*}{7}
      \mat{k}_1 &= f(t +
        && {\color{blue}0} h,
        && \mat{x}_k)
        &&
        &&
        &&
        && \\
      \mat{k}_2 &= f(t +
        && {\color{blue}\frac{1}{2}} h,
        && \mat{x}_k + h (
        && {\color{deeporange}\frac{1}{2}} \mat{k}_1))
        &&
        &&
        && \\
      \mat{k}_3 &= f(t +
        && {\color{blue}\frac{1}{2}} h,
        && \mat{x}_k + h (
        && {\color{deeporange}0} \mat{k}_1 +
        && {\color{deeporange}\frac{1}{2}} \mat{k}_2))
        &&
        && \\
      \mat{k}_4 &= f(t +
        && {\color{blue}1} h,
        && \mat{x}_k + h (
        && {\color{deeporange}0} \mat{k}_1 +
        && {\color{deeporange}0} \mat{k}_2 +
        && {\color{deeporange}1} \mat{k}_3))
        && \\
      \mat{x}_{k+1} &=
        &&
        && \mat{x}_k + h (
        && {\color{deepgreen}\frac{1}{6}} \mat{k}_1 +
        && {\color{deepgreen}\frac{1}{3}} \mat{k}_2 +
        && {\color{deepgreen}\frac{1}{3}} \mat{k}_3 +
        && {\color{deepgreen}\frac{1}{6}} \mat{k}_4)
    \end{alignat*}
  \end{minipage}
  \quad
  \begin{minipage}{0.35\linewidth}
    \centering
    \begin{equation*}
      \renewcommand\arraystretch{1.2}
      \begin{array}{c|cccc}
        {\color{blue}0} \\
        {\color{blue}\frac{1}{2}} & {\color{deeporange}\frac{1}{2}} \\
        {\color{blue}\frac{1}{2}} & {\color{deeporange}0}           & {\color{deeporange}\frac{1}{2}} \\
        {\color{blue}1}           & {\color{deeporange}0}           & {\color{deeporange}0}           & {\color{deeporange}1} \\
        \hline
                                  & {\color{deepgreen}\frac{1}{6}}  & {\color{deepgreen}\frac{1}{3}}  & {\color{deepgreen}\frac{1}{3}} & {\color{deepgreen}\frac{1}{6}}
      \end{array}
    \end{equation*}
  \end{minipage}
\end{center}

Remove zeroed out terms.
\begin{align*}
  \mat{k}_1 &= f(t, \mat{x}_k) \\
  \mat{k}_2 &= f(t + \frac{1}{2} h, \mat{x}_k + h \frac{1}{2} \mat{k}_1) \\
  \mat{k}_3 &= f(t + \frac{1}{2} h, \mat{x}_k + h \frac{1}{2} \mat{k}_2) \\
  \mat{k}_4 &= f(t + h, \mat{x}_k + h \mat{k}_3) \\
  \mat{x}_{k+1} &= \mat{x}_k + h \left(
    \frac{1}{6} \mat{k}_1 +
    \frac{1}{3} \mat{k}_2 +
    \frac{1}{3} \mat{k}_3 +
    \frac{1}{6} \mat{k}_4\right)
  \intertext{$\frac{1}{6}$ is usually factored out of the last equation to
    reduce the number of floating point operations.}
  \mat{x}_{k+1} &= \mat{x}_k + h \frac{1}{6} (
    \mat{k}_1 + 2\mat{k}_2 + 2\mat{k}_3 + \mat{k}_4)
\end{align*}

In FRC, our differential equations are of the form
$\dot{\mat{x}} = f(\mat{x}, \mat{u})$ where $\mat{u}$ is held constant between
timesteps. Since it's time-invariant, we can ignore the time argument of the
integration method. This gives theorem \ref{thm:rk4}.
\begin{theorem}[Runge-Kutta fourth-order integration]
  \label{thm:rk4}

  Given the differential equation $\dot{\mat{x}} = f(\mat{x}_k, \mat{u}_k)$,
  this method solves for $\mat{x}_{k+1}$ at $h$ seconds in the future.
  $\mat{u}$ is assumed to be held constant between timesteps.
  \begin{center}
    \begin{minipage}{0.35\linewidth}
      \centering
      \begin{align*}
        \mat{k}_1 &= f(\mat{x}_k, \mat{u}_k) \\
        \mat{k}_2 &= f(\mat{x}_k + h \frac{1}{2}\mat{k}_1, \mat{u}_k) \\
        \mat{k}_3 &= f(\mat{x}_k + h \frac{1}{2}\mat{k}_2, \mat{u}_k) \\
        \mat{k}_4 &= f(\mat{x}_k + h \mat{k}_3, \mat{u}_k) \\
        \mat{x}_{k+1} &= \mat{x}_k + h \frac{1}{6} (\mat{k}_1 + 2\mat{k}_2 +
          2\mat{k}_3 + \mat{k}_4)
      \end{align*}
    \end{minipage}
    \quad
    \begin{minipage}{0.35\linewidth}
      \centering
      \begin{equation*}
        \renewcommand\arraystretch{1.2}
        \begin{array}{c|cccc}
          0 \\
          \frac{1}{2} & \frac{1}{2} \\
          \frac{1}{2} & 0 & \frac{1}{2} \\
          1 & 0 & 0 & 1 \\
          \hline
          & \frac{1}{6} & \frac{1}{3} & \frac{1}{3} & \frac{1}{6}
        \end{array}
      \end{equation*}
    \end{minipage}
  \end{center}
\end{theorem}

Here's a reference implementation.
\begin{coderemote}{cpp}{snippets/RK4.cpp}
  \caption{RK4 implementation in C++}
\end{coderemote}

Other methods of Runge-Kutta integration exist with various properties
\cite{bib:wiki_rk4}, but the one presented here is popular for its high accuracy
relative to the amount of floating point operations (FLOPs) it requires.

\subsection{Dormand-Prince method}
\index{numerical integration!Dormand-Prince}

Dormand-Prince (RKDP) is a fourth-order method with fifth-order error checking.
It uses an adaptive stepsize to enforce an upper bound on the integration error.
\begin{theorem}[Dormand-Prince integration]
  \label{thm:rkdp}

  Given the differential equation $\dot{\mat{x}} = f(\mat{x}_k, \mat{u}_k)$,
  this method solves for $\mat{x}_{k+1}$ at $h$ seconds in the future.
  $\mat{u}$ is assumed to be held constant between timesteps. It has the
  following Butcher tableau.
  \begin{equation*}
    \renewcommand\arraystretch{1.2}
    \begin{array}{c|ccccccc}
      0 \\
      \frac{1}{5} & \frac{1}{5} \\
      \frac{3}{10} & \frac{3}{40} & \frac{9}{40} \\
      \frac{4}{5} & \frac{44}{45} & -\frac{56}{15} & \frac{32}{9} \\
      \frac{8}{9} & \frac{19372}{6561} & -\frac{25360}{2187} &
        \frac{64448}{6561} & -\frac{212}{729} \\
      1 & \frac{9017}{3168} & -\frac{355}{33} & \frac{46732}{5247} &
        \frac{49}{176} & -\frac{5103}{18656} \\
      1 & \frac{35}{384} & 0 & \frac{500}{1113} & \frac{125}{192} &
        -\frac{2187}{6784} & \frac{11}{84} \\
      \hline
      & \frac{35}{384} & 0 & \frac{500}{1113} & \frac{125}{192} &
        -\frac{2187}{6784} & \frac{11}{84} & 0 \\
      & \frac{5179}{57600} & 0 & \frac{7571}{16695} & \frac{393}{640} &
        -\frac{92097}{339200} & \frac{187}{2100} & \frac{1}{40}
    \end{array}
  \end{equation*}
  The first row of coefficients below the table divider gives the fifth-order
  accurate solution. The second row gives an alternative solution which,
  when subtracted from the first solution, gives the error estimate.
\end{theorem}

Here's a reference implementation.
\begin{coderemote}{cpp}{snippets/RKDP.cpp}
  \caption{RKDP implementation in C++}
\end{coderemote}

