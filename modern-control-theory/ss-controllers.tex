\chapterimage{ss-controllers.jpg}{Night sky above Dufour Street in Santa Cruz, CA}

\chapter{State-space controllers}
\begin{remark}
  Chapters from here on use the \texttt{frccontrol} Python package to
  demonstrate the concepts discussed and perform the complex math required. See
  appendix \ref{ch:installing_python_packages} for how to install it.
\end{remark}

When we want to command a \gls{system} to a set of \glspl{state}, we design a
controller with certain \glspl{control law} to do it. PID controllers use the
system \glspl{output} with proportional, integral, and derivative
\glspl{control law}. In state-space, we also have knowledge of the system
\glspl{state} so we can do better.

Modern control theory uses state-space representation to model and control
systems. State-space representation models \glspl{system} as a set of
\gls{state}, \gls{input}, and \gls{output} variables related by first-order
differential equations that describe how the \gls{system}'s \gls{state} changes
over time given the current \glspl{state} and \glspl{input}.

\renewcommand*{\chapterpath}{\partpath/ss-controllers}
\section{From PID control to model-based control}
\index{PID control}

As mentioned before, controls engineers have a more general framework to
describe control theory than just PID control. PID controller designers are
focused on fiddling with controller parameters relating to the current, past,
and future \gls{error} rather than the underlying system \glspl{state}. Integral
control is a commonly used tool, and some people use integral action as the
majority of the control action. While this approach works in a lot of
situations, it is an incomplete view of the world.

Model-based control has a completely different mindset. Controls designers using
model-based control care about developing an accurate \gls{model} of the
\gls{system}, then driving the \glspl{state} they care about to zero (or to a
\gls{reference}). Integral control is added with $u_{error}$ estimation if
needed to handle \gls{model} uncertainty, but we prefer not to use it because
its response is hard to tune and some of its destabilizing dynamics aren't
visible during simulation.

Why use model-based control in FRC? Poor build season schedule management often
leads to the software team:
\begin{enumerate}
  \item Not getting enough time to verify basic functionality and test/tune
    feedback controllers.
  \item Spending dedicated software testing time troubleshooting
    mechanical/electrical issues within recently integrated subsystems instead.
\end{enumerate}

Model-based control (one of the focuses of this book) avoids both problems
because it lets software teams test basic functionality in simulation much
earlier in the build season and tune their feedback controllers automatically.

\section{What is a dynamical system?}

A dynamical system is a \gls{system} whose motion varies according to a set of
differential equations. A dynamical system is considered \textit{linear} if the
differential equations describing its dynamics consist only of linear operators.
Linear operators are things like constant gain multiplications, derivatives, and
integrals. You can define reasonably accurate linear \glspl{model} for pretty
much everything you'll see in FRC with just those relations.

But let's say you have a DC motor hooked up to a power supply and you applied a
constant voltage to it from rest. The motor approaches a steady-state angular
velocity, but the shape of the angular velocity curve over time isn't a line. In
fact, it's a decaying exponential curve akin to
\begin{equation*}
  \omega = \omega_{max}\left(1 - e^{-t}\right)
\end{equation*}

where $\omega$ is the angular velocity and $\omega_{max}$ is the maximum angular
velocity. If DC motors are said to behave linearly, then why is this?

Linearity refers to a \gls{system}'s equations of motion, not its time domain
response. The equation defining the motor's change in angular velocity over time
looks like
\begin{equation*}
  \dot{\omega} = -a\omega + bV
\end{equation*}

where $\dot{\omega}$ is the derivative of $\omega$ with respect to time, $V$ is
the input voltage, and $a$ and $b$ are constants specific to the motor. This
equation, unlike the one shown before, is actually linear because it only
consists of multiplications and additions relating the \gls{input} $V$ and
current \gls{state} $\omega$.

Also of note is that the relation between the input voltage and the angular
velocity of the output shaft is a linear regression. You'll see why if you model
a DC motor as a voltage source and generator producing back-EMF (in the equation
above, $bV$ corresponds to the voltage source and $-a\omega$ corresponds to the
back-EMF). As you increase the input voltage, the back-EMF increases linearly
with the motor's angular velocity. If there was a friction term that varied with
the angular velocity squared (air resistance is one example), the relation from
input to output would be a curve. Friction that scales with just the angular
velocity would result in a lower maximum angular velocity, but because that term
can be lumped into the back-EMF term, the response is still linear.

\section{State-space notation}

\subsection{What is state-space?}

Recall from last chapter that 2D space has two axes: $x$ and $y$. We represent
locations within this space as a pair of numbers packaged in a vector, and each
coordinate is a measure of how far to move along the corresponding axis.
State-space is a Cartesian coordinate system with an axis for each \gls{state}
variable, and we represent locations within it the same way we do for 2D space:
with a list of numbers in a vector. Each element in the vector corresponds to a
\gls{state} of the \gls{system}.

In addition to the \gls{state}, \glspl{input} and \glspl{output} are represented
as vectors. Since the mapping from the current \glspl{state} and \glspl{input}
to the change in \gls{state} is a system of equations, it's natural to write it
in matrix form.

\subsection{Benefits over classical control}

State-space notation provides a more convenient and compact way to model and
analyze \glspl{system} with multiple \glspl{input} and \glspl{output}. For a
\gls{system} with $p$ \glspl{input} and $q$ \glspl{output}, we would have to
write $q \times p$ transfer functions to represent it. Not only is the resulting
algebra unwieldy, but it only works for linear \glspl{system}. Including nonzero
initial conditions complicates the algebra even more. State-space representation
uses the time domain instead of the Laplace domain, so it can model nonlinear
\glspl{system}\footnote{This book focuses on analysis and control of linear
\glspl{system}. See chapter \ref{ch:nonlinear_control} for more on nonlinear
control.} and trivially supports nonzero initial conditions.

If modern control theory is so great and classical control theory isn't needed
to use it, why learn classical control theory at all? We teach classical control
theory because it provides a framework within which to understand results from
the mathematical machinery of modern control as well as vocabulary with which to
communicate that understanding. For example, faster poles (poles moved to the
left in the s-plane) mean faster decay, and oscillation means there is at least
one pair of complex conjugate poles. Not only can you describe what happened
succinctly, but you know why it happened from a theoretical perspective.

\subsection{Definition}

Below are the continuous and discrete versions of state-space notation.

\begin{definition}[State-space notation]%
  \index{State-space controllers!open-loop}

  \begin{align}
    \dot{\mtx{x}} &= \mtx{A}\mtx{x} + \mtx{B}\mtx{u} \label{eq:ss_ctrl_x} \\
    \mtx{y} &= \mtx{C}\mtx{x} + \mtx{D}\mtx{u} \label{eq:ss_ctrl_y} \\
    \nonumber \\
    \mtx{x}_{k+1} &= \mtx{A}\mtx{x}_k + \mtx{B}\mtx{u}_k \label{eq:ssz_ctrl_x} \\
    \mtx{y}_k &= \mtx{C}\mtx{x}_k + \mtx{D}\mtx{u}_k \label{eq:ssz_ctrl_y}
  \end{align}

  \begin{figurekey}
    \begin{tabular}{llll}
      $\mtx{A}$ & system matrix      & $\mtx{x}$ & state vector \\
      $\mtx{B}$ & input matrix       & $\mtx{u}$ & input vector \\
      $\mtx{C}$ & output matrix      & $\mtx{y}$ & output vector \\
      $\mtx{D}$ & feedthrough matrix &  &  \\
    \end{tabular}
  \end{figurekey}
\end{definition}

\begin{booktable}
  \begin{tabular}{|ll|ll|}
    \hline
    \rowcolor{headingbg}
    \textbf{Matrix} & \textbf{Rows $\times$ Columns} &
    \textbf{Matrix} & \textbf{Rows $\times$ Columns} \\
    \hline
    $\mtx{A}$ & states $\times$ states & $\mtx{x}$ & states $\times$ 1 \\
    $\mtx{B}$ & states $\times$ inputs & $\mtx{u}$ & inputs $\times$ 1 \\
    $\mtx{C}$ & outputs $\times$ states & $\mtx{y}$ & outputs $\times$ 1 \\
    $\mtx{D}$ & outputs $\times$ inputs &  &  \\
    \hline
  \end{tabular}
  \caption{State-space matrix dimensions}
  \label{tab:ss_matrix_dims}
\end{booktable}

In the continuous case, the change in \gls{state} and the \gls{output} are
linear combinations of the \gls{state} vector and the \gls{input} vector. The
$\mtx{A}$ and $\mtx{B}$ matrices are used to map the \gls{state} vector
$\mtx{x}$ and the \gls{input} vector $\mtx{u}$ to a change in the \gls{state}
vector $\dot{\mtx{x}}$. The $\mtx{C}$ and $\mtx{D}$ matrices are used to map the
\gls{state} vector $\mtx{x}$ and the \gls{input} vector $\mtx{u}$ to an
\gls{output} vector $\mtx{y}$.

\section{Eigenvalues and stability}

If a system is stable, its output will tend toward equilibrium (steady-state)
over time. For a general system $\dot{\mat{x}} = f(\mat{x}, \mat{u})$,
equilibrium points are points where $\dot{\mat{x}} = \mat{0}$. If we let
$\mat{x} = \mat{0}$ and $\mat{u} = \mat{0}$ in
$\dot{\mat{x}} = \mat{A}\mat{x} + \mat{B}\mat{u}$, we can see that
$\dot{\mat{x}} = \mat{0}$, so $\mat{x} = \mat{0}$ is an equilibrium point.

We'd like to know whether all possible unforced system trajectories
($\mat{u} = \mat{0}$) move toward or away from the equilibrium point. If we
solve the system of linear differential equations
$\dot{\mat{x}} = \mat{A}\mat{x}$, we get
$\mat{x}(t) = e^{\mat{A}t} \mat{x}_0$.\footnote{Section
\ref{sec:linear_system_zoh} will explain why the matrix exponential
$e^{\mat{A}t}$ shows up here.} $e^{\mat{A}t}$ is the superposition of
$e^{\lambda_j t}$ terms where $\{\lambda_j\}$ is the set of $\mat{A}$'s
eigenvalues.\footnote{We're handwaving why this is the case, but it's a
consequence of $e^{\mat{A}t}$ being diagonalizable.}

For now, let's consider when all the eigenvalues are real numbers.
\begin{equation*}
  \begin{cases}
    \lambda_j < 0, & e^{\lambda_j t} \text{ decays to zero (stable)}
      \\
    \lambda_j = 0, & e^{\lambda_j t} = 1 \text{ (marginally stable)} \\
    \lambda_j > 0, & e^{\lambda_j t} \text{ grows to infinity (unstable)}
  \end{cases}
\end{equation*}

So the system tends toward the equilibrium point (i.e., it's stable) if
$\lambda_j < 0$ for all $j$.

Now let's consider when the eigenvalues are complex numbers. What does that mean
for the system response? Let $\lambda_j = a_j + b_j i$. Each of the exponential
terms in the solution can be written as
\begin{equation*}
  e^{\lambda_j t} = e^{(a_j + b_j i)t} = e^{a_j t} e^{i b_j t}
\end{equation*}

The complex exponential can be rewritten using Euler's formula.\footnote{Euler's
formula may seem surprising at first, but it's rooted in the fact that complex
exponentials are rotations in the complex plane around the origin. If you can
imagine walking around the unit circle traced by that rotation, you'll notice
the real part of your position oscillates between $-1$ and $1$ over time. That
is, complex exponentials manifest as oscillations in real life.
\begin{center}
  \qrcode{https://youtu.be/ZxYOEwM6Wbk} \\
  ``What is Euler's formula actually saying? | Ep. 4 Lockdown live math" (51
    minutes) \\
  \footnotesize 3Blue1Brown \\
  \url{https://youtu.be/ZxYOEwM6Wbk}
\end{center}
}
\begin{equation*}
  e^{i b_j t} = \cos(b_j t) + i \sin(b_j t)
\end{equation*}

Therefore,
\begin{equation*}
  e^{\lambda_j t} = e^{a_j t} (\cos(b_j t) + i \sin(b_j t))
\end{equation*}

When the eigenvalue's imaginary part $b_j \neq 0$, it contributes oscillation to
the real part's response.

\index{stability!poles}
The eigenvalues of $\mat{A}$ are called \textit{poles}.\footnote{This name comes
from classical control theory. See subsection \ref{subsec:poles_and_zeroes} for
more.} Figure \ref{fig:impulse_response_eig} shows the \glspl{impulse response}
in the time domain for \glspl{system} with various pole locations in the complex
plane (real numbers on the x-axis and imaginary numbers on the y-axis). Each
response has an initial condition of $1$.
\begin{bookfigure}
  \begin{tikzpicture}[auto, >=latex']
  % \draw [help lines] (-4,-2) grid (4,4);

  % Draw main axes
  \draw[->] (-4,0) -- (4,0) node[below] {\small Re($\sigma$)};
  \draw[->] (0,-2) -- (0,4) node[right] {\small Im($j\omega$)};

  % Stable: e^-1.75t * cos(1.75wt) (80/3*w for readability)
  \drawtimeplot{-2.125cm}{2.5cm}{0.125cm}{0.44375cm}{
    exp(-1.75 * \x) * cos(80/3 * 1.75 * deg(\x))}
  \drawpole{-1.75cm}{1.75cm}

  % Stable: e^-2.5t
  \drawtimeplot{-2.25cm}{0.75cm}{0.125cm}{0.125cm}{exp(-2 * \x)}
  \drawpole{-2cm}{0cm}

  % Stable: e^-t
  \drawtimeplot{-1.125cm}{-0.75cm}{0.125cm}{0.125cm}{exp(-\x)}
  \drawpole{-1cm}{0cm}

  % Marginally stable: cos(wt) (80/3*w for readability)
  \drawtimeplot{-0.75cm}{1.125cm}{0.125cm}{0.44375cm}{cos(80/3 * deg(\x))}
  \drawpole{0cm}{1cm}

  % Marginally stable cos(2wt) (80/3*w for readability)
  \drawtimeplot{0cm}{2.75cm}{0.125cm}{0.44375cm}{cos(80/3 * 2 * deg(\x))}
  \drawpole{0cm}{2cm}

  % Integrator
  \drawtimeplot{0.25cm}{-0.75cm}{0.125cm}{0.125cm}{1}
  \drawpole{0cm}{0cm}

  % Unstable: e^t
  \drawtimeplot{1.125cm}{0.75cm}{0.125cm}{0.125cm}{exp(\x)}
  \drawpole{1cm}{0cm}

  % Unstable: e^2t
  \drawtimeplot{2.25cm}{-0.75cm}{0.125cm}{0.125cm}{exp(2 * \x)}
  \drawpole{2cm}{0cm}

  % Unstable: e^0.75t * cos(1.75wt) (80/3*w for readability)
  \drawtimeplot{1.5cm}{2.25cm}{0.125cm}{0.44375cm}{
    exp(0.75 * \x) * cos(80/3 * 1.75 * deg(\x))}
  \drawpole{0.75cm}{1.75cm}

  % LHP and RHP labels
  \draw (-3.5,1.5) node {LHP};
  \draw (3.5,1.5) node {RHP};

  % Stable and unstable labels
  \draw (-2.5,3.5) node {\small Stable};
  \draw (2.5,3.5) node {\small Unstable};
\end{tikzpicture}

  \caption{Impulse response vs pole location}
  \label{fig:impulse_response_eig}
\end{bookfigure}

Poles in the left half-plane (LHP) are stable; the \gls{system}'s output may
oscillate but it converges to steady-state. Poles on the imaginary axis are
marginally stable; the \gls{system}'s output oscillates at a constant amplitude
forever. Poles in the right half-plane (RHP) are unstable; the \gls{system}'s
output grows without bound.
\begin{remark}
  Imaginary poles always come in complex conjugate pairs (e.g., $-2 + 3i$,
  $-2 - 3i$).
\end{remark}

\section{Controllability and observability}

\subsection{Controllability matrix}

\index{controller design!controllability}
A \gls{system} is controllable if it can be steered from any \gls{state} to any
\gls{state} by a finite sequence of admissible \glspl{input}.

The controllability matrix can be used to determine if a system is controllable.
\begin{theorem}[Controllability]
  A continuous \gls{time-invariant} linear state-space \gls{model} is
  controllable if and only if
  \begin{equation}
    \rank\left(
    \begin{bmatrix}
      \mat{B} & \mat{A}\mat{B} & \cdots & \mat{A}^{n-1}\mat{B}
    \end{bmatrix}
    \right) = n
    \label{eq:ctrl_rank}
  \end{equation}

  where rank is the number of linearly independent rows in a matrix and $n$ is
  the number of \glspl{state}.
\end{theorem}

The controllability matrix in equation \eqref{eq:ctrl_rank} being rank-deficient
means the \glspl{input} cannot apply transforms along all axes in the
state-space; the transformation the matrix represents is collapsed into a lower
dimension.

The condition number of the controllability matrix $\mathcal{C}$ is defined as
$\frac{\sigma_{max}(\mathcal{C})}{\sigma_{min}(\mathcal{C})}$ where
$\sigma_{max}$ is the maximum singular
value\footnote{\label{footn:singular_val}Singular values are a generalization of
eigenvalues for nonsquare matrices.} and $\sigma_{min}$ is the minimum singular
value. As this number approaches infinity, one or more of the \glspl{state}
becomes uncontrollable. This number can also be used to tell us which actuators
are better than others for the given \gls{system}; a lower condition number
means that the actuators have more control authority.

\subsection{Controllability Gramian}
\index{controller design!controllability Gramian}

While the rank of the observability matrix can tell us whether the system is
controllable, it won't tell us which specific states are controllable or how
controllable. The controllability Gramian can be used to determine these things.

If $\mat{A}$ is stable, the controllability Gramian $\mat{W}_c$ is the unique
solution to the following continuous Lyapunov equation.
\begin{equation}
  \mat{A}\mat{W}_c + \mat{W}_c\mat{A}\T + \mat{B}\mat{B}\T = 0
\end{equation}

Alternatively,
\begin{equation}
  \mat{W}_c =
    \int_0^\infty e^{\mat{A}\tau} \mat{B}\mat{B}\T e^{\mat{A}\T\tau} \,d\tau
\end{equation}

If the solution is positive definite, the system is controllable. The
eigenvalues of $\mat{W}_c$ represent how controllable their respective states
are (larger means more controllable).

\subsection{Controllability of specific states}

If you want to know if a specific state is controllable, first find its
corresponding eigenvalue $\lambda$ in $\mat{A}$. Then, that state is
controllable if
\begin{equation}
  \rank\left(
  \begin{bmatrix}
    \lambda\mat{I} - \mat{A} & \mat{B}
  \end{bmatrix}\right) = n
\end{equation}

where $n$ is the number of \glspl{state}.

\subsection{Stabilizability}
\label{subsec:stabilizability}
\index{controller design!stabilizability}

Stabilizability is a weaker form of controllability. A system is considered
stabilizable if one of the following conditions is true:
\begin{enumerate}
  \item All uncontrollable states can be stabilized
  \item All unstable states are controllable
\end{enumerate}

\subsection{Observability matrix}

\index{observer design!observability}
A \gls{system} is observable if the \gls{state}, whatever it may be, can be
inferred from a finite sequence of \glspl{output}.

Observability and controllability are mathematical duals; controllability proves
that a sequence of \glspl{input} exists that drives the \gls{system} to any
\gls{state}, and observability proves that a sequence of \glspl{output} exists
that drives the \gls{state} estimate to any true \gls{state}.

The observability matrix can be used to determine if a system is observable.
\begin{theorem}[Observability]
  A continuous \gls{time-invariant} linear state-space \gls{model} is observable
  if and only if
  \begin{equation}
    \rank\left(
    \begin{bmatrix}
      \mat{C} \\
      \mat{C}\mat{A} \\
      \vdots \\
      \mat{C}\mat{A}^{n-1}
    \end{bmatrix}\right) = n \label{eq:obsv_rank}
  \end{equation}

  where rank is the number of linearly independent rows in a matrix and $n$ is
  the number of \glspl{state}.
\end{theorem}

The observability matrix in equation \eqref{eq:obsv_rank} being rank-deficient
means the \glspl{output} do not contain contributions from every \gls{state}.
That is, not all \glspl{state} are mapped to a linear combination in the
\gls{output}. Therefore, the \glspl{output} alone are insufficient to estimate
all the \glspl{state}.

The condition number of the observability matrix $\mathcal{O}$ is defined as
$\frac{\sigma_{max}(\mathcal{O})}{\sigma_{min}(\mathcal{O})}$ where
$\sigma_{max}$ is the maximum singular value\footref{footn:singular_val} and
$\sigma_{min}$ is the minimum singular value. As this number approaches
infinity, one or more of the \glspl{state} becomes unobservable. This number can
also be used to tell us which sensors are better than others for the given
\gls{system}; a lower condition number means the \glspl{output} produced by the
sensors are better indicators of the \gls{system} \gls{state}.

\subsection{Observability Gramian}
\index{observer design!observability Gramian}

While the rank of the observability matrix can tell us whether the system is
observable, it won't tell us which specific states are observable or how
observable. The observability Gramian can be used to determine these things.

If $\mat{A}$ is stable, the observability Gramian $\mat{W}_o$ is the unique
solution to the following continuous Lyapunov equation.
\begin{equation}
  \mat{A}\T\mat{W}_o + \mat{W}_o\mat{A} + \mat{C}\T\mat{C} = 0
\end{equation}

Alternatively,
\begin{equation}
  \mat{W}_o =
    \int_0^\infty e^{\mat{A}\T\tau} \mat{C}\T\mat{C} e^{\mat{A}\tau} \,d\tau
\end{equation}

If the solution is positive definite, the system is observable. The eigenvalues
of $\mat{W}_o$ represent how observable their respective states are (larger
means more observable).

\subsection{Observability of specific states}

If you want to know if a specific state is observable, first find its
corresponding eigenvalue $\lambda$ in $\mat{A}$. Then, that state is
observable if
\begin{equation}
  \rank\left(
  \begin{bmatrix}
    \lambda\mat{I} - \mat{A} \\
    \mat{C}
  \end{bmatrix}\right) = n
\end{equation}

where $n$ is the number of \glspl{state}.

\subsection{Detectability}
\label{subsec:detectability}
\index{observer design!detectability}

Detectability is a weaker form of observability. A system is considered
detectable if one of the following conditions is true:
\begin{enumerate}
  \item All unobservable states are stable
  \item All unstable states are observable
\end{enumerate}

\section{Closed-loop controller}
\index{state-space controllers!discrete closed-loop}

With the \gls{control law} $\mat{u}_k = \mat{K}(\mat{r}_k - \mat{x}_k)$, we can
derive the closed-loop state-space equations. We'll discuss where this
\gls{control law} comes from in subsection \ref{sec:lqr}.

First is the \gls{state} update equation. Substitute the \gls{control law} into
equation \eqref{eq:disc_ss_x}.
\begin{align}
  \mat{x}_{k+1} &= \mat{A}\mat{x}_k + \mat{B}\mat{K}(\mat{r}_k - \mat{x}_k)
    \nonumber \\
  \mat{x}_{k+1} &= \mat{A}\mat{x}_k + \mat{B}\mat{K}\mat{r}_k -
    \mat{B}\mat{K}\mat{x}_k \nonumber \\
  \mat{x}_{k+1} &= (\mat{A} - \mat{B}\mat{K})\mat{x}_k + \mat{B}\mat{K}\mat{r}_k
    \label{eq:disc_ss_ctrl_x}
  \intertext{Now for the \gls{output} equation. Substitute the \gls{control law}
    into equation \eqref{eq:disc_ss_y}.}
  \mat{y}_k &= \mat{C}\mat{x}_k + \mat{D}(\mat{K}(\mat{r}_k - \mat{x}_k))
    \nonumber \\
  \mat{y}_k &= \mat{C}\mat{x}_k + \mat{D}\mat{K}\mat{r}_k -
    \mat{D}\mat{K}\mat{x}_k \nonumber \\
  \mat{y}_k &= (\mat{C} - \mat{D}\mat{K})\mat{x}_k + \mat{D}\mat{K}\mat{r}_k
\end{align}

\index{stability!eigenvalues}
Instead of commanding the \gls{system} to a \gls{state} using the vector
$\mat{u}_k$ directly, we can now specify a vector of desired \glspl{state}
through $\mat{r}_k$ and the \gls{controller} will choose values of $\mat{u}_k$
for us over time to make the \gls{system} converge to the \gls{reference}.

The eigenvalues of $\mat{A} - \mat{B}\mat{K}$ are the poles of the closed-loop
\gls{system}. Therefore, the rate of convergence and stability of the
closed-loop \gls{system} can be changed by moving the poles via the eigenvalues
of $\mat{A} - \mat{B}\mat{K}$. $\mat{A}$ and $\mat{B}$ are inherent to the
\gls{system}, but $\mat{K}$ can be chosen arbitrarily by the controller
designer. For equation \eqref{eq:disc_ss_ctrl_x} to reach steady-state, the
eigenvalues of $\mat{A} - \mat{B}\mat{K}$ must be in the left-half plane.
\begin{booktable}
  \begin{tabular}{|lll|}
    \hline
    \rowcolor{headingbg}
    \textbf{Symbol} & \textbf{Name} & \textbf{Rows $\times$ Columns} \\
    \hline
    $\mat{A}$ & system matrix & states $\times$ states \\
    $\mat{B}$ & input matrix & states $\times$ inputs \\
    $\mat{C}$ & output matrix & outputs $\times$ states \\
    $\mat{D}$ & feedthrough matrix & outputs $\times$ inputs \\
    $\mat{K}$ & controller gain matrix & inputs $\times$ states \\
    $\mat{r}$ & \gls{reference} vector & states $\times$ 1 \\
    $\mat{x}$ & state vector & states $\times$ 1 \\
    $\mat{u}$ & input vector & inputs $\times$ 1 \\
    $\mat{y}$ & output vector & outputs $\times$ 1 \\
    \hline
  \end{tabular}
  \caption{Controller matrix dimensions}
\end{booktable}

\section{Pole placement}
\index{controller design!pole placement}

This is the practice of placing the poles of a closed-loop \gls{system} directly
to produce a desired response. Python Control offers several pole placement
algorithms for generating controller or observer gains from a set of poles.

Since all our applications will be discrete \glspl{system}, we'll place poles in
the discrete domain (the z-plane). The s-plane's LHP maps to the inside of a
unit circle (see figure \ref{fig:s2z_mapping_pp}).
\begin{bookfigure}
  \begin{minisvg}{2}{build/modern-control-theory/discrete-state-space-control/s_plane}
  \end{minisvg}
  \hfill
  \begin{minisvg}{2}{build/modern-control-theory/discrete-state-space-control/z_plane}
  \end{minisvg}
  \caption{Mapping of axes from s-plane (left) to z-plane (right)}
  \label{fig:s2z_mapping_pp}
\end{bookfigure}

Pole placement should only be used if you know what you're doing. It's much
easier to let LQR place the poles for you, which we'll discuss next.

\chapterimage{appendices.jpg}{Sunset in an airplane over New Mexico}

\chapter{Linear-quadratic regulator}
\label{ch:deriv_lqr}

This appendix will go into more detail on the linear-quadratic regulator's
derivation and interesting applications.

\renewcommand*{\chapterpath}{\partpath/linear-quadratic-regulator}
\section{Derivation}

Let there be a discrete time linear \gls{system} defined as
\begin{equation}
  \mat{x}_{k+1} = \mat{A}\mat{x}_k + \mat{B}\mat{u}_k
\end{equation}

with the cost functional
\begin{equation*}
  J = \sum_{k=0}^\infty
    \begin{bmatrix}
      \mat{x}_k \\
      \mat{u}_k
    \end{bmatrix}\T
    \begin{bmatrix}
      \mat{Q} & \mat{N} \\
      \mat{N}\T & \mat{R}
    \end{bmatrix}
    \begin{bmatrix}
      \mat{x}_k \\
      \mat{u}_k
    \end{bmatrix}
\end{equation*}

where $J$ represents a trade-off between \gls{state} excursion and
\gls{control effort} with the weighting factors $\mat{Q}$, $\mat{R}$, and
$\mat{N}$. $\mat{Q}$ is the weight matrix for \gls{error}, $\mat{R}$ is the
weight matrix for \gls{control effort}, and $\mat{N}$ is a cross weight matrix
between \gls{error} and \gls{control effort}. $\mat{N}$ is commonly utilized
when penalizing the output in addition to the state and input.
\begin{align*}
  J &= \sum_{k=0}^\infty
    \begin{bmatrix}
      \mat{x}_k \\
      \mat{u}_k
    \end{bmatrix}\T
    \begin{bmatrix}
      \mat{Q}\mat{x}_k + \mat{N}\mat{u}_k \\
      \mat{N}\T\mat{x}_k + \mat{R}\mat{u}_k
    \end{bmatrix} \\
  J &= \sum_{k=0}^\infty
    \begin{bmatrix}
      \mat{x}_k\T & \mat{u}_k\T
    \end{bmatrix}
    \begin{bmatrix}
      \mat{Q}\mat{x}_k + \mat{N}\mat{u}_k \\
      \mat{N}\T\mat{x}_k + \mat{R}\mat{u}_k
    \end{bmatrix} \\
  J &= \sum_{k=0}^\infty
    (\mat{x}_k\T (\mat{Q}\mat{x}_k + \mat{N}\mat{u}_k) +
      \mat{u}_k\T (\mat{N}\T\mat{x}_k + \mat{R}\mat{u}_k)) \\
  J &= \sum_{k=0}^\infty
    (\mat{x}_k\T\mat{Q}\mat{x}_k + \mat{x}_k\T\mat{N}\mat{u}_k +
      \mat{u}_k\T\mat{N}\T\mat{x}_k + \mat{u}_k\T\mat{R}\mat{u}_k) \\
  J &= \sum_{k=0}^\infty
    (\mat{x}_k\T\mat{Q}\mat{x}_k + \mat{x}_k\T\mat{N}\mat{u}_k +
      \mat{x}_k\T\mat{N}\mat{u}_k\T + \mat{u_k}\T\mat{R}\mat{u}_k) \\
  J &= \sum_{k=0}^\infty
    (\mat{x}_k\T\mat{Q}\mat{x}_k + 2\mat{x}_k\T\mat{N}\mat{u}_k +
      \mat{u}_k\T\mat{R}\mat{u}_k) \\
  J &= \sum_{k=0}^\infty
    (\mat{x}_k\T\mat{Q}\mat{x}_k + \mat{u}_k\T\mat{R}\mat{u}_k +
      2\mat{x}_k\T\mat{N}\mat{u}_k)
\end{align*}

The feedback \gls{control law} which minimizes $J$ subject to the constraint
$\mat{x}_{k+1} = \mat{A}\mat{x}_k + \mat{B}\mat{u}_k$ is
\begin{equation*}
  \mat{u}_k = -\mat{K}\mat{x}_k
\end{equation*}

where $\mat{K}$ is given by
\begin{equation*}
  \mat{K} = (\mat{R} + \mat{B}\T\mat{S}\mat{B})^{-1}
    (\mat{B}\T\mat{S}\mat{A} + \mat{N}\T)
\end{equation*}

and $\mat{S}$ is found by solving the discrete time algebraic Riccati equation
defined as
\begin{equation*}
  \mat{A}\T\mat{S}\mat{A} - (\mat{A}\T\mat{S}\mat{B} + \mat{N})
    (\mat{R} + \mat{B}\T\mat{S}\mat{B})^{-1}
    (\mat{B}\T\mat{S}\mat{A} + \mat{N}\T) + \mat{Q} = 0
\end{equation*}

or alternatively
\begin{equation*}
  \mathcal{A}\T\mat{S}\mathcal{A} - \mathcal{A}\T\mat{S}\mat{B}
    (\mat{R} + \mat{B}\T\mat{S}\mat{B})^{-1} \mat{B}\T\mat{S}\mathcal{A} +
    \mat{Q} = 0
\end{equation*}

with
\begin{align*}
  \mathcal{A} &= \mat{A} - \mat{B}\mat{R}^{-1}\mat{N}\T \\
  \mathcal{Q} &= \mat{Q} - \mat{N}\mat{R}^{-1}\mat{N}\T
\end{align*}

If there is no cross-correlation between \gls{error} and \gls{control effort},
$\mat{N}$ is a zero matrix and the cost functional simplifies to
\begin{equation*}
  J = \sum_{k=0}^\infty (\mat{x}_k\T\mat{Q}\mat{x}_k +
    \mat{u}_k\T\mat{R}\mat{u}_k)
\end{equation*}

The feedback \gls{control law} which minimizes this $J$ subject to
$\mat{x}_{k+1} = \mat{A}\mat{x}_k + \mat{B}\mat{u}_k$ is
\begin{equation*}
  \mat{u}_k = -\mat{K}\mat{x}_k
\end{equation*}

where $\mat{K}$ is given by
\begin{equation*}
  \mat{K} = (\mat{R} + \mat{B}\T\mat{S}\mat{B})^{-1} \mat{B}\T\mat{S}\mat{A}
\end{equation*}

and $\mat{S}$ is found by solving the discrete time algebraic Riccati equation
defined as
\begin{equation*}
  \mat{A}\T\mat{S}\mat{A} - \mat{A}\T\mat{S}\mat{B}
    (\mat{R} + \mat{B}\T\mat{S}\mat{B})^{-1} \mat{B}\T\mat{S}\mat{A} +
    \mat{Q} = 0
\end{equation*}

Snippet \ref{lst:lqr} computes the infinite horizon, discrete time LQR.
\begin{coderemote}{Python}{snippets/lqr.py}
  \caption{Infinite horizon, discrete time LQR computation in Python}
  \label{lst:lqr}
\end{coderemote}

Other formulations of LQR for finite horizon and discrete time can be seen on
Wikipedia \cite{bib:wiki_lqr}.

MIT OpenCourseWare has a rigorous proof of the results shown above
\cite{bib:lqr_derivs}.

\section{State feedback with output cost}
\index{controller design!linear-quadratic regulator!state feedback with output
  cost}

LQR is normally used for state feedback on
\begin{align*}
  \mat{x}_{k+1} &= \mat{A}\mat{x}_k + \mat{B}\mat{u}_k \\
  \mat{y}_k &= \mat{C}\mat{x}_k + \mat{D}\mat{u}_k
\end{align*}

with the cost functional
\begin{equation*}
  J = \sum_{k=0}^\infty (\mat{x}_k\T\mat{Q}\mat{x}_k +
    \mat{u}_k\T\mat{R}\mat{u}_k)
\end{equation*}

However, we may not know how to select costs for some of the states, or we don't
care what certain internal states are doing. We can address this by writing the
cost functional in terms of the output vector instead of the state vector. Not
only can we make our output contain a subset of states, but we can use any other
cost metric we can think of as long as it's representable as a linear
combination of the states and inputs.\footnote{We'll see this later on in
section \ref{sec:implicit_model_following} when we define the cost metric as
deviation from the behavior of another model.}

For state feedback with an output cost, we want to minimize the following cost
functional.
\begin{align*}
  J &= \sum_{k=0}^\infty (\mat{y}_k\T\mat{Q}\mat{y}_k +
    \mat{u}_k\T\mat{R}\mat{u}_k)
  \intertext{Substitute in the expression for $\mat{y}_k$.}
  J &= \sum_{k=0}^\infty ((\mat{C}\mat{x}_k + \mat{D}\mat{u}_k)\T\mat{Q}
    (\mat{C}\mat{x}_k + \mat{D}\mat{u}_k) + \mat{u}_k\T\mat{R}\mat{u}_k)
  \intertext{Apply the transpose to the left-hand side of the $\mat{Q}$ term.}
  J &= \sum_{k=0}^\infty ((\mat{x}_k\T\mat{C}\T + \mat{u}_k\T\mat{D}\T)\mat{Q}
    (\mat{C}\mat{x}_k + \mat{D}\mat{u}_k) + \mat{u}_k\T\mat{R}\mat{u}_k)
  \intertext{Factor out $\begin{bmatrix}\mat{x}_k \\ \mat{u}_k\end{bmatrix}\T$
    from the left side and $\begin{bmatrix}\mat{x}_k \\ \mat{u}_k\end{bmatrix}$
    from the right side of each term.}
  J &= \sum_{k=0}^\infty \left(
    \begin{bmatrix}
      \mat{x}_k \\
      \mat{u}_k
    \end{bmatrix}\T
    \begin{bmatrix}
      \mat{C}\T \\
      \mat{D}\T
    \end{bmatrix}
    \mat{Q}
    \begin{bmatrix}
      \mat{C} &
      \mat{D}
    \end{bmatrix}
    \begin{bmatrix}
      \mat{x}_k \\
      \mat{u}_k
    \end{bmatrix} +
    \begin{bmatrix}
      \mat{x}_k \\
      \mat{u}_k
    \end{bmatrix}\T
    \begin{bmatrix}
      \mat{0} & \mat{0} \\
      \mat{0} & \mat{R}
    \end{bmatrix}
    \begin{bmatrix}
      \mat{x}_k \\
      \mat{u}_k
    \end{bmatrix}
    \right) \\
  J &= \sum_{k=0}^\infty \left(
    \begin{bmatrix}
      \mat{x}_k \\
      \mat{u}_k
    \end{bmatrix}\T
    \left(
    \begin{bmatrix}
      \mat{C}\T \\
      \mat{D}\T
    \end{bmatrix}
    \mat{Q}
    \begin{bmatrix}
      \mat{C} &
      \mat{D}
    \end{bmatrix} +
    \begin{bmatrix}
      \mat{0} & \mat{0} \\
      \mat{0} & \mat{R}
    \end{bmatrix}
    \right)
    \begin{bmatrix}
      \mat{x}_k \\
      \mat{u}_k
    \end{bmatrix}
    \right)
  \intertext{Multiply in $\mat{Q}$.}
  J &= \sum_{k=0}^\infty \left(
    \begin{bmatrix}
      \mat{x}_k \\
      \mat{u}_k
    \end{bmatrix}\T
    \left(
    \begin{bmatrix}
      \mat{C}\T\mat{Q} \\
      \mat{D}\T\mat{Q}
    \end{bmatrix}
    \begin{bmatrix}
      \mat{C} &
      \mat{D}
    \end{bmatrix} +
    \begin{bmatrix}
      \mat{0} & \mat{0} \\
      \mat{0} & \mat{R}
    \end{bmatrix}
    \right)
    \begin{bmatrix}
      \mat{x}_k \\
      \mat{u}_k
    \end{bmatrix}
    \right)
  \intertext{Multiply matrices in the left term together.}
  J &= \sum_{k=0}^\infty \left(
    \begin{bmatrix}
      \mat{x}_k \\
      \mat{u}_k
    \end{bmatrix}\T
    \left(
    \begin{bmatrix}
      \mat{C}\T\mat{Q}\mat{C} & \mat{C}\T\mat{Q}\mat{D} \\
      \mat{D}\T\mat{Q}\mat{C} & \mat{D}\T\mat{Q}\mat{D}
    \end{bmatrix} +
    \begin{bmatrix}
      \mat{0} & \mat{0} \\
      \mat{0} & \mat{R}
    \end{bmatrix}
    \right)
    \begin{bmatrix}
      \mat{x}_k \\
      \mat{u}_k
    \end{bmatrix}
    \right)
\end{align*}

Add the terms together.
\begin{equation}
  J = \sum_{k=0}^\infty
  \begin{bmatrix}
    \mat{x}_k \\
    \mat{u}_k
  \end{bmatrix}\T
  \begin{bmatrix}
    \underbrace{\mat{C}\T\mat{Q}\mat{C}}_{\mat{Q}} &
    \underbrace{\mat{C}\T\mat{Q}\mat{D}}_{\mat{N}} \\
    \underbrace{\mat{D}\T\mat{Q}\mat{C}}_{\mat{N}\T} &
    \underbrace{\mat{D}\T\mat{Q}\mat{D} + \mat{R}}_{\mat{R}}
  \end{bmatrix}
  \begin{bmatrix}
    \mat{x}_k \\
    \mat{u}_k
  \end{bmatrix}
\end{equation}

Thus, state feedback with an output cost can be defined as the following
optimization problem.
\begin{theorem}[Linear-quadratic regulator with output cost]
  \begin{align}
    \min_{\mat{u}_k} &\sum_{k=0}^\infty
    \begin{bmatrix}
      \mat{x}_k \\
      \mat{u}_k
    \end{bmatrix}\T
    \begin{bmatrix}
      \underbrace{\mat{C}\T\mat{Q}\mat{C}}_{\mat{Q}} &
      \underbrace{\mat{C}\T\mat{Q}\mat{D}}_{\mat{N}} \\
      \underbrace{\mat{D}\T\mat{Q}\mat{C}}_{\mat{N}\T} &
      \underbrace{\mat{D}\T\mat{Q}\mat{D} + \mat{R}}_{\mat{R}}
    \end{bmatrix}
    \begin{bmatrix}
      \mat{x}_k \\
      \mat{u}_k
    \end{bmatrix}
    \nonumber \\
    \text{subject to } &\mat{x}_{k+1} = \mat{A}\mat{x}_k + \mat{B}\mat{u}_k
  \end{align}

  The optimal control policy $\mat{u}_k^*$ is $\mat{K}(\mat{r}_k - \mat{x}_k)$
  where $\mat{r}_k$ is the desired state. Note that the $\mat{Q}$ in
  $\mat{C}\T\mat{Q}\mat{C}$ is outputs $\times$ outputs instead of states
  $\times$ states. $\mat{K}$ can be computed via the typical LQR equations based
  on the algebraic Ricatti equation.
\end{theorem}

If the output is just the state vector, then $\mat{C} = \mat{I}$,
$\mat{D} = \mat{0}$, and the cost functional simplifies to that of LQR with a
state cost.
\begin{equation*}
  J = \sum_{k=0}^\infty
  \begin{bmatrix}
    \mat{x}_k \\
    \mat{u}_k
  \end{bmatrix}\T
  \begin{bmatrix}
    \mat{Q} & \mat{0} \\
    \mat{0} & \mat{R}
  \end{bmatrix}
  \begin{bmatrix}
    \mat{x}_k \\
    \mat{u}_k
  \end{bmatrix}
\end{equation*}

\section{Implicit model following}
\label{sec:implicit_model_following}
\index{controller design!linear-quadratic regulator!implicit model following}

If we want to design a feedback controller that erases the dynamics of our
system and makes it behave like some other system, we can use \textit{implicit
model following}. This is used on the Blackhawk helicopter at NASA Ames research
center when they want to make it fly like experimental aircraft (within the
limits of the helicopter's actuators, of course).

\subsection{Following reference system matrix}

Let the original system dynamics be
\begin{align*}
  \mat{x}_{k+1} &= \mat{A}\mat{x}_k + \mat{B}\mat{u}_k \\
  \mat{y}_k &= \mat{C}\mat{x}_k
\end{align*}

and the desired system dynamics be $\mat{z}_{k+1} = \mat{A}_{ref}\mat{z}_k$.
\begin{align*}
  \mat{y}_{k+1} &= \mat{C}\mat{x}_{k+1} \\
  \mat{y}_{k+1} &= \mat{C}(\mat{A}\mat{x}_k + \mat{B}\mat{u}_k) \\
  \mat{y}_{k+1} &= \mat{C}\mat{A}\mat{x}_k + \mat{C}\mat{B}\mat{u}_k
\end{align*}

We want to minimize the following cost functional.
\begin{equation*}
  J = \sum_{k=0}^\infty \left((\mat{y}_{k+1} - \mat{z}_{k+1})\T \mat{Q}
    (\mat{y}_{k+1} - \mat{z}_{k+1}) + \mat{u}_k\T\mat{R}\mat{u}_k\right)
\end{equation*}

We'll be measuring the desired system's state, so let $\mat{y} = \mat{z}$.
\begin{align*}
  \mat{z}_{k+1} &= \mat{A}_{ref}\mat{y}_k \\
  \mat{z}_{k+1} &= \mat{A}_{ref}\mat{C}\mat{x}_k
\end{align*}

Therefore,
\begin{align*}
  \mat{y}_{k+1} - \mat{z}_{k+1} &=
    \mat{C}\mat{A}\mat{x}_k + \mat{C}\mat{B}\mat{u}_k -
    (\mat{A}_{ref}\mat{C}\mat{x}_k) \\
  \mat{y}_{k+1} - \mat{z}_{k+1} &=
    (\mat{C}\mat{A} - \mat{A}_{ref}\mat{C})\mat{x}_k + \mat{C}\mat{B}\mat{u}_k
\end{align*}

Substitute this into the cost functional.
\begin{align*}
  J &= \sum_{k=0}^\infty \left((\mat{y}_{k+1} - \mat{z}_{k+1})\T \mat{Q}
    (\mat{y}_{k+1} - \mat{z}_{k+1}) + \mat{u}_k\T\mat{R}\mat{u}_k\right) \\
  J &= \sum_{k=0}^\infty \left(
    ((\mat{C}\mat{A} - \mat{A}_{ref}\mat{C})\mat{x}_k + \mat{C}\mat{B}\mat{u}_k)\T
    \mat{Q}
    ((\mat{C}\mat{A} - \mat{A}_{ref}\mat{C})\mat{x}_k + \mat{C}\mat{B}\mat{u}_k) \right. \\
  &\qquad \left. + \mat{u}_k\T\mat{R}\mat{u}_k\right)
\end{align*}

Apply the transpose to the left-hand side of the $\mat{Q}$ term.
\begin{align*}
  J &= \sum_{k=0}^\infty \left(
    (\mat{x}_k\T(\mat{C}\mat{A} - \mat{A}_{ref}\mat{C})\T + \mat{u}_k\T(\mat{C}\mat{B})\T)
    \mat{Q}
    ((\mat{C}\mat{A} - \mat{A}_{ref}\mat{C})\mat{x}_k + \mat{C}\mat{B}\mat{u}_k) \right. \\
  &\qquad \left. + \mat{u}_k\T\mat{R}\mat{u}_k\right)
\end{align*}

Factor out $\begin{bmatrix}\mat{x}_k \\ \mat{u}_k\end{bmatrix}\T$ from the left
side and $\begin{bmatrix}\mat{x}_k \\ \mat{u}_k\end{bmatrix}$ from the right
side of each term.
\begin{align*}
  J &= \sum_{k=0}^\infty \left(
    \begin{bmatrix}
      \mat{x}_k \\
      \mat{u}_k
    \end{bmatrix}\T
    \begin{bmatrix}
      (\mat{C}\mat{A} - \mat{A}_{ref}\mat{C})\T \\
      (\mat{C}\mat{B})\T
    \end{bmatrix}
    \mat{Q}
    \begin{bmatrix}
      \mat{C}\mat{A} - \mat{A}_{ref}\mat{C} &
      \mat{C}\mat{B}
    \end{bmatrix}
    \begin{bmatrix}
      \mat{x}_k \\
      \mat{u}_k
    \end{bmatrix} \right. \\
  &\qquad \left. +
    \begin{bmatrix}
      \mat{x}_k \\
      \mat{u}_k
    \end{bmatrix}\T
    \begin{bmatrix}
      \mat{0} & \mat{0} \\
      \mat{0} & \mat{R}
    \end{bmatrix}
    \begin{bmatrix}
      \mat{x}_k \\
      \mat{u}_k
    \end{bmatrix}
    \right) \\
  J &= \sum_{k=0}^\infty \left(
    \begin{bmatrix}
      \mat{x}_k \\
      \mat{u}_k
    \end{bmatrix}\T
    \left(
    \begin{bmatrix}
      (\mat{C}\mat{A} - \mat{A}_{ref}\mat{C})\T \\
      (\mat{C}\mat{B})\T
    \end{bmatrix}
    \mat{Q}
    \begin{bmatrix}
      \mat{C}\mat{A} - \mat{A}_{ref}\mat{C} &
      \mat{C}\mat{B}
    \end{bmatrix}
    \right.\right. \\
  &\qquad \left.\left. +
    \begin{bmatrix}
      \mat{0} & \mat{0} \\
      \mat{0} & \mat{R}
    \end{bmatrix}
    \right)
    \begin{bmatrix}
      \mat{x}_k \\
      \mat{u}_k
    \end{bmatrix}
    \right)
\end{align*}

Multiply in $\mat{Q}$.
\begin{align*}
  J &= \sum_{k=0}^\infty \left(
    \begin{bmatrix}
      \mat{x}_k \\
      \mat{u}_k
    \end{bmatrix}\T
    \left(
    \begin{bmatrix}
      (\mat{C}\mat{A} - \mat{A}_{ref}\mat{C})\T\mat{Q} \\
      (\mat{C}\mat{B})\T\mat{Q}
    \end{bmatrix}
    \begin{bmatrix}
      \mat{C}\mat{A} - \mat{A}_{ref}\mat{C} &
      \mat{C}\mat{B}
    \end{bmatrix} \right.\right. \\
  &\qquad \left.\left. +
    \begin{bmatrix}
      \mat{0} & \mat{0} \\
      \mat{0} & \mat{R}
    \end{bmatrix}
    \right)
    \begin{bmatrix}
      \mat{x}_k \\
      \mat{u}_k
    \end{bmatrix}
    \right)
\end{align*}

Multiply matrices in the left term together.
\begin{align*}
  J &= \sum_{k=0}^\infty \\
  &\left(
    \begin{bmatrix}
      \mat{x}_k \\
      \mat{u}_k
    \end{bmatrix}\T
    \left(
    \begin{bmatrix}
      (\mat{C}\mat{A} - \mat{A}_{ref}\mat{C})\T\mat{Q}(\mat{C}\mat{A} - \mat{A}_{ref}\mat{C}) &
      (\mat{C}\mat{A} - \mat{A}_{ref}\mat{C})\T\mat{Q}(\mat{C}\mat{B}) \\
      (\mat{C}\mat{B})\T\mat{Q}(\mat{C}\mat{A} - \mat{A}_{ref}\mat{C}) &
      (\mat{C}\mat{B})\T\mat{Q}(\mat{C}\mat{B})
    \end{bmatrix} \right.\right. \\
  &\qquad \left.\left. +
    \begin{bmatrix}
      \mat{0} & \mat{0} \\
      \mat{0} & \mat{R}
    \end{bmatrix}
    \right)
    \begin{bmatrix}
      \mat{x}_k \\
      \mat{u}_k
    \end{bmatrix}
    \right)
\end{align*}

Add the terms together.
\begin{align}
  J &= \sum_{k=0}^\infty \nonumber \\
  &\begin{bmatrix}
    \mat{x}_k \\
    \mat{u}_k
  \end{bmatrix}\T
  \begin{bsmallmatrix}
    \underbrace{(\mat{C}\mat{A} - \mat{A}_{ref}\mat{C})\T\mat{Q}
      (\mat{C}\mat{A} - \mat{A}_{ref}\mat{C})}_{\mat{Q}} &
    \underbrace{(\mat{C}\mat{A} - \mat{A}_{ref}\mat{C})\T\mat{Q}
      (\mat{C}\mat{B})}_{\mat{N}} \\
    \underbrace{(\mat{C}\mat{B})\T\mat{Q}
      (\mat{C}\mat{A} - \mat{A}_{ref}\mat{C})}_{\mat{N}\T} &
    \underbrace{(\mat{C}\mat{B})\T\mat{Q}(\mat{C}\mat{B}) + \mat{R}}_{\mat{R}}
  \end{bsmallmatrix}
  \begin{bmatrix}
    \mat{x}_k \\
    \mat{u}_k
  \end{bmatrix}
\end{align}

Thus, implicit model following can be defined as the following optimization
problem.
\begin{theorem}[Implicit model following]
  \begin{align}
    \mat{u}_k^* = \argmin_{\mat{u}_k} &\sum_{k=0}^\infty
    \begin{bmatrix}
      \mat{x}_k \\
      \mat{u}_k
    \end{bmatrix}\T
    \begin{bmatrix}
      \mat{Q}_{imf} & \mat{N}_{imf} \\
      \mat{N}_{imf}\T & \mat{R}_{imf}
    \end{bmatrix}
    \begin{bmatrix}
      \mat{x}_k \\
      \mat{u}_k
    \end{bmatrix}
    \nonumber \\
    \text{subject to } &\mat{x}_{k+1} = \mat{A}\mat{x}_k + \mat{B}\mat{u}_k
  \end{align}
  where
  \begin{align*}
    \mat{Q}_{imf} &= (\mat{C}\mat{A} - \mat{A}_{ref}\mat{C})\T\mat{Q}
      (\mat{C}\mat{A} - \mat{A}_{ref}\mat{C}) \\
    \mat{N}_{imf} &= (\mat{C}\mat{A} - \mat{A}_{ref}\mat{C})\T\mat{Q}
      (\mat{C}\mat{B}) \\
    \mat{R}_{imf} &= (\mat{C}\mat{B})\T\mat{Q}(\mat{C}\mat{B}) + \mat{R}
  \end{align*}

  The optimal control policy $\mat{u}_k^*$ is $-\mat{K}\mat{x}_k$. $\mat{K}$ can
  be computed via the typical LQR equations based on the algebraic Ricatti
  equation.
\end{theorem}

The control law $\mat{u}_{imf,k} = -\mat{K}\mat{x}_k$ makes
$\mat{x}_{k+1} = \mat{A}\mat{x}_k + \mat{B}\mat{u}_{imf,k}$ match the open-loop
response of $\mat{z}_{k+1} = \mat{A}_{ref}\mat{z}_k$.

If the original and desired system have the same states, then
$\mat{C} = \mat{I}$ and the cost functional simplifies to
\begin{equation}
  J = \sum_{k=0}^\infty
  \begin{bmatrix}
    \mat{x}_k \\
    \mat{u}_k
  \end{bmatrix}\T
  \begin{bmatrix}
    \underbrace{(\mat{A} - \mat{A}_{ref})\T\mat{Q}
      (\mat{A} - \mat{A}_{ref})}_{\mat{Q}} &
    \underbrace{(\mat{A} - \mat{A}_{ref})\T\mat{Q}\mat{B}}_{\mat{N}} \\
    \underbrace{\mat{B}\T\mat{Q}(\mat{A} - \mat{A}_{ref})}_{\mat{N}\T} &
    \underbrace{\mat{B}\T\mat{Q}\mat{B} + \mat{R}}_{\mat{R}}
  \end{bmatrix}
  \begin{bmatrix}
    \mat{x}_k \\
    \mat{u}_k
  \end{bmatrix}
\end{equation}

\subsection{Following reference input matrix}

The feedback control law above makes the open-loop system behave like
$\mat{A}_{ref}$, but the input dynamics are still that of the original system.
Here's how to make the input dynamics behave like $\mat{B}_{ref}$. We want to
find the $\mat{u}_{imf,k}$ that makes the real model follow the reference model.
\begin{align*}
  \mat{x}_{k+1} &= \mat{A}\mat{x}_k + \mat{B}\mat{u}_{imf,k} \\
  \mat{z}_{k+1} &= \mat{A}_{ref}\mat{z}_k + \mat{B}_{ref}\mat{u}_k
\end{align*}

Let $\mat{x} = \mat{z}$.
\begin{align*}
  \mat{x}_{k+1} &= \mat{z}_{k+1} \\
  \mat{A}\mat{x}_k + \mat{B}\mat{u}_{imf,k} &= \mat{A}_{ref}\mat{x}_k +
    \mat{B}_{ref}\mat{u}_k \\
  \mat{B}\mat{u}_{imf,k} &= \mat{A}_{ref}\mat{x}_k - \mat{A}\mat{x}_k +
    \mat{B}_{ref}\mat{u}_k \\
  \mat{B}\mat{u}_{imf,k} &= (\mat{A}_{ref} - \mat{A})\mat{x}_k +
    \mat{B}_{ref}\mat{u}_k \\
  \mat{u}_{imf,k} &= \mat{B}^+ ((\mat{A}_{ref} - \mat{A})\mat{x}_k +
    \mat{B}_{ref}\mat{u}_k) \\
  \mat{u}_{imf,k} &= -\mat{B}^+ (\mat{A} - \mat{A}_{ref})\mat{x}_k +
    \mat{B}^+ \mat{B}_{ref}\mat{u}_k
\end{align*}

The first term makes the open-loop poles match that of the reference model, and
the second term makes the input behave like that of the reference model.

\section{Time delay compensation}
\index{controller design!linear-quadratic regulator!time delay compensation}

Linear-Quadratic regulator controller gains tend to be aggressive, and in the
presence of sensor delay, they may be unstable (see figure
\ref{fig:elevator_time_delay_no_comp}). However, if we know the amount of delay,
we can design the controller for a projected version of the undelayed model.
That is, we can compute the control based on where the system will be after the
time delay (see figure \ref{fig:elevator_time_delay_comp}).
\begin{bookfigure}
  \begin{minisvg}{2}{build/\chapterpath/elevator_time_delay_no_comp}
    \caption{Elevator response at 5ms sample period with 50ms of output lag
      (uncompensated controller gains)}
    \label{fig:elevator_time_delay_no_comp}
  \end{minisvg}
  \hfill
  \begin{minisvg}{2}{build/\chapterpath/elevator_time_delay_comp}
    \caption{Elevator response at 5ms sample period with 50ms of output lag
      (compensated controller gains)}
    \label{fig:elevator_time_delay_comp}
  \end{minisvg}
\end{bookfigure}

This method of delay compensation seems to work best for second-order systems.
Figure \ref{fig:drivetrain_time_delay_no_comp} shows time delay for a drivetrain
velocity system. Figure \ref{fig:drivetrain_time_delay_comp} shows that
compensating the controller gain significantly reduces the feedback gain. Mainly
feedforward is acting for a predominantly open-loop response, which has poor
disturbance rejection.
\begin{bookfigure}
  \begin{minisvg}{2}{build/\chapterpath/drivetrain_time_delay_no_comp}
    \caption{Drivetrain response at 1ms sample period with 40ms of output lag
      (uncompensated controller gain)}
    \label{fig:drivetrain_time_delay_no_comp}
  \end{minisvg}
  \hfill
  \begin{minisvg}{2}{build/\chapterpath/drivetrain_time_delay_comp}
    \caption{Drivetrain response at 1ms sample period with 40ms of output lag
      (compensated controller gain)}
    \label{fig:drivetrain_time_delay_comp}
  \end{minisvg}
\end{bookfigure}

For this controller to be optimal, we need to have a time-varying control gain
to give the system an initial kick in the right direction. Since we are
effectively controlling the state some time in the future, the state
exponentially converges to zero and so does the control gain. However, the
inputs we use to project into the future start at zero since there's no history.
This is reflected by equation (10) in the original paper
\cite{bib:lqr_time_delay}. As the input buffer fills up, the controller gain
converges to the steady-state one we're using. Fixing the source of the time
delay is always preferred in general, but especially in cases that require a
constant controller gain.

We'll show how to derive this controller gain compensation for continuous and
discrete systems.

\subsection{Continuous case}

The continuous linear system is defined as
\begin{equation*}
  \dot{\mtx{x}} = \mtx{A}\mtx{x}(t) + \mtx{B}\mtx{u}(t)
\end{equation*}

Let the controller for this system be
\begin{equation*}
  \mtx{u}(t) = -\mtx{K}\mtx{x}(t)
\end{equation*}

Substitute this into the continuous model.
\begin{align*}
  \dot{\mtx{x}} &= \mtx{A}\mtx{x}(t) + \mtx{B}\mtx{u}(t) \\
  \dot{\mtx{x}} &= \mtx{A}\mtx{x}(t) + \mtx{B}(-\mtx{K}\mtx{x}(t)) \\
  \dot{\mtx{x}} &= \mtx{A}\mtx{x}(t) - \mtx{B}\mtx{K}\mtx{x}(t) \\
  \dot{\mtx{x}} &= (\mtx{A} - \mtx{B}\mtx{K}) \mtx{x}(t)
\end{align*}

Let $L$ be the amount of time delay in seconds. Take the matrix exponential from
the current time $t$ to $L$ in the future.
\begin{equation}
  \mtx{x}(t + L) = e^{(\mtx{A} - \mtx{B}\mtx{K})L} \mtx{x}(t)
    \label{eq:continuous_advance_state_by_delay}
\end{equation}

We can avoid the time delay if we compute the control based on the plant $L$
seconds in the future using equation
\eqref{eq:continuous_advance_state_by_delay}. Therefore, the latency-compensated
controller is
\begin{align}
  \mtx{u}(t) &= -\mtx{K}\mtx{x}(t + L) \nonumber \\
  \mtx{u}(t) &= -\mtx{K} e^{(\mtx{A} - \mtx{B}\mtx{K})L} \mtx{x}(t)
\end{align}

\subsection{Discrete case}

The discrete linear system is defined as
\begin{equation*}
  \mtx{x}_{k+1} = \mtx{A}\mtx{x}_k + \mtx{B}\mtx{u}_k
\end{equation*}

Let the controller for this system be
\begin{equation*}
  \mtx{u}_k = -\mtx{K}\mtx{x}_k
\end{equation*}

Substitute this into the discrete model.
\begin{align*}
  \mtx{x}_{k+1} &= \mtx{A}\mtx{x}_k + \mtx{B}\mtx{u}_k \\
  \mtx{x}_{k+1} &= \mtx{A}\mtx{x}_k + \mtx{B}(-\mtx{K}\mtx{x}_k) \\
  \mtx{x}_{k+1} &= \mtx{A}\mtx{x}_k - \mtx{B}\mtx{K}\mtx{x}_k \\
  \mtx{x}_{k+1} &= (\mtx{A} - \mtx{B}\mtx{K}) \mtx{x}_k
\end{align*}

Let $T$ be the duration between timesteps in seconds and $L$ be the amount of
time delay in seconds. $\frac{L}{T}$ gives the number of timesteps represented
by $L$.
\begin{equation}
  \mtx{x}_{k+L} = (\mtx{A} - \mtx{B}\mtx{K})^\frac{L}{T} \mtx{x}_k
    \label{eq:discrete_advance_state_by_delay}
\end{equation}

We can avoid the time delay if we compute the control based on the plant $L$
seconds in the future using equation \eqref{eq:discrete_advance_state_by_delay}.
Therefore, the latency-compensated controller is
\begin{align}
  \mtx{u}_k &= -\mtx{K}\mtx{x}_{k+L} \nonumber \\
  \mtx{u}_k &= -\mtx{K} (\mtx{A} - \mtx{B}\mtx{K})^\frac{L}{T} \mtx{x}_k
    \label{eq:discrete_delay_comp_control_law}
\end{align}

If the delay $L$ isn't a multiple of the sample period $T$ in equation
\eqref{eq:discrete_delay_comp_control_law}, we have to evaluate a fractional
matrix power, which can be tricky. If $\mtx{A} - \mtx{B}\mtx{K}$ is
diagonalizable, we can obtain an exact answer for
$(\mtx{A} - \mtx{B}\mtx{K})^\frac{L}{T}$ by decomposing
$\mtx{A} - \mtx{B}\mtx{K}$ into $\mtx{P}\mtx{D}\mtx{P}^{-1}$ where $\mtx{D}$ is
a diagonal matrix, computing $\mtx{D}^\frac{L}{T}$ as each diagonal element
raised to $\frac{L}{T}$, then recomposing
$\mtx{P}\mtx{D}^\frac{L}{T}\mtx{P}^{-1}$. If $\mtx{A} - \mtx{B}\mtx{K}$ isn't
diagonalizable, we'll have to approximate the matrix power by rounding
$\frac{L}{T}$ to the nearest integer. This approximation gets worse as
$L \bmod T$ approaches $\frac{T}{2}$.


\section{Model augmentation}

This section will teach various tricks relating to state-space representation
aimed at demystifying the matrix algebra at play.

\section{Feedforward}

Feedback control can be effective for \gls{reference} \gls{tracking} (making a
\gls{system}'s output follow a desired \gls{reference} signal), but it's a
reactionary measure; the \gls{system} won't start applying \gls{control effort}
until the \gls{system} is already behind. If we could tell the \gls{controller}
about the desired movement and required input beforehand, the \gls{system} could
react quicker and the feedback \gls{controller} could do less work. A
\gls{controller} that feeds information forward into the \gls{plant} like this
is called a \gls{feedforward controller}.

A \gls{feedforward controller} injects information about the \gls{system}'s
dynamics (like a \gls{model} does) or the desired movement. The feedforward
handles parts of the control actions we already know must be applied to make a
\gls{system} track a \gls{reference}, then feedback compensates for what we do
not or cannot know about the \gls{system}'s behavior at runtime.

There are two types of feedforwards: model-based feedforward and feedforward for
unmodeled dynamics. The first solves a mathematical model of the system for the
inputs required to meet desired velocities and accelerations. The second
compensates for unmodeled forces or behaviors directly so the feedback
controller doesn't have to. Both types can facilitate simpler feedback
controllers; we'll cover examples of each in later chapters.

\section{Integral control}
\label{sec:integral_control}

A common way of implementing integral control is to add an additional
\gls{state} that is the integral of the \gls{error} of the variable intended to
have zero \gls{steady-state error}.

There are two drawbacks to this method. First, there is integral windup on a
unit \gls{step input}. That is, the integrator accumulates even if the
\gls{system} is \gls{tracking} the \gls{model} correctly. The second is
demonstrated by an example from Jared Russell of FRC team 254. Say there is a
position/velocity trajectory for some \gls{plant} to follow. Without integral
control, one can calculate a desired $\mtx{K}\mtx{x}$ to use as the
\gls{control input}. As a result of using both desired position and velocity,
\gls{reference} \gls{tracking} is good. With integral control added, the
\gls{reference} is always the desired position, but there is no way to tell the
controller the desired velocity.

Consider carefully whether integral control is necessary. One can get relatively
close without integral control, and integral adds all the issues listed above.
Below, it is assumed that the controls designer has determined that integral
control will be worth the inconvenience.

There are three methods FRC team 971 has used over the years:

\begin{enumerate}
  \item Augment the \gls{plant} as described earlier. For an arm, one would add
    an ``integral of position" state.
  \item Add an integrator to the output of the controller, then estimate the
    \gls{control effort} being applied. 971 has called this Delta U control. The
    upside is that it doesn't have the windup issue described above; the
    integrator only acts if the \gls{system} isn't behaving like the
    \gls{model}, which was the original intent. The downside is working with it
    is very confusing.
  \item Estimate the ``error" in the \gls{control input} (the difference between
    what was applied versus what was observed to happen) via the \gls{observer}
    and compensate for it.
\end{enumerate}

We'll present the first and third methods since the third is strictly better
than the second.

\subsection{Plant augmentation}
\index{Integral control!plant augmentation}

We want to augment the \gls{system} with an integral term that integrates the
\gls{error} $\mtx{e} = \mtx{r} - \mtx{y} = \mtx{r} - \mtx{C}\mtx{x}$.

\begin{align*}
  \mtx{x}_I &= \int \mtx{e} \,dt \\
  \dot{\mtx{x}}_I &= \mtx{e} = \mtx{r} - \mtx{C}\mtx{x}
\end{align*}

The \gls{plant} is augmented as

\begin{align*}
  \dot{\begin{bmatrix}
    \mtx{x} \\
    \mtx{x}_I
  \end{bmatrix}} &=
  \begin{bmatrix}
    \mtx{A} & \mtx{0} \\
    -\mtx{C} & \mtx{0}
  \end{bmatrix}
  \begin{bmatrix}
    \mtx{x} \\
    \mtx{x}_I
  \end{bmatrix} +
  \begin{bmatrix}
    \mtx{B} \\
    \mtx{0}
  \end{bmatrix}
  \mtx{u} +
  \begin{bmatrix}
    \mtx{0} \\
    \mtx{I}
  \end{bmatrix}
  \mtx{r} \\
  \dot{\begin{bmatrix}
    \mtx{x} \\
    \mtx{x}_I
  \end{bmatrix}} &=
  \begin{bmatrix}
    \mtx{A} & \mtx{0} \\
    -\mtx{C} & \mtx{0}
  \end{bmatrix}
  \begin{bmatrix}
    \mtx{x} \\
    \mtx{x}_I
  \end{bmatrix} +
  \begin{bmatrix}
    \mtx{B} & \mtx{0} \\
    \mtx{0} & \mtx{I}
  \end{bmatrix}
  \begin{bmatrix}
    \mtx{u} \\
    \mtx{r}
  \end{bmatrix}
\end{align*}

The controller is augmented as

\begin{align*}
  \mtx{u} &= \mtx{K} (\mtx{r} - \mtx{x}) - \mtx{K}_I\mtx{x}_I \\
  \mtx{u} &=
  \begin{bmatrix}
    \mtx{K} & \mtx{K}_I
  \end{bmatrix}
  \left(\begin{bmatrix}
    \mtx{r} \\
    \mtx{0}
  \end{bmatrix} -
  \begin{bmatrix}
    \mtx{x} \\
    \mtx{x}_I
  \end{bmatrix}\right)
\end{align*}

\subsection{U error estimation}
\label{subsec:u_error_estimation}
\index{Integral control!U error estimation}

Given the desired \gls{input} produced by a \gls{controller}, unmodeled
\glspl{disturbance} may cause the observed behavior of a \gls{system} to deviate
from its \gls{model}. U error estimation estimates the difference between the
desired \gls{input} and a hypothetical \gls{input} that makes the \gls{model}
match the observed behavior. This value can be added to the \gls{control input}
to make the \gls{controller} compensate for unmodeled \glspl{disturbance} and
make the \gls{model} better predict the \gls{system}'s future behavior.

First, we'll consider the one-dimensional case. Let $u_{error}$ be the
difference between the \gls{input} actually applied to a \gls{system} and the
desired \gls{input}. The $u_{error}$ term is then added to the \gls{system} as
follows.

\begin{equation*}
  \dot{x} = Ax + B\left(u + u_{error}\right)
\end{equation*}

$u + u_{error}$ is the hypothetical \gls{input} actually applied to the
\gls{system}.

\begin{equation*}
  \dot{x} = Ax + Bu + Bu_{error}
\end{equation*}

The following equation generalizes this to a multiple-input \gls{system}.

\begin{equation*}
  \dot{\mtx{x}} = \mtx{A}\mtx{x} + \mtx{B}\mtx{u} + \mtx{B}_{error}u_{error}
\end{equation*}

where $\mtx{B}_{error}$ is a column vector that maps $u_{error}$ to changes in
the rest of the \gls{state} the same way $\mtx{B}$ does for $\mtx{u}$.
$\mtx{B}_{error}$ is only a column of $\mtx{B}$ if $u_{error}$ corresponds to an
existing \gls{input} within $\mtx{u}$.

Given the above equation, we'll augment the \gls{plant} as

\begin{align*}
  \dot{\begin{bmatrix}
    \mtx{x} \\
    u_{error}
  \end{bmatrix}} &=
  \begin{bmatrix}
    \mtx{A} & \mtx{B}_{error} \\
    \mtx{0} & \mtx{0}
  \end{bmatrix}
  \begin{bmatrix}
    \mtx{x} \\
    u_{error}
  \end{bmatrix} +
  \begin{bmatrix}
    \mtx{B} \\
    \mtx{0}
  \end{bmatrix}
  \mtx{u} \\
  \mtx{y} &= \begin{bmatrix}
    \mtx{C} & 0
  \end{bmatrix} \begin{bmatrix}
    \mtx{x} \\
    u_{error}
  \end{bmatrix} + \mtx{D}\mtx{u}
\end{align*}

Notice how the \gls{state} is augmented with $u_{error}$. With this \gls{model},
the \gls{observer} will estimate both the \gls{state} and the $u_{error}$ term.
The controller is augmented similarly. $\mtx{r}$ is augmented with a zero for
the goal $u_{error}$ term.

\begin{align*}
  \mtx{u} &= \mtx{K} \left(\mtx{r} - \mtx{x}\right) - \mtx{k}_{error}u_{error}
    \\
  \mtx{u} &=
  \begin{bmatrix}
    \mtx{K} & \mtx{k}_{error}
  \end{bmatrix}
  \left(\begin{bmatrix}
    \mtx{r} \\
    0
  \end{bmatrix} -
  \begin{bmatrix}
    \mtx{x} \\
    u_{error}
  \end{bmatrix}\right)
\end{align*}

where $\mtx{k}_{error}$ is a column vector with a $1$ in a given row if
$u_{error}$ should be applied to that \gls{input} or a $0$ otherwise.

This process can be repeated for an arbitrary \gls{error} which can be corrected
via some linear combination of the \glspl{input}.

\section{Case study: flywheel PID control}
\index{PID control!flywheel (modern control)}

PID controllers typically control voltage to a motor in FRC independent of the
equations of motion of that motor. For position PID control, large values of
$K_p$ can lead to overshoot and $K_d$ is commonly used to reduce overshoots.
Let's consider a flywheel controlled with a standard PID controller. Why
wouldn't $K_d$ provide damping for velocity overshoots in this case?

PID control is designed to control second-order and first-order \glspl{system}
well. It can be used to control a lot of things, but struggles when given higher
order \glspl{system}. It has three degrees of freedom. Two are used to place the
two poles of the \gls{system}, and the third is used to remove steady-state
error. With higher order \glspl{system} like a one input, seven \gls{state}
\gls{system}, there aren't enough degrees of freedom to place the \gls{system}'s
poles in desired locations. This will result in poor control.

The math for PID doesn't assume voltage, a motor, etc. It defines an output
based on derivatives and integrals of its input. We happen to use it for motors
because it actually works pretty well for it because motors are second-order
\glspl{system}.

The following math will be in continuous time, but the same ideas apply to
discrete time. This is all assuming a velocity controller.

Our simple motor model hooked up to a mass is
\begin{align}
  V &= IR + \frac{\omega}{K_v} \label{eq:steady-state_error_ss_flywheel_1} \\
  \tau &= I K_t \label{eq:steady-state_error_ss_flywheel_2} \\
  \tau &= J \frac{d\omega}{dt} \label{eq:steady-state_error_ss_flywheel_3}
\end{align}

For an explanation of where these equations come from, read section
\ref{sec:dc_brushed_motor}.

First, we'll solve for $\frac{d\omega}{dt}$ in terms of $V$.

Substitute equation \eqref{eq:steady-state_error_ss_flywheel_2} into equation
\eqref{eq:steady-state_error_ss_flywheel_1}.
\begin{align}
  V &= IR + \frac{\omega}{K_v} \nonumber \\
  V &= \left(\frac{\tau}{K_t}\right) R + \frac{\omega}{K_v} \nonumber
  \intertext{Substitute in equation
    \eqref{eq:steady-state_error_ss_flywheel_3}.}
  V &= \frac{\left(J \frac{d\omega}{dt}\right)}{K_t} R + \frac{\omega}{K_v}
    \nonumber \\
  \intertext{Solve for $\frac{d\omega}{dt}$.}
  V &= \frac{J \frac{d\omega}{dt}}{K_t} R + \frac{\omega}{K_v} \nonumber \\
  V - \frac{\omega}{K_v} &= \frac{J \frac{d\omega}{dt}}{K_t} R \nonumber \\
  \frac{d\omega}{dt} &= \frac{K_t}{JR} \left(V - \frac{\omega}{K_v}\right)
    \nonumber \\
  \underbrace{\frac{d\omega}{dt}}_{\dot{\mat{x}}} &=
    \underbrace{-\frac{K_t}{JRK_v}}_{\mat{A}} \underbrace{\omega}_{\mat{x}} +
    \underbrace{\frac{K_t}{JR}}_{\mat{B}} \underbrace{V}_{\mat{u}}
\end{align}

There's one stable open-loop pole at $-\frac{K_t}{JRK_v}$. Let's try a simple P
controller.
\begin{align*}
  \mat{u} &= \mat{K} (\mat{r} - \mat{x}) \\
  V &= K_p (\omega_{goal} - \omega)
\end{align*}

Closed-loop models have the form
$\dot{\mat{x}} = (\mat{A} - \mat{B}\mat{K})\mat{x} + \mat{B}\mat{K}\mat{r}$.
Therefore, the closed-loop poles are the eigenvalues of
$\mat{A} - \mat{B}\mat{K}$.
\begin{align*}
  \dot{\mat{x}} &= (\mat{A} - \mat{B}\mat{K})\mat{x} + \mat{B}\mat{K}\mat{r}
    \\
  \dot{\omega} &= \left(\left(-\frac{K_t}{JRK_v}\right) -
    \left(\frac{K_t}{JR}\right)(K_p)\right)\omega +
    \left(\frac{K_t}{JR}\right)(K_p)(\omega_{goal}) \\
  \dot{\omega} &= -\left(\frac{K_t}{JRK_v} + \frac{K_t K_p}{JR}\right)\omega +
    \frac{K_t K_p}{JR}\omega_{goal}
\end{align*}

This closed-loop flywheel model has one pole at
$-\left(\frac{K_t}{JRK_v} + \frac{K_t K_p}{JR}\right)$. It therefore only needs
one P controller to place that pole anywhere on the real axis. A derivative
term is unnecessary on an ideal flywheel. It may compensate for unmodeled
dynamics such as accelerating projectiles slowing the flywheel down, but that
effect may also increase recovery time; $K_d$ drives the acceleration to zero in
the undesired case of negative acceleration as well as well as the actually
desired case of positive acceleration.

This analysis assumes that the motor is well coupled to the mass and that the
time constant of the inductor is small enough that it doesn't factor into the
motor equations. The latter is a pretty good assumption, as shown by the slight
wiggle in figure \ref{fig:cs_ss_highfreq_unstable_step} compared to figure
\ref{fig:cs_ss_highfreq_stable_step}. If more mass is added to the motor
armature, the response timescales increase and the inductance matters even less.
\begin{bookfigure}
  \begin{minisvg}{2}{build/figs/highfreq_unstable_step}
    \caption{Step response of second-order DC brushed motor plant augmented with
      position ($L = 230$ μH)}
    \label{fig:cs_ss_highfreq_unstable_step}
  \end{minisvg}
  \hfill
  \begin{minisvg}{2}{build/figs/highfreq_stable_step}
    \caption{Step response of first-order DC brushed motor plant augmented with
      position}
    \label{fig:cs_ss_highfreq_stable_step}
  \end{minisvg}
\end{bookfigure}

Subsection \ref{subsec:input_error_estimation} covers a superior compensation
method that avoids zeroes in the \gls{controller}, doesn't act against the
desired control action, and facilitates better \gls{tracking}.

