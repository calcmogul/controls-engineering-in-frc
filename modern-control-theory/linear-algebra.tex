\chapterimage{linear-algebra.jpg}{Grass clearing by Interdisciplinary Sciences building and Thimann Labs at UCSC}

\chapter{Linear algebra}

Modern control theory borrows concepts from linear algebra. At first, linear
algebra may appear very abstract, but there are simple geometric intuitions
underlying it. We'll be linking to 3Blue1Brown's
\href{https://www.3blue1brown.com/topics/linear-algebra}{\textit{Essence of
linear algebra}} video series because it's better at conveying that intuition
than static text.
\begin{bookfigure}
  \qrcode{https://www.3blue1brown.com/lessons/eola-preview} \\
  ``Essence of linear algebra preview" (5 minutes) \\
  \footnotesize 3Blue1Brown \\
  \url{https://www.3blue1brown.com/lessons/eola-preview}
\end{bookfigure}

\renewcommand*{\chapterpath}{\partpath/linear-algebra}
\section{Vectors}
\index{linear algebra!vectors}

Watch the ``Vectors, what even are they?" video from 3Blue1Brown's
\textit{Essence of linear algebra} series (5 minutes) or read the associated
lesson on 3blue1brown.com
\cite{bib:3b1b_linalg_vectors}.

\section{Linear combinations, span, and basis vectors}
\index{linear algebra!basis vectors}
\index{linear algebra!linear combination}

Watch the ``Linear combination, span, and basis vectors" video from
3Blue1Brown's \textit{Essence of linear algebra} series (10 minutes)
\cite{bib:linalg_linear_combinations}.

\section{Linear transformations and matrices}

This section focuses on what linear transformations look like in the case of two
dimensions and how they relate to the idea of a matrix-vector multiplication.

\begin{equation*}
  \begin{bmatrix}
    \textcolor{red}{1} & \textcolor{orange}{-3} \\
    \textcolor{green}{2} & \textcolor{cyan}{4}
  \end{bmatrix}
  \begin{bmatrix}
    \textcolor{blue}{5} \\
    \textcolor{purple}{7}
  \end{bmatrix} = \begin{bmatrix}
    (\textcolor{red}{1})(\textcolor{blue}{5}) +
      (\textcolor{orange}{-3})(\textcolor{purple}{7}) \\
    (\textcolor{green}{2})(\textcolor{blue}{5}) +
      (\textcolor{cyan}{4})(\textcolor{purple}{7})
  \end{bmatrix}
\end{equation*}

In particular, we want to show you a way to think about matrix-vector
multiplication that doesn't rely on memorization of the procedure shown above.

\subsection{What is a linear transformation?}
\index{Matrices!linear transformation}

To start, let's just parse this term ``linear transformation". ``Transformation"
is essentially another name for ``function". It's something that takes in inputs
and returns an output for each one. Specifically in the context of linear
algebra, we consider transformations that take in some vector and spit out
another vector.

\begin{bookfigure}
  \begin{tikzpicture}[auto, >=latex']
    % Place the nodes
    \node [name=input] {
      $\begin{bmatrix}
        5 \\
        7
      \end{bmatrix}$
    };
    \node [name=inputlabel, below=of input] {Vector input};
    \node [name=func, right=of input] {$L(\vec{v})$};
    \node [name=output, right=of func] {
      $\begin{bmatrix}
        2 \\
        -3
      \end{bmatrix}$
    };
    \node [name=outputlabel, below=of output] {Vector output};

    % Connect the nodes
    \draw [arrow] (input) -- node {} (func);
    \draw [arrow] (func) -- node {} (output);
  \end{tikzpicture}
\end{bookfigure}

So why use the word ``transformation" instead of ``function" if they mean the
same thing? It's to be suggestive of a certain way to visualize this
input-output relation. You see, a great way to understand functions of vectors
is to use movement. If a transformation takes some input vector to some output
vector, we imagine that input vector moving over to the output vector. Then to
understand the transformation as a whole, we might imagine watching every
possible input vector move over to its corresponding output vector. it gets
really crowded to think about all the vectors all at once, where each one is an
arrow. Therefore, as we mentioned in the previous section, it's useful to
conceptualize each vector as a single point where its tip sits rather than an
arrow. To think about a transformation taking every possible input vector to
some output vector, we watch every point in space moving to some other point.

The effect of various transformations moving around all of the points in space
gives the feeling of compressing and morphing space itself. As you can imagine
though, arbitrary transformations can look complicated. Luckily, linear algebra
limits itself to a special type of transformation, ones that are easier to
understand, called ``linear" transformations. Visually speaking, a
transformation is linear if it has two properties: all straight lines must
remain as such, and the origin must remain fixed in place. In general, you
should think of linear transformations as keeping grid lines parallel and evenly
spaced.

\subsection{Describing transformations numerically}

Some transformations are simple to think about, like rotations about the origin.
Others are more difficult to describe with words. So how could one describe
these transformations numerically? If you were, say, programming some animations
to make a video teaching the topic, what formula could you give the computer so
that if you give it the coordinates of a vector, it would return the coordinates
of where that vectors lands?

\begin{bookfigure}
  \begin{tikzpicture}[auto, >=latex']
    % Place the nodes
    \node [name=input] {
      $\begin{bmatrix}
        x_{in} \\
        y_{in}
      \end{bmatrix}$
    };
    \node [name=func, right=of input] {$????$};
    \node [name=output, right=of func] {
      $\begin{bmatrix}
        x_{out} \\
        y_{out}
      \end{bmatrix}$
    };

    % Connect the nodes
    \draw [arrow] (input) -- node {} (func);
    \draw [arrow] (func) -- node {} (output);
  \end{tikzpicture}
\end{bookfigure}

You only need to record where the two basis vectors, $\hat{i}$ and $\hat{j}$,
each land, and everything else will follow from that. For example, consider the
vector $v$ with coordinates $(-1, 2)$, meaning $\vec{v} = -1\hat{i} + 2\hat{j}$.
If we play some transformation and follow where all three of these vectors go,
the property that grid lines remain parallel and evenly spaced has a really
important consequence: the place where $\vec{v}$ lands will be $-1$ times the
vector where $\hat{i}$ landed plus $2$ times the vector where $\hat{j}$ landed.
In other words, it started off as a certain linear combination of $\hat{i}$ and
$\hat{j}$ and it ends up as that same linear combination of where those two
vectors landed. This means you can deduce where $\vec{v}$ must go based only on
where $\hat{i}$ and $\hat{j}$ each land. For this transformation, $\hat{i}$
lands on the coordinates $(1, -2)$ and $\hat{j}$ lands on the x-axis at the
coordinates $(3, 0)$.

\begin{align*}
  \text{Transformed } \vec{v} &= -1(\text{Transformed } \hat{i}) +
    2(\text{Transformed } \hat{j}) \\
  \text{Transformed } \vec{v} &= -1\begin{bmatrix}
    1 \\
    -2
  \end{bmatrix} + 2\begin{bmatrix}
    3 \\
    0
  \end{bmatrix}
\end{align*}

Adding that all together, you can deduce that $\vec{v}$ has to land on the
vector $(5, 2)$.

\begin{align*}
  \text{Transformed } \vec{v} &= \begin{bmatrix}
    -1(1) + 2(3) \\
    -1(-2) + 2(0)
  \end{bmatrix} \\
  \text{Transformed } \vec{v} &= \begin{bmatrix}
    5 \\
    2
  \end{bmatrix}
\end{align*}

This is a good point to pause and ponder, because it's pretty important. This
gives us a technique to deduce where any vectors land, so long as we have a
record of where $\hat{i}$ and $\hat{j}$ each land, without needing to watch the
transformation itself.

Given a vector with more general coordinates $x$ and $y$, it will land on $x$
times the vector where $\hat{i}$ lands $(1, -2)$, plus $y$ times the vector
where $\hat{j}$ lands $(3, 0)$. Carrying out that sum, you see that it lands at
$(1x + 3y, -2x + 0y)$.

\begin{equation*}
  \begin{array}{cc}
    \hat{i} \rightarrow \begin{bmatrix}
      1 \\
      -2
    \end{bmatrix} &
    \hat{j} \rightarrow \begin{bmatrix}
      3 \\
      0
    \end{bmatrix}
  \end{array}
\end{equation*}

\begin{equation*}
  \begin{bmatrix}
    x \\
    y
  \end{bmatrix} \rightarrow x\begin{bmatrix}
    1 \\
    -2
  \end{bmatrix} + y\begin{bmatrix}
    3 \\
    0
  \end{bmatrix} = \begin{bmatrix}
    1x + 3y \\
    -2x + 0y
  \end{bmatrix}
\end{equation*}

Given any vector, this formula will describe where that vector lands.

What all of this is saying is that a two dimensional linear transformation is
completely described by just four numbers: the two coordinates for where
$\hat{i}$ lands and the two coordinates for where $\hat{j}$ lands. It's common
to package these coordinates into a two-by-two grid of numbers, called a
two-by-two matrix, where you can interpret the columns as the two special
vectors where $\hat{i}$ and $\hat{j}$ each land. If $\hat{i}$ lands on the
vector $(3, -2)$ and $\hat{j}$ lands on the vector $(2, 1)$, this two-by-two
matrix would be

\begin{equation*}
  \begin{bmatrix}
    3 & 2 \\
    -2 & 1
  \end{bmatrix}
\end{equation*}

If you're given a two-by-two matrix describing a linear transformation and some
specific vector, say $(5, 7)$, and you want to know where that linear
transformation takes that vector, you can multiply the coordinates of the vector
by the corresponding columns of the matrix, then add together the result.

\begin{equation*}
  \begin{bmatrix}
    3 & 2 \\
    -2 & 1
  \end{bmatrix}
  \begin{bmatrix}
    5 \\
    7
  \end{bmatrix} = 5\begin{bmatrix}
    3 \\
    -2
  \end{bmatrix} + 7\begin{bmatrix}
    2 \\
    1
  \end{bmatrix}
\end{equation*}

This corresponds with the idea of adding the scaled versions of our new basis
vectors.

Let's see what this looks like in the most general case where your matrix has
entries $a$, $b$, $c$, $d$.

\begin{equation*}
  \begin{bmatrix}
    a & b \\
    c & d
  \end{bmatrix}
\end{equation*}

Remember, this matrix is just a way of packaging the information needed to
describe a linear transformation. Always remember to interpret that first
column, $(a, c)$, as the place where the first basis vector lands and that
second column, $(b, d)$, as the place where the second basis vector lands.

When we apply this transformation to some vector $(x, y)$, the result will be
$x$ times $(a, c)$ plus $y$ times $(b, d)$. Together, this gives a vector
$(ax + by, cx + dy)$.

\begin{equation*}
  \begin{bmatrix}
    a & b \\
    c & d
  \end{bmatrix} \begin{bmatrix}
    x \\
    y
  \end{bmatrix} = x\begin{bmatrix}
    a \\
    c
  \end{bmatrix} + y\begin{bmatrix}
    b \\
    d
  \end{bmatrix} = \begin{bmatrix}
    ax + by \\
    cx + dy
  \end{bmatrix}
\end{equation*}

You could even define this as matrix-vector multiplication when you put the
matrix on the left of the vector like it's a function. Then, you could make high
schoolers memorize this, without showing them the crucial part that makes it
feel intuitive (yes, that was sarcasm). Isn't it more fun to think about these
columns as the transformed versions of your basis vectors and to think about the
result as the appropriate linear combination of those vectors?

\subsection{Examples of linear transformations}

Let's practice describing a few linear transformations with matrices. For
example, if we rotate all of space $90\degree$ counterclockwise then $\hat{i}$
lands on the coordinates $(0, 1)$ and $\hat{j}$ lands on the coordinates
$(-1, 0)$. So the matrix we end up with has the columns $(0, 1)$, $(-1, 0)$.

\begin{equation*}
  \begin{bmatrix}
    0 & -1 \\
    1 & 0
  \end{bmatrix}
\end{equation*}

To ascertain what happens to any vector after a $90\degree$ rotation, you could
just multiply its coordinates by this matrix.

\begin{equation*}
  \begin{bmatrix}
    0 & -1 \\
    1 & 0
  \end{bmatrix} \begin{bmatrix}
    x \\
    y
  \end{bmatrix}
\end{equation*}

Here's a fun transformation with a special name, called a ``shear". In it,
$\hat{i}$ remains fixed so the first column of the matrix is $(1, 0)$, but
$\hat{j}$ moves over to the coordinates $(1, 1)$ which become the second column
of the matrix.

\begin{equation*}
  \begin{bmatrix}
    1 & 1 \\
    0 & 1
  \end{bmatrix}
\end{equation*}

And, at the risk of being redundant here, figuring out how shear transforms a
given vector comes down to multiplying this matrix by that vector.

\begin{equation*}
  \begin{bmatrix}
    1 & 1 \\
    0 & 1
  \end{bmatrix} \begin{bmatrix}
    x \\
    y
  \end{bmatrix}
\end{equation*}

Let's say we want to go the other way around, starting with a matrix, say with
columns $(1, 2)$ and $(3, 1)$, and we want to deduce what its transformation
looks like. Pause and take a moment to see if you can imagine it.

\begin{equation*}
  \begin{bmatrix}
    1 & 3 \\
    2 & 1
  \end{bmatrix}
\end{equation*}

One way to do this is to first move $\hat{i}$ to $(1, 2)$. Then, move $\hat{j}$
to $(3, 1)$, always moving the rest of space in such a way that that keeps grid
lines parallel and evenly spaced.

Suppose that the vectors that $\hat{i}$ and $\hat{j}$ land on are linearly
dependent as in the following matrix (that is, it has linearly dependent
columns).

\begin{equation*}
  \begin{bmatrix}
    2 & -2 \\
    1 & -1
  \end{bmatrix}
\end{equation*}

If you recall from last section, this means that one vector is a scaled version
of the other, so that linear transformation compresses all of 2D space onto the
line where those two vectors sit. This is also known as the one-dimensional span
of those two linearly dependent vectors.

To sum up, linear transformations are a way to move around space such that the
grid lines remain parallel and evenly spaced and such that the origin remains
fixed. Delightfully, these transformations can be described using only a handful
of numbers: the coordinates of where each basis vector lands. Matrices give us a
language to describe these transformations where the columns represent those
coordinates and matrix-vector multiplication is just a way to compute what that
transformation does to a given vector. The important take-away here is that
every time you see a matrix, you can interpret it as a certain transformation of
space. Once you really digest this idea, you're in a great position to
understand linear algebra deeply. Almost all of the topics coming up, from
matrix multiplication to determinants, eigenvalues, etc. will become easier to
understand once you start thinking about matrices as transformations of space.

\begin{remark}
  See the corresponding \textit{Essence of linear algebra} video for a more
  visual presentation (11 minutes)
  \cite{bib:linalg_linear_transformations_and_matrices}.
\end{remark}

\section{Matrix multiplication as composition}
\index{Matrices!multiplication}

Often-times you find yourself wanting to describe the effect of applying one
transformation and then another. For example, you may want to describe what
happens when you first rotate the plane $90\degree$ counterclockwise then apply
a shear. The overall effect here, from start to finish, is another linear
transformation distinct from the rotation and the shear. This new linear
transformation is commonly called the ``composition" of the two separate
transformations we applied, and like any linear transformation, it can be
described with a matrix all its own by following $\hat{i}$ and $\hat{j}$. In
this example, the ultimate landing spot for $\hat{i}$ after both transformations
is $(1, 1)$, so that's the first column of the matrix. Likewise, $\hat{j}$
ultimately ends up at the location $(-1, 0)$, so we make that the second column
of the matrix.

\begin{equation*}
  \begin{bmatrix}
    1 & -1 \\
    1 & -0
  \end{bmatrix}
\end{equation*}

This new matrix captures the overall effect of applying a rotation then a shear
but as one single action rather than two successive ones.

Here's one way to think about that new matrix: if you were to feed some vector
through the rotation then the shear, the long way to compute where it ends up is
to, first, multiply it on the left by the rotation matrix; then, take whatever
you get and multiply that on the left by the shear matrix.

\begin{equation*}
  \begin{bmatrix}
    1 & 1 \\
    0 & 1
  \end{bmatrix}\left(
  \begin{bmatrix}
    0 & -1 \\
    1 & 0
  \end{bmatrix}
  \begin{bmatrix}
    x \\
    y
  \end{bmatrix}\right)
\end{equation*}

This is, numerically speaking, what it means to apply a rotation then a shear to
a given vector, but the result should be the same as just applying this new
composition matrix we found to that same vector. This applies to any vector
because this new matrix is supposed to capture the same overall effect as the
rotation-then-shear action.

\begin{equation*}
  \begin{bmatrix}
    1 & 1 \\
    0 & 1
  \end{bmatrix}\left(
  \begin{bmatrix}
    0 & -1 \\
    1 & 0
  \end{bmatrix}
  \begin{bmatrix}
    x \\
    y
  \end{bmatrix}\right) =
  \begin{bmatrix}
    1 & -1 \\
    1 & 0
  \end{bmatrix} \begin{bmatrix}
    x \\
    y
  \end{bmatrix}
\end{equation*}

Based on how things are written down here, it's reasonable to call this new
matrix the ``product" of the original two matrices.

\begin{equation*}
  \begin{bmatrix}
    1 & 1 \\
    0 & 1
  \end{bmatrix}
  \begin{bmatrix}
    0 & -1 \\
    1 & 0
  \end{bmatrix} =
  \begin{bmatrix}
    1 & -1 \\
    1 & 0
  \end{bmatrix}
\end{equation*}

We can think about how to compute that product more generally in just a moment,
but it's way too easy to get lost in the forest of numbers. Always remember
that multiplying two matrices like this has the geometric meaning of applying
one transformation then another.

One oddity here is that we are reading the transformations from right to left;
you first apply the transformation represented by the matrix on the right, then
you apply the transformation represented by the matrix on the left. This stems
from function notation, since we write functions on the left of variables, so
every time you compose two functions, you always have to read it right to left.

Let's look at another example. Take the matrix with columns $(1, 1)$ and
$(-2, 0)$.

\begin{equation*}
  M_1 = \begin{bmatrix}
    1 & -2 \\
    1 & 0
  \end{bmatrix}
\end{equation*}

Next, take the matrix with columns $(0, 1)$ and $(2, 0)$.

\begin{equation*}
  M_2 = \begin{bmatrix}
    0 & 2 \\
    1 & 0
  \end{bmatrix}
\end{equation*}

The total effect of applying $M_1$ then $M_2$ gives us a new transformation, so
let's find its matrix. First, we need to determine where $\hat{i}$ goes. After
applying $M_1$, the new coordinates of $\hat{i}$, by definition, are given by
that first column of $M_1$, namely, $(1, 1)$. To see what happens after applying
$M_2$, multiply the matrix for $M_2$ by that vector $(1, 1)$. Working it out the
way described in the last section, you'll get the vector $(2, 1)$.

\begin{equation*}
  \begin{bmatrix}
    0 & 2 \\
    1 & 0
  \end{bmatrix}
  \begin{bmatrix}
    1 \\
    1
  \end{bmatrix} = 1
  \begin{bmatrix}
    0 \\
    1
  \end{bmatrix} + 1
  \begin{bmatrix}
    2 \\
    0
  \end{bmatrix} =
  \begin{bmatrix}
    2 \\
    1
  \end{bmatrix}
\end{equation*}

This will be the first column of the composition matrix. Likewise, to follow
$\hat{j}$, the second column of $M_1$ tells us that it first lands on $(-2, 0)$.
Then, when we apply $M_2$ to that vector, you can work out the matrix-vector
product to get $(0, -2)$.

\begin{equation*}
  \begin{bmatrix}
    0 & 2 \\
    1 & 0
  \end{bmatrix}
  \begin{bmatrix}
    -2 \\
    0
  \end{bmatrix} = -2
  \begin{bmatrix}
    0 \\
    1
  \end{bmatrix} + 0
  \begin{bmatrix}
    2 \\
    0
  \end{bmatrix} =
  \begin{bmatrix}
    0 \\
    -2
  \end{bmatrix}
\end{equation*}

This will be the second column of the composition matrix.

\begin{equation*}
  \begin{bmatrix}
    0 & 2 \\
    1 & 0
  \end{bmatrix}
  \begin{bmatrix}
    1 & -2 \\
    1 & 0
  \end{bmatrix} =
  \begin{bmatrix}
    2 & 0 \\
    1 & -2
  \end{bmatrix}
\end{equation*}

\subsection{General matrix multiplication}

Let's go through that same process again, but this time, we'll use variable
entries in each matrix, just to show that the same line of reasoning works for
any matrices. This is more symbol-heavy, but it should be pretty satisfying for
anyone who has previously been taught matrix multiplication the more rote way.

\begin{equation*}
  \begin{bmatrix}
    a & b \\
    c & d
  \end{bmatrix}
  \begin{bmatrix}
    e & f \\
    g & h
  \end{bmatrix} =
  \begin{bmatrix}
    ? & ? \\
    ? & ?
  \end{bmatrix}
\end{equation*}

To follow where $\hat{i}$ goes, start by looking at the first column of the
matrix on the right, since this is where $\hat{i}$ initially lands. Multiplying
that column by the matrix on the left is how you can tell where the intermediate
version of $\hat{i}$ ends up after applying the second transformation.

\begin{equation*}
  \begin{bmatrix}
    a & b \\
    c & d
  \end{bmatrix}
  \begin{bmatrix}
    e \\
    g
  \end{bmatrix} = e
  \begin{bmatrix}
    a \\
    c
  \end{bmatrix} + g
  \begin{bmatrix}
    b \\
    d
  \end{bmatrix} =
  \begin{bmatrix}
    ae + bg \\
    ce + dg
  \end{bmatrix}
\end{equation*}

So the first column of the composition matrix will always equal the left matrix
times the first column of the right matrix. Likewise, $\hat{j}$ will always
initially land on the second column of the right matrix, so multiplying by this
second column will give its final location, and hence, that's the second column
of the composition matrix.

\begin{equation*}
  \begin{bmatrix}
    a & b \\
    c & d
  \end{bmatrix}
  \begin{bmatrix}
    f \\
    h
  \end{bmatrix} = f
  \begin{bmatrix}
    a \\
    c
  \end{bmatrix} + h
  \begin{bmatrix}
    b \\
    d
  \end{bmatrix} =
  \begin{bmatrix}
    af + bh \\
    cf + dh
  \end{bmatrix}
\end{equation*}

So the complete composition matrix is

\begin{equation*}
  \begin{bmatrix}
    a & b \\
    c & d
  \end{bmatrix}
  \begin{bmatrix}
    e & f \\
    g & h
  \end{bmatrix} =
  \begin{bmatrix}
    ae + bg & af + bh \\
    ce + dg & cf + dh
  \end{bmatrix}
\end{equation*}

Notice there's a lot of symbols here, and it's common to be taught this formula
as something to memorize along with a certain algorithmic process to help
remember it. Before memorizing that process, you should get in the habit of
thinking about what matrix multiplication really represents: applying one
transformation after another. This will give you a much better conceptual
framework that makes the properties of matrix multiplication much easier to
understand.

\subsection{Matrix multiplication associativity}

For example, here's a question: does it matter what order we put the two
matrices in when we multiply them? Let's think through a simple example. Take a
shear which fixes $\hat{i}$ and moves $\hat{j}$ over to the right, and a
$90\degree$ rotation. If you first do the shear then rotate, we can see that
$\hat{i}$ ends up at $(0, 1)$ and $\hat{j}$ ends up at $(-1, 1)$. Both are
generally pointing close together. If you first rotate then do the shear,
$\hat{i}$ ends up over at $(1, 1)$ and $\hat{j}$ is off on a different direction
at $(-1, 0)$ and they're pointing farther apart. The overall effect here is
clearly different, so evidently, order totally does matter. Notice by thinking
in terms of transformations, that's the kind of thing that you can do in your
head by visualizing. No matrix multiplication necessary.

Let's consider trying to prove that matrix multiplication is associative. This
means that if you have three matrices $A$, $B$, and $C$, and you multiply them
all together, it shouldn't matter if you first compute $A$ times $B$ then
multiply the result by $C$, or if you first multiply $B$ times $C$ then multiply
that result by $A$ on the left. In other words, it doesn't matter where you put
the parenthesis.

\begin{equation*}
  (AB)C \stackrel{?}{=} A(BC)
\end{equation*}

If you try to work through this numerically, it's horrible, and unenligthening
for that matter. However, when you think about matrix multiplication as applying
one transformation after another, this property is just trivial. Can you see
why? What it's saying is that if you first apply $C$ then $B$, then $A$, it's
the same as applying $C$, then $B$ then $A$. There's nothing to prove, you're
just applying the same three things one after the other all in the same order.
This might feel like cheating, but it's not. This is a valid proof that matrix
multiplication is associative, and even better than that, it's a good
explanation for why that property should be true.

\begin{remark}
  See the corresponding \textit{Essence of Linear Algebra} video for a more
  visual presentation (10 minutes)
  \cite{bib:linalg_matrix_multiplication_as_composition}.
\end{remark}

\section{The determinant}
\index{Matrices!determinant}

So, moving forward, we will be assuming you have a visual understanding of
linear transformations and how they're represented with matrices.

\subsection{Scaling areas}

If you think about a couple linear transformations, you might notice how some of
them seem to stretch space out while others compress it. It's useful for
understanding these transformations to measure exactly how much it stretches or
compresses things (more specifically, to measure the factor by which areas are
scaled). For example, look at the matrix with columns $(3, 0)$, and $(0, 2)$.

\begin{equation*}
  \begin{bmatrix}
    3 & 0 \\
    0 & 2
  \end{bmatrix}
\end{equation*}

It scales $\hat{i}$ by a factor of $3$ and scales $\hat{j}$ by a factor of $2$.
Now, if we focus our attention on the one-by-one square whose bottom sits on
$\hat{i}$ and whose left side sits on $\hat{j}$, after the transformation, this
turns into a $2$ by $3$ rectangle. Since this region started out with area $1$
and ended up with area $6$, we can say the linear transformation has scaled its
area by a factor of $6$. Compare that to a shear whose matrix has columns
$(1, 0)$ and $(1, 1)$ meaning $\hat{i}$ stays in place and $\hat{j}$ moves over
to $(1, 1)$.

\begin{equation*}
  \begin{bmatrix}
    1 & 1 \\
    0 & 1
  \end{bmatrix}
\end{equation*}

That same unit square determined by $\hat{i}$ and $\hat{j}$ gets slanted and
turned into a parallelogram, but the area of that parallelogram is still $1$
since its base and height each continue to each have length $1$. Even though
this transformation pushes things about, it seems to leave areas unchanged (at
least in the case of that one uint square).

Actually though, if you know how much the area of that one single unit square
changes, you can know how the area of any possible region in space changes.
First off, notice that whatever happens to one square in the grid has to happen
to any other square in the grid no matter the size. This follows from the fact
that grid lines remain parallel and evenly spaced. Then, any shape that's not a
grid square can be approximated by grid squares pretty well with arbitrarily
good approximations if you use small enough grid squares. So, since the areas of
all those tiny grid squares are being scaled by some single amount, the area of
the shape as a whole will also be scaled by that same single amount.

\subsection{Exploring the determinant}

This special scaling factor, the factor by which a linear transformation changes
any area, is called the \textit{determinant} of that transformation. We'll show
how to compute the determinant of a transformation using its matrix later on,
but understanding what it represents is much more important than the
computation. For example, the determinant of a transformation would be $3$ if
that transformation increases the area of the region by a factor of $3$.

\begin{equation*}
  det\left(\begin{bmatrix}
    0 & 2 \\
    -1.5 & 1
  \end{bmatrix}\right) = 3
\end{equation*}

The determinant of a matrix is commonly denoted by vertical bars instead of
square brackets.

\begin{equation*}
  \begin{vmatrix}
    0 & 2 \\
    -1.5 & 1
  \end{vmatrix} = 3
\end{equation*}

The determinant of a transformation would be $\frac{1}{2}$ if it compresses all
areas by a factor of $\frac{1}{2}$.

\begin{equation*}
  \begin{vmatrix}
    0.5 & 0.5 \\
    -0.5 & 0.5
  \end{vmatrix} = 0.5
\end{equation*}

The determinant of a 2D transformation is zero if it compresses all of space
onto a line or even onto a single point since then, the area of any region would
become zero.

\begin{equation*}
  \begin{vmatrix}
    4 & 2 \\
    2 & 1
  \end{vmatrix} = 0
\end{equation*}

That last example proved to be pretty important. It means checking if the
determinant of a given matrix is zero will give a way of computing whether the
transformation associated with that matrix compresses everything into a smaller
dimension.

This analogy so far isn't quite right. The full concept of a determinant allows
for negative values.

\begin{equation*}
  \begin{vmatrix}
    1 & 2 \\
    3 & 4
  \end{vmatrix} = -2
\end{equation*}

What would scaling an area by a negative amount even mean? This has to do with
the idea of orientation. A 2D transformation with a negative determinant
essentially flips space over. Any transformations that do this are said to
``invert the orientation of space". Another way to think about it is in terms of
$\hat{i}$ and $\hat{j}$. In their starting positions, $\hat{j}$ is to the left
of $\hat{i}$. If, after a transformation, $\hat{j}$ is now on the right side of
$\hat{i}$, the orientation of space has been inverted. Whenever this happens,
the determinant will be negative. The absolute value of the determinant still
tells you the factor by which areas have been scaled.

For example, the matrix with columns $(1, 1)$ and $(2, -1)$ encodes a
transformation that has determinant $-3$.

\begin{equation*}
  \begin{vmatrix}
    1 & 2 \\
    1 & -1
  \end{vmatrix} = -3
\end{equation*}

This means that space gets flipped over and areas are scaled by a factor of $3$.

Why would this idea of a negative area scaling factor be a natural way to
describe orientation-flipping? Think about the series of transformations you get
by slowly letting $\hat{i}$ rotate closer and closer to $\hat{j}$. As $\hat{i}$
gets closer, all the areas in space are getting compressed more and more meaning
the determinant approaches zero. Once $\hat{i}$ lines up perfectly with
$\hat{j}$, the determinant is zero. Then, if $\hat{i}$ continues, doesn't it
feel natural for the determinant to keep decreasing into negative numbers?

\subsection{The determinant in 3D}

That's the understanding of determinants in two dimensions. What should it mean
for three dimensions? The determinant of a $3 \times 3$ matrix tells you how
much volumes get scaled. A determinant of zero would mean that all of space is
compressed onto something with zero volume meaning either a flat plane, a line,
or in the most extreme case, a single point. This means that the columns of the
matrix are linearly dependent.

What should negative determinants mean for three dimensions? One way to describe
orientation in 3D is with the right-hand rule. Point the forefinger of your
right hand in the direction if $\hat{i}$, stick out your middle finger in the
direction of $\hat{j}$, and notice how when you point your thumb up, it is in
the direction of $\hat{k}$. If you can still do that after the transformation,
orientation has not changed and the determinant is positive. Otherwise, if after
the transformation it only makes sense to do that with your left hand,
orientation has been flipped and the determinant is negative.

\subsection{Computing the determinant}

How do you actually compute the determinant? For a $2 \times 2$ matrix with
entries $a$, $b$, $c$, $d$, the formula is as follows.

\begin{equation*}
  \begin{vmatrix}
    a & b \\
    c & d
  \end{vmatrix} = ad - bc
\end{equation*}

Here's part of an intution for where this formula comes from. Let's say that the
terms $b$ and $c$ were both zero. Then, the term $a$ tells you how much
$\hat{i}$ is stretched in the x-direction and the term $d$ tells you how much
$\hat{j}$ is stretched in the y-direction. Since those other terms are zero, it
should make sense that $ad$ gives the area of the rectangle that the unit square
turns into. Even if only one of $b$ or $c$ are zero, you'll have a parallelogram
with a base of $a$ and a height $d$, so the area should still be $ad$. Loosely
speaking, if both $b$ and $c$ are nonzero, then that $bc$ term tells you how
much this parallelogram is stretched or compressed in the diagonal direction.

If you feel like computing determinants by hand is something that you need to
know (you won't for this book), the only way to get it down is to just practice
it with a few. This is all triply true for 3D determinants. There is a formula,
and if you feel like that's something you need to know, you should practice with
a few matrices.

\begin{equation*}
  \begin{vmatrix}
    a & b & c \\
    d & e & f \\
    g & h & i
  \end{vmatrix} =
  a \begin{vmatrix}
    e & f \\
    h & i
  \end{vmatrix}
  - b \begin{vmatrix}
    d & f \\
    g & i
  \end{vmatrix}
  + c \begin{vmatrix}
    d & e \\
    g & h
  \end{vmatrix}
\end{equation*}

We don't think those computations fall within the essence of linear algebra, but
understanding what the determinant represents falls within that essence.

\begin{remark}
  See the corresponding \textit{Essence of Linear Algebra} video for a more
  visual presentation (10 minutes) \cite{bib:linalg_the_determinant}.
\end{remark}

\section{Inverse matrices, column space, and null space}
\index{matrices!linear systems}
\index{matrices!inverse}
\index{matrices!rank}

Watch the ``Inverse matrices, column space, and null space" video from
3Blue1Brown's \textit{Essence of linear algebra} series (12 minutes)
\cite{bib:linalg_inverse_matrices_column_space_and_null_space}.

\section{Nonsquare matrices}
\begin{bookfigure}
  \qrcode{https://www.3blue1brown.com/lessons/nonsquare-matrices} \\
  ``Nonsquare matrices as transformations between dimensions" (4 minutes) \\
  \footnotesize 3Blue1Brown \\
  \url{https://www.3blue1brown.com/lessons/nonsquare-matrices}
\end{bookfigure}

\section{Eigenvectors and eigenvalues}
\index{matrices!eigenvalues}
\begin{bookfigure}
  \qrcode{https://www.3blue1brown.com/lessons/eigenvalues} \\
  ``Eigenvectors and eigenvalues" (17 minutes) \\
  \footnotesize 3Blue1Brown \\
  \url{https://www.3blue1brown.com/lessons/eigenvalues}
\end{bookfigure}

\section{Miscellaneous notation and operators}

This book works with two-dimensional matrices in the sense that they only have
rows and columns. The dimensionality of these matrices is specified by row
first, then column. For example, a matrix with two rows and three columns would
be a two-by-three matrix. A square matrix has the same number of rows as
columns. Matrices commonly use capital letters while vectors use lowercase
letters.

\subsection{Special constant matrices}

$\mat{I}$ is the identity matrix, a typically square matrix with ones along its
diagonal and zeroes elsewhere. $\mat{0}$ is a matrix filled with zeroes and
$\mat{1}$ is a matrix filled with ones. An optional subscript ${m \times n}$
denotes the matrix having $m$ rows and $n$ columns.
\begin{equation*}
  \mat{I}_{3 \times 3} =
  \begin{bmatrix}
    1 & 0 & 0 \\
    0 & 1 & 0 \\
    0 & 0 & 1
  \end{bmatrix}
  \quad
  \mat{0}_{3 \times 2} =
  \begin{bmatrix}
    0 & 0 \\
    0 & 0 \\
    0 & 0
  \end{bmatrix}
  \quad
  \mat{1}_{3 \times 2} =
  \begin{bmatrix}
    1 & 1 \\
    1 & 1 \\
    1 & 1
  \end{bmatrix}
\end{equation*}

\subsection{Operators}

\subsubsection{Transpose}
\index{matrices!transpose}
The $\T$ in $\mat{A}\T$ denotes transpose, which flips the matrix across its
diagonal such that the rows become columns and vice versa.

A symmetric matrix is equal to its transpose.

\subsubsection{Pseudoinverse}
\index{matrices!pseudoinverse}
The $^+$ in $\mat{B}^+$ denotes the Moore-Penrose pseudoinverse given by
$\mat{B}^+ = (\mat{B}\T\mat{B})^{-1}\mat{B}\T$. The pseudoinverse is used when
the matrix is nonsquare and thus not invertible to produce a close approximation
of an inverse in the least squares sense.

\subsubsection{Diagonal}
\index{matrices!diagonal}
A diagonal matrix has elements along its diagonal and zeroes elsewhere (e.g.,
the identity matrix). Let
$\mat{x} = \begin{bmatrix}x_1 & \ldots & x_n\end{bmatrix}\T$.
\begin{equation*}
  \diag(\mat{x}) = \diag(x_1,\, \ldots,\, x_n) =
  \begin{bmatrix}
    x_1 & 0 & \cdots & 0 \\
    0 & x_2 & & \vdots \\
    \vdots & & \ddots & 0 \\
    0 & \cdots & 0 & x_n
  \end{bmatrix}
\end{equation*}

A block diagonal matrix has matrices along its diagonal. $\diag()$ works
similarly for constructing one. Let
$\mat{A} = \begin{bsmallmatrix}1 & 2\\3 & 4\end{bsmallmatrix}$ and
$\mat{B} = \begin{bsmallmatrix}1 & 2 & 3\\4 & 5 & 6\\7 & 8 & 9\end{bsmallmatrix}$.
\begin{equation*}
  \diag(\mat{A}, \mat{B}) =
  \begin{bmatrix}
    \mat{A} & \mat{0} \\
    \mat{0} & \mat{B}
  \end{bmatrix} =
  \begin{bmatrix}
    1 & 2 & 0 & 0 & 0 \\
    3 & 4 & 0 & 0 & 0 \\
    0 & 0 & 1 & 2 & 3 \\
    0 & 0 & 4 & 5 & 6 \\
    0 & 0 & 7 & 8 & 9
  \end{bmatrix}
\end{equation*}

Operations on the $\diag()$ argument are applied element-wise.
\begin{equation*}
  \diag\left(\frac{1}{\mat{x}^2}\right) =
  \begin{bmatrix}
    \frac{1}{x_1^2} & 0 & \cdots & 0 \\
    0 & \frac{1}{x_2^2} & & \vdots \\
    \vdots & & \ddots & 0 \\
    0 & \cdots & 0 & \frac{1}{x_n^2}
  \end{bmatrix}
\end{equation*}

\subsubsection{Trace}
\index{matrices!trace}
$\tr(\mat{A})$ denotes the trace of the square matrix $\mat{A}$, which is
defined as the sum of the elements on the main diagonal (top-left to
bottom-right).

\section{Matrix definiteness}
\begin{booktable}
  \begin{tabular}{|cccc|}
    \hline
    \rowcolor{headingbg}
    \textbf{Type} & \textbf{Eigenvalues} & \textbf{Notation} &
      \textbf{Definition} \\
    \hline
    Negative definite &
      All $\lambda < 0$ &
      $\mat{M} < \mat{0}$ &
      $\mat{x}\T\mat{M}\mat{x} < 0$ for all $\mat{x}$ \\
    Negative semidefinite &
      All $\lambda \leq 0$ &
      $\mat{M} \leq \mat{0}$ &
      $\mat{x}\T\mat{M}\mat{x} \leq 0$ for all $\mat{x}$ \\
    Positive semidefinite &
      All $\lambda \geq 0$ &
      $\mat{M} \geq \mat{0}$ &
      $\mat{x}\T\mat{M}\mat{x} \geq 0$ for all $\mat{x}$ \\
    Positive definite &
      All $\lambda > 0$ &
      $\mat{M} > \mat{0}$ &
      $\mat{x}\T\mat{M}\mat{x} > 0$ for all $\mat{x}$ \\
    Indefinite &
      Positive and negative &
      N/A &
      $\exists \mat{x}, \mat{y} \ni \mat{x}\T\mat{M}\mat{x} < 0 < \mat{y}\T\mat{M}\mat{y}$ \\
    \hline
  \end{tabular}
  \caption{Types of matrix definiteness. Let $\mat{M}$ be a matrix and let
    $\mat{x}$ and $\mat{y}$ be nonzero vectors.}
\end{booktable}

\section{Common control theory matrix equations}

Here's some common matrix equations from control theory we'll use later on.
Solvers for them exist in \href{https://github.com/RobotLocomotion/drake}{Drake}
(C++) and \href{https://github.com/scipy/scipy}{SciPy} (Python).

\subsection{Continuous algebraic Riccati equation (CARE)}
\index{algebraic Riccati equation!continuous}

The continuous algebraic Riccati equation (DARE) appears in the solution to the
continuous time LQ problem.
\begin{equation}
  \mat{A}\mat{X} + \mat{X}\mat{A} - \mat{X}\mat{B}\mat{R}^{-1}\mat{B}\T\mat{X} +
    \mat{Q} = 0
\end{equation}

\subsection{Discrete algebraic Riccati equation (DARE)}
\index{algebraic Riccati equation!discrete}

The discrete algebraic Riccati equation (CARE) appears in the solution to the
discrete time LQ problem.
\begin{equation}
  \mat{X} = \mat{A}\T\mat{X}\mat{A} - (\mat{A}\T\mat{X}\mat{B})(\mat{R} +
    \mat{B}\T\mat{X}\mat{B})^{-1} \mat{B}\T\mat{X}\mat{A} + \mat{Q}
\end{equation}

\subsection{Continuous Lyapunov equation}
\index{Lyapunov equation!continuous}

The continuous Lyapunov equation appears in controllability/observability
analysis of continuous time systems.
\begin{equation}
  \mat{A}\mat{X} + \mat{X}\mat{A}\T + \mat{Q} = 0
\end{equation}

\subsection{Discrete Lyapunov equation}
\index{Lyapunov equation!discrete}

The discrete Lyapunov equation appears in controllability/observability analysis
of discrete time systems.
\begin{equation}
  \mat{A}\mat{X}\mat{A}\T - \mat{X} + \mat{Q} = 0
\end{equation}

\section{Matrix calculus}
\label{sec:matrix_calculus}

Matrix calculus uses partial derivatives. See subsection
\ref{subsec:partial_derivatives} for how they work.

We'll need a vector-valued function to demonstrate some common operations in
matrix calculus. Let $\mat{f}(\mat{x}, \mat{u})$ be a vector-valued function
defined as
\begin{equation*}
  \mat{f}(\mat{x}, \mat{u}) =
  \begin{bmatrix}
    f_1(\mat{x}, \mat{u}) \\
    \vdots \\
    f_m(\mat{x}, \mat{u})
  \end{bmatrix}
  \text{where }
  \mat{x} =
  \begin{bmatrix}
    x_1 \\
    \vdots \\
    x_m
  \end{bmatrix}
  \text{and }
  \mat{u} =
  \begin{bmatrix}
    u_1 \\
    \vdots \\
    u_n
  \end{bmatrix}
\end{equation*}

\subsection{Jacobian}
\index{matrices!Jacobian}

The Jacobian is the first-order partial derivative of a vector-valued function
with respect to one of its vector arguments. The columns of the Jacobian of
$\mat{f}$ are filled with partial derivatives of $\mat{f}$'s rows with respect
to each of the argument's elements. For example, the Jacobian of $\mat{f}$ with
respect to $\mat{x}$ is
\begin{equation*}
  \frac{\partial \mat{f}(\mat{x}, \mat{u})}{\partial \mat{x}} =
  \begin{bmatrix}
    \frac{\partial \mat{f}(\mat{x}, \mat{u})}{\partial x_1} & \hdots &
      \frac{\partial \mat{f}(\mat{x}, \mat{u})}{\partial x_m}
  \end{bmatrix} =
  \begin{bmatrix}
    \frac{\partial f_1}{\partial x_1} & \hdots &
      \frac{\partial f_1}{\partial x_m} \\
    \vdots & \ddots & \vdots \\
    \frac{\partial f_m}{\partial x_1} & \hdots &
      \frac{\partial f_m}{\partial x_m}
  \end{bmatrix}
\end{equation*}

$\frac{\partial f_1}{\partial x_1}$ is the partial derivative of the first row
of $\mat{f}$ with respect to the first row of $\mat{x}$, and so on for all rows
of $\mat{f}$ and $\mat{x}$. This has $m^2$ permutations and thus produces a
square matrix.

The Jacobian of $\mat{f}$ with respect to $\mat{u}$ is
\begin{equation*}
  \frac{\partial \mat{f}(\mat{x}, \mat{u})}{\partial \mat{u}} =
  \begin{bmatrix}
    \frac{\partial \mat{f}(\mat{x}, \mat{u})}{\partial u_1} & \hdots &
      \frac{\partial \mat{f}(\mat{x}, \mat{u})}{\partial u_n}
  \end{bmatrix} =
  \begin{bmatrix}
    \frac{\partial f_1}{\partial u_1} & \hdots &
      \frac{\partial f_1}{\partial u_n} \\
    \vdots & \ddots & \vdots \\
    \frac{\partial f_m}{\partial u_1} & \hdots &
      \frac{\partial f_m}{\partial u_n}
  \end{bmatrix}
\end{equation*}

$\frac{\partial f_1}{\partial u_1}$ is the partial derivative of the first row
of $\mat{f}$ with respect to the first row of $\mat{u}$, and so on for all rows
of $\mat{f}$ and $\mat{u}$. This has $m \times n$ permutations and can produce a
nonsquare matrix if $m \neq n$.

\subsection{Hessian}
\index{matrices!Hessian}

The Hessian is the second-order partial derivative of a vector-valued function
with respect to one of its vector arguments. For example, the Hessian of
$\mat{f}$ with respect to $\mat{x}$ is
\begin{equation*}
  \frac{\partial^2 \mat{f}(\mat{x}, \mat{u})}{\partial \mat{x}^2} =
  \begin{bmatrix}
    \frac{\partial^2 \mat{f}(\mat{x}, \mat{u})}{\partial x_1^2} & \hdots &
      \frac{\partial^2 \mat{f}(\mat{x}, \mat{u})}{\partial x_m^2}
  \end{bmatrix} =
  \begin{bmatrix}
    \frac{\partial^2 f_1}{\partial x_1^2} & \hdots &
      \frac{\partial^2 f_1}{\partial x_m^2} \\
    \vdots & \ddots & \vdots \\
    \frac{\partial^2 f_m}{\partial x_1^2} & \hdots &
      \frac{\partial^2 f_m}{\partial x_m^2}
  \end{bmatrix}
\end{equation*}

and the Hessian of $\mat{f}$ with respect to $\mat{u}$ is
\begin{equation*}
  \frac{\partial^2 \mat{f}(\mat{x}, \mat{u})}{\partial \mat{u}^2} =
  \begin{bmatrix}
    \frac{\partial^2 \mat{f}(\mat{x}, \mat{u})}{\partial u_1^2} & \hdots &
      \frac{\partial^2 \mat{f}(\mat{x}, \mat{u})}{\partial u_n^2}
  \end{bmatrix} =
  \begin{bmatrix}
    \frac{\partial^2 f_1}{\partial u_1^2} & \hdots &
      \frac{\partial^2 f_1}{\partial u_n^2} \\
    \vdots & \ddots & \vdots \\
    \frac{\partial^2 f_m}{\partial u_1^2} & \hdots &
      \frac{\partial^2 f_m}{\partial u_n^2}
  \end{bmatrix}
\end{equation*}

