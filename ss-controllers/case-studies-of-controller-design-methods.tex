\section{Case studies of controller design methods}

This example uses the following second-order model for a CIM motor (a DC brushed
motor).

\begin{align*}
  \begin{array}{cccc}
    \mtx{A} = \begin{bmatrix}
      -\frac{b}{J} & \frac{K_t}{J} \\
      -\frac{K_e}{L} & -\frac{R}{L}
    \end{bmatrix} &
    \mtx{B} = \begin{bmatrix}
      0 \\
      \frac{1}{L}
    \end{bmatrix} &
    \mtx{C} = \begin{bmatrix}
      1 & 0
    \end{bmatrix} &
    \mtx{D} = \begin{bmatrix}
      0
    \end{bmatrix}
  \end{array}
\end{align*}

Figure \ref{fig:case_study_pp_lqr} shows the response using poles placed at
$(0.1, 0)$ and $(0.9, 0)$ and LQR with the following cost matrices.

\begin{align*}
  \begin{array}{cc}
    \mtx{Q} = \begin{bmatrix}
      \frac{1}{20^2} & 0 \\
      0 & \frac{1}{40^2}
    \end{bmatrix} &
    \mtx{R} = \begin{bmatrix}
      \frac{1}{12^2}
    \end{bmatrix}
  \end{array}
\end{align*}

\begin{svg}{build/code/case_study_pp_lqr}
  \caption{Second-order CIM motor response with pole placement and LQR}
  \label{fig:case_study_pp_lqr}
\end{svg}

LQR selected poles at $(0.593, 0)$ and $(0.955, 0)$. Notice with pole placement
that as the current pole moves left, the control effort becomes more aggressive.
