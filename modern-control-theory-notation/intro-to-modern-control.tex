\chapterimage{intro-to-modern-control.jpg}{OPERS field at UCSC}

\chapter{Introduction to modern control}

\begin{remark}
  Chapters from here on use Python Control to demonstrate the concepts discussed
  and perform the complex math required. See appendix
  \ref{ch:installing_python_control} for how to install it.
\end{remark}

Modern control theory uses state-space representation to model and control
systems. State-space representation models \glspl{system} as a set of
\gls{state}, \gls{input}, and \gls{output} variables related by first-order
differential equations that describe how the \gls{system}'s \gls{state} changes
over time given the current \glspl{state} and \glspl{input}.

\renewcommand*{\chapterpath}{\partpath/intro-to-modern-control}
\section{From PID control to model-based control}
\index{PID control}

As mentioned before, controls engineers have a more general framework to
describe control theory than just PID control. PID controller designers are
focused on fiddling with controller parameters relating to the current, past,
and future \gls{error} rather than the underlying system \glspl{state}. Integral
control is a commonly used tool, and some people use integral action as the
majority of the control action. While this approach works in a lot of
situations, it is an incomplete view of the world.

Model-based control has a completely different mindset. Controls designers using
model-based control care about developing an accurate \gls{model} of the
\gls{system}, then driving the \glspl{state} they care about to zero (or to a
\gls{reference}). Integral control is added with $u_{error}$ estimation if
needed to handle \gls{model} uncertainty, but we prefer not to use it because
its response is hard to tune and some of its destabilizing dynamics aren't
visible during simulation.

Why use model-based control in FRC? Poor build season schedule management often
leads to the software team:
\begin{enumerate}
  \item Not getting enough time to verify basic functionality and test/tune
    feedback controllers.
  \item Spending dedicated software testing time troubleshooting
    mechanical/electrical issues within recently integrated subsystems instead.
\end{enumerate}

Model-based control (one of the focuses of this book) avoids both problems
because it lets software teams test basic functionality in simulation much
earlier in the build season and tune their feedback controllers automatically.

\section{What is a dynamical system?}

A dynamical system is a \gls{system} whose motion varies according to a set of
differential equations. A dynamical system is considered \textit{linear} if the
differential equations describing its dynamics consist only of linear operators.
Linear operators are things like constant gain multiplications, derivatives, and
integrals. You can define reasonably accurate linear \glspl{model} for pretty
much everything you'll see in FRC with just those relations.

But let's say you have a DC motor hooked up to a power supply and you applied a
constant voltage to it from rest. The motor approaches a steady-state angular
velocity, but the shape of the angular velocity curve over time isn't a line. In
fact, it's a decaying exponential curve akin to
\begin{equation*}
  \omega = \omega_{max}\left(1 - e^{-t}\right)
\end{equation*}

where $\omega$ is the angular velocity and $\omega_{max}$ is the maximum angular
velocity. If DC motors are said to behave linearly, then why is this?

Linearity refers to a \gls{system}'s equations of motion, not its time domain
response. The equation defining the motor's change in angular velocity over time
looks like
\begin{equation*}
  \dot{\omega} = -a\omega + bV
\end{equation*}

where $\dot{\omega}$ is the derivative of $\omega$ with respect to time, $V$ is
the input voltage, and $a$ and $b$ are constants specific to the motor. This
equation, unlike the one shown before, is actually linear because it only
consists of multiplications and additions relating the \gls{input} $V$ and
current \gls{state} $\omega$.

Also of note is that the relation between the input voltage and the angular
velocity of the output shaft is a linear regression. You'll see why if you model
a DC motor as a voltage source and generator producing back-EMF (in the equation
above, $bV$ corresponds to the voltage source and $-a\omega$ corresponds to the
back-EMF). As you increase the input voltage, the back-EMF increases linearly
with the motor's angular velocity. If there was a friction term that varied with
the angular velocity squared (air resistance is one example), the relation from
input to output would be a curve. Friction that scales with just the angular
velocity would result in a lower maximum angular velocity, but because that term
can be lumped into the back-EMF term, the response is still linear.

\section{State-space notation}

\subsection{What is state-space?}

Recall from last chapter that 2D space has two axes: $x$ and $y$. We represent
locations within this space as a pair of numbers packaged in a vector, and each
coordinate is a measure of how far to move along the corresponding axis.
State-space is a Cartesian coordinate system with an axis for each \gls{state}
variable, and we represent locations within it the same way we do for 2D space:
with a list of numbers in a vector. Each element in the vector corresponds to a
\gls{state} of the \gls{system}.

In addition to the \gls{state}, \glspl{input} and \glspl{output} are represented
as vectors. Since the mapping from the current \glspl{state} and \glspl{input}
to the change in \gls{state} is a system of equations, it's natural to write it
in matrix form.

\subsection{Benefits over classical control}

State-space notation provides a more convenient and compact way to model and
analyze \glspl{system} with multiple \glspl{input} and \glspl{output}. For a
\gls{system} with $p$ \glspl{input} and $q$ \glspl{output}, we would have to
write $q \times p$ transfer functions to represent it. Not only is the resulting
algebra unwieldy, but it only works for linear \glspl{system}. Including nonzero
initial conditions complicates the algebra even more. State-space representation
uses the time domain instead of the Laplace domain, so it can model nonlinear
\glspl{system}\footnote{This book focuses on analysis and control of linear
\glspl{system}. See chapter \ref{ch:nonlinear_control} for more on nonlinear
control.} and trivially supports nonzero initial conditions.

If modern control theory is so great and classical control theory isn't needed
to use it, why learn classical control theory at all? We teach classical control
theory because it provides a framework within which to understand results from
the mathematical machinery of modern control as well as vocabulary with which to
communicate that understanding. For example, faster poles (poles moved to the
left in the s-plane) mean faster decay, and oscillation means there is at least
one pair of complex conjugate poles. Not only can you describe what happened
succinctly, but you know why it happened from a theoretical perspective.

\subsection{Definition}

Below are the continuous and discrete versions of state-space notation.

\begin{definition}[State-space notation]%
  \index{State-space controllers!open-loop}

  \begin{align}
    \dot{\mtx{x}} &= \mtx{A}\mtx{x} + \mtx{B}\mtx{u} \label{eq:ss_ctrl_x} \\
    \mtx{y} &= \mtx{C}\mtx{x} + \mtx{D}\mtx{u} \label{eq:ss_ctrl_y} \\
    \nonumber \\
    \mtx{x}_{k+1} &= \mtx{A}\mtx{x}_k + \mtx{B}\mtx{u}_k \label{eq:ssz_ctrl_x} \\
    \mtx{y}_k &= \mtx{C}\mtx{x}_k + \mtx{D}\mtx{u}_k \label{eq:ssz_ctrl_y}
  \end{align}

  \begin{figurekey}
    \begin{tabular}{llll}
      $\mtx{A}$ & system matrix      & $\mtx{x}$ & state vector \\
      $\mtx{B}$ & input matrix       & $\mtx{u}$ & input vector \\
      $\mtx{C}$ & output matrix      & $\mtx{y}$ & output vector \\
      $\mtx{D}$ & feedthrough matrix &  &  \\
    \end{tabular}
  \end{figurekey}
\end{definition}

\begin{booktable}
  \begin{tabular}{|ll|ll|}
    \hline
    \rowcolor{headingbg}
    \textbf{Matrix} & \textbf{Rows $\times$ Columns} &
    \textbf{Matrix} & \textbf{Rows $\times$ Columns} \\
    \hline
    $\mtx{A}$ & states $\times$ states & $\mtx{x}$ & states $\times$ 1 \\
    $\mtx{B}$ & states $\times$ inputs & $\mtx{u}$ & inputs $\times$ 1 \\
    $\mtx{C}$ & outputs $\times$ states & $\mtx{y}$ & outputs $\times$ 1 \\
    $\mtx{D}$ & outputs $\times$ inputs &  &  \\
    \hline
  \end{tabular}
  \caption{State-space matrix dimensions}
  \label{tab:ss_matrix_dims}
\end{booktable}

In the continuous case, the change in \gls{state} and the \gls{output} are
linear combinations of the \gls{state} vector and the \gls{input} vector. The
$\mtx{A}$ and $\mtx{B}$ matrices are used to map the \gls{state} vector
$\mtx{x}$ and the \gls{input} vector $\mtx{u}$ to a change in the \gls{state}
vector $\dot{\mtx{x}}$. The $\mtx{C}$ and $\mtx{D}$ matrices are used to map the
\gls{state} vector $\mtx{x}$ and the \gls{input} vector $\mtx{u}$ to an
\gls{output} vector $\mtx{y}$.

\section{Controllability}
\index{controller design!controllability}

A \gls{system} is controllable if it can be steered from any \gls{state} to any
\gls{state} by a finite sequence of admissible \glspl{input}.
\begin{theorem}[Controllability]
  A continuous \gls{time-invariant} linear state-space \gls{model} is
  controllable if and only if
  \begin{equation}
    \rank\left(
    \begin{bmatrix}
      \mat{B} & \mat{A}\mat{B} & \cdots & \mat{A}^{n-1}\mat{B}
    \end{bmatrix}
    \right) = n
    \label{eq:ctrl_rank}
  \end{equation}

  where rank is the number of linearly independent rows in a matrix and $n$ is
  the number of \gls{state} variables.
\end{theorem}

The matrix in equation \eqref{eq:ctrl_rank} being rank-deficient means the
\glspl{input} cannot apply transforms along all axes in the state-space; the
transformation the matrix represents is collapsed into a lower dimension.

The condition number of the controllability matrix $\mathcal{C}$ is defined as
$\frac{\sigma_{max}(\mathcal{C})}{\sigma_{min}(\mathcal{C})}$ where
$\sigma_{max}$ is the maximum singular
value\footnote{\label{footn:singular_val}Singular values are a generalization of
eigenvalues for nonsquare matrices.} and $\sigma_{min}$ is the minimum singular
value. As this number approaches infinity, one or more of the \glspl{state}
becomes uncontrollable. This number can also be used to tell us which actuators
are better than others for the given \gls{system}; a lower condition number
means that the actuators have more control authority.

\section{Observability}
\index{controller design!observability}

A \gls{system} is observable if the \gls{state}, whatever it may be, can be
inferred from a finite sequence of \glspl{output}.

Observability and controllability are mathematical duals; controllability proves
that a sequence of \glspl{input} exists that drives the \gls{system} to any
\gls{state}, and observability proves that a sequence of \glspl{output} exists
that drives the \gls{state} estimate to any true \gls{state}.
\begin{theorem}[Observability]
  A continuous \gls{time-invariant} linear state-space \gls{model} is observable
  if and only if
  \begin{equation}
    \rank\left(
    \begin{bmatrix}
      \mat{C} \\
      \mat{C}\mat{A} \\
      \vdots \\
      \mat{C}\mat{A}^{n-1}
    \end{bmatrix}\right) = n \label{eq:obsv_rank}
  \end{equation}

  where rank is the number of linearly independent rows in a matrix and $n$ is
  the number of \gls{state} variables.
\end{theorem}

The matrix in equation \eqref{eq:obsv_rank} being rank-deficient means the
\glspl{output} do not contain contributions from every \gls{state}. That is, not
all \glspl{state} are mapped to a linear combination in the \gls{output}.
Therefore, the \glspl{output} alone are insufficient to estimate all the
\glspl{state}.

The condition number of the observability matrix $\mathcal{O}$ is defined as
$\frac{\sigma_{max}(\mathcal{O})}{\sigma_{min}(\mathcal{O})}$ where
$\sigma_{max}$ is the maximum singular value\footref{footn:singular_val} and
$\sigma_{min}$ is the minimum singular value. As this number approaches
infinity, one or more of the \glspl{state} becomes unobservable. This number can
also be used to tell us which sensors are better than others for the given
\gls{system}; a lower condition number means the \glspl{output} produced by the
sensors are better indicators of the \gls{system} \gls{state}.

