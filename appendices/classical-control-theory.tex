\chapterimage{appendices.jpg}{Sunset in an airplane over New Mexico}

\chapter{Classical control theory}

This appendix is provided for those who are curious about a lower-level
interpretation of control systems. It describes what a transfer function is and
shows how they can be used to analyze dynamical systems. Emphasis is placed on
the geometric intuition of this analysis rather than the frequency domain math.
Many tools exclusive to classical control theory (root locus, Bode plots,
Nyquist plots, etc.) aren't useful for or relevant to the examples presented in
the main chapters, so they would serve only to complicate the learning process.

Classical control theory is interesting in that one can perform stability and
robustness analyses and design reasonable controllers for systems on the back of
a napkin. It's also useful for controlling systems which don't have a model. One
can generate a Bode plot of a system by feeding in sine waves of increasing
frequency and recording the amplitude of the output oscillations. This data can
be used to create a transfer function or lead and lag compensators can be
applied directly based on the Bode plot. However, computing power is much more
plentiful nowadays; we should take advantage of this in the design phase and use
the more modern tools that enables when it makes sense.

\renewcommand*{\chapterpath}{\partpath/classical-control-theory}
\section{Classical vs modern control theory}

State-space notation provides a more convenient and compact way to model and
analyze \glspl{system} with multiple \glspl{input} and \glspl{output}. For a
\gls{system} with $p$ \glspl{input} and $q$ \glspl{output}, we would have to
write $q \times p$ transfer functions to represent it. Not only is the resulting
algebra unwieldy, but it only works for linear \glspl{system}. Including nonzero
initial conditions complicates the algebra even more. State-space representation
uses the time domain instead of the Laplace domain, so it can model nonlinear
\glspl{system}\footnote{This book primarily focuses on analysis and control of
linear \glspl{system}. See chapter \ref{ch:nonlinear_control} for more on
nonlinear control.} and trivially supports nonzero initial conditions.

If modern control theory is so great and classical control theory isn't needed
to use it, why learn classical control theory at all? We teach classical control
theory because it provides a framework within which to understand results from
the mathematical machinery of modern control as well as vocabulary with which to
communicate that understanding. For example, faster poles (poles moved to the
left in the s-plane) mean faster decay, and oscillation means there is at least
one pair of complex conjugate poles. Not only can you describe what happened
succinctly, but you know why it happened from a theoretical perspective.

This book uses LQR and modern control over, say, loop shaping with Bode and
Nyquist plots because we have accurate dynamical models to leverage, and LQR
allows directly expressing what the author is concerned with optimizing: state
excursion relative to control effort. Applying lead and lag compensators, while
effective for robust controller design, doesn't provide the same expressive
power.

\chapterimage{transfer-functions.jpg}{Treeline by Crown/Merril bus stop at UCSC}

\chapter{Transfer functions}

This chapter briefly discusses what transfer functions are, how the locations of
poles and zeroes affect \gls{system response} and stability, and how controllers
affect pole locations. As long as you gain understanding of those concepts,
don't worry too much about not being able to follow the math presented here; we
won't use transfer functions in modern control theory (it's mainly provided for
completeness). This chapter is intended to provide a framework within which to
understand results from the mathematical machinery of modern control as well as
vocabulary to communicate that understanding.

\section{Laplace transform}

For an introduction to Laplace transforms and the geometric intuition behind
transfer functions, we recommend watching Zach Star's video ``What does the
Laplace Transform really tell us? A visual explanation (plus applications)" (21
minutes) \cite{bib:laplace_transform}. An optional, more mathematical
introduction is presented in appendix \ref{ch:laplace-domain-analysis} for
completeness.

\renewcommand*{\chapterpath}{\partpath/transfer-functions}
\section{Parts of a transfer function}

A transfer function maps an input coordinate to an output coordinate in the
Laplace domain. These can be obtained by applying the Laplace transform to a
differential equation and rearranging the terms to obtain a ratio of the output
variable to the input variable. Equation \eqref{eq:transfer_func} is an example
of a transfer function.

\begin{equation} \label{eq:transfer_func}
  H(s) = \frac{\overbrace{(s-9+9i)(s-9-9i)}^{zeroes}}
    {\underbrace{s(s+10)}_{poles}}
\end{equation}

\subsection{Poles and zeroes}

The roots of factors in the numerator of a transfer function are called
\textit{zeroes} because they make the transfer function approach zero. Likewise,
the roots of factors in the denominator of a transfer function are called
\textit{poles} because they make the transfer function approach infinity; on a
3D graph, these look like the poles of a circus tent (see figure
\ref{fig:tf_3d}).

When the factors of the denominator are broken apart using partial fraction
expansion into something like $\frac{A}{s + a} + \frac{B}{s + b}$, the constants
$A$ and $B$ are called residues, which determine how much each pole contributes
to the \gls{system response}.

The factors representing poles are each the Laplace transform of a decaying
exponential\footnote{We are handwaving Laplace transform derivations because
they are complicated and neither relevant nor useful.}. That means the time
domain responses of \glspl{system} comprise decaying exponentials (e.g.,
$y = e^{-t}$).

\begin{svg}{build/\chapterpath/tf_3d}
  \caption{Equation \eqref{eq:transfer_func} plotted in 3D}
  \label{fig:tf_3d}
\end{svg}

\begin{remark}
  Imaginary poles and zeroes always come in complex conjugate pairs (e.g.,
  $-2 + 3i$, $-2 - 3i$).
\end{remark}

\index{stability!poles and zeroes}
The locations of the closed-loop poles in the complex plane determine the
stability of the \gls{system}. Each pole represents a frequency mode of the
\gls{system}, and their location determines how much of each response is induced
for a given input frequency. Figure \ref{fig:impulse_response_poles} shows the
\glspl{impulse response} in the time domain for transfer functions with various
pole locations. They all have an initial condition of $1$.

\begin{bookfigure}
  \begin{tikzpicture}[auto, >=latex']
    % \draw [help lines] (-4,-2) grid (4,4);

    % Draw main axes
    \draw[->] (-4,0) -- (4,0) node[below] {\small Re($\sigma$)};
    \draw[->] (0,-2) -- (0,4) node[right] {\small Im($j\omega$)};

    % Stable: e^-1.75t * cos(1.75wt) (80/3*w for readability)
    \drawtimeplot{-2.125cm}{2.5cm}{0.125cm}{0.44375cm}{
      exp(-1.75 * \x) * cos(80/3 * 1.75 * deg(\x))}
    \drawpole{-1.75cm}{1.75cm}

    % Stable: e^-2.5t
    \drawtimeplot{-2.25cm}{0.75cm}{0.125cm}{0.125cm}{exp(-2 * \x)}
    \drawpole{-2cm}{0cm}

    % Stable: e^-t
    \drawtimeplot{-1.125cm}{-0.75cm}{0.125cm}{0.125cm}{exp(-\x)}
    \drawpole{-1cm}{0cm}

    % Marginally stable: cos(wt) (80/3*w for readability)
    \drawtimeplot{-0.75cm}{1.125cm}{0.125cm}{0.44375cm}{cos(80/3 * deg(\x))}
    \drawpole{0cm}{1cm}

    % Marginally stable cos(2wt) (80/3*w for readability)
    \drawtimeplot{0cm}{2.75cm}{0.125cm}{0.44375cm}{cos(80/3 * 2 * deg(\x))}
    \drawpole{0cm}{2cm}

    % Integrator
    \drawtimeplot{0.25cm}{-0.75cm}{0.125cm}{0.125cm}{1}
    \drawpole{0cm}{0cm}

    % Unstable: e^t
    \drawtimeplot{1.125cm}{0.75cm}{0.125cm}{0.125cm}{exp(\x)}
    \drawpole{1cm}{0cm}

    % Unstable: e^2t
    \drawtimeplot{2.25cm}{-0.75cm}{0.125cm}{0.125cm}{exp(2 * \x)}
    \drawpole{2cm}{0cm}

    % Unstable: e^0.75t * cos(1.75wt) (80/3*w for readability)
    \drawtimeplot{1.5cm}{2.25cm}{0.125cm}{0.44375cm}{
      exp(0.75 * \x) * cos(80/3 * 1.75 * deg(\x))}
    \drawpole{0.75cm}{1.75cm}

    % LHP and RHP labels
    \draw (-3.5,1.5) node {LHP};
    \draw (3.5,1.5) node {RHP};

    % Stable and unstable labels
    \draw (-2.5,3.5) node {\small Stable};
    \draw (2.5,3.5) node {\small Unstable};
  \end{tikzpicture}

  \caption{Impulse response vs pole location}
  \label{fig:impulse_response_poles}
\end{bookfigure}

\begin{booktable}
  \begin{tabular}{|ll|}
    \hline
    \rowcolor{headingbg}
    \textbf{Location} & \textbf{Stability} \\
    \hline
    Left Half-plane (LHP) & Stable \\
    Imaginary axis & Marginally stable \\
    Right Half-plane (RHP) & Unstable \\
    \hline
  \end{tabular}

  \caption{Pole location and stability}
\end{booktable}

When a \gls{system} is stable, its output may oscillate but it converges to
steady-state. When a \gls{system} is marginally stable, its output oscillates at
a constant amplitude forever. When a \gls{system} is unstable, its output grows
without bound.

\subsection{Nonminimum phase zeroes}

While poles in the RHP are unstable, the same is not true for zeroes. They can
be characterized by the \gls{system} initially moving in the wrong direction
before heading toward the \gls{reference}. Since the poles always move toward
the zeroes, zeroes impose a ``speed limit" on the \gls{system response} because
it takes a finite amount of time to move the wrong direction, then change
directions.

One example of this type of \gls{system} is bicycle steering. Try riding a
bicycle without holding the handle bars, then poke the right handle; the bicycle
turns right. Furthermore, if one is holding the handlebars and wants to turn
left, rotating the handlebars counterclockwise will make the bicycle fall toward
the right. The rider has to lean into the turn and overpower the nonminimum
phase dynamics to go the desired direction.

Another example is a Segway. To move forward by some distance, the Segway must
first roll backward to rotate the Segway forward. Once the Segway starts falling
in that direction, it begins rolling forward to avoid falling over until
it reaches the target distance. At that point, the Segway increases its forward
speed to pitch backward and slow itself down. To come to a stop, the Segway
rolls backward again to level itself out.

\subsection{Pole-zero cancellation}
\label{subsec:pole-zero_cancellation}

Pole-zero cancellation occurs when a pole and zero are located at the same place
in the s-plane. This effectively eliminates the contribution of each to the
\gls{system} dynamics. By placing poles and zeroes at various locations (this is
done by placing transfer functions in series), we can eliminate undesired
\gls{system} dynamics. While this may appear to be a useful design tool at
first, there are major caveats. Most of these are due to \gls{model} uncertainty
resulting in poles which aren't in the locations the controls designer expected.

Notch filters are typically used to dampen a specific range of frequencies in
the \gls{system response}. If its band is made too narrow, it can still leave the
undesirable dynamics, but now you can no longer measure them in the response.
They are still happening, but they are what's called \textit{unobservable}.

Never pole-zero cancel unstable or nonminimum phase dynamics. If the \gls{model}
doesn't quite reflect reality, an attempted pole cancellation by placing a
nonminimum phase zero results in the pole still moving to the zero placed next
to it. You have the same dynamics as before, but the pole is also stuck where it
is no matter how much \gls{feedback gain} is applied. For an attempted
nonminimum phase zero cancellation, you have effectively placed an unstable pole
that's unobservable. This means the \gls{system} will be going unstable and
blowing up, but you won't be able to detect this and react to it.

Keep in mind when making design decisions that the \gls{model} likely isn't
perfect. The whole point of feedback control is to be robust to this kind of
uncertainty.

\section{Transfer functions in feedback}

For \glspl{controller} to \glslink{regulator}{regulate} a \gls{system} or
\glslink{tracking}{track} a reference, they must be placed in positive or
negative feedback with the \gls{plant} (whether to use positive or negative
depends on the \gls{plant} in question). Stable feedback loops attempt to make
the \gls{output} equal the \gls{reference}.

\begin{bookfigure}
  \begin{tikzpicture}[auto, >=latex']
    % Place the blocks
    \node [name=input] {$X(s)$};
    \node [sum, right=of input] (sum) {};
    \node [block, right=of sum] (K) {$K$};
    \node [block, right=of K] (G) {$G$};
    \node [right=of G] (output) {$Y(s)$};
    \node [block, below=of $(K)!0.5!(G)$] (H) {$H$};

    % Connect the nodes
    \draw [arrow] (input) -- node[pos=0.85] {$+$} (sum);
    \draw [arrow] (sum) -- node {} (K);
    \draw [arrow] (K) -- node {} (G);
    \draw [arrow] (G) -- node[name=y] {} (output);
    \draw [arrow] (y) |- (H);
    \draw [arrow] (H) -| node[pos=0.97, right] {$-$} (sum);
  \end{tikzpicture}

  \caption{Feedback controller block diagram}
  \label{fig:feedback_controller_block_diagram}

  \begin{figurekey}
    \begin{tabulary}{\linewidth}{LLLL}
      $X(s)$ & input & $H$ & measurement transfer function \\
      $K$ & controller gain & $Y(s)$ & output \\
      $G$ & plant transfer function & & \\
    \end{tabulary}
  \end{figurekey}
\end{bookfigure}

The transfer function of figure \ref{fig:feedback_controller_block_diagram}, a
\gls{control system} diagram with feedback, from input to output is

\begin{equation}
  G_{cl}(s) = \frac{Y(s)}{X(s)} = \frac{KG}{1 + KGH}
\end{equation}

The numerator is the \gls{open-loop gain} and the denominator is one plus the
gain around the feedback loop, which may include parts of the
\gls{open-loop gain} (see appendix \ref{sec:deriv_tf_feedback} for a
derivation). As another example, the transfer function from the input to the
\gls{error} is

\begin{equation}
  G_{cl}(s) = \frac{E(s)}{X(s)} = \frac{1}{1 + KGH}
\end{equation}

The roots of the denominator of $G_{cl}(s)$ are different from those of the
open-loop transfer function $KG(s)$. These are called the closed-loop poles.


\chapterimage{appendices.jpg}{Sunset in an airplane over New Mexico}

\chapter{Laplace domain analysis}
\label{ch:laplace-domain-analysis}

This appendix uses Laplace transforms and transfer functions to analyze
properties of control systems like \gls{steady-state error}.

These case studies cover various aspects of PID control using the algebraic
approach of transfer functions. For this, we'll be using equation
\eqref{eq:pid_tf}, the transfer function for a PID controller.
\begin{equation}
  K(s) = K_p + \frac{K_i}{s} + K_ds \label{eq:pid_tf}
\end{equation}

First, we need to define what Laplace transforms and transfer functions are,
which is rooted in the concept of orthogonal projections.

\renewcommand*{\chapterpath}{\partpath/laplace-domain-analysis}
\section{Projections}

Imagine a two-dimensional Euclidean space $\mathbb{R}^2$ (i.e., the standard x-y
plane). Ordinarily, we notate points in this plane by their components in the
set of basis vectors $\{\hat{i}, \hat{j}\}$, where $\hat{i}$ (pronounced i-hat)
is the unit vector in the $+x$ direction and $\hat{j}$ is the unit vector in the
$+y$ direction.

How do we find the coordinates of a given vector $v$ in this basis? So long as
the basis is \textit{orthogonal} (i.e., the basis vectors are at right angles to
each other), we simply take the \textit{orthogonal projection} of $v$ onto
$\hat{i}$ and $\hat{j}$. Intuitively, this means finding ``the amount of $v$
that points in the direction of $\hat{i}$ or $\hat{j}$". More formally, we can
calculate it with the dot product - the projection of $v$ onto any other vector
$w$ is equal to $\frac{v \cdot w}{|w|}$. (Since $\hat{i}$ and $\hat{j}$ are
\textit{unit vectors}, we can see simply that the coordinates of $v$ are
$v \cdot \hat{i}$ and $v \cdot \hat{j}$. We can also see that ``orthogonal" can
be defined as "has zero dot product".)

But we can use this same process to find the coordinates of $v$ in \textit{any}
orthogonal basis. For example, imagine the basis
$\{\hat{i} + \hat{j}, \hat{i} - \hat{j}\}$ - the coordinates in this basis are
given by $\frac{v \cdot (\hat{i} + \hat{j})}{\sqrt{2}}$ and
$\frac{v \cdot (\hat{i} - \hat{j})}{\sqrt{2}}$. Let us now "unwrap" the formula
for dot product and look a bit more closely.

\begin{equation*}
  \frac{v \cdot (\hat{i} + \hat{j})}{\sqrt{2}} = \frac{1}{\sqrt{2}} \sum_n v_n
    (\hat{i} + \hat{j})_n
\end{equation*}

So, what have we really done to change coordinates? We expanded both $v$ and
$\hat{i} + \hat{j}$ in a basis, multiplied their components, and added them up.

Now the previous example was only a change of coordinates in a
finite-dimensional vector space. However, as we will see, the core idea does not
change much when we move to more complicated structures. Observe the formula for
the Fourier transform.

\begin{equation*}
  \hat{f}(\xi) = \int_{-\infty}^\infty f(x) e^{-2\pi ix \xi} \,dx
    \text{ where } \xi \in \mathbb{R}
\end{equation*}

This is fundamentally the same formula we had before. $f(x)$ has taken the place
of $v_n$, $e^{-2\pi ix \xi}$ has taken the place of $(\hat{i} + \hat{j})_n$, and
the sum over $n$ has turned into an integral over $dx$, but the underlying
concept is the same. To change coordinates in a \textit{function space}, we
simply take the orthogonal projection onto our new basis \textit{functions}. In
the case of the Fourier transform, the function basis is the family of functions
of the form $f(x) = e^{-2\pi ix \xi} \text{ for } \xi \in \mathbb{R}$. Since
these functions are oscillatory at a frequency determined by $\xi$, we can think
of this as a ``frequency basis".

Now, the Laplace transform is somewhat more complicated - as it turns out, the
Fourier basis is orthogonal, so the analogy to the simpler vector space holds
almost-precisely. The Laplace transform is \textit{not} orthogonal, so we can't
interpret it \textit{strictly} as a change of coordinates in the traditional
sense. However, the intuition is the same: we are taking the orthogonal
projection of our original function onto the functions of our new basis set.

\begin{equation*}
  F(s) = \int_0^\infty f(t) e^{-st} \,dt, \text{ where } s \in \mathbb{C}
\end{equation*}

Here, it becomes obvious that the Laplace transform is a \textit{generalization}
of the Fourier transform in that the basis family is strictly larger (we have
allowed the ``frequency" parameter to take \textit{complex} values, as opposed
to merely \textit{real} values). The upshot of this is that the Laplace basis
contains functions that grow and decay, while the Fourier basis does not.

\section{Fourier transform}

The Fourier transform decomposes a function of time into its component
frequencies. Each of these frequencies is part of what's called a
\textit{basis}. These basis waveforms can be multiplied by their respective
contribution amount and summed to produce the original signal (this weighted sum
is called a linear combination). In other words, the Fourier transform provides
a way for us to determine, given some signal, what frequencies can we add
together and in what amounts to produce the original signal.

Think of an Fmajor4 chord which has the notes $F_4$ ($349.23\,Hz$), $A_4$
($440\,Hz$), and $C_4$ ($261.63\,Hz$). The waveform over time looks like figure
\ref{fig:fourier_chord}.

\begin{svg}{build/code/fourier_chord}
  \caption{Frequency decomposition of Fmajor4 chord}
  \label{fig:fourier_chord}
\end{svg}

Notice how this complex waveform can be represented just by three frequencies.
They show up as Dirac delta functions\footnote{The Dirac delta function is zero
everywhere except at the origin. The nonzero region has an infinitesimal width
and has a height such that the area within that region is $1$.} in the frequency
domain with the area underneath them equal to their contribution (see figure
\ref{fig:fourier_chord_fft}).

\begin{svg}{build/code/fourier_chord_fft}
  \caption{Fourier transform of Fmajor4 chord}
  \label{fig:fourier_chord_fft}
\end{svg}

Since Euler's identity states that $e^{i\theta} = \cos\theta + i\sin\theta$, we
can represent frequencies as complex numbers on a number line (we will use
$e^{j\omega} = \cos\omega + j\sin\omega$ for notational consistency going
forward). For example, $400\,Hz$ would be $e^{j400}$. The frequency domain just
uses the complex exponent $400j$.

\subsection{Laplace transform}

The Laplace domain is a generalization of the frequency domain that has the
frequency ($j\omega$) on the imaginary y-axis and a real number on the x-axis,
yielding a two-dimensional coordinate system. We represent coordinates in this
space as a complex number $s = \sigma + j\omega$. The real part $\sigma$
corresponds to the x-axis and the imaginary part $j\omega$ corresponds to the
y-axis (see figure \ref{fig:laplace_domain}).
\begin{bookfigure}
  \begin{tikzpicture}[auto, >=latex']
    %\draw [help lines] (-4,-2) grid (4,4);

    % Draw main axes
    \draw[<->] (-4,0) -- (4,0) node[below] {\small Re($\sigma$)};
    \draw[<->] (0,-3) -- (0,3) node[right] {\small Im($j\omega$)};
  \end{tikzpicture}

  \caption{Laplace domain}
  \label{fig:laplace_domain}
\end{bookfigure}

To extend our analogy of each coordinate being represented by some basis, we now
have the y coordinate representing the oscillation frequency of the
\gls{system response} (the frequency domain) and also the x coordinate
representing the speed at which that oscillation decays and the \gls{system}
converges to zero (i.e., a decaying exponential). Figure
\ref{fig:impulse_response_poles} shows this for various points.

If we move the component frequencies in the Fmajor4 chord example parallel to
the real axis to $\sigma = -25$, the resulting time domain response attenuates
according to the decaying exponential $e^{-25t}$ (see figure
\ref{fig:laplace_chord_attenuating}).
\begin{svg}{build/\sectionpath/laplace_chord_attenuating}
  \caption{Fmajor4 chord at $\sigma = 0$ and $\sigma = -25$}
  \label{fig:laplace_chord_attenuating}
\end{svg}

Note that this explanation as a basis isn't exact because the Laplace basis
isn't orthogonal (that is, the x and y coordinates affect each other and have
cross-talk). In the frequency domain, we had a basis of sine waves that we
represented as delta functions in the frequency domain. Each frequency
contribution was independent of the others. In the Laplace domain, this is not
the case; a pure exponential is $\frac{1}{s - a}$ (a rational function where $a$
is a real number) instead of a delta function. This function is nonzero at
points that aren't actually frequencies present in the time domain. Figure
\ref{fig:laplace_chord_3d} demonstrates this, which shows the Laplace transform
of the Fmajor4 chord plotted in 3D.
\begin{svg}{build/\sectionpath/laplace_chord_3d}
  \caption{Laplace transform of Fmajor4 chord plotted in 3D}
  \label{fig:laplace_chord_3d}
\end{svg}

Notice how the values of the function around each component frequency decrease
according to $\frac{1}{\sqrt{x^2 + y^2}}$ in the $x$ and $y$ directions (in just
the $x$ direction, it would be $\frac{1}{x}$).

\subsection{Laplace transform definition}

The Laplace transform of a function $f(t)$ is defined as
\begin{equation*}
  \mathcal{L}\{f(t)\} = F(s) = \int_0^\infty f(t) e^{-st} \,dt
\end{equation*}

We won't be computing any Laplace transforms by hand using this formula
(everyone in the real world looks these up in a table anyway). Common Laplace
transforms (assuming zero initial conditions) are shown in table
\ref{tab:common_laplace_transforms}. Of particular note are the Laplace
transforms for the derivative, unit step\footnote{The unit step $u(t)$ is
defined as $0$ for $t < 0$ and $1$ for $t \ge 0$.}, and exponential decay. We
can see that a derivative is equivalent to multiplying by $s$, and an integral
is equivalent to multiplying by $\frac{1}{s}$.
\begin{booktable}
  \begin{tabular}{|ccc|}
    \hline
    \rowcolor{headingbg}
    & \textbf{Time domain} & \textbf{Laplace domain} \\
    \hline
    Linearity & $a\,f(t) + b\,g(t)$ & $a\,F(s) + b\,G(s)$ \\
    Convolution & $(f * g)(t)$ & $F(s) \,G(s)$ \\
    Derivative & $f'(t)$ & $s \,F(s)$ \\
    $n^{th}$ derivative & $f^{(n)}(t)$ & $s^n \,F(s)$ \\
    Unit step & $u(t)$ & $\frac{1}{s}$ \\
    Ramp & $t \,u(t)$ & $\frac{1}{s^2}$ \\
    Exponential decay & $e^{-\alpha t} u(t)$ & $\frac{1}{s + \alpha}$ \\
    \hline
  \end{tabular}
  \caption{Common Laplace transforms and Laplace transform properties with zero
    initial conditions}
  \label{tab:common_laplace_transforms}
\end{booktable}

\section{Case study: steady-state error}
\index{steady-state error}

To demonstrate the problem of \gls{steady-state error}, we will use a DC brushed
motor controlled by a velocity PID controller. A DC brushed motor has a transfer
function from voltage ($V$) to angular velocity ($\dot{\theta}$) of
\begin{equation}
  G(s) = \frac{\dot{\Theta}(s)}{V(s)} = \frac{K}{(Js+b)(Ls+R)+K^2}
\end{equation}

First, we'll try controlling it with a P controller defined as
\begin{equation*}
  K(s) = K_p
\end{equation*}

When these are in unity feedback, the transfer function from the input voltage
to the error is
\begin{align*}
  \frac{E(s)}{V(s)} &= \frac{1}{1 + K(s)G(s)} \\
  E(s) &= \frac{1}{1 + K(s)G(s)} V(s) \\
  E(s) &= \frac{1}{1 + (K_p) \left(\frac{K}{(Js+b)(Ls+R)+K^2}\right)} V(s) \\
  E(s) &= \frac{1}{1 + \frac{K_p K}{(Js+b)(Ls+R)+K^2}} V(s)
\end{align*}

The steady-state of a transfer function can be found via
\begin{equation}
  \lim_{s\to0} sH(s)
\end{equation}

since steady-state has an input frequency of zero.
\begin{align}
  e_{ss} &= \lim_{s\to0} sE(s) \nonumber \\
  e_{ss} &= \lim_{s\to0} s \frac{1}{1 + \frac{K_p K}{(Js+b)(Ls+R)+K^2}} V(s)
    \nonumber \\
  e_{ss} &= \lim_{s\to0} s \frac{1}{1 + \frac{K_p K}{(Js+b)(Ls+R)+K^2}}
    \frac{1}{s} \nonumber \\
  e_{ss} &= \lim_{s\to0} \frac{1}{1 + \frac{K_p K}{(Js+b)(Ls+R)+K^2}}
    \nonumber \\
  e_{ss} &= \frac{1}{1 + \frac{K_p K}{(J(0)+b)(L(0)+R)+K^2}} \nonumber \\
  e_{ss} &= \frac{1}{1 + \frac{K_p K}{bR+K^2}} \label{eq:ss_nonzero}
\end{align}

Notice that the \gls{steady-state error} is nonzero. To fix this, an integrator
must be included in the controller.
\begin{equation*}
  K(s) = K_p + \frac{K_i}{s}
\end{equation*}

The same steady-state calculations are performed as before with the new
controller.
\begin{align*}
  \frac{E(s)}{V(s)} &= \frac{1}{1 + K(s)G(s)} \\
  E(s) &= \frac{1}{1 + K(s)G(s)} V(s) \\
  E(s) &= \frac{1}{1 + \left(K_p + \frac{K_i}{s}\right)
    \left(\frac{K}{(Js+b)(Ls+R)+K^2}\right)} \left(\frac{1}{s}\right) \\
  e_{ss} &= \lim_{s\to0} s \frac{1}{1 + \left(K_p + \frac{K_i}{s}\right)
    \left(\frac{K}{(Js+b)(Ls+R)+K^2}\right)} \left(\frac{1}{s}\right) \\
  e_{ss} &= \lim_{s\to0} \frac{1}{1 + \left(K_p + \frac{K_i}{s}\right)
    \left(\frac{K}{(Js+b)(Ls+R)+K^2}\right)} \\
  e_{ss} &= \lim_{s\to0} \frac{1}{1 + \left(K_p + \frac{K_i}{s}\right)
    \left(\frac{K}{(Js+b)(Ls+R)+K^2}\right)} \frac{s}{s} \\
  e_{ss} &= \lim_{s\to0} \frac{s}{s + \left(K_p s + K_i\right)
    \left(\frac{K}{(Js+b)(Ls+R)+K^2}\right)} \\
  e_{ss} &= \frac{0}{0 + (K_p (0) + K_i)
    \left(\frac{K}{(J(0)+b)(L(0)+R)+K^2}\right)} \\
  e_{ss} &= \frac{0}{K_i \frac{K}{bR+K^2}}
\end{align*}

The denominator is nonzero, so $e_{ss} = 0$. Therefore, an integrator is
required to eliminate \gls{steady-state error} in all cases for this
\gls{model}.

It should be noted that $e_{ss}$ in equation \eqref{eq:ss_nonzero} approaches
zero for $K_p = \infty$. This is known as a bang-bang controller. In practice,
an infinite switching frequency cannot be achieved, but it may be close enough
for some performance specifications.

\section{Case study: flywheel PID control}
\index{PID control!flywheel (modern control)}

PID controllers typically control voltage to a motor in FRC independent of the
equations of motion of that motor. For position PID control, large values of
$K_p$ can lead to overshoot and $K_d$ is commonly used to reduce overshoots.
Let's consider a flywheel controlled with a standard PID controller. Why
wouldn't $K_d$ provide damping for velocity overshoots in this case?

PID control is designed to control second-order and first-order \glspl{system}
well. It can be used to control a lot of things, but struggles when given higher
order \glspl{system}. It has three degrees of freedom. Two are used to place the
two poles of the \gls{system}, and the third is used to remove steady-state
error. With higher order \glspl{system} like a one input, seven \gls{state}
\gls{system}, there aren't enough degrees of freedom to place the \gls{system}'s
poles in desired locations. This will result in poor control.

The math for PID doesn't assume voltage, a motor, etc. It defines an output
based on derivatives and integrals of its input. We happen to use it for motors
because it actually works pretty well for it because motors are second-order
\glspl{system}.

The following math will be in continuous time, but the same ideas apply to
discrete time. This is all assuming a velocity controller.

Our simple motor model hooked up to a mass is
\begin{align}
  V &= IR + \frac{\omega}{K_v} \label{eq:steady-state_error_ss_flywheel_1} \\
  \tau &= I K_t \label{eq:steady-state_error_ss_flywheel_2} \\
  \tau &= J \frac{d\omega}{dt} \label{eq:steady-state_error_ss_flywheel_3}
\end{align}

For an explanation of where these equations come from, read section
\ref{sec:dc_brushed_motor}.

First, we'll solve for $\frac{d\omega}{dt}$ in terms of $V$.

Substitute equation \eqref{eq:steady-state_error_ss_flywheel_2} into equation
\eqref{eq:steady-state_error_ss_flywheel_1}.
\begin{align}
  V &= IR + \frac{\omega}{K_v} \nonumber \\
  V &= \left(\frac{\tau}{K_t}\right) R + \frac{\omega}{K_v} \nonumber
  \intertext{Substitute in equation
    \eqref{eq:steady-state_error_ss_flywheel_3}.}
  V &= \frac{\left(J \frac{d\omega}{dt}\right)}{K_t} R + \frac{\omega}{K_v}
    \nonumber \\
  \intertext{Solve for $\frac{d\omega}{dt}$.}
  V &= \frac{J \frac{d\omega}{dt}}{K_t} R + \frac{\omega}{K_v} \nonumber \\
  V - \frac{\omega}{K_v} &= \frac{J \frac{d\omega}{dt}}{K_t} R \nonumber \\
  \frac{d\omega}{dt} &= \frac{K_t}{JR} \left(V - \frac{\omega}{K_v}\right)
    \nonumber \\
  \underbrace{\frac{d\omega}{dt}}_{\dot{\mat{x}}} &=
    \underbrace{-\frac{K_t}{JRK_v}}_{\mat{A}} \underbrace{\omega}_{\mat{x}} +
    \underbrace{\frac{K_t}{JR}}_{\mat{B}} \underbrace{V}_{\mat{u}}
\end{align}

There's one stable open-loop pole at $-\frac{K_t}{JRK_v}$. Let's try a simple P
controller.
\begin{align*}
  \mat{u} &= \mat{K} (\mat{r} - \mat{x}) \\
  V &= K_p (\omega_{goal} - \omega)
\end{align*}

Closed-loop models have the form
$\dot{\mat{x}} = (\mat{A} - \mat{B}\mat{K})\mat{x} + \mat{B}\mat{K}\mat{r}$.
Therefore, the closed-loop poles are the eigenvalues of
$\mat{A} - \mat{B}\mat{K}$.
\begin{align*}
  \dot{\mat{x}} &= (\mat{A} - \mat{B}\mat{K})\mat{x} + \mat{B}\mat{K}\mat{r}
    \\
  \dot{\omega} &= \left(\left(-\frac{K_t}{JRK_v}\right) -
    \left(\frac{K_t}{JR}\right)(K_p)\right)\omega +
    \left(\frac{K_t}{JR}\right)(K_p)(\omega_{goal}) \\
  \dot{\omega} &= -\left(\frac{K_t}{JRK_v} + \frac{K_t K_p}{JR}\right)\omega +
    \frac{K_t K_p}{JR}\omega_{goal}
\end{align*}

This closed-loop flywheel model has one pole at
$-\left(\frac{K_t}{JRK_v} + \frac{K_t K_p}{JR}\right)$. It therefore only needs
one P controller to place that pole anywhere on the real axis. A derivative
term is unnecessary on an ideal flywheel. It may compensate for unmodeled
dynamics such as accelerating projectiles slowing the flywheel down, but that
effect may also increase recovery time; $K_d$ drives the acceleration to zero in
the undesired case of negative acceleration as well as well as the actually
desired case of positive acceleration.

This analysis assumes that the motor is well coupled to the mass and that the
time constant of the inductor is small enough that it doesn't factor into the
motor equations. The latter is a pretty good assumption, as shown by the slight
wiggle in figure \ref{fig:cs_ss_highfreq_unstable_step} compared to figure
\ref{fig:cs_ss_highfreq_stable_step}. If more mass is added to the motor
armature, the response timescales increase and the inductance matters even less.
\begin{bookfigure}
  \begin{minisvg}{2}{build/figs/highfreq_unstable_step}
    \caption{Step response of second-order DC brushed motor plant augmented with
      position ($L = 230$ μH)}
    \label{fig:cs_ss_highfreq_unstable_step}
  \end{minisvg}
  \hfill
  \begin{minisvg}{2}{build/figs/highfreq_stable_step}
    \caption{Step response of first-order DC brushed motor plant augmented with
      position}
    \label{fig:cs_ss_highfreq_stable_step}
  \end{minisvg}
\end{bookfigure}

Subsection \ref{subsec:input_error_estimation} covers a superior compensation
method that avoids zeroes in the \gls{controller}, doesn't act against the
desired control action, and facilitates better \gls{tracking}.


\section{s-plane to z-plane}

Transfer functions are converted to impulse responses using the Z-transform. The
s-plane's LHP maps to the inside of a unit circle in the z-plane. Table
\ref{tab:s2z_mapping} contains a few common points and figure
\ref{fig:s2z_mapping} shows the mapping visually.
\begin{booktable}
  \begin{tabular}{|cc|}
    \hline
    \rowcolor{headingbg}
    \textbf{s-plane} & \textbf{z-plane} \\
    \hline
    $(0, 0)$ & $(1, 0)$ \\
    imaginary axis & edge of unit circle \\
    $(-\infty, 0)$ & $(0, 0)$ \\
    \hline
  \end{tabular}
  \caption{Mapping from s-plane to z-plane}
  \label{tab:s2z_mapping}
\end{booktable}
\begin{bookfigure}
  \begin{minisvg}{2}{build/figs/s_plane}
  \end{minisvg}
  \hfill
  \begin{minisvg}{2}{build/figs/z_plane}
  \end{minisvg}
  \caption{Mapping of complex plane from s-plane (left) to z-plane (right)}
  \label{fig:s2z_mapping}
\end{bookfigure}

\subsection{Discrete system stability}

Eigenvalues of a \gls{system} that are within the unit circle are stable. To
demonstrate this, consider the discrete system $x_{k + 1} = ax_k$ where $a$ is a
complex number. $|a| < 1$ will make $x_{k + 1}$ converge to zero.

\subsection{Discrete system behavior}

Figure \ref{fig:disc_impulse_response_poles} shows the \glspl{impulse response}
in the time domain for \glspl{system} with various pole locations in the complex
plane (real numbers on the x-axis and imaginary numbers on the y-axis). Each
response has an initial condition of $1$.
\begin{bookfigure}
  \begin{tikzpicture}[auto, >=latex']
  % \draw [help lines] (-4,-4) grid (4,4);

  % Draw main axes
  \draw[->] (-4,0) -- (4,0) node[below] {\small Re};
  \draw[->] (0,-4) -- (0,4) node[right] {\small Im};

  % Unit circle
  \draw[black] (0,0) circle (3cm);

  % Exponent for the given x plot coordinate:
  %
  %   exp(at) = x
  %   at = ln(x)
  %   a = ln(x)/t
  %
  % t = 50 ms

  % LHP integrator
  \drawdiscretecoordplot{-3cm - 0.166cm}{-0.75cm}{0.125cm}{0.44375cm}
    {(0,1) (0.05,-1) (0.1,1) (0.15,-1) (0.2,1) (0.25,-1) (0.3,1) (0.35,-1)
     (0.4,1) (0.45,-1) (0.5,1)}
  \drawpole{-3cm}{0cm}

  % LHP stable: xₖ₊₁ = −2/3xₖ
  \drawdiscretecoordplot{-2cm}{0.75cm}{0.125cm}{0.44375cm}
    {(0,1) (0.05,-0.667) (0.1,0.444) (0.15,-0.296) (0.2,0.198) (0.25,-0.132)
     (0.3,0.088) (0.35,-0.059) (0.4,0.039) (0.45,-0.026) (0.5,0.017)}
  \drawpole{-2cm}{0cm}

  % LHP stable: xₖ₊₁ = −1/3xₖ
  \drawdiscretecoordplot{-1cm + 0.166cm}{-0.75cm}{0.125cm}{0.44375cm}
    {(0,1) (0.05,-0.333) (0.1,0.111) (0.15,-0.037) (0.2,0.012) (0.25,-0.004)
     (0.3,0.001) (0.35,0) (0.4,0) (0.45,0) (0.5,0)}
  \drawpole{-1cm}{0cm}

  % LHP stable: xₖ₊₁ = (0.333 + 0.5i)xₖ
  \drawdiscretecoordplot{-1cm}{2.25cm}{0.125cm}{0.44375cm}
    {(0,1) (0.05,-0.333) (0.1,-0.139) (0.15,0.213) (0.2,-0.092) (0.25,-0.016)
     (0.3,0.044) (0.35,-0.023) (0.4,0) (0.45,0.009) (0.5,-0.006)}
  \drawpole{-1cm}{1.5cm}

  % LHP unstable: xₖ₊₁ = (−1 − 0.75i)xₖ
  \drawdiscretecoordplot{-2.25cm}{-2.25cm}{0.125cm}{0.44375cm}
    {(0,1) (0.05,-1) (0.1,0.4375) (0.15,0.6875) (0.2,-2.059) (0.25,3.043)
     (0.3,-2.869) (0.35,0.984) (0.4,2.515) (0.45,-6.568) (0.5,9.206)}
  \drawpole{-3cm}{-2.25cm}

  % RHP stable: exp(-10.192t) cos(19.665wt) (40/3*w for readability)
  %
  % a = ln(0.333 + 0.5i)/0.05 = -10.192 + 19.665i
  \drawdiscretetimeplot{1cm}{2.25cm}{0.125cm}{0.44375cm}{
    exp(-10.192 * \x) * cos(40/3 * 19.665 * deg(\x))}
  \drawpole{1cm}{1.5cm}

  % RHP stable: exp(-21.97t)
  %
  % a = ln(1/3)/0.05 = -21.97
  \drawdiscretetimeplot{1cm - 0.166cm}{-0.75cm}{0.125cm}{0.125cm}{exp(-21.97 * \x)}
  \drawpole{1cm}{0cm}

  % RHP stable: exp(-8.109t)
  %
  % a = ln(2/3)/0.05 = -8.109
  \drawdiscretetimeplot{2cm}{0.75cm}{0.125cm}{0.125cm}{exp(-8.109 * \x)}
  \drawpole{2cm}{0cm}

  % RHP marginally stable: cos(15.707wt)
  %
  % a = ln(√2 + √2i)/0.05 = 15.707i
  \drawdiscretetimeplot{3 * (0.707cm + 0.25cm)}{3 * 0.707cm + 0.375cm}{0.125cm}{0.44375cm}{cos(15.707 * deg(\x))}
  \drawpole{3 * 0.707cm}{3 * 0.707cm}

  % RHP integrator
  \drawdiscretetimeplot{3cm + 0.166cm}{-0.75cm}{0.125cm}{0.125cm}{exp(0 * \x)}
  \drawpole{3cm}{0cm}

  % RHP unstable: exp(4.463t) cos(-12.87wt)
  %
  % a = ln(1 - 0.75i)/0.05 = 4.463 - 12.87i
  \drawdiscretetimeplot{2.25cm}{-2.25cm}{0.125cm}{0.44375cm}{exp(4.463 * \x) * cos(-12.87 * deg(\x))}
  \drawpole{3cm}{-2.25cm}

  % LHP and RHP labels
  \draw (-3.5,1.5) node {LHP};
  \draw (3.5,1.5) node {RHP};

  % Stable and unstable labels
  \fill[white] (-0.5,-2.25) rectangle (0.5,-1.75);
  \draw (0,-2) node {\small Stable};
  \draw (2.5,3.5) node {\small Unstable};
  \draw (-2.5,3.5) node {\small Unstable};
  \draw (-2.5,-3.5) node {\small Unstable};
  \draw (2.5,-3.5) node {\small Unstable};
\end{tikzpicture}

  \caption{Discrete impulse response vs pole location}
  \label{fig:disc_impulse_response_poles}
\end{bookfigure}

As $\omega$ increases in $s = j\omega$, a pole in the z-plane moves around the
perimeter of the unit circle. Once it hits $\frac{\omega_s}{2}$ (half the
sampling frequency) at $(-1, 0)$, the pole wraps around. This is due to poles
faster than the sample frequency folding down to below the sample frequency
(that is, higher frequency signals \textit{alias} to lower frequency ones).

Placing the poles at $(0, 0)$ produces a \textit{deadbeat controller}. An
$\rm N^{th}$-order deadbeat controller decays to the \gls{reference} in N
timesteps. While this sounds great, there are other considerations like
\gls{control effort}, \gls{robustness}, and \gls{noise immunity}.

Poles in the left half-plane cause jagged outputs because the frequency of the
\gls{system} dynamics is above the Nyquist frequency (twice the sample
frequency). The \glslink{discretization}{discretized} signal doesn't have enough
samples to reconstruct the continuous \gls{system}'s dynamics. See figures
\ref{fig:z_oscillations_1p} and \ref{fig:z_oscillations_2p} for examples.
\begin{bookfigure}
  \begin{minisvg}{2}{build/\chapterpath/z_oscillations_1p}
    \caption{Single poles in various locations in z-plane}
    \label{fig:z_oscillations_1p}
  \end{minisvg}
  \hfill
  \begin{minisvg}{2}{build/\chapterpath/z_oscillations_2p}
    \caption{Complex conjugate poles in various locations in z-plane}
    \label{fig:z_oscillations_2p}
  \end{minisvg}
\end{bookfigure}

\section{Phase loss}

However, \gls{discretization} has drawbacks. Since a microcontroller performs
discrete steps, there is a sample delay that introduces phase loss in the
controller. Phase loss is the reduction of \gls{phase margin}\footnote{See
section \ref{sec:gain_phase_margin} for an explanation of phase margin.} that
occurs in digital implementations of feedback controllers from sampling the
continuous \gls{system} at discrete time intervals. As the sample rate of the
controller decreases, the \gls{phase margin} decreases according to
$-\frac{T}{2}\omega$ where $T$ is the sample period and $\omega$ is the
frequency of the \gls{system} dynamics. Instability occurs if the
\gls{phase margin} of the \gls{system} reaches zero. Large amounts of phase loss
can make a stable controller in the continuous domain become unstable in the
discrete domain. Here are a few ways to combat this.

\begin{itemize}
  \item Run the controller with a high sample rate.
  \item Designing the controller in the analog domain with enough
    \gls{phase margin} to compensate for any phase loss that occurs as part of
    \gls{discretization}.
  \item Convert the \gls{plant} to the digital domain and design the controller
    completely in the digital domain.
\end{itemize}

