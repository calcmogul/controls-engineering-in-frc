\subsection{Projections}

Consider a two-dimensional Euclidean space $\mathbb{R}^2$ shown in figure
\ref{fig:euclidean_space_R2} (each $\mathbb{R}$ is a dimension whose domain is
the set of real numbers, so $\mathbb{R}^2$ is the standard x-y plane).
\begin{bookfigure}
  \begin{tikzpicture}[auto, >=latex']
    %\draw [help lines] (-4,-2) grid (4,4);

    % Draw main axes
    \draw[<->] (-4,0) -- (4,0) node[below] {\small $x$};
    \draw[<->] (0,-3) -- (0,3) node[right] {\small $y$};
  \end{tikzpicture}

  \caption{Euclidean space $\mathbb{R}^2$}
  \label{fig:euclidean_space_R2}
\end{bookfigure}

Ordinarily, we notate points in this plane by their components in the set of
basis vectors $\{\hat{i}, \hat{j}\}$, where $\hat{i}$ (pronounced i-hat) is the
unit vector in the positive $x$ direction and $\hat{j}$ is the unit vector in
the positive $y$ direction. Figure \ref{fig:basis_vectors_R2} shows an example
vector $\mat{v}$ in this basis.
\begin{bookfigure}
  \begin{tikzpicture}[auto, >=latex']
    %\draw [help lines] (-4,-2) grid (4,4);

    % Draw main axes
    \draw[<->] (-4,0) -- (4,0) node[below] {\small $x$};
    \draw[<->] (0,-3) -- (0,3) node[right] {\small $y$};

    % Draw vector v
    \draw[->,line width=0.4mm] (0,0) -- (2,1.5) node[above,right]
      {$\mat{v} = 2\hat{i} + 1.5\hat{j}$};

    % Draw basis
    \draw[->,line width=0.4mm] (0,0) -- node[below] {$\hat{i}$} (1,0);
    \draw[->,line width=0.4mm] (0,0) -- node[left] {$\hat{j}$} (0,1);

    % Draw unit vector measurements
    \draw[dashed,gray] (0,1.5) -- node {$2$} (2,1.5);
    \draw[dashed,gray] (2,0) -- node[right] {$1.5$} (2,1.5);
  \end{tikzpicture}

  \caption{$\mat{v}$ with basis set $\{\hat{i}, \hat{j}\}$}
  \label{fig:basis_vectors_R2}
\end{bookfigure}

How do we find the coordinates of $\mat{v}$ in this basis mathematically? As
long as the basis is \textit{orthogonal} (i.e., the basis vectors are at right
angles to each other), we simply take the \textit{orthogonal projection} of
$\mat{v}$ onto $\hat{i}$ and $\hat{j}$. Intuitively, this means finding ``the
amount of $\mat{v}$ that points in the direction of $\hat{i}$ or $\hat{j}$".
Note that a set of orthogonal vectors have a dot product of zero with respect to
each other.

More formally, we can calculate projections with the dot product - the
projection of $\mat{v}$ onto any other vector $\mat{w}$ is as follows.
\begin{equation*}
  \text{proj}_\mat{w} \mat{v} = \frac{\mat{v} \cdot \mat{w}}{|\mat{w}|}
\end{equation*}

Since $\hat{i}$ and $\hat{j}$ are \textit{unit vectors}, their magnitudes are
$1$ so the coordinates of $\mat{v}$ are $\mat{v} \cdot \hat{i}$ and
$\mat{v} \cdot \hat{j}$.

We can use this same process to find the coordinates of $\mat{v}$ in
\textit{any} orthogonal basis. For example, imagine the basis
$\{\hat{i} + \hat{j}, \hat{i} - \hat{j}\}$ - the coordinates in this basis are
given by $\frac{\mat{v} \cdot (\hat{i} + \hat{j})}{\sqrt{2}}$ and
$\frac{\mat{v} \cdot (\hat{i} - \hat{j})}{\sqrt{2}}$. Let's ``unwrap" the
formula for dot product and look a bit more closely.
\begin{equation*}
  \frac{\mat{v} \cdot (\hat{i} + \hat{j})}{\sqrt{2}} =
    \frac{1}{\sqrt{2}} \sum_{i=0}^n \mat{v}_i (\hat{i} + \hat{j})_i
\end{equation*}

where the subscript $i$ denotes which component of each vector and $n$ is the
total number of components. To change coordinates, we expanded both $\mat{v}$
and $\hat{i} + \hat{j}$ in a basis, multiplied their components, and added them
up. Here's a more concrete example. Let $\mat{v} = 2\hat{i} + 1.5\hat{j}$ from
figure \ref{fig:basis_vectors_R2}. First, we'll project $\mat{v}$ onto the
$\hat{i} + \hat{j}$ basis vector.
\begin{align*}
  \frac{\mat{v} \cdot (\hat{i} + \hat{j})}{\sqrt{2}} &=
    \frac{1}{\sqrt{2}} (2\hat{i} \cdot \hat{i} + 1.5\hat{j} \cdot \hat{j}) \\
  \frac{\mat{v} \cdot (\hat{i} + \hat{j})}{\sqrt{2}} &=
    \frac{1}{\sqrt{2}} (2 + 1.5) \\
  \frac{\mat{v} \cdot (\hat{i} + \hat{j})}{\sqrt{2}} &= \frac{3.5}{\sqrt{2}} \\
  \frac{\mat{v} \cdot (\hat{i} + \hat{j})}{\sqrt{2}} &= \frac{3.5\sqrt{2}}{2} \\
  \frac{\mat{v} \cdot (\hat{i} + \hat{j})}{\sqrt{2}} &= 1.75\sqrt{2}
\end{align*}

Next, we'll project $\mat{v}$ onto the $\hat{i} - \hat{j}$ basis vector.
\begin{align*}
  \frac{\mat{v} \cdot (\hat{i} - \hat{j})}{\sqrt{2}} &=
    \frac{1}{\sqrt{2}} (2\hat{i} \cdot \hat{i} - 1.5\hat{j} \cdot \hat{j}) \\
  \frac{\mat{v} \cdot (\hat{i} - \hat{j})}{\sqrt{2}} &=
    \frac{1}{\sqrt{2}} (2 - 1.5) \\
  \frac{\mat{v} \cdot (\hat{i} - \hat{j})}{\sqrt{2}} &= \frac{0.5}{\sqrt{2}} \\
  \frac{\mat{v} \cdot (\hat{i} - \hat{j})}{\sqrt{2}} &= \frac{0.5\sqrt{2}}{2} \\
  \frac{\mat{v} \cdot (\hat{i} - \hat{j})}{\sqrt{2}} &= 0.25\sqrt{2}
\end{align*}

Figure \ref{fig:alternative_basis} shows this result geometrically with respect
to the basis $\{\hat{i} + \hat{j}, \hat{i} - \hat{j}\}$.
\begin{bookfigure}
  \begin{tikzpicture}[auto, >=latex']
    %\draw [help lines] (-4,-2) grid (4,4);

    % Draw main axes
    \draw[<->] (-4,0) -- (4,0) node[below] {\small $x$};
    \draw[<->] (0,-3) -- (0,3) node[right] {\small $y$};

    % Draw vector v
    \draw[->,line width=0.4mm] (0,0) -- (2,1.5) node[above,right] {$\mat{v}$};

    % Draw basis
    \draw[dashed,gray] (0,0) -- (1.75,1.75);
    \draw[->,line width=0.4mm] (0,0) -- node {$\hat{i} + \hat{j}$} (1,1);
    \draw[->,line width=0.4mm] (0,0) -- node[below,left]
      {$\hat{i} - \hat{j}$} (1,-1);

    % Draw projected v measurements
    \draw[dashed,gray] (1.75,1.75) -- node {$0.25\sqrt{2}$} (2,1.5);
    \draw[dashed,gray] (0.25,-0.25) -- node[below,right]
      {$1.75\sqrt{2}$} (2,1.5);
  \end{tikzpicture}

  \caption{$\mat{v}$ with basis $\{\hat{i} + \hat{j}, \hat{i} - \hat{j}\}$}
  \label{fig:alternative_basis}
\end{bookfigure}

The previous example was only a change of coordinates in a finite-dimensional
vector space. However, as we will see, the core idea does not change much when
we move to more complicated structures. Observe the formula for the Fourier
transform.
\begin{equation*}
  \hat{f}(\xi) = \int_{-\infty}^\infty f(x) e^{-2\pi ix \xi} \,dx
    \text{ where } \xi \in \mathbb{R}
\end{equation*}

This is fundamentally the same formula we had before. $f(x)$ has taken the place
of $v_n$, $e^{-2\pi ix \xi}$ has taken the place of $(\hat{i} + \hat{j})_i$, and
the sum over $i$ has turned into an integral over $dx$, but the underlying
concept is the same. To change coordinates in a \textit{function space}, we
simply take the orthogonal projection onto our new basis \textit{functions}. In
the case of the Fourier transform, the function basis is the family of functions
of the form $f(x) = e^{-2\pi ix \xi} \text{ for } \xi \in \mathbb{R}$. Since
these functions are oscillatory at a frequency determined by $\xi$, we can think
of this as a ``frequency basis".
\begin{remark}
  Watch the ``Abstract vector spaces" video from 3Blue1Brown's \textit{Essence
  of linear algebra} series (17 minutes)
  \cite{bib:3b1b_linalg_abstract_vector_spaces} for a more geometric
  introduction to using functions as a basis.
\end{remark}

Now, the Laplace transform is somewhat more complicated - as it turns out, the
Fourier basis is orthogonal, so the analogy to the simpler vector space holds
almost-precisely. The Laplace transform is \textit{not} orthogonal, so we can't
interpret it \textit{strictly} as a change of coordinates in the traditional
sense. However, the intuition is the same: we are taking the orthogonal
projection of our original function onto the functions of our new basis set.
\begin{equation*}
  F(s) = \int_0^\infty f(t) e^{-st} \,dt, \text{ where } s \in \mathbb{C}
\end{equation*}

Here, it becomes obvious that the Laplace transform is a \textit{generalization}
of the Fourier transform in that the basis family is strictly larger (we have
allowed the ``frequency" parameter to take \textit{complex} values, as opposed
to merely \textit{real} values). As a result, the Laplace basis contains
functions that grow and decay, while the Fourier basis does not.
