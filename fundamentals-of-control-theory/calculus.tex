\chapterimage{system-modeling.jpg}{Hills by northbound freeway between Santa Maria and Ventura}

\chapter{Calculus}

This book uses derivatives and integrals occasionally to represent small changes
in values over small changes in time and the infinitesimal sum of values over
time respectively. The formulas and tables presented here are all you'll need to
carry through with the math in later chapters.

If you are interested in more information after reading this chapter,
3Blue1Brown does a fantastic job of introducing them in his \textit{Essence of
calculus} series \cite{bib:3b1b_calculus}. We recommend reading or watching
chapters 1 through 3 and 7 through 11 for a solid foundation. The Taylor series
(presented in chapter 11) will be used in chapter
\ref{ch:discrete_state-space_control}.

\renewcommand*{\chapterpath}{\partpath/calculus}
\section{Derivatives}

Derivatives are expressions for the slope of a curve at arbitrary points. Common
notations for this operation on a function like $f(x)$ include
\begin{booktable}
  \begin{tabular}{|cc|}
    \hline
    \rowcolor{headingbg}
    \multicolumn{1}{|c}{\textbf{Leibniz notation}} &
      \multicolumn{1}{c|}{\textbf{Lagrange notation}} \\
    \hline
    $\frac{d}{dx} f(x)$ & $f'(x)$ \\
    $\frac{d^2}{dx^2} f(x)$ & $f''(x)$ \\
    $\frac{d^3}{dx^3} f(x)$ & $f'''(x)$ \\
    $\frac{d^4}{dx^4} f(x)$ & $f^{(4)}(x)$ \\
    $\frac{d^n}{dx^n} f(x)$ & $f^{(n)}(x)$ \\
    \hline
  \end{tabular}
  \caption{Notation for derivatives of $f(x)$}
\end{booktable}

Lagrange notation is usually voiced as ``f prime of x", ``f double-prime of x",
etc.

\subsection{Power rule}
\begin{align*}
  f(x) &= x^n \\
  f'(x) &= nx^{n - 1}
\end{align*}

\subsection{Product rule}

This is for taking the derivative of the product of two expressions.
\begin{align*}
  h(x) &= f(x)g(x) \\
  h'(x) &= f'(x)g(x) + f(x)g'(x)
\end{align*}

\subsection{Chain rule}

This is for taking the derivative of nested expressions.
\begin{align*}
  h(x) &= f(g(x)) \\
  h'(x) &= f'(g(x)) \cdot g'(x)
\end{align*}

For example
\begin{align*}
  h(x) &= \left(3x + 2\right)^5 \\
  h'(x) &= 5\left(3x + 2\right)^4 \cdot \left(3\right) \\
  h'(x) &= 15\left(3x + 2\right)^4
\end{align*}

\subsection{Partial derivatives}
\label{subsec:partial_derivatives}

A partial derivative of a function of several variables is its derivative with
respect to one of those variables, with the others held constant (as opposed to
the total derivative, in which all variables are allowed to vary). Partial
derivatives use $\partial$ instead of $d$ in Leibniz notation.

For example, let $h(x, y) = 3xy + 2x$. For the partial derivative with respect
to $x$, $y$ is treated as a constant.
\begin{equation*}
  \frac{\partial h(x, y)}{\partial x} = 3y + 2
\end{equation*}

For the partial derivative with respect to $y$, $x$ is treated as a constant, so
the second term becomes zero.
\begin{equation*}
  \frac{\partial h(x, y)}{\partial y} = 3x
\end{equation*}

\section{Integrals}

The integral is the inverse operation of the derivative and calculates the area
under a curve. Here is an example of one based on table
\ref{tab:common_derivatives_and_integrals}.
\begin{align*}
  \int e^{at} \,dt \\
  \frac{1}{a}e^{at} + C
\end{align*}

The arbitrary constant $C$ is needed because when you take a derivative,
constants are discarded because vertical offsets don't affect the slope. When
performing the inverse operation, we don't have enough information to determine
the constant.

However, we can provide bounds for the integration.
\begin{align*}
  &\int_0^t e^{at} \,dt \\
  &\left.\left(\frac{1}{a}e^{at} + C\right)\right\vert_0^t \\
  &\left(\frac{1}{a}e^{at} + C\right) -
    \left(\frac{1}{a}e^{a \cdot 0} + C\right) \\
  &\left(\frac{1}{a}e^{at} + C\right) - \left(\frac{1}{a} + C\right) \\
  &\frac{1}{a}e^{at} + C - \frac{1}{a} - C \\
  &\frac{1}{a}e^{at} - \frac{1}{a}
\end{align*}

When we do this, the constant cancels out.

\section{Change of variables}
\label{sec:calculus_change_of_variables}

Change of variables is a technique for simplifying problems in which expressions
are replaced with new variables to make the problem more tractable. This can
mean either the problem is more straightforward or it matches a common form for
which tools for finding solutions are readily available. Here's an example of
integration which utilizes it.
\begin{equation*}
  \int \cos\omega t \,dt
\end{equation*}

Let $u = \omega t$.
\begin{align*}
  du &= \omega \,dt \\
  dt &= \frac{1}{\omega} \,du
\end{align*}

Now substitute the expressions for $u$ and $dt$ in.
\begin{align*}
  &\int \cos u \,\frac{1}{\omega} \,du \\
  &\frac{1}{\omega} \int \cos u \,du \\
  &\frac{1}{\omega} \sin u + C \\
  &\frac{1}{\omega} \sin\omega t + C
\end{align*}

Another example, which will be relevant when we actually cover state-space
notation ($\dot{\mat{x}} = \mat{A}\mat{x} + \mat{B}\mat{u}$), is a closed-loop
state-space system.
\begin{align*}
  \dot{\mat{x}} &= (\mat{A} - \mat{B}\mat{K})\mat{x} + \mat{B}\mat{K}\mat{r} \\
  \dot{\mat{x}} &= \mat{A}_{cl}\mat{x} + \mat{B}_{cl}\mat{u}_{cl}
\end{align*}

where $\mat{A}_{cl} = \mat{A} - \mat{B}\mat{K}$, $\mat{B}_{cl} = \mat{B}\mat{K}$,
and $\mat{u}_{cl} = \mat{r}$. Since it matches the form of the open-loop system,
all the same analysis tools will work with it.

\section{Tables}

\subsection{Common derivatives and integrals}
\begin{booktable}
  \begin{tabular}{|ccc|}
    \hline
    \rowcolor{headingbg}
    \multicolumn{1}{|c}{\textbf{$\int f(x) \,dx$}} &
      \multicolumn{1}{c}{\textbf{$f(x)$}} &
      \multicolumn{1}{c|}{\textbf{$f'(x)$}} \\
    \hline
    $ax$ & $a$ & $0$ \\
    $\frac{1}{2}ax^2$ & $ax$ & $a$ \\
    $\frac{1}{a + 1}x^{a + 1}$ & $x^a$ & $ax^{a - 1}$ \\
    $\frac{1}{a}e^{ax}$ & $e^{ax}$ & $ae^{ax}$ \\
    $-\cos(x)$ & $\sin(x)$ & $\cos(x)$ \\
    $\sin(x)$ & $\cos(x)$ & $-\sin(x)$ \\
    $\cos(x)$ & $-\sin(x)$ & $-\cos(x)$ \\
    $-\sin(x)$ & $-\cos(x)$ & $\sin(x)$ \\
    \hline
  \end{tabular}
  \caption{Common derivatives and integrals}
  \label{tab:common_derivatives_and_integrals}
\end{booktable}

