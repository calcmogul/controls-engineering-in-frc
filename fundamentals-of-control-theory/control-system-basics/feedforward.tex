\section{Feedforward}

Feedback control can be effective for \gls{reference} \gls{tracking} (making a
\gls{system}'s output follow a desired \gls{reference} signal), but it's a
reactionary measure; the \gls{system} won't start applying \gls{control effort}
until the \gls{system} is already behind. If we could tell the \gls{controller}
about the desired movement and required input beforehand, the \gls{system} could
react quicker and the feedback \gls{controller} could do less work. A
\gls{controller} that feeds information forward into the \gls{plant} like this
is called a \gls{feedforward controller}.

A \gls{feedforward controller} injects information about the \gls{system}'s
dynamics (like a \gls{model} does) or the desired movement. The feedforward
handles parts of the control actions we already know must be applied to make a
\gls{system} track a \gls{reference}, then feedback compensates for what we do
not or cannot know about the \gls{system}'s behavior at runtime.

There are two types of feedforwards: model-based feedforward and feedforward for
unmodeled dynamics. The first solves a mathematical model of the system for the
inputs required to meet desired velocities and accelerations. The second
compensates for unmodeled forces or behaviors directly so the feedback
controller doesn't have to. Both types can facilitate simpler feedback
controllers; we'll cover examples of each in later chapters.
